\section{Inversion formulas}
\label{sect:inversion}


\begin{defn}
Let $\delta_0$ be the function $\delta_0: \theta \ZZ \rarr \CC$
given by $\delta_0(k\theta)=\delta_{k,0}$ for $k\in\ZZ$.
Clearly $\delta_0$ belongs to \KZeven\@. Next let $T_0$ denote the map
from \HC\ into $\FZ \tens \HC$ given by tensoring with $\delta_0$ on the
left, i.e.\ $T_0(g)=\delta_0 \tens g$.
Finally, for any $k\in\ZZ$, let $P_k$ denote the map
$P_k : \FZ \rarr \FZ$ given by  $(P_k f)(l\theta)=\delta_{k,l} f(l\theta)$
for $l\in\ZZ$ and $f\in\FZ$.
\end{defn}

The following lemma is trivial but important; similarly to
lemma \ref{lemma:slice_map:parity}, it follows from (\ref{eq:def:Hcore:parity}).

\begin{lemma}
The  mapping $T_0$ maps $\Hcore_\tau$ into $\calL(\Hcore_\tau)$.
\end{lemma}

\begin{defn}
First recall definition \ref{def:YaH}\@.
Now for any $m,k\in\ZZ$, we define a linear mapping $\Htil{m}{k}$
from $\Hcore_\nu$ into $\calL(\Hcore_\tau)$ as follows:
$$ \Htil{m}{k} \;=\;
   \kq(m,k)^{-1} \: (\Gamma^{-1} \tens \Omega)^{-k}\,
            T_0\, \Lambda_\tau \, \holH{m} \, \Lambda_\nu^{-1}. $$
Observe that because of the previous lemma, $\Htil{m}{k}$ indeed ends up in
$\calL(\Hcore_\tau)$.
\end{defn}


\begin{lemma}
Take $m,k,l\in\ZZ$. Then $\YaH{m}{k} \,\Htil{m}{l} = \delta_{k,l} \,\id$ and\/
$\Htil{m}{k} \,\YaH{m}{k} = P_k \tens \id$ and hence
$$ \sum_{k\in\ZZ}\: \Htil{m}{k}\, \YaH{m}{k} X  \:=\: X $$
for any $X\in \calL(\Hcore_\tau)$.
Notice that the last formula makes sense because of lemma \ref{lemma:finite_support}\@.
\end{lemma}
{\em proof.}
Recall $\holH{m}^{2} = \id$ and observe that
$$ T_0 (\evzero \tens \id) = P_0 \tens \id
             \andspace{4em}
(\evzero \tens \id) (\Gamma^{-1} \tens \Omega)^{k-l}T_0 = \delta_{k,l} \, \id.$$



\begin{prop}  \label{prop:Eq2:Fourier:bijections}
The Fourier transforms $F_R$ and $F_L$
(cf.\ definition \ref{def:Eq2:Fourier_transforms}) are bijections from
$\Uq\!\left(\calL(\Hcore_\tau)\right)$ onto $\Aq(\Hcore_\nu)$.
Their inverses are given by
$$\begin{array}{lcr}
F_R^{-1} \left(  \alpha^l \gamma^m \,g(\gamma^*\gamma) \vertL \right)
&=&
\Upsilon \left(\Htil{m}{l-m}\,g \right) b^m   \vertXL
\\
F_R^{-1} \left(  \alpha^l (\gamma^*)^m \,g(\gamma^*\gamma) \vertL \right)
&=&
\Upsilon \left(\Htil{-m}{l+m}\,g \right) c^m  \vertXL
\\
F_L^{-1} \left(  \alpha^l \gamma^m \,g(\gamma^*\gamma) \vertL \right)
&=&
q^{-2l} \: \Upsilon \left(\Htil{m}{l-m}\,g \right) b^m  \vertXL
\\
F_L^{-1} \left(  \alpha^l (\gamma^*)^m \,g(\gamma^*\gamma) \vertL \right)
&=&
q^{-2l} \: \Upsilon \left(\Htil{-m}{l+m}\,g \right) c^m  \vertXL
\end{array} $$
\end{prop}


\begin{lemma} \label{lemma:Eq2:lambdaiszeta}
We have $\pair{1_{\Uq}}{F_{L}(x)} = \varphi(x)$
for any $x\in \Uq\!\left(\calL(\Hcore_\tau)\vertM\right)$.
\end{lemma}
\begin{proof}
Using proposition \ref{prop:moment:link_with_Hankel}\ we get for
all $X\in \calL(\Hcore_\tau)$ and $m\in \NN$ that
\begin{eqnarray*}
\lefteqn{ \left\langle 1_{\Uq} ,\: F_L\left(\vertM\Upsilon(X)\, b^m\right)
          \right\rangle}\\
&=&
\sum_{k\in\ZZ} \;  \left\langle 1_{\Uq} ,\: (q^2 \alpha)^{k+m}\, \gamma^m
               \left(\YaH{m}{k}\,X\right)(\gamma^*\gamma) \right\rangle \\
&=&
\sum_{k\in\ZZ} \; \delta_{m,0} \, q^{2k} \left(\YaH{0}{k}\,X\right)(0) \\
&=&
\sum_{k\in\ZZ} \; \delta_{m,0} \, q^{2k} \left(\holH{0} \, \Lambda_\tau^{-1}\,
         (\evzero \tens \id)\,(\Gamma^{-1} \tens \Omega)^k X \vertL\right)(0) \\
&=&
\sum_{k\in\ZZ} \; \delta_{m,0} \, q^{2k}  \frac{1}{1-q^2} \int_0^\infty
     \! \left(\Lambda_\tau^{-1}\, (\evzero \tens \id)\,
           (\Gamma^{-1} \tens \Omega)^k X \vertL\right)(x) \: d_{q^2}x  \\
&=&
\sum_{k\in\ZZ} \; \delta_{m,0} \, q^{2k}  \frac{1}{1-q^2} \int_0^\infty
      X(k\theta,\, \tau q^k x) \: d_{q^2}x  \\
&=&
\sum_{k\in\ZZ} \; \delta_{m,0} \, q^{2k}
        \sum_{n\in\ZZ}   X(k\theta,\, \tau q^k q^{2n}) \, q^{2n}  \\
&=&
\delta_{m,0} \!\! \sum_{(k,l)\, \in \,\mathfrak{S}}
         X(k\theta,\, \tau q^l) \, q^{k+l}  \\
&=& \varphi\left(\vertM\Upsilon(X)\, b^m\right).
\end{eqnarray*}
At the last but one equality we rearranged the sums according to the rule $k+2n=l$.
Similarly we can treat $\Upsilon(X)\, c^m$.
\end{proof}
\vspace{2ex}

The previous lemma provides the condition in
\mbox{\cite[proposition 2.4.2]{Jeroen:QE2:haar}}, yielding

\begin{prop} \label{prop:Eq2:Fourier:inverse}
$\;G_{LR}=S F_{L}^{-1}$ is an {\sc lr} Fourier transform from
$\Aq(\Hcore_\nu)$ to $\Uq\!\left(\calL(\Hcore_\tau)\right)$.
\end{prop}

From the general observations in \mbox{\cite{Jeroen:QE2:haar}}\
we also obtain all the other Fourier transforms from $\Aq(\Hcore_\nu)$ to
$\Uq\!\left(\calL(\Hcore_\tau)\right)$, as well as the appropriate Plancherel
identities for $G_{LR}$ and $G_{RL}$ (observe however that $G_{LL}$ and $G_{RR}$
do {\em not\/} obey a Plancherel formula).


\section{Conclusions}

\begin{thm}
Take $0<q<1$ and define $\tau = q^{-1}$ and $\nu = (q^{-1}-q)^{-2}$.
Then the system
$$\left( \Uq\!\left(\calL(\Hcore_\tau)\vertM\right) \: \subseteq\:
         \Uq\!\left(\calL({\mathcal S}_\tau(\RR^+;q^2)) \vertM \right),
                \varphi, \psi;\;    \Uq,\,\Aq;\;
                \Aq(\Hcore_\nu)\: \subseteq\:
         \Aq\!\left({\mathcal S}_\nu(\RR^+;q^2)\vertM \right),
          \omega \vertL \right)$$
is a Plancherel context (see \cite{Jeroen:QE2:haar}).
\end{thm}
\begin{proof}
Let's show conditions (i-iv) of \cite[definition 2.5.5]{Jeroen:QE2:haar}\@.
(i) follows from propositions
\ref{prop:Eq2:Fourier_transforms},
\ref{prop:Eq2:Fourier:bijections}\ and \ref{prop:Eq2:Fourier:inverse}\@.
(ii) follows from theorem \ref{thm:Eq2:Plancherel}\ together with some results in
\cite{Jeroen:QE2:haar}\@. (iii) was shown in lemma \ref{lemma:Eq2:lambdaiszeta}\@.
Also (iv) is not so hard to prove.
\end{proof}


\paragraph{Further investigation}
From \cite[lemmas 2.3.1 and 2.6.1]{Jeroen:QE2:haar}\ we now know that
there is a natural duality between
$\Uq\!\left(\calL(\Hcore_\tau)\vertM\right)$ and $\Aq(\Hcore_\nu)$,
for instance given by $\pair{x}{y} = \omega(F_L(x)\,y) $
for $x \in \Uq\!\left(\calL(\Hcore_\tau)\vertM\right)$ and $y \in \Aq(\Hcore_\nu)$.
This formula for the pairing can be made very explicit simply by
plugging in definitions \ref{def:Eq2:Fourier_transforms},
\ref{def:YaH}, \ref{def:Hmpair}\ and \ref{def:qHankeltransform:unitary},
together with proposition \ref{prop:exist:holomorphic_qHankel}\
and the definition of the Haar functional $\omega$.
This should yield some very concrete expressions for the pairing, e.g.
\begin{eqnarray}
  \lefteqn{  \left\langle \vertL\Upsilon(X)\, b^m, \;
     \alpha^l (\gamma^*)^n \,g(\gamma^*\gamma)  \right\rangle
     \hspace{1em}=\hspace{1em}
     \delta_{m,n} \: \kq(m,-m-l)\: \nu^m \, q^{ml} \ldots } \\
  &\ldots & \sum_{r,k\in \ZZ}   q^{(m+2)(k+r)} \: \J{m}{q^{k+r}} \;   g(\nu\, q^{2r+2l}) \;
       X\!\left(-m\theta-l\theta, \: \tau\, q^{2k-m-l} \vertM
           \right) \hspace{1em} \label{eq:Eq2:induced_pairing}
\end{eqnarray}
for any and $l\in \ZZ$, $m,n\in \NN$, $g \in \Hcore_\nu$ and $X\in \calL(\Hcore_\tau)$.
The interesting point about this formula is that it no longer
depends on the use of {\em entire\/} functions, i.e.\ $X$ and $g$
may be any functions living on the appropriate (discrete!) sets,
provided they satisfy some elementary summability criterion.
So it may be possible to take (\ref{eq:Eq2:induced_pairing}) as the
{\em point to start\/} the construction of the quantum $E(2)$
group, i.e.\ making (\ref{eq:Eq2:induced_pairing}) into a {\em definition\/}
for the pairing\ldots





%%%%%%%%%%%%%%%%%%%%%%%%%%%%%%
%\begin{lemma}
%For any $m,k\in\ZZ$ we have
%$$\begin{array}{lclcl}
%(\Gamma \tens \Omega^{-1}) \, \Htil{m}{k}
%&\vertXL =&
%q^{\frac{1}{2}|m|}\:  \Htil{m}{k+1}
%\\
%(\Phi \tens \id) \, \Htil{m}{k}
%&\vertXL =&
%q^{-\frac{1}{2}k}\:  \Htil{m}{k}
%\\
%(\id \tens \Omega^2) \, \Htil{m}{k}
%&\vertXL =&
%q^{-2|m|-2}\: \Htil{m}{k} \Omega^{-2}
%\\
%\Htil{m+1}{k}\, D_{q^2}
%&\vertXL =&
%q^{\frac{1}{2} -\frac{1}{2}k}\:  \Htil{m}{k}\, \Omega^2
%\\
%\Htil{m-1}{k} \, \Omega \nabq{m}
%&\vertXL =&
%-q^{-\frac{1}{2} -\frac{1}{2}k -m}\: (\id \tens \Psi)\, \Htil{m}{k}
%\\
%\Htil{-m+1}{k} \, \Omega^{-1} \nabq{m}
%&\vertXL =&
%q^{\frac{3}{2} -\frac{1}{2}k +m}\: (\id \tens \Psi)\, \Htil{-m}{k}
%\\
%\Htil{-m-1}{k}\, D_{q^2}
%&\vertXL =&
%- q^{-\frac{3}{2} -\frac{1}{2}k}\:  \Htil{-m}{k}
%\end{array}$$
%\end{lemma}
%%%%%%%%%%%%%%%%%%%%%%%%%
