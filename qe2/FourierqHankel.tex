

\section{$q$-Exponentials \&\ Hahn-Exton $q$-Bessel functions}
\label{sec:qExp_and_qBessel}
\begin{abs} \rm
First we recall some basic notions from $q$-calculus, together with a
well-known $q$-analogue of the exponential function.
Hahn-Exton $q$-Bessel functions were introduced in
\cite{KoornwSwartt,Swarttouw,VaksKor:QE2}\ and first related to the
quantum $E(2)$ group in \cite{Koelink:QE2}\@. In
\cite{KoornwSwartt}\ they were used as a kernel for $q$-analogues
of Hankel transforms. In these papers, Hahn-Exton $q$-Bessel
functions are defined in terms of $q$-hypergeometric series. For
our purposes, we can rely entirely on a few elementary
properties---summarized below---of these $q$-Bessel functions, and
we shall never have to deal with their definition directly; for
the sake of completeness however, we give an {\em ad hoc\/}
definition below.
\hfill $\star$
\end{abs}


\begin{defn} \label{def:shifted_factorial:qExp}
Whenever $q\in\RR$ with $0<q<1$, $m\in \NN$ and $z\in \CC$,
we define the {\em $q$-shifted factorial\/} $(z;q)_m$ by
$$ (z;q)_m = \prod_{k=0}^{m-1} (1-q^k z) \hspace{20mm} (z;q)_0 =1.  $$
The limit $m\rarr \infty$ is well defined and yields
\begin{equation}\label{eq:shifted_factorial:infty}
(z;q)_\infty \: = \:\lim_{m\rarr \infty} (z;q)_m \:= \:\prod_{k=0}^{\infty} (1-q^k z).
\end{equation}
Now the $q$-Exponential $E_q : \CC \rarr \CC$ is defined by
$E_q(z)=(-z;q)_\infty$.
\end{defn}

\begin{remark} \rm
According to the Weierstra\ss\ theory of entire functions and canonical
products (\ref{eq:shifted_factorial:infty})
defines an entire function in $z$, having simple zeros at the
points $q^{-k}$ for $k\in \NN$, and no zeros elsewhere.
To ensure convergence of (\ref{eq:shifted_factorial:infty}) it is
essential that $0<q<1$. It should be noted that there exist
other $q$-analogues of the exponential function as well, but the
above one is the most suitable for our purposes.
\hfill $\star$
\end{remark}

\begin{prop} \label{prop:def:qExp}
The $q$-Exponential $E_q$ is entire and has the following power
series at the origin:
\begin{equation}\label{eq:qExp:powerseries}
  E_q(z)\:=\; \sum_{n=0}^\infty \:\frac{q^{\frac{1}{2}n(n-1)}}{(q;q)_n} \, z^n
\end{equation}
for any $z\in \CC$. Furthermore $E_q(-q^{-k})=(q^{-k};q)_\infty=0$
for all $k\in \NN$.
\end{prop}
\begin{proof}
See above remark; also (\ref{eq:qExp:powerseries}) is a standard result in $q$-calculus.
\end{proof}

\begin{lemma} \label{lemma:def:qBessel}
Whenever $n \in \ZZ$ and $q\in\RR$ with $0<q<1$, the power series
\begin{equation}\label{eq:def:qBessel}
  J_n(z;q) \:=\; \sum_{k=0}^{\infty} \:
  \frac{(-1)^k\,q^{\frac{1}{2}k(k-1)}\,q^k\,(q^{k+n+1};q)_\infty}{(q;q)_\infty \,(q;q)_k}\:z^{2k+n}
\end{equation}
defines an entire function in $z$, having a zero of order $|n|$ at the origin.
This function is said to be the {\em Hahn-Exton $q$-Bessel\/} function of order $n$.
\end{lemma}
\begin{proof}
First notice that (\ref{eq:def:qBessel}) is indeed a {\em power\/}
series---not a genuine Laurent series---when $n$ is negative,
because $(q^{k+n+1};q)_\infty=0$ whenever $n < -k \leq 0$.
So if $n<0$, the $(k=|n|)$-term will be the first non-zero term
in the series.
To prove that (\ref{eq:def:qBessel}) converges for all $z\in \CC$,
we appeal to the ratio test: fix any $z\in \CC$ and let
$a_k$ denote the $k$-term of the above series, then $a_k$
is non-zero (provided $k\geq|n|$ when $n$ is negative) and
$$ \left| \frac{a_{k+1}}{a_k} \right| \:=\:
    q^{k+1}\: \frac{(q^{k+n+2};q)_\infty}{(q^{k+n+1};q)_\infty} \:
              \frac{(q;q)_k}{(q;q)_{k+1}}\: |z|^2
    \:=\: \frac{q^{k+1}\, |z|^2}{(1-q^{k+n+1})\, (1- q^{k+1})},  $$
which tends to zero as $k\rarr\infty$, since $0<q<1$.
\end{proof}

\begin{remark} \rm
Notice Hahn-Exton $q$-Bessel functions take real values on the real line,
since all the coefficients in (\ref{eq:def:qBessel}) are real.
It should be noted that it is also possible to consider Hahn-Exton
$q$-Bessel functions of non-integer order. However, those
of integer order have some interesting properties not shared
by those of non-integer order.

Furthermore, given $q$ we shall usually deal with $q^2$-Bessel functions
henceforth---rather than with $q$-Bessel functions.
\end{remark}

\begin{prop} \label{prop:qbessel:properties}
The Hahn-Exton $q^2$-Bessel functions satisfy the following
recurrence and symmetry relations,
for all $n,m \in \ZZ$ and $z\in\CC$,
\begin{eqnarray}
 z\,\J{n-1}{z} &=& \J{n}{z} - q^n\,\J{n}{qz} \vertXL
     \label{eq:qBessel:recurrence1} \\
 -q^{-n} z\, \J{n+1}{z} &=& \J{n}{q^{-1}z} - q^{-n} \J{n}{z} \vertXL
    \label{eq:qBessel:recurrence2} \\
 \J{n}{q^m} &=& \J{m}{q^n} \vertXL
    \label{eq:qBessel:symmetry}\\
 \J{-n}{z} &=& (-q)^n\, \J{n}{q^n z} \vertXL
    \label{eq:qBessel:inversion}
\end{eqnarray}
and respectively Hansen-Lommel and $q$-Hankel orthogonality relations
\begin{eqnarray}
 \sum_{k\in \ZZ} \: q^{2k}\, \J{n+k}{z} \,\J{m+k}{z}
  &=& \delta_{n,m} \,q^{-2n} \hspace{2em}  (|qz|<1)
      \label{eq:qBessel:Hansen_Lommel} \\
 \sum_{k\in \ZZ} \: q^{2k} \,\J{r}{q^{n+k}} \,\J{r}{q^{m+k}}
  &=& \delta_{n,m} \,q^{-2n} \hspace{2em}  (r\in\ZZ)
      \label{eq:qBessel:qHankel}
\end{eqnarray}
where the sums in (\ref{eq:qBessel:Hansen_Lommel}) and (\ref{eq:qBessel:qHankel})
are absolutely convergent.
\end{prop}
\begin{proof}
The recurrence relations (\ref{eq:qBessel:recurrence1}) and (\ref{eq:qBessel:recurrence2})
are equivalent with formulas (4.8) and (4.6) in
\cite{KoelinkSwart:qBessel:zeros}, the latter being in turn
related to $q$-derivatives of Hahn-Exton $q$-Bessel functions.
It is however not too hard to derive (\ref{eq:qBessel:recurrence1})
directly from the power series expression given in
\cite[proof of prop.\ 2.1]{KoornwSwartt}\@.
The other relations can be found in
\cite{Koelink:thesis,Koelink:QE2,KoornwSwartt,Swarttouw}\@.
Be aware that (\ref{eq:qBessel:Hansen_Lommel}) is only valid when
$|qz|<1$, and observe how (\ref{eq:qBessel:qHankel})
follows from (\ref{eq:qBessel:Hansen_Lommel}) via
(\ref{eq:qBessel:symmetry}) for $r\geq 0$, whereas proving the $r<0$ case of
(\ref{eq:qBessel:qHankel}) moreover involves (\ref{eq:qBessel:inversion}).
Also notice (\ref{eq:qBessel:inversion}) interchanges
(\ref{eq:qBessel:recurrence1}) and (\ref{eq:qBessel:recurrence2}).
\end{proof}


\section{$q$-Hankel transformation}  \label{sec:qHankel}
\begin{abs} \rm
$q$-analogues of Hankel transforms with Hahn-Exton $q^2$-Bessel
kernels first appeared explicitly in \cite{KoornwSwartt}, whereas
in \cite{Vainerman}\ they were mentioned in relation to
the quantum $E(2)$ group. The starting point for our theory will
be the definition as proposed in \cite{KoornwSwartt}\@.
We recall this definition below---slightly reformulated in
an $L^2$-language. Then we shall study the possibility of
transforming {\em entire\/} functions into {\em entire\/}
functions, a feature which is
crucial\footnote{in the sense that we want to stick to our semi-algebraic
description\cite{Jeroen:QE2:haar}\ of the quantum $E(2)$ group,
the {\em algebraic\/} nature of which depends heavily on the use of entire functions
in constructing a functional calculus.}\ for our applications to the
quantum $E(2)$ group.
Furthermore we investigate how \mbox{$q$-Hankel}\ transforms behave
w.r.t.\ $q$-differentiation and multiplication. Next we
will search for eigenfunctions of these $q$-Hankel transforms;
these will turn out to be functions of \mbox{$q^2$-exponential}\ type.
Eventually we shall establish a link between \mbox{$q$-Hankel}\ transformation and
\mbox{$q^2$-moments}, yielding positive answer to a particular \lq moment problem\rq.
\hfill $\star$
\end{abs}

\paragraph{The $L^2$-theory}
Let $\RR_q^+ \equiv (\RR_q^+,m_q)$ denote the discrete set $\{ q^k \mid k\in\ZZ \}$
endowed with the measure $m_q$ which assigns weight $q^{2k}$ to the
point $q^k$, for any $k\in\ZZ$. So integration w.r.t.\ $m_q$ yields
$$ \int_{\RR_q^+} f\,dm_q \:=\: \sum_{k\in \ZZ} f(q^k)\,q^{2k} $$
for $f\in L^1(\RR_q^+)$. Now the orthogonality relations (\ref{eq:qBessel:qHankel})
can be reformulated in an $L^2$-language:

\begin{prop}
Fix any $m\in\ZZ$ and define functions $\basis{m}{k}$ on $\RR_q^+$ by
$$  \basis{m}{k}(x) \: =\: q^k \,\J{m}{q^k x} \hspace{5em} (k\in \ZZ,\: x\in\RR_q^+). $$
Then $(\basis{m}{k})_{k\in \ZZ}$ is an orthonormal basis ({\sc onb}) in
the Hilbert space \Ltwoq\@.
\end{prop}

On the other hand, we also have a canonical {\sc onb} for \Ltwoq,
say $(d_k)_{k\in \ZZ}$ with $d_k = q^{-k} \delta_{q^k}$. Here $\delta_{q^k}$
denotes the characteristic function of the singleton $\{q^k\}$.

\begin{defn} \label{def:qHankeltransform:unitary}
Given $m\in\ZZ$, let $H_m$ be the unitary transformation of \Ltwoq\
which maps the {\sc onb} $(\basis{m}{k})_{k\in \ZZ}$
into the {\sc onb} $(d_k)_{k\in \ZZ}$, i.e.\
$ H_m f \:=\:\sum_{k\in \ZZ} \:\scal{f}{\basis{m}{k}}\,d_k $,
or equivalently,
\begin{equation}\label{eq:qHankeltransform:def}
  (H_m f)(q^k) \;=\; q^{-k} \scal{f}{\basis{m}{k}}
          \;=\;\sum_{n\in \ZZ} \: q^{2n} \J{m}{q^{n+k}} f(q^n)
\end{equation}
for $f\in \Ltwoq$ and $k\in\ZZ$.
$H_m$ is said to be the {\em $q$-Hankel transform\/} of order $m$.
\end{defn}

\begin{prop} \label{prop:qHankelSquare_is_id}
$\: H_m^2 = \id$ for all $m\in\ZZ$.
\end{prop}
\begin{proof}
It suffices to show that $H_m d_j = \basis{m}{j}$ for all $j\in \ZZ$.
For $k \in \ZZ$ we have
$$ (H_m d_j)(q^k)  \:=\: \sum_{n\in \ZZ} q^{2n} \J{m}{q^{n+k}} d_j(q^n)
     \:=\:  q^{2j} \J{m}{q^{j+k}} q^{-j} \:=\:  q^j \J{m}{q^j q^k}
     \,=\,  \basis{m}{j}(q^k).  $$
\end{proof}


\paragraph{Holomorphic $q$-Hankel transformation}
In the previous paragraph we considered functions living on a
discrete space $\RR_q^+$. Now we shall try
to \lq interpolate\rq\ these functions on $\RR_q^+$ by entire
functions (whenever possible). But first we recall the \lq Schwartz-like\rq\ spaces
of entire functions, as defined in \cite[example 3.4.1.2]{Jeroen:QE2:haar}.

\begin{defn} \label{ex:Schwartz-like_space}
Whenever $r$ is a real number with $r>0$, we define
$$ {\mathcal S}_r(\RR^+;q)
    \;=\: \left\{ f \in \HC \left|\vertL \right.
          \mbox{for all $n \in \NN$, the set
          $\left\{\,f(r q^k)\,q^{nk} \right\}_{_{\scriptstyle k \in \ZZ}}$
          is bounded} \right\}.$$
In case $r=1$, the subscript $r$ will be suppressed in our notation.
\end{defn}

Also recall (cf.\ \cite[\S 3.2]{Jeroen:QE2:haar}) the linear
operators $\Omega$ and $\Psi$ from \HC\ into \HC\ defined by
$(\Omega f)(z) = f(qz)$ and $(\Psi f)(z) = z f(z)$ for
$f\in\HC$ and $z\in\CC$
(notice $\Omega$ is invertible whereas $\Psi$ is not).
We now moreover introduce the operator $K$ on \HC, defined by
$(Kf)(z)=f(z^2)$. Observe that $K$ maps \Swqbis\ into \Swq\@.
Also the commutation rules $\Omega K = K\Omega^2$ and
$\Psi^2 K = K \Psi$ are easily verified.

Furthermore, let $R_q$ denote the restriction map from \HC\ into
the space of all functions on $\RR_q^+$. Since $\RR_q^+$ admits a
point of accumulation, the identity theorem for holomorphic functions
implies $R_q$ to be injective.
Observe that $R_q$ maps \Swq\ into \Ltwoq\@. The proof of this involves
the fact that entire functions are bounded on---let's say---the unit interval;
also recall that $\RR_q^+ \equiv (\RR_q^+,m_q)$ was endowed with a particular
measure---{\em not\/} the counting measure.



\begin{abs} \label{abs:holomorphic_qHankel}\rm
Our next aim is to find, for any $m\in\ZZ$, a sufficiently large subspace
$\holS{m}$ of \Swqbis\ together with a linear mapping $\holH{m}$ from $\holS{m}$
into $\holS{m}$, such that the following diagram commutes:
\begin{equation}\label{eq:diagram:holomorphic_qHankel}
\begin{CD}
  \holS{m} @>\id>>
  \Swqbis  @>{\textstyle K}>>
  \Swq     @>{\textstyle \Psi^{|m|}}>>
  \Swq     @>{\textstyle R_q}>>
  \Ltwoq \\
%%%%%%%%%%%%%%%
  @V{\textstyle\holH{m}}VV @. @. @. @VV{\textstyle H_m}V  \\
%%%%%%%%%%%%%%%
  \holS{m} @>>\id>
  \Swqbis  @>>{\textstyle K}>
  \Swq     @>>{\textstyle \Psi^{|m|}}>
  \Swq     @>>{\textstyle R_q}>
  \Ltwoq
\end{CD}
\end{equation}
The map $\holH{m}$ will then be called the {\em holomorphic\/} $q$-Hankel transform of order $m$,
because it transforms entire functions into entire functions.
However the above scheme involves more than merely the passage to entire functions:
the operators $K$ and $\Psi^{|m|}$ appearing in the diagram will ensure
the system $(\holH{m})_{m\in\ZZ}$ to have the proper behaviour w.r.t.\
$q^2$-differentiation (cf.\ proposition \ref{prop:holqHankel:qdiff}).
Iterating the above diagram immediately reveals that $\holH{m}^2=\id$
(recall that $H_m^2=\id$, cf.\ proposition \ref{prop:qHankelSquare_is_id}).
\hfill $\star$
\end{abs}

\begin{defn*} \label{def:Hmpair}
For all $m\in\ZZ$ we define $\auxH{m} : \Swqbis \rarr \Ltwoq$
by \mbox{$\auxH{m} = H_m R_q \Psi^{|m|} K$}\@. Now a pair $(f,g)$
of entire functions is said to be an {\em \Hmpair\/} whenever
\begin{enumerate}
\item $f\in \Swqbis$ and $\auxH{m} f = R_q g$,
\item $g$ is an even (resp.\ odd) function whenever $m$ is an even (resp.\ odd) number,
\item $g$ has a zero at the origin of order at least $|m|$.
\end{enumerate}
Finally we define $\holS{m}$ to be the following subspace of \Swqbis:
$$ \holS{m} \:= \: \left\{ f\in \Swqbis \:\left|\:
                \begin{array}{l}
                   \mbox{there exists an entire function $g$ such} \\
                   \mbox{that $(f,g)$ is an \Hmpair}
                \end{array}\!\right.\right\}$$
\end{defn*}

\begin{remark} \label{rem:not_constructive} \rm
The maps $\auxH{m}$ and the notion of an \Hmpair\ shall have no
role in the eventual picture; however they will turn out to be
very convenient in formulating some intermediate results. Roughly
speaking we want to consider entire functions whose $q$-Hankel
transforms on $\RR_q^+$ admit interpolation by entire functions.
The above approach may seem to be rather descriptive and certainly
not very constructive at this moment, since it does not provide a
criterion to determine whether a function $f$ belongs to
$\holS{m}$ or not---at least not without computing $\auxH{m} f$
first. In this way however, we want to avoid (or at least
postpone) some technical questions which do not have too high
priority from our perspective. Just to give a hint of the kind of
trouble we run into when we actually try to {\em construct\/}
such an \Hmpair\ $(f,g)$, let's write out in detail the equation
$\auxH{m} f = R_q g$ appearing in the above definition:
\begin{equation}\label{eq:remark:trouble:qpower}
 g(q^k) \:=\: (H_m R_q \Psi^{|m|} K f)(q^k)
        \:=\:  \sum_{n\in \ZZ} \: q^{2n} \J{m}{q^{n+k}}\, q^{n|m|} f(q^{2n})
\end{equation}
for any $k\in\ZZ$. Formally replacing $q^k$ with $z\in\CC$ we obtain
\begin{equation}\label{eq:remark:trouble:entire}
   g(z)  \:=\;  \sum_{n\in \ZZ} \: q^{2n} \J{m}{q^n z}\, q^{n|m|} f(q^{2n})
     \hspace{5em} \mbox{for any $z\in\CC$}.
\end{equation}
Now we want (\ref{eq:remark:trouble:entire}) to define an entire
function---so we need the sum to converge uniformly on
compact sets---and that's precisely the point where things become very tricky.
Indeed there turns out to be a huge difference between
(\ref{eq:remark:trouble:qpower}) and (\ref{eq:remark:trouble:entire})
on the technical level. The fact is that our Hahn-Exton $q^2$-Bessel
functions are behaving very well as far as we evaluate them in in $q$-powers only,
whereas they are not quite so innocent in the rest of the complex plane.
For instance, the orthogonality relation (\ref{eq:qBessel:qHankel})
reveals that $\J{m}{q^{n+k}}$ tends to zero as $n \rarr -\infty$, but in general
when $z\in \CC$ is not an integer power of $q$, $\J{m}{q^n z}$ will grow very
rapidly as $n \rarr -\infty$.
However the \lq interpolation\rq\ strategy we have chosen for will allow us to
proceed quite far with only {\em partial\/} answers (like for instance lemma
\ref{lemma:compactsupp_in_holS}) to the question of convergence mentioned above,
and therefore avoiding a lot of trouble.
Indeed precisely those results that {\em we\/} are looking for
(i.e.\ those which are useful w.r.t.\ the quantum $E(2)$ group) will come
almost for free in our approach, but of course we shall also pay a price:
some properties one might expect to hold could be hard to prove within our setting
(cf.\ remark \ref{rem:Q:strict_inclusions}).
Let's conclude the present remark by stating that we {\em won't\/} give a full
explicit description of the spaces $\holS{m}$, whereas we {\em will\/}
show---constructively---that they contain plenty of elements.
\hfill $\star$
\end{remark}



\paragraph{$q$-Hankel transforms and $q$-differentiation}
Whenever $f \in H(\CC)$ the function
\begin{equation}\label{eq:def:qderivative}
  \CC_0 \rarr \CC : z \mapsto \frac{f(z)-f(qz)}{(1-q)z}
\end{equation}
is holomorphic and obviously has a removable singularity at the origin.
Hence (\ref{eq:def:qderivative}) defines an {\em entire\/} function again,
denoted $D_q f$. This yields a linear operator $D_q: \HC \rarr \HC$
which is usually referred to as $q$-differentiation.
In \cite[\S 3.3]{Jeroen:QE2:haar}\ we also considered the $q$-difference
operators $\nabq{m}$ on \HC, defined by
\begin{equation}\label{eq:def:nabq}
 \nabq{m} \:=\;  \frac{1}{q-q^{-1}}
        \,\left( q^m\, \Omega  - q^{-m}\, \Omega^{-1} \vertL\right)
\end{equation}
for $m\in\NN$.
Notice \Swqbis\ is not \mbox{$\nabq{m}$-invariant};
however $\nabq{m} \Omega$ does leave \Swqbis\ invariant.
It's also possible to prove that \Swqbis\ is $D_{q^2}$-invariant,
although definition \ref{ex:Schwartz-like_space}\ does not make
any reference to $q^2$-derivatives at
all\footnote{which is rather peculiar for a \lq Schwartz-like\rq\ space,
isn't it? The fact $D_{q^2}$-invariance holds automatically here, is related to the
discrete nature of $q^2$-differentiation.} (cf.\ \cite[\S 3.4]{Jeroen:QE2:haar}).
Now the following makes sense:

\begin{lemma} \label{lemma:auxH:qdiff}
For all $f\in \Swqbis$ and $m,k\in \ZZ$ with $m\geq 1$ we have
\begin{eqnarray}
(\auxH{m} D_{q^2} f)(q^k)
      &=& -\frac{q^{-1}}{q^{-1}-q} \; q^k (\auxH{m-1} \Omega^2 f)(q^k)
       \label{eq:auxH:Dq2:pos}\\
(\auxH{m-1} \nabq{m} \Omega f)(q^k)
      &=& \frac{1}{q^{-1}-q} \; q^{-m}\, q^k (\auxH{m} f)(q^k)
       \label{eq:auxH:nabq:pos} \\
(\auxH{-m} D_{q^2} f)(q^k)
      &=&  \frac{q^{-1}}{q^{-1}-q} \; q^k (\auxH{-m+1} f)(q^k)
       \label{eq:auxH:Dq2:neg} \\
(\auxH{-m+1} \nabq{m} \Omega f)(q^k)
      &=& -\frac{1}{q^{-1}-q} \; q^m\: q^k (\auxH{-m} \Omega^2 f)(q^k)
      \label{eq:auxH:nabq:neg}
\end{eqnarray}
\end{lemma}
\begin{proof}
Take any $f\in \Swqbis$. Assuming $m \geq 1$, we have for all $k\in \ZZ$
\begin{eqnarray*}
\lefteqn{(\auxH{m} D_{q^2} f)(q^k)
  \hspace{1em} = \hspace{1em}
      (H_m R_q \Psi^{|m|} K D_{q^2} f)(q^k)} \\
  \vertXL
  &=& \sum_{n\in \ZZ} \: q^{2n} \J{m}{q^{n+k}}\: (R_q \Psi^{|m|} K D_{q^2} f)(q^n)\\
  &=& \sum_{n\in \ZZ} \: q^{2n} \J{m}{q^{n+k}}\: q^{nm}\,(D_{q^2} f)(q^{2n}) \\
  &=& \sum_{n\in \ZZ} \: q^{2n} \J{m}{q^{n+k}}\: q^{nm}\,
      \frac{f(q^{2n})-f(q^2 q^{2n})}{(1-q^2) q^{2n}} \\
  &\stackrel{(\sharp)}{=}&
   \frac{1}{1-q^2}\,  \sum_{n\in \ZZ} \: \J{m}{q^{n+k}}\: q^{nm} f(q^{2n})
  \;-\;\frac{1}{1-q^2}\, \sum_{n\in \ZZ} \: \J{m}{q^{n+k}}\:
              q^{nm} f\!\left(q^{2n+2}\right) \\
  &\stackrel{(*)}{=}&
      \frac{1}{1-q^2}\, \sum_{n\in \ZZ} \:
           \left(q^m \J{m}{q^{n+1+k}} - \J{m}{q^{n+k}}\vertL\right)
            q^{nm} f\!\left(q^{2n+2}\right)\\
  &\stackrel{(\ref{eq:qBessel:recurrence1})}{=}&
       -\frac{1}{1-q^2}\, \sum_{n\in \ZZ} \:
         q^{n+k} \J{m-1}{q^{n+k}}\: q^{nm} \:(\Omega^2 f)(q^{2n}) \\
  &=& -\frac{1}{1-q^2}\, q^k \sum_{n\in \ZZ} \:
         q^{2n} \J{m-1}{q^{n+k}} \: q^{n(m-1)}\: (\Omega^2 f)(q^{2n}) \\
  &=& -\frac{q^{-1}}{q^{-1}-q}\, q^k (\auxH{m-1} \Omega^2 f)(q^k).
\end{eqnarray*}
$(*)$ relies on replacing $n$ by $n+1$ in {\em only one\/} of the two summations
in the {\sc rhs} of $(\sharp)$. In this respect it is
crucial (!) that both sums in themselves converge
absolutely\footnote{To appreciate this, do consider for a moment the
$m=0$ case (which we are excluding in the text above). Indeed, if $m$ were zero, the
sums in the {\sc rhs} of $(\sharp)$ would actually {\em diverge\/}
(as $n \rarr +\infty$) unless $f(0)=0$, since $\J{0}{0}\neq 0$.
Nevertheless the {\sc lhs} of $(\sharp)$ would still exist.}. To see this,
consider the following observations: as far as $n \rarr +\infty$ is concerned,
absolute summability follows from the fact that $f$ and $\J{m}{\cdot\,}$
are bounded on a neighborhood of the origin, together with the
assumption that $m\geq 1$. To deal with $n \rarr -\infty$, we have to use
that $Kf$ is in the Schwartz-like space \Swq, which roughly means that
$f(q^{2n})= (Kf)(q^n)$ tends to zero very rapidly when $n \rarr -\infty$.
Furthermore we need some bound for $\J{m}{q^{n+k}}$ when $n \rarr -\infty$,
which can easily be obtained from the orthogonality relations
(\ref{eq:qBessel:qHankel}). We conclude that $(\sharp)$ indeed constitutes a
valid operation. In the above computation we also used recurrence relation
(\ref{eq:qBessel:recurrence1}) of proposition
\ref{prop:qbessel:properties}\@. This proves (\ref{eq:auxH:Dq2:pos}).
The proof of (\ref{eq:auxH:nabq:neg}) proceeds similarly:
\begin{eqnarray*}
\lefteqn{\left(\auxH{-m+1} \nabq{m} \Omega f\right)(q^k)
    \hspace{1em} = \hspace{1em}
     \left(H_{-m+1} R_q \Psi^{|-m+1|} K \nabq{m} \Omega f\right)(q^k)} \\
   \vertXL
    &=& \sum_{n\in \ZZ} \: q^{2n} \J{-m+1}{q^{n+k}}\:
             q^{n(m-1)} \left(\nabq{m} \Omega f\right)(q^{2n}) \\
    &=& \frac{1}{q-q^{-1}}\, \sum_{n\in \ZZ} \: q^{2n} \J{-m+1}{q^{n+k}}\: q^{n(m-1)}
           \left( q^m f\left(q^2 q^{2n}\right) - \,q^{-m} f(q^{2n})\vertL \right) \\
    &=& \frac{1}{q-q^{-1}}\, \sum_{n\in \ZZ} \: q^{2n} q^{n(m-1)} \!
        \left(q^m \J{-m+1}{q^{n+k}}
             - q \J{-m+1}{q^{n+1+k}}\vertL\right)\! f\!\left(q^{2n+2}\right)\\
    &=& \frac{1}{q-q^{-1}} \sum_{n\in \ZZ} q^{2n} q^{n(m-1)}
         q^m \left(\J{-m+1}{q^{n+k}} - q^{-m+1} \J{-m+1}{q^{n+1+k}}\vertL \right)
               (\Omega^2 f)(q^{2n})\\
    &\stackrel{(\ref{eq:qBessel:recurrence1})}{=}&
       \frac{1}{q-q^{-1}} \, \sum_{n\in \ZZ} \: q^{2n} q^{n(m-1)}
          q^m q^{n+k} \J{-m}{q^{n+k}} \: (\Omega^2 f)(q^{2n})\\
    &=& -\frac{1}{q^{-1}-q}\, q^m \: q^k (\auxH{-m} \Omega^2 f)(q^k).
\end{eqnarray*}
Formulas (\ref{eq:auxH:nabq:pos}) and (\ref{eq:auxH:Dq2:neg}) are shown
similarly, now relying on recurrence relation (\ref{eq:qBessel:recurrence2}).
\end{proof}


\paragraph{Construction and basic properties}
Now we are about to construct the holomorphic $q$-Hankel transform
announced in abstract \ref{abs:holomorphic_qHankel}\@.
Furthermore we translate the intermediate results of lemma \ref{lemma:auxH:qdiff}\
into there final form (proposition \ref{prop:holqHankel:qdiff})
and prove some basic properties.

\begin{lemma} \label{lemma:transform_in_schwartz_again}
Let $m\in\ZZ$. If $(f,g)$ is an \Hmpair, then $g\in \Swq$.
\end{lemma}
\begin{proof}
First consider the case $m\geq 0$.
Let $(f,g)$ be an \Hmpair\ and take any $n\in \NN$.
Iterating (\ref{eq:auxH:Dq2:pos}) $n$ times yields, for all $k\in \ZZ$,
$$ \left(\auxH{m+n} \left(D_{q^2} \Omega^{-2}\right)^{\! n} \! f\right)(q^k)
      = \left(-\frac{q^{-1}}{q^{-1}-q}\right)^n  q^{nk} \left(\auxH{m}f\right)(q^k) $$
It follows that for all $n\in \NN$ and $k\in \ZZ$
$$ q^{nk} g(q^k) = (q^2-1)^n \:\left(\auxH{m+n} f_n\right)(q^k) $$
where $f_n=(D_{q^2} \Omega^{-2})^n \! f$.
Now \Swqbis\ is $\{\Omega^{\pm 2}, D_{q^2} \}$-invariant,
hence all the $f_n$ are still in \Swqbis\@. Since $\auxH{m+n}$
maps \Swqbis\ into \Ltwoq, we get
$\sum_{k\in\ZZ} |q^{nk}g(q^k)|^2 q^{2k} < \infty$ for any $n\in \NN$.
From this one can easily derive that $g$ belongs to \Swq\@.
In case $m$ is negative we proceed analogously, but now using (\ref{eq:auxH:Dq2:neg}).
\end{proof}


\begin{prop} \label{prop:exist:holomorphic_qHankel}
For any $m\in\ZZ$ there exists a unique linear map
$\holH{m}$ from $\holS{m}$ into $\holS{m}$,
such that diagram (\ref{eq:diagram:holomorphic_qHankel})
in abstract \ref{abs:holomorphic_qHankel}\ commutes.
In particular, if $(f,g)$ is an \Hmpair, then $f\in\holS{m}$ and
$\Psi^{|m|} K \holH{m}f =g$. Furthermore, $\holH{m}^2=\id$.
\end{prop}
\begin{proof}
Uniqueness is obvious (recall $R_q$ is injective). To prove existence,
take any $m\in\ZZ$ and $f \in\holS{m}$. By definition there exists
an entire function $g$ such that $(f,g)$ is an \Hmpair\@.
Now the function
$$ \CC_0 \rarr \CC : z \mapsto \frac{1}{z^{|m|}}\, g(z)$$
is holomorphic, and has a removable singularity at the origin because
of (iii) in definition \ref{def:Hmpair}\@.
Furthermore, this function will always be {\em even\/} because of (ii)
in definition \ref{def:Hmpair}\@.
Hence there exists an entire function, denoted $\holH{m}f$, such that
$$ (\holH{m}f)(z^2)\: =\: \frac{1}{z^{|m|}}\, g(z) \hspace{5em} (z\in \CC_0) $$
or equivalently, $\Psi^{|m|} K \holH{m}f = g$.
From lemma \ref{lemma:transform_in_schwartz_again}\ we know that $g\in \Swq$,
and it follows easily that $\holH{m}f$ belongs to \Swqbis\@.
But we actually want $\holH{m}f\in \holS{m}$, and
in this respect we claim that $(\holH{m}f, \Psi^{|m|} K f)$ is an \Hmpair\@.
Indeed, items (ii-iii) of definition \ref{def:Hmpair}\ are
obviously fulfilled; to complete (i), observe that
$$ \auxH{m} \holH{m} f \:=\: H_m R_q \Psi^{|m|} K \holH{m} f  \:=\: H_m R_q g
   \:=\: H_m \auxH{m} f \:=\: H_m^2 R_q \Psi^{|m|} K f  \:=\: R_q \Psi^{|m|} K f. $$
Here we have used that $H_m^2=\id$ (proposition \ref{prop:qHankelSquare_is_id}).
Thus we have constructed a linear map $\holH{m}$ from $\holS{m}$ into
$\holS{m}$ satisfying all our requirements.
\end{proof}


\begin{prop} \label{prop:holS:Omega_invariant}
For any $m\in \ZZ$, the space $\holS{m}$ is $\Omega^{\pm 2}$-invariant and
\begin{equation}\label{eq:holS:Omega_invariant}
  \holH{m} \Omega^2 \;=\; q^{-2(|m|+1)}\, \Omega^{-2} \holH{m}
\end{equation}
\end{prop}
\begin{proof}
Let $(f,g)$ be any \Hmpair\@; we claim
$(\Omega^2 f,\, q^{-|m|-2}\Omega^{-1} g)$ is an \Hmpair\ as well.
First observe \Swqbis\ is $\Omega^{\pm 2}$-invariant.
Now from $\Psi^{|m|} K \Omega^2 = q^{-|m|} \Omega \Psi^{|m|} K$
it follows that, for all $k\in \ZZ$,
\begin{eqnarray*}
(\auxH{m} \Omega^2 f)(q^k)
  &=& (H_m R_q \Psi^{|m|} K \Omega^2 f)(q^k)
  \hspace{1em} = \hspace{1em}
      q^{-|m|}\,(H_m R_q \Omega \Psi^{|m|} K f)(q^k) \\
  \vertXL
  &=& q^{-|m|} \sum_{n\in \ZZ} \: q^{2n} \J{m}{q^{n+k}}\,(R_q \Omega \Psi^{|m|} K f)(q^n)\\
  &\stackrel{(*)}{=}&
      q^{-|m|} \sum_{n\in \ZZ} \: q^{2(n-1)} \J{m}{q^{n-1+k}}\,(\Psi^{|m|} K f)(q^{n-1} q)\\
  &=& q^{-|m|-2} \,(\auxH{m} f)(q^{k-1})
  \hspace{1em} = \hspace{1em}
  q^{-|m|-2}\, (\Omega^{-1} g)(q^k)
\end{eqnarray*}
In (*) we replaced the summation index $n$ by $n-1$.
This proves item (i) of definition \ref{def:Hmpair}, whereas items (ii-iii)
are obvious here. Hence by proposition \ref{prop:exist:holomorphic_qHankel},
$\Omega^2 f$ belongs to $\holS{m}$ and
$$ \Psi^{|m|} K \holH{m} \Omega^2 f \:=\:  q^{-|m|-2}\Omega^{-1} g
      \: =\:  q^{-|m|-2}\Omega^{-1} \Psi^{|m|} K \holH{m} f
      \: =\:  q^{-|m|-2} q^{-|m|} \Psi^{|m|} K \Omega^{-2} \holH{m} f.  $$
Here we used the commutation rules for $\Psi$, $K$, and $\Omega$.
Canceling $\Psi^{|m|} K$ yields the result.
\end{proof}

\vspace{2ex}
In \cite{Jeroen:QE2:haar}\ the space \HC\ of entire functions was
endowed with a natural $^*$-operation, denoted by $\til$ and
defined as $\tilde{f}(z) = \overline{f(\overline{z})}$. Now we study its
behaviour w.r.t.\ $q$-Hankel transformation:

\begin{prop} \label{prop:qHankel:tilde}
Take any $m\in\ZZ$. The space $\holS{m}$ is $\til$-invariant and\/
$\til\!$ commutes with $\holH{m}$.
\end{prop}
\begin{proof}
Observe that $\til$ commutes with $\Psi$ and $K$.
Furthermore $H_m$ obviously commutes with complex conjugation,
since Hahn-Exton $q$-Bessel functions take real values on the real line.
We also have $R_q \tilde{f} = \overline{R_q f}$ for any $f\in\HC$.
It is clear that if $(f,g)$ is an \Hmpair, then so is $(\tilde{f},\tilde{g})$.
The result now follows easily from proposition \ref{prop:exist:holomorphic_qHankel}.
\end{proof}

\begin{prop} \label{prop:qHankel:inverseorder}
Take any $m\in\ZZ$. Then $\holS{-m} = \holS{m}$ and
$\holH{-m} = (-q)^m q^{m|m|}\Omega^{2m}\holH{m}$.
In this respect we also have two \lq auxiliary\rq\ results worth mentioning:
\begin{equation}\label{eq:qHankel:inverseorder}
   (H_{-m} f)(q^k) = (-q)^m (H_m f)(q^{k+m})
      \hspace{4em}
   (\auxH{-m} g)(q^k) = (-q)^m (\auxH{m} g)(q^{k+m})
\end{equation}
for all $f\in \Ltwoq$, $g\in \Swqbis$ and $k\in\ZZ$.
\end{prop}
\begin{proof}
The first formula in (\ref{eq:qHankel:inverseorder}) follows
easily from (\ref{eq:qBessel:inversion}), whereas the second is an
immediate consequence of the first one.
Now let $(f,g)$ be any \Hmpair\@; then it is easy to see that
$(f, (-q)^m \Omega^m g)$ is a \mbox{$(-m;q)$-Hankel} pair.
Hence according to proposition \ref{prop:exist:holomorphic_qHankel}\
we have $f\in \holS{-m}$ and
$$ \Psi^{|-m|} K \holH{-m}f \:=\: (-q)^m \Omega^m g
      \:=\: (-q)^m \Omega^m \Psi^{|m|} K \holH{m}f
      \:=\: (-q)^m q^{m|m|} \Psi^{|m|} K \Omega^{2m} \holH{m} f. $$
Canceling $\Psi^{|m|} K$ yields the result.
\end{proof}

\vspace{1ex}
The next proposition will play a key-role throughout the entire paper:

\begin{prop} \label{prop:holqHankel:qdiff}
Take any $m\in \NN$ with $m\geq 1$. Then
$$\begin{array}{rclrcrlr}
D_{q^2} \holS{m-1} & \!\! \subseteq&  \!\! \holS{m} &
   \holH{m} D_{q^2} f &\!\! =&  \!\! \displaystyle -\frac{q^{-1}}{q^{-1}-q}&
              \!\!\! \holH{m-1} \Omega^2 f & \;(f\in \holS{m-1}) \\
\nabq{m} \Omega \, \holS{m}& \!\! \subseteq&  \!\! \holS{m-1} &
   \holH{m-1} \nabq{m} \Omega f & \!\! =&  \!\!  \displaystyle
          \frac{1}{q^{-1}-q} &  \!\!\! q^{-m}\: \Psi \holH{m}f & \;(f\in \holS{m}) \\
D_{q^2} \holS{-m+1} & \!\! \subseteq& \!\! \holS{-m} &
   \holH{-m} D_{q^2}f & \!\! =& \!\! \displaystyle  \frac{q^{-1}}{q^{-1}-q} &
    \!\!\! \holH{-m+1}f & \;(f\in \holS{-m+1})  \\
\nabq{m} \Omega^{-1}  \holS{-m} & \!\! \subseteq& \!\! \holS{-m+1} \;\; &
   \holH{-m+1} \nabq{m} \Omega^{-1}f & \!\!=&  \!\!\displaystyle -\frac{1}{q^{-1}-q} &
      \!\!\!   q^m\: \Psi \holH{-m} f & (f\in \holS{-m})
\end{array}$$
\end{prop}
\begin{proof}
Take any $f\in \holS{m-1}$. Then by proposition
\ref{prop:holS:Omega_invariant}, also $\Omega^2 f$ belongs to $\holS{m-1}$.
Let $(\Omega^2 f,g)$ be an \mbox{$(m-1;q)$-Hankel} pair.
With (\ref{eq:auxH:Dq2:pos}) of lemma \ref{lemma:auxH:qdiff}\ it
follows easily that
$$ \left(D_{q^2}f,\:-\frac{q^{-1}}{q^{-1}-q} \,\Psi g \right) $$
is an \Hmpair, and hence $D_{q^2}f$ belongs to $\holS{m}$.
Furthermore, since $m$ and $m-1$ are positive, we have
(cf.\ proposition \ref{prop:exist:holomorphic_qHankel})
$$ \Psi^m K \holH{m} D_{q^2}f \;=\; -\frac{q^{-1}}{q^{-1}-q}\, \Psi g
    \;=\; -\frac{q^{-1}}{q^{-1}-q}\, \Psi \Psi^{m-1} K \holH{m-1} \Omega^2 f $$
Canceling $\Psi^m K$ yields the first formula. The other formulas
are more or less analogous---let's also have a look at the last one:
take any $f\in \holS{-m}$ and let $(f,g)$ be a \mbox{$(-m;q)$-Hankel} pair.
From (\ref{eq:auxH:nabq:neg}) we obtain that
$$ \left(\nabq{m} \Omega^{-1}f, \; -\frac{1}{q^{-1}-q} \, q^m\, \Psi g \right)$$
is a \mbox{$(-m+1;q)$-Hankel} pair, hence $\nabq{m} \Omega^{-1}f \in \holS{-m+1}$ and
$$ \Psi^{|-m+1|} K \holH{-m+1} \nabq{m} \Omega^{-1} f
  \;=\;  -\frac{1}{q^{-1}-q} \, q^m\, \Psi g
  \;=\;  -\frac{1}{q^{-1}-q} \, q^m\,
  \underbrace{\Psi \Psi^{|-m|} K}_{\Psi^{m-1} K \Psi} \holH{-m} f $$
Canceling $\Psi^{m-1} K$ yields the last formula.
\end{proof}

\begin{remark} \rm
Notice proposition \ref{prop:qHankel:inverseorder}\ (accompanied
by proposition \ref{prop:holS:Omega_invariant}) transforms the
first two formulas of proposition \ref{prop:holqHankel:qdiff}\
into the remaining ones and vice versa. In a similar way,
(\ref{eq:qHankel:inverseorder}) interchanges
(\ref{eq:auxH:Dq2:pos}-\ref{eq:auxH:Dq2:neg}) as well as
(\ref{eq:auxH:nabq:pos}-\ref{eq:auxH:nabq:neg}).
$\hfill \star$
\end{remark}



\paragraph{The spaces $\holS{m}$ do contain many functions}
The above theory would collapse in case the spaces $\holS{m}$ would
turn out to be trivial; fortunately we have the following

\begin{lemma} \label{lemma:compactsupp_in_holS}
Let $f$ be an entire function having the property that there
exists an integer $n_0$ such that $f(q^{2n})=0$ for all integers $n<n_0$.
Then $f \in \holS{m}$ for any $m\in \ZZ$.
\end{lemma}
\begin{proof}
Observe that our assumptions imply $f$ to be in \Swqbis\ in the first place.
Then fix any $m\in \ZZ$. We claim that the series
\begin{equation}\label{eq:lemma:qqE:series}
    g(z) = \frac{1}{z^{|m|}}\: \sum_{n\in\ZZ}
          q^{2n}\, \J{m}{q^n z} \, q^{|m|n} \, f(q^{2n})
\end{equation}
converges\footnote{Notice that convergence in (\ref{eq:lemma:qqE:series})\
is quite a different matter than for instance in (\ref{eq:qHankeltransform:def}).
Indeed in (\ref{eq:lemma:qqE:series}) our $q$-Bessel functions are
to be evaluated in any complex number, whereas (\ref{eq:qHankeltransform:def})
only deals with powers of $q$.
See also remark \ref{rem:not_constructive}}\ absolutely for all $z\in \CC_0$ and defines an
{\em entire\/} function $g$ (the singularity at the origin being removable).
Once this fact established, it will be easy to see that $(f, \Psi^{|m|} g)$
is an \Hmpair, and hence $f \in \holS{m}$.
So let's investigate the above series; according to lemma \ref{lemma:def:qBessel}\
it is possible to define, for any integer $n$, an entire function $g_n$ such that
$$ g_n (z) =  (q^n z)^{-|m|} \, \J{m}{q^n z} \,f(q^{2n}) $$
for $z\in \CC_0$.
The series (\ref{eq:lemma:qqE:series}) can now be rewritten as follows:
\begin{equation}\label{eq:lemma:qqE:series:rewrite}
  g = \sum_{n=n_0}^\infty q^{2n(|m|+1)}\,g_n
\end{equation}
Since every $g_n$ is entire, it suffices  to show the latter series
converges uniformly on compact sets.
Therefore, pick any number $r>0$ and let $D(r)$ denote the disk
$\{z\in \CC \mid |z| \leq r\}$.
Once again appealing to lemma \ref{lemma:def:qBessel}, it is clear that there exists
a bound, say $M_r>0$, such that $|x^{-|m|}\J{m}{x}| \leq M_r$ whenever
$0 < |x| \leq q^{n_0}r$. On the other hand also $f$ is entire, hence
bounded on compact sets; so we can find a bound $N>0$ such that
$|f(x)| \leq N$ for all $x$ in the interval $[0,q^{2n_0}]$.
Since $0<q<1$, it follows that $|g_n(z)| \leq M_r N$ for any $n \geq n_0$
and any $z\in D(r)$. In other words, the family $\{g_n\}_{n=n_0}^\infty$
is uniformly bounded on $D(r)$.
Since $\sum_{n=n_0}^\infty q^{2(|m|+1)n}$ is a convergent geometric series,
(\ref{eq:lemma:qqE:series:rewrite}) yields an entire function $g$ which
satisfies (\ref{eq:lemma:qqE:series}).
Now we still have to show that $(f, \Psi^{|m|} g)$
is indeed an \Hmpair\@. Only item (ii) of definition \ref{def:Hmpair}\
requires some explanation; from the power series expansion (\ref{eq:def:qBessel})
of the Hahn-Exton $q$-Bessel functions, it is clear that
$\J{m}{\cdot\,}$ is even (resp.\ odd) whenever $m$ is even (resp.\ odd).
It follows that $g_n$ is always even (for any $n$) and consequently
also $g$ is even. Eventually, $\Psi^{|m|} g$ satisfies item (ii) of
definition \ref{def:Hmpair}, whereas (i) is straight\-forward and (iii) is obvious.
\end{proof}


\begin{remark} \rm
With a more thorough study using the proper estimates for the
Hahn-Exton $q$-Bessel functions involved here, it should probably
be possible to relax the conditions on  the function $f$ a little bit,
yet still ensuring the proper convergence in (\ref{eq:lemma:qqE:series}).
From the theory of entire functions and canonical products, it is nevertheless
clear that there exist plenty of functions satisfying the conditions of the
previous lemma. The $q$-version of the exponential function, as
described in section \S\ref{sec:qExp_and_qBessel}, provides an important example:
$\hfill \star$
\end{remark}



\begin{cor} \label{cor:qqE_in_holS}
Let \qqE\ denote the entire function given by
$\qqE(z) = E_{q^2}(-z) = (z;q^2)_\infty$.
Then $\qqE \in \holS{m}$ for any $m\in \ZZ$.
\end{cor}

\paragraph{One single \lq core\rq\ susceptible to $q$-Hankel transforms of any order}
The problem with the system $(\holS{m},\holH{m})_{m\in \ZZ}$
is that we have to keep track of the order $m$ at any time.
It would certainly be convenient to have some kind of order-independent
domain on which all the $\holH{m}$ can act nicely;
so let's try to find such a \lq core\rq.

\begin{prop} \label{prop:qHankel:nesting}
If $n,m\in \ZZ$ and $|n| \leq |m|$, then $\holS{n} \subseteq \holS{m}$.
\end{prop}
\begin{proof}
Recalling proposition \ref{prop:qHankel:inverseorder},
it suffices to prove that $\holS{m-1} \subseteq \holS{m}$ for all $m\geq 1$.
So let's take any $m\in \NN$ with $m\geq 1$ and any $g\in \holS{m-1}$.
Then $\Omega^{-2} \holH{m-1}g$ still belongs to $\holS{m-1}$,
hence we can apply the first formula of proposition \ref{prop:holqHankel:qdiff}\
with $f=\Omega^{-2} \holH{m-1}g$, yielding
\begin{equation}\label{eq:qHankel:nesting:proof}
     \holH{m} D_{q^2} \Omega^{-2} \holH{m-1} g
      \;=\: -\frac{q^{-1}}{q^{-1}-q} \holH{m-1} \Omega^2 \Omega^{-2} \holH{m-1}g
      \;=\: -\frac{q^{-1}}{q^{-1}-q} \, g.
\end{equation}
We conclude that $g$ belongs to $\holS{m}$.
\end{proof}



\begin{remark} \label{rem:Q:strict_inclusions} \rm
At present we lack an example which shows the inclusions in
proposition \ref{prop:qHankel:nesting} to be {\em strict\/} inclusions,
and as a matter of fact the spaces $\holS{m}$ for various $m\in\ZZ$ are not unlikely
to coincide. For instance from the second formula in proposition
\ref{prop:holqHankel:qdiff}\ one can derive that
$\Psi \holS{m} \subseteq \holS{m-1}$ for any $m\geq 1$, which yields an indication
for the reversed inclusions. In our approach to holomorphic $q$-Hankel transformation
however, it won't be easy to settle this question; but do we really
care?\footnote{no, not really; it will become clear very soon that we can perfectly
well proceed without knowing whether the $\holS{m}$ coincide or not. }
$\hfill \star$
\end{remark}



\begin{defn*} \label{def:Hcore}
We introduce two more subspaces of \HC\ as follows:
\begin{eqnarray*}
  \Hintersect &=& \bigcap_{m\in\ZZ} \holS{m}  \\
  \Hcore      &=& \left\{ f \in \Hintersect \left| \vertL \right.
                  \holH{m}f, \, D_{q^2}^m f \in \Hintersect
                  \mbox{ for all } m\in \NN  \right\} \\
\end{eqnarray*}
\end{defn*}

Observe that $\Hcore  \subseteq \Hintersect \subseteq \Swqbis$.
Also notice the definition of \Hcore\ only refers to {\em positive\/}
order $q$-Hankel transforms, which is justifiable in view of
the propositions \ref{prop:qHankel:inverseorder}\ and \ref{prop:holS:Omega_invariant}\@.
Of course $\Hintersect$ is nothing but $\holS{0}$ (cf.\ proposition \ref{prop:qHankel:nesting}).
However we prefer to use this new symbol $\Hintersect$ (rather than $\holS{0}$)
to emphasize that it will be considered in relation to $q$-Hankel transforms
of {\em any\/} order---not just $\holH{0}$.
Now the problem with $\Hintersect$ is that it might
(cf.\ remark \ref{rem:Q:strict_inclusions})
not be invariant under $\holH{m}$ when $m \neq 0$, and moreover it
is not clear either whether $\Hintersect$ is invariant under $D_{q^2}$.
That's of course exactly the reason why we have introduced the space \Hcore\@.
Indeed we can prove the following

\begin{prop} \label{prop:Hcore:invariance}
\hspace{2pt} \Hcore\ is invariant under $\holH{m}$ for any $m\in \ZZ$.
Furthermore, \Hcore\ is also invariant under $\til$, $\Omega^{\pm 2}$,
$\Psi$ and  $D_{q^2}$.
\end{prop}
\begin{proof}
Obviously $\Hintersect$ is $\Omega^{\pm 2}$-invariant
(cf.\ proposition \ref{prop:holS:Omega_invariant}).
From (\ref{eq:holS:Omega_invariant}) and the commutation rule
$D_{q^2} \Omega^2 = q^2 \Omega^2 D_{q^2}$ it follows that \Hcore\ is
invariant under $\Omega^{\pm 2}$.

Next we will show $D_{q^2}$-invariance; take any $f\in\Hcore$ and put $g=D_{q^2}f$.
Choose any $m\in \NN$ and first assume that $m\geq 1$.
Since $f\in \Hintersect \subseteq \holS{m-1}$, it follows
from the first formula in proposition \ref{prop:holqHankel:qdiff}\ that
$\holH{m}g = ({\rm scalar})\, \holH{m-1} \Omega^2 f$.
Now the latter belongs to $\Hintersect$ because $f\in\Hcore$.
Now consider the $m=0$ case. Since $f\in\Hcore$, we have
$g = D_{q^2}f \in \Hintersect = \holS{0}$, and hence
$\holH{0} g \in \Hintersect$.
So we have shown that $\holH{m}g \in \Hintersect$ for any $m\in\NN$,
whereas clearly $D_{q^2}^m g = D_{q^2}^{m+1} f \in \Hintersect$
for any $m\in\NN$. We conclude that $g\in\Hcore$.

Now we will show by induction on $m\in\NN$ that \Hcore\ is invariant under
$\holH{m}$. Notice it is sufficient to consider positive $m$ only,
because of proposition \ref{prop:qHankel:inverseorder}\@.
Let's first consider the $m=0$ case; take any $f\in\Hcore$, put $g=\holH{0}f$
and observe that $g\in\holS{0}=\Hintersect$.
We shall prove that $g$ belongs to \Hcore\ again;
therefore, choose any $n\in \NN$ and consider $\holH{n} g$ and $D_{q^2}^n g$.
When $n=0$, we obtain $\holH{0} g = \holH{0}^2 f = f \in \Hintersect$.
To deal with $n\geq 1$, rewrite the first formula in proposition
\ref{prop:holqHankel:qdiff}\ as
\begin{equation}\label{eq:holqHankel:qdiff:rewritten}
   \holH{n}  \:=\: ({\rm scalar})\, D_{q^2} \Omega^{-2} \holH{n-1}
\end{equation}
(see also: equation (\ref{eq:qHankel:nesting:proof}) above). Iterating this $n$ times yields
\begin{equation} \label{eq:holqHankel:qdiff:iterated}
   \holH{n}  \:=\: ({\rm scalar})\, (D_{q^2} \Omega^{-2})^n \holH{0}
             \:=\: ({\rm scalar})\, \Omega^{-2n} D_{q^2}^n \holH{0}
\end{equation}
We used the commutation rule for $D_{q^2}$ and $\Omega^{-2}$.
The exact value of the scalars involved here could be computed easily, though
they are not relevant for our purposes (except for the fact they are all non-zero).
Since $g$ belongs to $\holS{0}$ we may apply (\ref{eq:holqHankel:qdiff:iterated})
to it, yielding
$$ \holH{n} g \:=\: ({\rm scalar})\, \Omega^{-2n} D_{q^2}^n f. $$
Now the latter is in $\Hintersect$ because $f\in\Hcore$.
We have shown that $\holH{n} g \in \Hintersect$ for all $n\in \NN$.
On the other hand, (\ref{eq:holqHankel:qdiff:iterated}) is also useful in
the other direction:
$$ D_{q^2}^n g \:=\: D_{q^2}^n \holH{0} f \:=\: ({\rm scalar})\, \Omega^{2n} \holH{n}f. $$
It follows that $D_{q^2}^n g \in \Hintersect$ for all $n\in \NN$,
which completes the $m=0$ case. Now we proceed by induction on $m$.
Let's assume \Hcore\ to be invariant under $\holH{m-1}$ for some $m\geq 1$,
and prove that this implies invariance under $\holH{m}$.
Applying (\ref{eq:holqHankel:qdiff:rewritten}) to an arbitrarily chosen $f\in\Hcore$ yields
$$  \holH{m} f \:=\: ({\rm scalar})\, D_{q^2} \Omega^{-2} \holH{m-1} f. $$
By hypothesis $\holH{m-1} f$ belongs to \Hcore\ again, and since
we have already shown that \Hcore\ is invariant under $\Omega^{-2}$ and
$D_{q^2}$, the result follows.

Invariance under $\til$ follows easily from proposition \ref{prop:qHankel:tilde}\
and the fact that $\til$ commutes with $D_{q^2}$,
so it only remains to prove that \Hcore\ is $\Psi$-invariant.
Therefore consider the $m=1$ case of the second formula in
proposition \ref{prop:holqHankel:qdiff}\@. Plugging in (\ref{eq:def:nabq}) and
canceling some scalars, the formula can be rewritten as
$$ \holH{0} (\id - q^2 \Omega^2) \holH{1} g \:=\: \Psi g $$
for $g\in\holS{1}$ and a fortiori for $g\in\Hcore$.
Since we have already established that \Hcore\ is invariant under
$\holH{0}$, $\holH{1}$ and $\Omega^2$, the result follows.
\end{proof}


\begin{remark} \rm
Again (cf.\ remark \ref{rem:not_constructive}) our approach here
is far from being explicit: still we lack any criterion which
describes the spaces \Hintersect\ and \Hcore\ in a direct way; we
merely {\em define\/} \Hcore\ to be the largest space on which one
can happily take iterated $q$-Hankel transforms of any order. It's
very likely that one could do better than this, unraveling all the
details of $q$-Hankel transformation: for instance---instead of
restricting ourselves to this \lq core\rq\ \Hcore\ of functions
susceptible to $q$-Hankel transformation of arbitrary order---one
could maybe obtain a richer theory if one sticks to the original
system $(\holS{m},\holH{m})_{m\in \ZZ}$, keeping track of the
orders at any time\ldots Our main concern however, is to proceed
into harmonic analysis on the quantum $E(2)$ group. Nevertheless
we will show---constructively---that \Hcore\ still contains an
important class of functions. $\hfill \star$
\end{remark}


\paragraph{An eigenfunction of $q$-Hankel transformation}
Below we shall prove that the entire function $z \mapsto E_{q^2}(-q^2 z)$
belongs to the space \Hcore\ constructed in the previous paragraph,
and moreover, that it constitutes an eigenfunction of all positive
order holomorphic $q$-Hankel transforms. This function of $q^2$-exponential
type plays a role similar to the
Gaussian\footnote{and as a matter of fact, this $q^2$-exponential does amount
to a $q$-Gaussian if one takes into account the \lq{\em squaring\/} operator\rq\ $K$
appearing in diagram (\ref{eq:diagram:holomorphic_qHankel}).}
in ordinary Fourier analysis on the real line.

\begin{lemma}
Let \qE\ denote the entire function given by $\qE(z) = E_q(-z) = (z;q)_\infty$.
Then
$$ D_q \qE \:=\: -\frac{1}{1-q}\, \Omega \qE. $$
\end{lemma}
\begin{proof}
We could easily derive this from the power series (\ref{eq:qExp:powerseries})
but it is even more convenient to use the product representation:
indeed (\ref{eq:shifted_factorial:infty}) implies
$\qE(z) = (1-z)\qE(qz)$ for all $z\in \CC$, and
$$ (D_q \qE)(z) \;=\;  \frac{\qE(z) - \qE(qz)}{(1-q)z}
                \;=\; \frac{-z\,\qE(qz)}{(1-q)z}
                \;=\; -\frac{1}{1-q} \, (\Omega \qE)(z). $$
for $z\neq 0$. Extending the result to $z=0$ by continuity completes the proof.
\end{proof}

\begin{cor} \label{cor:qqderivative:qqE}
Replacing $q$ by $q^2$ and (consequently!) $\Omega$ by $\Omega^2$, we get
$$  D_{q^2} \qqE \:=\: -\frac{1}{1-q^2}\, \Omega^2 \qqE. $$
\end{cor}

\begin{prop} \label{prop:qdifferential_equation:qexp}
The function $\Omega^2 \qqE = (\,\cdot\,q^2 ;q^2)_\infty$ satisfies the
$q^2$-differential equation
\begin{equation}\label{eq:qdifferential_equation:qexp}
   D_{q^2} f \;=\; -\frac{q^2}{1-q^2}\, \Omega^2 f
     \hspace{4em} \mbox{($f$ entire).}
\end{equation}
Moreover the solution of this $q^2$-differential equation is
essentially unique, in the sense that any entire function $f$
satisfying (\ref{eq:qdifferential_equation:qexp}) must be a scalar
multiple of\/ $\Omega^2 \qqE$.
\end{prop}
\begin{proof}
Combining corollary \ref{cor:qqderivative:qqE}\ with the obvious
commutation rule for $D_{q^2}$ and $\Omega^2$ yields
$$ D_{q^2} \Omega^2 \qqE \:=\: q^2 \Omega^2 D_{q^2} \qqE
             \:=\: -\frac{q^2}{1-q^2}\, \Omega^2 \Omega^2 \qqE. $$
To prove the uniqueness statement, let's evaluate (\ref{eq:qdifferential_equation:qexp})
in $q^{2n}$ for any $n\in\NN$. We get
$$ \frac{f(q^{2n})-f(q^2 q^{2n})}{(1-q^2) q^{2n}}
            \;=\;  -\frac{q^2}{1-q^2}\, f(q^2 q^{2n}) $$
Putting $x_n = f(q^{2n})$ for $n\in\NN$, we obtain a sequence $(x_n)_{n=0}^\infty$
of complex numbers satisfying the following recurrence relation:
$$ x_n \,=\: (1-q^{2n} q^2)\, x_{n+1}. $$
Iterating this relation yields
\begin{equation}\label{eq:uniqueness:recurrence_relation}
   x_0 \:=\: (1-q^2)(1-q^2 q^2) \ldots (1-q^{2(n-1)} q^2) \, x_n
       \:=\: (q^2;q^2)_n \, x_n.
\end{equation}
It follows that the sequence $(x_n)_n$ is completely determined by
the value of $x_0$. In other words, (\ref{eq:qdifferential_equation:qexp}) and
$f(1)$ determine the value of $f$ at the points $q^{2n}$ ($n\in\NN$).
Since $q^{2n} \rarr 0$ as $n\rarr\infty$, we may invoke the identity theorem
for holomorphic functions and draw the conclusion (notice that $f(1)=0$ implies $f=0$,
and observe that (\ref{eq:qdifferential_equation:qexp}) is {\em linear\/} in $f$).
\end{proof}

It is also instructive to observe (\ref{eq:shifted_factorial:infty}) implies that
$x_n = (q^{2n} q^2; q^2)_\infty = (\Omega^2 \qqE)(q^{2n})$ indeed satisfies the
above recurrence relation (\ref{eq:uniqueness:recurrence_relation}).

\begin{lemma}
Let $m$ be any non-negative integer. If $f$ is a solution of (\ref{eq:qdifferential_equation:qexp})
then so is $\holH{m}f$ (notice $f\in \Hintersect$ because of proposition
\ref{prop:qdifferential_equation:qexp}\ and corollary \ref{cor:qqE_in_holS},
so the statement makes sense).
\end{lemma}
\begin{proof}
With the commutation rule $D_{q^2} \Omega^2 = q^2 \Omega^2 D_{q^2}$
the first formula in proposition \ref{prop:holqHankel:qdiff}\ can be
rewritten in the form
\begin{equation}\label{eq:holqHankel:qdiff:rewritten:bis}
    D_{q^2} \holH{m-1} = -\frac{q^2}{1-q^2}\, \Omega^2 \holH{m}
\end{equation}
for $m\geq 1$. Replacing $m$ with $m+1$ and applying this to a solution $f$ of
(\ref{eq:qdifferential_equation:qexp}) yields
$$  D_{q^2} \holH{m} f
     \;=\; -\frac{q^2}{1-q^2}\, \Omega^2 \holH{m+1} f
     \;\stackrel{(\ref{eq:qdifferential_equation:qexp})}{=}\;
             \Omega^2 \holH{m+1} \Omega^{-2} D_{q^2} f
     \;=\; q^2 \Omega^2 \holH{m+1} D_{q^2} \Omega^{-2} f $$
for any $m\geq 0$. Once again applying the first formula in
proposition \ref{prop:holqHankel:qdiff}\ yields
\begin{equation}\label{eq:HolHmf:solution}
   D_{q^2} \holH{m} f  =  -\frac{q^2}{1-q^2}\, \Omega^2 \holH{m} f
\end{equation}
which means that $\holH{m}f$ is a solution to (\ref{eq:qdifferential_equation:qexp}).
\end{proof}

\begin{remark} \rm
The above lemma does not extend to negative $m$. To see this,
apply proposition \ref{prop:qHankel:inverseorder}\ to (\ref{eq:HolHmf:solution}).
The reason for this remarkable lack of symmetry is not so clear at the moment.
$\hfill \star$
\end{remark}

\begin{prop} \label{prop:qqE:eigenfunction_qHankel}
$\;\Omega^2 \qqE$ is an eigenfunction of all $\holH{m}$ with $m\geq 0$.
\end{prop}
\begin{proof}
Combine the above lemma with proposition \ref{prop:qdifferential_equation:qexp}\@.
\end{proof}

\vspace{2ex}
It is probably not too hard to compute the corresponding eigenvalues.
Furthermore it's likely possible to construct the other eigenfunctions as
well, and one might expect them to be of the form (\ref{eq:many_functions_in_Hcore}) below.
These eigenfunctions should then play a role quite similar to Hermite
functions in classical Fourier analysis.
However from our interest, the main issue here is that it finally becomes clear that
our \lq core\rq\ \Hcore\ is non-trivial; indeed it does contain an important class of
functions:

\begin{cor} \label{cor:qExp_in_Hcore}
For any polynomial $P$ and any integer $k$, the entire function
\begin{equation}\label{eq:many_functions_in_Hcore}
  \CC \rarr \CC : \: z \,\mapsto \, P(z)\, E_{q^2}(-q^{2k} z)
\end{equation}
belongs to the space \Hcore\ of definition \ref{def:Hcore}.
\end{cor}
\begin{proof}
We already established that $\Omega^2 \qqE$ belongs to $\Hintersect$
(cf.\ corollary \ref{cor:qqE_in_holS}) and from
proposition \ref{prop:qqE:eigenfunction_qHankel}\
it is clear that $\holH{m} \Omega^2 \qqE$ is in $\Hintersect$ as well,
for any $m\in \NN$. Furthermore, the condition of lemma
\ref{lemma:compactsupp_in_holS}\ is obviously invariant under $q^2$-differentiation,
hence $D_{q^2}^m \Omega^2 \qqE$ still belongs to $\Hintersect$, for all $m\in \NN$.
We conclude that $\Omega^2 \qqE$ is contained in \Hcore\@.
Eventually, the invariance properties of \Hcore\
(cf.\ proposition \ref{prop:Hcore:invariance}) yield the result.
\end{proof}




\paragraph{The $q$-moment problem for the space $\Hintersect$}
will play a key-role in establishing {\em uniqueness\/} of Fourier transforms
for quantum E(2). Moreover, one of the intermediate results in
the present paragraph shall also be crucial for the {\em construction\/} of
these Fourier transforms.
In classical moment problems the typical question reads as follows:
suppose a function of some particular class has vanishing moments;
may we then conclude that the function itself vanishes everywhere?
In this paragraph however we shall consider \mbox{$q$-moments},
i.e.\ moments computed w.r.t.\ a well-known $q$-analogue of the integral.
Restricting to functions in the space $\Hintersect$ of definition \ref{def:Hcore},
our $q$-moment problem has positive answer---although the proof will turn out
to be quite tricky.

\begin{defn}
Let $f$ be any complex valued function defined on some subset of
$\CC$ containing the points $q^{n}$ with $n\in \ZZ$.
The $q$-integral (or {\em Jackson\/} integral) of $f$ is then defined by
$$ \int_0^\infty f(x)\, d_q x  \;=\;  (1-q) \sum_{n\in \ZZ} \: f(q^{n}) \, q^{n} $$
provided the summation in the {\sc rhs} converges absolutely.
\end{defn}

Notice this is different from integration on $\RR_q^+ \equiv (\RR_q^+,m_q)$
as introduced at the beginning of section \S \ref{sec:qHankel}\@.
Furthermore we shall actually be dealing with $q^2$-integration rather
than with $q$-integration. First we establish a link between
$q^2$-moments and $q$-Hankel transformation:

\begin{prop} \label{prop:moment:link_with_Hankel}
Take any $m\in \NN$ and $f\in\holS{m}$. Then
$$ (1-q^2)\:(q^2;q^2)_m \, (\holH{m}f)(0) \;=\, \int_0^\infty x^m f(x)\, d_{q^2}x.  $$
Observe that the $q^2$-integral in the {\sc rhs} is well-defined
since $f$ belongs to $\Swqbis$.
\end{prop}
\begin{proof}
Let $(f,g)$ be an \Hmpair\@. According to proposition
\ref{prop:exist:holomorphic_qHankel}\ (see proof)
we have $(\holH{m}f)(z^2) = z^{-m} g(z)$ for all $z\in \CC_0$, and in particular
we obtain for $k \in \ZZ$ that
\begin{eqnarray}
(\holH{m}f)(q^{2k}) &=& q^{-mk} g(q^k)
   \hspace{1em}=\hspace{1em}
      q^{-mk} (H_m R_q \Psi^m K f)(q^k)  \\
   &\stackrel{(\ref{eq:qHankeltransform:def})}{=}&
     \sum_{n\in \ZZ} \: q^{-m(n+k)} \J{m}{q^{n+k}}\, q^{2(m+1)n}  f(q^{2n}) \\
   &=&
     \sum_{n\in \ZZ} \: \varJ{m}(q^{n+k}) \, q^{2(m+1)n}  f(q^{2n})
     \label{eq:momentq:full_series}
\end{eqnarray}
where $\varJ{m}$ is the {\em entire\/} function satisfying
$\varJ{m}(z)= z^{-m} \J{m}{z}$ for $z\in \CC_0$
(the singularity at the origin being removable, cf.\ lemma \ref{lemma:def:qBessel}).
For any $k,n \in \ZZ$, let $t_{n,k}$ denote the number
$$ t_{n,k}  \:=\; \varJ{m}(q^{n+k}) \, q^{2(m+1)n}  f(q^{2n}). $$
Since $\holH{m}f$ is entire, it is a fortiori continuous at the origin,
and therefore
\begin{equation}\label{eq:moment:holHmf_at zero:limit:full_summ}
   (\holH{m}f)(0) \:=\; \lim_{k \rarr +\infty} (\holH{m}f)(q^{2k})
        \:=\; \lim_{k \rarr +\infty}
               \left(\vertL \textstyle \sum_{n\in \ZZ} \, t_{n,k}\right)
\end{equation}
Now comes the tricky part: we have to compute the above limit,
which roughly speaking amounts to interchanging the limit and the summation.
Therefore the summation shall be split into two parts, the splitting being
$k$-dependent in a quite peculiar way, as follows: for any $k\in\NN$ we define
\begin{equation}\label{eq:moment:series:splitting}
   a_k = \sum_{n=-\infty}^{n_k-1}   t_{n,k}
                      \andspace{5em}
   b_k = \sum_{n=n_k}^{+\infty}     t_{n,k}
\end{equation}
where $n_k$ denotes the integer such that
\begin{equation}\label{eq:moment:defn:smallest_int:nk}
  n_k -1   \:\leq\:  \frac{-k}{4(m+1)}   \: < \:  n_k.
\end{equation}
Both series in (\ref{eq:moment:series:splitting}) should converge because
together they constitute the series (\ref{eq:momentq:full_series}),
the latter being absolutely convergent since it arose from a well-defined
$q$-Hankel transform (cf.\ the $L^2$-theory). It is however instructive
to investigate convergence of the summations in (\ref{eq:moment:series:splitting})
directly: the one that defines $a_k$ converges, because $f$ belongs to $\Swqbis$
and because $\varJ{m}(q^{n+k})$ tends to zero as $n \rarr -\infty$. The
series defining $b_k$ converges since $f$ and $\varJ{m}$ are
bounded in the neighborhood of the origin.
Now (\ref{eq:moment:holHmf_at zero:limit:full_summ}) can be rewritten as
\begin{equation}\label{eq:moment:lim:ak+bk}
   (\holH{m}f)(0) \:=\; \lim_{k \rarr +\infty} (a_k + b_k)
\end{equation}
and we claim the limits of $a_k$ and $b_k$ to exist {\em separately}.

To compute $\lim_{k \rarr +\infty} b_k$ we introduce one more
sequence $(c_k)_{k=0}^\infty$ of numbers, as follows
$$   c_k = \sum_{n=n_k}^{+\infty}  \varJ{m}(0) \, q^{2(m+1)n}\, f(q^{2n}). $$
Observe the definitions of $b_k$ and $c_k$ are quite similar,
and the sums defining them converge absolutely for quite the same reasons.
Now contrary to the sequence $(b_k)_{k=0}^\infty$,
the limit of $(c_k)_{k=0}^\infty$ is easily computed; indeed as $k \rarr +\infty$,
then $n_k \rarr -\infty$, and hence
\begin{equation}\label{eq:moment:limit:ck}
 \lim_{k \rarr +\infty} c_k
     \:= \:  \varJ{m}(0)  \lim_{j \rarr -\infty}
             \left(\, \sum_{n=j}^{+\infty} \, q^{2mn}\, f(q^{2n}) \, q^{2n}\right)
     \:= \:  \frac{\varJ{m}(0)}{1-q^2} \: \int_0^\infty x^m f(x)\, d_{q^2}x.
\end{equation}
Our next goal is to prove that
\begin{equation}\label{eq:moment:lim:bk-ck:zero}
  \lim_{k \rarr +\infty} (b_k - c_k) \, = 0.
\end{equation}
Since $\varJ{m}$ is entire, we can find a bound $N_1>0$ such that
\begin{equation}\label{eq:moment:estimate:varJ}
  |\varJ{m}(z) - \varJ{m}(0)| \; \leq  \; N_1 |z|  \hspace{4em}  \mbox{for all $|z| \leq 1$.}
\end{equation}
On the other hand $f$ belongs to $\Swqbis$, so we also have a bound,
say $N_2>0$, such that $|f(q^{2n})| \leq N_2$ for all $n\in \ZZ$.
Observe the condition $n\geq n_k$ together with $k\geq 0$ and (\ref{eq:moment:defn:smallest_int:nk})
implies $n+k$ to be positive, and hence the estimate (\ref{eq:moment:estimate:varJ})
to be valid for $z=q^{n+k}$. Consequently
\begin{eqnarray*}
   |b_k - c_k| & \leq &  \sum_{n=n_k}^{+\infty}
         \left| \vertM \varJ{m}(q^{n+k}) - \varJ{m}(0)  \right|
         \, q^{2(m+1)n} \, |f(q^{2n})|   \\
      & \leq &  N_1 N_2  \sum_{n=n_k}^{+\infty}  q^{n+k} q^{2(m+1)n} \\
      & \leq &  N_1 N_2 \: q^{\frac{1}{2}k}  \sum_{n=n_k}^{+\infty}  q^n
                \underbrace{q^{2(m+1)n + \frac{1}{2}k}}_{\leq \,1}
\end{eqnarray*}
Now the summation range and (\ref{eq:moment:defn:smallest_int:nk}) imply
$2(m+1)n + \frac{1}{2}k$ to be positive, and therefore
$$ |b_k - c_k|  \;\leq\;   N_1 N_2 \: q^{\frac{1}{2}k} \sum_{n=n_k}^{+\infty} q^n
                \;=\;      N_1 N_2 \: q^{\frac{1}{2}k} q^{n_k} \sum_{n=0}^{+\infty} q^n
                \;=\;      \frac{N_1 N_2}{1-q} \, q^{\frac{1}{2}k + n_k} $$
for all $k\in\NN$. Now
$$ \frac{1}{2}k + n_k   \;>\;  \frac{1}{2}k + \frac{-k}{4(m+1)}
     \;=\;  \frac{2m+1}{4(m+1)} \, k,  $$
hence we found some non-zero positive numbers, say
$$ C= \frac{N_1 N_2}{1-q} \andspace{4em} \gamma = \frac{2m+1}{4(m+1)} $$
only depending on $q$, $m$ and $f$, such that $|b_k - c_k| \leq C q^{\gamma k}$
for all $k\geq 0$. This proves (\ref{eq:moment:lim:bk-ck:zero}).


Now we still have to deal with the $a_k$ in (\ref{eq:moment:lim:ak+bk}).
Observe $\varJ{m}$ is bounded in the neighborhood of the origin, whereas
$\varJ{m}(q^r)$ tends to zero as $r \rarr -\infty$. It follows
that there exists a bound $M_1>0$ such that $|\varJ{m}(q^r)| \leq M_1$
for all $r \in \ZZ$. On the other hand, since $f$ belongs to $\Swqbis$
we can find a number $M_2>0$ such that
$$ |q^{4(m+1)n} f(q^{2n})| \; \leq \: M_2
         \hspace{3em} \mbox{and hence} \hspace{3em}
   q^{2(m+1)n} \, |f(q^{2n})| \; \leq \: M_2 \, q^{-2(m+1)n}    $$
for all $n\in \ZZ$.
It follows that $|t_{n,k}| \leq M_1 M_2\, q^{-2(m+1)n}$ for all $k,n\in \ZZ$,
and hence
$$ |a_k| \: \leq \sum_{n=-\infty}^{n_k-1} |t_{n,k}|
         \; \leq \; M_1 M_2 \sum_{n=-\infty}^{n_k-1} q^{-2(m+1)n} $$
for all $k\in \NN$.
Substituting the summation index by $n=n_k-1-j$ we obtain
$$ |a_k| \: \leq \; M_1 M_2 \sum_{j=0}^{+\infty}  q^{-2(m+1)(n_k-1-j)}
  \:=\;  M_1 M_2\: q^{-2(m+1)(n_k-1)} \sum_{j=0}^{+\infty}  q^{2(m+1)j} $$
If we define $u=q^{2(m+1)}$, then $0<u<1$, and
$$ |a_k| \: \leq \;  M_1 M_2\: u^{-n_k} \,u  \sum_{j=0}^{+\infty} u^j
           \:=\;  M_1 M_2\, \frac{u}{1-u} \, u^{-n_k} $$
where $M_1$, $M_2$ and $u$ only depend on $q$, $m$ and $f$.
Recalling that $n_k \rarr -\infty$ when $k \rarr +\infty$,
it follows that $\lim_{k \rarr +\infty} a_k = 0$.
Putting all the pieces together, we obtain
$$ (\holH{m}f)(0) \;=\: \lim_{k \rarr +\infty} (a_k + b_k)
      \;=\: \underbrace{\lim_{k \rarr +\infty} a_k}_{=\,0}
       \: + \underbrace{\lim_{k \rarr +\infty} (b_k - c_k)}_{=\,0}
       \: + \underbrace{\lim_{k \rarr +\infty} c_k}_{(\ref{eq:moment:limit:ck})}.$$
Now it only remains to compute the number $\varJ{m}(0)$ appearing
inside equation (\ref{eq:moment:limit:ck}). It is clear however
that $\varJ{m}(0)$ is nothing but the $(k=0)$-coefficient in the
power series (\ref{eq:def:qBessel}; $q$ replaced with $q^2$) that
defines the Hahn-Exton $q^2$-Bessel functions, yielding
$$ \varJ{m}(0) \:=\:  \frac{(q^{2(m+1)};q^2)_\infty}{(q^2;q^2)_\infty}
               \:=\:  \frac{1}{(q^2;q^2)_m} $$
This completes the proof.
\end{proof}

\begin{lemma} \label{lemma:moment:all_qderivs_zero}
If $f$ is an entire function such that $(D_q^m f)(0) = 0$ for all $m\in \NN$,
then $f=0$.
\end{lemma}
\begin{proof}
Let's first introduce the following notion of $q$-factorials: for $n\in \NN$, put
$$ \varqfac{n}  \:=\: \frac{1-q^n}{1-q}
         \hspace{5em}
   \varqfac{0}! \:=\, 1 \hspace{5em}
   \varqfac{n}! \:=\: \varqfac{1} \varqfac{2} \ldots \varqfac{n}.  $$
Now let $f(z) = \sum_{k=0}^\infty a_k\, z^k$ be the power series expansion of $f$
around the origin. Then
$$ (D_q f)(z)
       \;=\; \frac{1}{(1-q)z} \left(\:
             \sum_{k=0}^\infty a_k\, z^k  - \sum_{k=0}^\infty a_k\, (q z)^k  \right)
       \;=\; \sum_{k=1}^\infty \: \varqfac{k} \, a_k \, z^{k-1} $$
for all $z\in \CC_0$ (and hence also for $z=0$).
Iterating this $m\in \NN$ times and then evaluating at the origin, we obtain
$(D_q^m f)(0) = \varqfac{m}! \, a_m$. The result follows.
\end{proof}


\begin{lemma}
If $f\in\Hintersect$ such that $(\holH{m}f)(0) = 0$ for all $m\in \NN$,
then $f=0$.
\end{lemma}
\begin{proof}
Recall the formula (\ref{eq:holqHankel:qdiff:iterated}) which was
derived from proposition \ref{prop:holqHankel:qdiff}, and apply it to $f$:
$$  \holH{m} f \:=\: ({\rm scalar})\, \Omega^{-2m} D_{q^2}^m \holH{0} f $$
for any $m\in\NN$. With our assumptions it follows that
$(D_{q^2}^m \holH{0} f)(0) = 0$ for all $m\in \NN$.
Since $\holH{0} f$ is entire, we may draw our conclusions from lemma
\ref{lemma:moment:all_qderivs_zero}\ ($q$ replaced with $q^2$).
\end{proof}

\vspace{2ex}
Combining this lemma with proposition \ref{prop:moment:link_with_Hankel}, we get

\begin{thm} \label{thm:qmoment}
If $f\in\Hintersect$ has vanishing $q^2$-moments, i.e.\
$\int_0^\infty x^m f(x)\, d_{q^2}x \,= 0$ for all $m\in\NN$, then $f=0$.
\end{thm}
