\section{Antipodes}


\begin{prop_sec}  \label{prop:antipodes:anti_homomorphisms}
If\/ \pairAB\ is a Hopf system, then its antipodes
$$ \SA : A \rarr \Env(\Aa)     \hspace{8em}    \SB : B \rarr \Env(\BB). $$
are anti-homomorphisms.
\end{prop_sec}

\begin{proof}
Take any $a,c\in A$ and $b\in B$ and write
$\rp^{-1}(a \tens b) = \sum_i \, p_i \tens q_i$ with $p_i \in A$ and $q_i \in B$.
Now observe that
\begin{eqnarray*}
      \pairM{\SA(a)\, \SA(c)}{b}
&=&
      \pairM{\SA(a)}{\SA(c) \lact b}
\\&=&
      \pairM{P^{-1}}{a \tens \left(\SA(c) \lact b\vertM\right)}
\\&=&
      \pairM{P^{-1} \left(1_\BB \tens \SA(c) \vertM\right)}{a \tens b}
\\&=&
      \pairM{1_\BB \tens \SA(c)}{\rp^{-1}(a \tens b)}
\\&=&
      \textstyle \sum_i \, \pairM{1_\BB}{p_i} \pairM{\SA(c)}{q_i}
\\&=&
      \textstyle \sum_i \, \pairM{1_\BB}{p_i} \pairM{P^{-1}}{c \tens q_i}
\\&=&
      \textstyle \sum_i \,
      \pairM{1_\BB \tens 1_\BB \tens 1_\Aa}{
             (\rp^{-1})_{13} (c \tens p_i \tens  q_i)}
\\&=&
      \pairM{1_\BB \tens 1_\BB \tens 1_\Aa}{
             (\rp^{-1})_{13} (\rp^{-1})_{23} (c \tens a \tens b)}
\\&\stackrel{(*)}{=}&
      \pairM{1_\BB \tens 1_\Aa}{
             (\mult{A} \tens \id) (\rp^{-1})_{13} (\rp^{-1})_{23} (c \tens a \tens b)}
\\&\stackrel{(\sharp)}{=}&
      \pairM{1_\BB \tens 1_\Aa}{ \rp^{-1} (\mult{A} \tens \id) (c \tens a \tens b)}
\\&=&
      \pairM{P^{-1}}{ca \tens b}
\\&=&
      \pairM{\SA(ca)}{b}.
\end{eqnarray*}
Observe how $(*)$ relies on $\epsA \simeq 1_\BB$ being a homomorphism
(proposition \ref{prop:counits_homomorphism}).
Proposition \ref{prop:twistings:lprp}.ii implies (\ref{eq:lemmma:twisting2}) with $R=\rp$.
This yields $(\sharp)$.
\end{proof}





\begin{lemma_sec} \label{lem:unital_over_range_S}
If\/ \pairAB\ is a Hopf system,
then\/ $\SB(B) \lact A = A = A \ract \SB(B)$.
In other words, $A$ is unital as an\/ $\SB(B)$-bimodule
\rm (recall that $\SB(B)$ is a subalgebra of $\Env(\BB)$ because of
     proposition \ref{prop:antipodes:anti_homomorphisms}).
\end{lemma_sec}
\begin{proof}
Choose any $a \in A$ and $d \in B$.
Also take $c \in A$ and $b \in B$ with $\pair{c}{b}=1$, and write
$\lp(a \tens b) = \sum_i \, p_i \tens q_i$ with $p_i \in A$ and $q_i \in B$.
Then we have
\begin{eqnarray*}
      \pair{a}{d}
&=&
      \pairM{d \tens c}{a \tens b}
\\&=&
      \textstyle \sum_i \,  \pairM{d \tens c}{\lp^{-1}(p_i \tens q_i)}
\\&=&
      \textstyle \sum_i \,  \pairM{P^{-1}}{(p_i \tens q_i) \ract (d \tens c)}
\\&=&
      \textstyle \sum_i \,  \pairM{p_i \ract d}{\SB(q_i \ract c)}
\\&=&
      \textstyle \sum_i \,  \pairM{p_i}{d\, \SB(q_i \ract c)}
\\&=&
      \textstyle \sum_i \,  \pairM{\SB(q_i \ract c) \lact p_i}{d}
\end{eqnarray*}
and hence $a = \sum_i \, \SB(q_i \ract c) \lact p_i$. The other case is similar.
\end{proof}
\vspace{2ex}


Recall that $\SA(A)$ is a subspace of $B'$ (cf.\ definition \ref{def:antipode}).
On the other hand $B'$ is a $B$-bimodule under canonical actions
(\S \ref{sect:def_act_context}) so the following makes sense:

\begin{lemma_sec}
Let\/ \pairAB\ be a Hopf system. Then\/ $\SA(A)$ is a sub-$B$-bimodule of\/ $B'$.
In fact we have for any\/ $a \in A$ and\/ $b \in B$ that
\begin{equation} \label{eq:lemma:antipode_and_actions}
  \SA(a) \ract b  \:=\:  \SA\left(\vertM  \SB(b) \lact a\right).
\end{equation}
\end{lemma_sec}

\begin{proof}
Anti-multiplicativity of\/ $\SB$ yields that for any $a \in A$ and $b,d \in B$
\begin{eqnarray*}
      \pairM{\SA(a) \ract b}{d}
  &=&
      \pairM{\SA(a)}{bd}
\\&=&
      \pairM{a}{\SB(bd)}
\\&=&
      \pairM{a}{\SB(d)\,\SB(b)}
\\&=&
      \pairM{\SB(b) \lact a}{\SB(d)}
\\&=&
      \pairM{\SA\left(\vertM  \SB(b) \lact a\right)}{d}
\end{eqnarray*}
and (\ref{eq:lemma:antipode_and_actions}) follows.
Similarly for the {\em left\/} actions of $B$ on $\SA(A)$.
%%%%%%%%%%%%
% It's instructive to notice that
% most of the above actually takes place within the the enveloping \context\
% $\tilde{\BB} \equiv \left(\vertM \Env(\BB); A,\pairing  \right)$
% as introduced in proposition \ref{prop:def:enveloping_context}\@.
%%%%%%%%%%%%%%
\end{proof}


\begin{cor_sec}  \label{cor:antipode_and_counit}
Let\/ \pairAB\ be a Hopf system.
If we assume that\/ $\SA(A) \subseteq A$, then\/ $\epsA \SA = \epsA$.
Similarly, $\SB(B) \subseteq B$ implies\/ $\epsB \SB = \epsB$.
\end{cor_sec}

\begin{proof}
Assume $\SA(A) \subseteq A$.
Then we may apply $\epsA$ to (\ref{eq:lemma:antipode_and_actions}), yielding
$$ \epsA \SA \left(\vertM  \SB(b) \lact a\right)
     \:=\:  \pairM{\SA(a)}{b}
     \:=\:  \pairM{a}{\SB(b)}
     \:=\:  \epsA \left(\vertM \SB(b) \lact a \right).   $$
According to lemma \ref{lem:unital_over_range_S}\ we have $\SB(B) \lact A = A$,
so the result follows.
\end{proof}
