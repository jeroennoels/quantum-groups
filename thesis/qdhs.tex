
\subsection{The quantum double as a Hopf system}

\begin{thm}
Let\/ \pairAB\ be a balanced\footnote{see definition \ref{def:balanced_Hopf_system}}\
regular Hopf system and define\/
\mbox{$R=\rp\lp^{-1}$}\ and\/ $T=R\flip$. Then\/
$\pair{A \ttens{T} B\op}{B \tens A} \,\equiv\, \pair{X}{Y}$
is a regular Hopf system.
Counits and antipodes are given by
\begin{equation}\label{eq:QD:counits_and_antipodes}
\begin{array}{lcl}
  \epsX = \epsA \tens \epsB   &\hspace{2em}&  \SX = R(\SA \tens \SB\op)    \\
  \epsY = \epsB \tens \epsA   &            &  \SY = (\SB \tens \SA\op)\, \piP \vertL
\end{array}
\end{equation}
Furthermore, if\/ $\phiA$ and\/ $\phiB$ are left invariant functionals on\/ $A$ and on\/ $B$,
then\/ $\phiA \tens \phiB$ is a left invariant functional on\/ $A \ttens{T} B\op$.
\end{thm}


\begin{proof}
Propositions \ref{prop:qdouble_is_idpa}\ and \ref{prop:qdouble:multiplicative}\
yield that $\pair{X}{Y}$ is a Hopf system, whereas
(\ref{eq:QD:counits_and_antipodes}) follows from
proposition \ref{prop:QD:dpa}\ and \mbox{lemma \ref{lem:QD:antipode}}\@.
Obviously the antipodes on $X$ and $Y$ are bijective,
hence $\pair{X}{Y}$ is regular.
The result on invariant functionals follows since $\YY = \BBAA$.
\end{proof}


\begin{defn}
The Hopf system $\pair{A \ttens{T} B\op}{B \tens A}$ in the above theorem
is called the {\em quantum double\/} of $\pair{A}{B\op}$.
\end{defn}

{\small
Notice that we have chosen to associate our quantum double to $\pair{A}{B\op}$ rather
than to \pairAB\@. Thus our definition is compatible with the ones in
literature \cite{Schmudgen,Majid,FonsSabine}\@.}



\begin{prop} \label{prop:QD:rmhs}
If\/ \pairAB\ is a regular \mhs,
then so is the quantum double\/ $\pair{A \ttens{T} B\op}{B \tens A}$.
\end{prop}
\begin{proof}
First recall that \mhs s are balanced, so we do have a quantum double.
Now take transposes of (\ref{eq:QD:alpha}) and (\ref{eq:QD:beta})
and recall that $\piP$ is dual to $R$.
Lemmas \ref{mhs:Gammas:bijective}\ and \ref{Mregularity_is_automatic}\
yield the result.
\end{proof}


\begin{cor}
If we adopt the setting of proposition \ref{prop:mha_is_rmhs}, then
$$ \left( A \ttens{T} \hat{A}\op, \;
       \flipsub{23} (\Delta \tens \hat{\Delta}) \right)
\itandspace{2em}
   \left( \hat{A} \tens A, \;
       \flipsub{23} (\piP)_{23} (\hat{\Delta} \tens \flip\Delta) \right)$$
are again regular \mha s with invariant functionals.
\end{cor}



\begin{remarks}
\item
By assumption $P$ and $P^{-1}$ belong to $M(\hat{A} \tens A)$.
Thus $(\piP)_{23}$ can be considered as an automorphism of
$M(\hat{A} \tens \hat{A} \tens A \tens A)$.
\item
We have to be a little bit careful with the comultiplication on $A \ttens{T} \hat{A}\op$.
A priori $\flipsub{23} (\Delta \tens \hat{\Delta})$ is to be considered
merely as a map from $A \tens \hat{A}\vertL$ into
$\vertL (\hat{A} \tens A \tens \hat{A} \tens A)'$.
It is then part of the assertion that this comultiplication
really ends up in the multiplier algebra of
$(A \ttens{T} \hat{A}\op)  \tens  (A \ttens{T} \hat{A}\op)$.
\item
The two objects obtained in this corollary are dual to each other and
can be considered as quantum doubles of $(A, \flip\Delta)$
within the category of regular \mha s with invariant functionals
\cite{FonsDra:QD}\@.
\end{remarks}



\begin{proof}
According to proposition \ref{prop:QD:rmhs}, the pair
$\pair{A \ttens{T} \hat{A}\op}{\hat{A} \tens A}$
is again a regular \mhs\@.
Theorem \ref{thm:mhs_yields_mha}\ yields the result.
\end{proof}
