

\chapter{$q$-calculus} \label{app:qcalc}


Fix any non-zero complex number $q$. For any $m \in \NN$, the {\em $q$-number\/}
$[m]_q$ is defined by
$$  [m]_q  \:=\:  \frac{q^m - q^{-m}}{q - q^{-1}}.  $$
We also consider the {\em $q$-factorial\/} $\qfac{m}$ given by
$$  \qfac{m}  \:=\:  [1]_q [2]_q \ldots [m]_q     \hspace{5em} \qfac{0} = 1, $$
and the {\em $q$-shifted factorial\/} $(a;q)_m$ defined by
$$ (a;q)_m    \:=\:  \prod_{k=0}^{m-1} (1-a q^k)  \hspace{5em} (a;q)_0 =1  $$
for any $a\in \CC$. These two are closely related by the formula \cite{Schmudgen}
$$ \qfac{m}   \:=\: \frac{q^{-\frac{1}{2}m(m-1)}}{(1-q^2)^m} \, (q^2;q^2)_m. $$



\paragraph{The $q$-Exponential function}
Let $q$ be any real number with $0<q<1$.

\begin{defn_chp} \label{def:shifted_factorial:qExp}
Take any $z\in \CC$ and recall the $q$-shifted factorials $(z;q)_m$ as introduced above.
Since $0<q<1$, the limit $m\rarr \infty$ is well defined and yields
\begin{equation}\label{eq:shifted_factorial:infty}
(z;q)_\infty  \:=\: \lim_{m\rarr \infty} (z;q)_m
              \:=\: \prod_{k=0}^{\infty} (1-q^k z).
\end{equation}
Now the $q$-Exponential $E_q : \CC \rarr \CC$ is defined by $E_q(z)=(-z;q)_\infty$.
\end{defn_chp}


\begin{remark_chp} \rm
According to Weierstra\ss\ theory of entire functions and canonical products,
(\ref{eq:shifted_factorial:infty}) defines an entire function in $z$,
having simple zeros at the points $q^{-k}$ for $k\in \NN$, and no zeros
elsewhere. Notice that it is essential that $0<q<1$ to ensure the convergence
of (\ref{eq:shifted_factorial:infty}).
It should be noted that there exist other $q$-analogues of the
exponential function as well, but the above one is the most suitable for our purposes.
\hfill $\star$
\end{remark_chp}


\begin{prop_chp} \label{prop:def:qExp}
The\/ $q$-Exponential\/ $E_q$ is entire and has the following power series at the origin:
\begin{equation}\label{eq:qExp:powerseries}
  E_q(z)\:=\; \sum_{n=0}^\infty \:\frac{q^{\frac{1}{2}n(n-1)}}{(q;q)_n} \, z^n
\end{equation}
for any\/ $z\in \CC$. Furthermore\/ $E_q(-q^{-k})=(q^{-k};q)_\infty=0$ for all\/ $k\in \NN$.
\end{prop_chp}

\begin{proof}
See above remark; also (\ref{eq:qExp:powerseries}) is a standard result in $q$-calculus.
\end{proof}





\paragraph{The $q$-derivative}
Whenever $f \in H(\CC)$ we consider the function
\begin{equation}\label{eq:def:qderivative}
  \CC_0 \rarr \CC : z \mapsto \frac{f(z)-f(qz)}{(1-q)z}
\end{equation}
which is holomorphic on $\CC_0$ and obviously has a removable
singularity at the origin. Hence (\ref{eq:def:qderivative})
actually defines an {\em entire\/} function again, which we denote
by $D_q f$. This yields a linear operator $D_q: \HC \rarr \HC$.
