
\begin{abs_chp}
Given any regular Hopf system we shall construct a quantum double, provided
an extra condition is fulfilled; this extra assumption is called
balancedness and will be investigated in the first paragraph.
Then we show that our quantum double is again a regular Hopf system.
In particular the construction applies to \mhs s,
and in that case the quantum double turns out to be again a \mhs\@.
Thus we obtain an alternative for the approach in \cite{FonsDra:QD}\@.
\end{abs_chp}



\subsection{Balanced Hopf systems}

Let \pairAB\ be any Hopf system, and recall the mappings $\GamL$ and $\GamR$
introduced in definition \ref{def:GammaLGammaR}\@.
Now it is not hard to see that $\GamL$ and $\GamR$ actually take values in the
weak\footnote{w.r.t.\
the duality \pairAB, of course. See appendix \ref{app:fubini} for details.}\
Fubini tensorproduct $B \fubtens A$. On the other
hand\footnote{cf.\ remark \ref{rem:weak_fubtens}.ii in appendix \ref{app:fubini}\@.}\
$B \fubtens A$ identifies naturally with the spaces $L(A)$ and $L(B)$
of weakly continuous linear operators on $A$ and $B$ respectively.
Explicitly, for any $a,p\in A$ and $b,q\in B$ we have
$$ \pairM{\GamL(a \tens b)}{p \tens q}  \:=\:  \pair{ap}{bq}  \:=\:  \pair{p}{bq \ract a}, $$
hence $\GamL(a \tens b)$ identifies with the operator $(b\,\cdot) \ract a$ on $B$.
Analogously:
\begin{center}
\begin{tabular}{cc|cccc}
&&& $L(A)$   &\hspace{1em}&   $L(B)$
\\ \hline
$\GamL(a \tens b)$  &&&  $(a\,\cdot) \ract b$  && $(b\,\cdot) \ract a$
\vertL\\
$\GamR(a \tens b)$  &&&  $b \lact (\cdot\,a)$ && $a \lact (\cdot\,b)$
\vertL
\end{tabular}
\end{center}
So the range of $\GamL$ identifies with the subspace of $L(B)$ spanned by the operators
$(b\,\cdot) \ract a$ with $a\in A$ and $b\in B$.
We shall denote this subspace of $L(B)$ by $\CL(B)$.
Similarly we define $\CR(B)$, $\CL(A)$ and
$\CR(A)$.\footnote{These spaces are comparable to the algebra $\mathcal{C}(V)$
associated to a multiplicative \mbox{unitary $V$}\ in the sense of \cite{BS}.}


Combining lemma \ref{lem:twisting}\ and proposition \ref{prop:twistings:lprp},
we obtain that $\TL=\rp^{-1}\flip$ is an $(A\op,B)$-twisting, whereas
$\TR=\lp^{-1}\flip$ is an $(A,B\op)$-twisting.
Hence according to proposition \ref{prop:twisted_tensorproduct}\
we have twisted tensorproducts $A\op\tens_{\TL}\! B$ and $A\tens_{\TR}\! B\op$.
Furthermore $L(A)$ and $L(B)$ are algebras under composition of operators,
so the following makes sense:


\begin{prop}
Let\/ \pairAB\ be any Hopf system. Then the mappings
$$ \GamL :  A\op\tens_{\TL}\! B  \rarr  L(B)\simeq L(A)\op
   \itandspace{1em}
   \GamR :  A\tens_{\TR}\! B\op  \rarr  L(B)\simeq L(A)\op $$
are algebra homomorphisms.
It follows that\/ $\CL(B)$ and\/ $\CR(B)$ are subalgebras of\/ $L(B)$.
Similarly\/ $\CL(A)$ and\/ $\CR(A)$ are subalgebras of\/ $L(A)$.
\end{prop}
\begin{proof}
Take any $a,c,p\in A$ and $b,d,q\in B$ and observe that
\begin{eqnarray*}
\pairM{p}{\left( \GamL(a \tens b) \circ \GamL(c \tens d) \vertM\right)(q)}
&=&
\pairM{p}{\left( \GamL(a \tens b) \vertM\right)(dq \ract c)}
\\&=&
\pairM{p}{b(dq \ract c) \ract a}
\\&=&
\pairM{ap \ract b}{dq \ract c}
\\&=&
\pairM{b\tens c}{\lp(ap \tens dq)}
\\&=&
\pairflipM{\rp\! \left(\rp^{-1}(c \tens b) \vertM\right)}{\lp(ap \tens dq)}
\\&\stackrel{(\ref{eq:muliplicative_actor})}{=}&
\pairM{P}{\rp^{-1}(c \tens b) \, (ap \tens dq)}
\\&=&
\pairM{P}{\left(\TL(b \tens c) \, (a \tens d) \vertM\right) (p \tens q)}
\\&\stackrel{(\ref{eq:def:Gammas})}{=}&
\pairM{\GamL\! \left((a \tens b) \rtimes (c \tens d) \vertM\right)}{p \tens q}
\end{eqnarray*}
where $\rtimes$ denotes the product in $A\op\tens_{\TL}\! B$.
The other cases are similar.
\end{proof}
\vspace{2ex}


Let us consider a second approach to these $\mathcal{C}$-algebras:
from (\ref{eq:Gammas:construction}) it is clear that the range of $\GamL$ is
also equal to $\lp\algtp(B\tens A)=(B\tens A)P$.
Now observe that
$$ \pairM{(b\tens a)P}{p\tens q}
   \:=\: \pairM{b\tens a}{\lp(p\tens q)}
   \:=\: \pairM{p \ract b}{q \ract a}
   \:=\: \pairM{p}{b(q \ract a)}   $$
for all $a,p\in A$ and $b,q\in B$.
It follows that $\CL(B)$ is also spanned by the operators
$b(\,\cdot\, \ract a)$ with $a\in A$ and $b\in B$.
Defining operators $\eta(a)$ and $\pi(b)$ \mbox{on $B$}\ by
$\eta(a)q=q \ract a$ and $\pi(b)q = bq$, our results can be summarized as follows:
$$ \Ran(\GamL)  \:\simeq\:  \CL(B)  \:=\: \eta(A)\,\pi(B)  \:=\: \pi(B)\,\eta(A)
          \:\subseteq\:  L(B).$$
Similar results hold for the other $\mathcal{C}$-algebras.
Another interesting observation is that when
$F(B)$ denotes the ideal of all {\em finite rank\/} operators in $L(B)$,
then lemma \ref{mhs:Gammas:bijective}\ can easily be reformulated as follows:
\begin{lemma}
A Hopf system\/ \pairAB\ is a \mhs\ if and only if\/
$\CL(B) = F(B) = \CR(B)$.
\end{lemma}



Whenever \pairAB\ is a Hopf system, we have an algebra homomorphism
$$ \piP : B \tens A \rarr \Pre(\BBAA): y \mapsto P y P^{-1} $$
(cf.\ remark \ref{rem:multipl:preview:QD}\ and
lemma \ref{lem:wu:commutation:invertibility}).
In order to construct a quantum double, however, we shall need $\piP$
to be actually an {\em auto\/}morphism of $B\tens A$:


\begin{lemma} \label{lem:QD:conditions}
Let\/ \pairAB\ be any Hopf system. The following are equivalent:
\begin{enumerate}
\item $\piP$ is an automorphism of\/ $B\tens A$
\item $(B\tens A)P = P(B\tens A)$
\item $\Ran(\GamL) = \Ran(\GamR)$
\item $\CL(B) = \CR(B)$.
\end{enumerate}
If\/ \pairAB\ is regular, then the above assertions are moreover equivalent with:
\begin{enumerate}
\item[v.]  The range of\/ $\GamL$ is invariant under\/ $\SB \fubtens \SA^{-1}$
           and\/ $\SB^{-1} \fubtens \SA$.
\item[vi.] $\SB \CL(B) \SB^{-1} \,=\, \CL(B)$.
\end{enumerate}
\rm Observe that (v) and (vi) make sense because $\SA$ and $\SB$ are invertible
elements in $L(A)$ and $L(B)$ respectively.
\end{lemma}
\begin{proof}
(i $\Leftrightarrow$ ii) is obvious.
(ii $\Leftrightarrow$ iii) follows from (\ref{eq:Gammas:construction}).
(iii $\Leftrightarrow$ iv) holds by definition.
Now assume \pairAB\ to be regular. We claim that
$$(\SB \fubtens \SA^{-1})\, \GamL (\SA \tens \SB^{-1}) \:=\: \GamR.$$
Indeed, for all $a,p\in A$ and $b,q\in B$ we have
\begin{eqnarray*}
\lefteqn{\pairM{(\SB \fubtens \SA^{-1})\,\GamL (\SA \tens \SB^{-1})(a \tens b)}{p \tens q}}
\\&=&
\pairM{\GamL \! \left( \SA(a) \tens \SB^{-1}(b)  \vertM\right)}{\SA(p) \tens  \SB^{-1}(q)}
\\&=&
\pairM{\SA(a) \, \SA(p)}{\SB^{-1}(b) \, \SB^{-1}(q)}
\\&=&
\pairM{pa}{qb}
\\&=&
\pairM{\GamR(a \tens b)}{p \tens q}
\end{eqnarray*}
It follows that the range of $\GamR$ equals the range of $(\SB \fubtens \SA^{-1}) \GamL$.
This proves (\mbox{iii $\Leftrightarrow$ v}).
Remark \ref{rem:fubtens_of_maps}.v in appendix \ref{app:fubini}\ yields (v $\Leftrightarrow$ vi).
\end{proof}



\begin{defn} \label{def:balanced_Hopf_system}
A Hopf system\/ \pairAB\ is said to be {\em balanced\/} whenever
it enjoys the equivalent conditions (i-iv) in lemma \ref{lem:QD:conditions}\@.
\end{defn}

Observe that e.g.\ every \mhs\ is balanced.
Also notice that if \pairAB\ is a balanced Hopf system,
then $P\in \Act(\BBAA)$ enjoys condition (iv) in
lemma \ref{lem:wu:commutation:invertibility}\@.
In particular it follows that $\lp$ and $\rp$ commute.

\begin{remark} \rm
The notion of balancedness is similar to S.\ Montgomery's \mbox{{\sc rl}-condition}
emerging in the duality theory for smash products \cite[\S 9.4]{Montgomery}\@.
When \pairAB\ is a dual pair of Hopf algebras, then $A$ is said to satisfy the
{\sc rl}-condition with respect to $B$ if the right action $(\cdot) \ract a$
of any $a\in A$ on $B$ can be expressed as a linear combination of operators
$b(c \lact \cdot\,)$ with $c\in A$ and $b\in B$.
So the {\sc rl}-condition roughly means that left actions can be expressed
in terms of right actions.
Our condition of balancedness seems to be more symmetric, but unfortunately also
more restrictive.
On the other hand, balancedness will appear to be a very obvious requirement in our
quantum double construction below.
\hfill $\star$
\end{remark}
