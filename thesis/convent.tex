
\section{Conventions}
\label{sec:conventions}

The reader should notice that assumptions are often fixed and kept in force
throughout a whole chapter, section or paragraph.

\paragraph{Linear spaces, duality}
Linear always means $\kk\,$-linear. We use $\tens$ to denote the algebraic
tensor product of linear spaces over \kk\@. Whenever $V$ and $W$ are linear
spaces, the flip map from $V \tens W$ onto $W \tens V$ shall be denoted by
$\flip$, hence $\flip(v\tens w) = w\tens v$. We always identify $V\tens\kk$ and
$\kk \tens V$ with $V$. The space of all linear functionals on $V$ is denoted
by $V'$, the canonical pairing by \pairing\@. In fact most pairings will be
denoted this way. We sometimes speak about a \lq vector space duality\rq\
instead of a \lq pairing\rq\@. We will almost exclusively deal with {\em
non-degenerate\/} pairings. Furthermore we shall not be rigorous concerning the
{\em order\/} in which to write a pairing \pairing\ between two spaces $V$ and
$W$, in the sense that the pairing of an element $v\in V$ with an element $w\in
W$ may be denoted by $\pair{v}{w}$ or $\pair{w}{v}$ interchangeably. A triplet
$(V,W,\pairing)$ will often be written in shorthand form as $\pair{V}{W}$. We
use the \mbox{superscript $\algtp$}\ to denote the (algebraic) transpose of a
linear map.


\paragraph{Algebras}
By an algebra we mean an associative algebra over \kk, not necessarily having
an identity. Multiplication in an algebra $E$ will often be considered as a
linear map $m\equiv \mult{E} : E \tens E \rarr E : x \tens y \mapsto xy$. The
product in $E$ is said to be {\em non-degenerate\/} whenever $$ xE=\{0\}
\mbox{ implies } x=0     \andspace{2em}
   Ex=\{0\}  \mbox{ implies } x=0
   \hspace{2em} (x\in E).  $$
The {\em opposite\/} product on $E$ is defined as $m\op = m\flip$. The
resulting algebra will be denoted by $E\op$.

\paragraph{Modules}
Let $E$ be any algebra. A linear space $\Om$ is a left $E$-module when it is
endowed with a linear map $\mu: E \tens \Om \rarr \Om$ satisfying $\mu (m \tens
\id) = \mu (\id \tens \mu)$. We denote $\mu(x \tens \om)$ by $x \lact \om$ or
sometimes $x\om$. Right $E$-modules are defined similarly. If $\Om$ is both a
left and right $E$-module such that $x\lact (\om \ract y) = (x\lact \om) \ract
y$ for all $x,y\in E$ and $\om \in \Om$, then $\Om$ is said to be an \Ebimod\@.

\paragraph{Unital modules, reduction}
A left $E$-module $\Om$ is said to be {\em unital\/} if its structure map $\mu$
is surjective. This replaces the usual condition $1\lact\om=\om$ for left
modules over algebras which do have a unit. We shall write $E \lact \Om$ to
denote the range of $\mu$. An \Ebimod\ $\Om$ is called unital if $E \lact \Om =
\Om = \Om \ract E$.

Given any \Ebimod\ $\Om$, one can pass to the subspace $\Om_0 = E \lact \Om
\ract E$, which is obviously a sub-\Ebimod\ of $\Om$. We say that $\Om_0$ is
the {\em reduction\/} of $\Om$ as an \Ebimod\@. Notice that if $\mult{E}$ is
surjective, i.e.\ $E^2=E$, then $\Om_0$ is unital. On the other hand, $\Om$
itself is unital if and only if $\Om=\Om_0$.

\paragraph{Locally convex spaces}
We are mainly dealing with weak and inductive topologies
\cite{Jarchow,Kothe,Schaefer,Treves}\@. We use the symbol $\sigma$ when
referring to some weak topology. The prefix \lq locally convex\rq\ will be
abbreviated by {\sc lc}\@.


\paragraph{Entire functions}
Let \HC\ denote the $^*$-algebra of entire functions, with pointwise
multiplication and $^*$-operation $\tilde{}$ defined by $\tilde{g}(z) =
\overline{g(\overline{z})}$. We shall use $\mu_n(g)$ to denote the $n$-th
coefficient of the Taylor series expansion of an entire function $g$ at the
origin. So whenever $g \in \HC$ we have \mbox{$g(z) = \sum_{n=0}^\infty
\mu_n(g)\,z^n$} for all $z\in \CC$. This yields, for every $n\in \NN$, a linear
functional $\mu_n$ on \HC\@.
