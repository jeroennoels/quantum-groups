
\section{Actor implementations of an algebra}

\begin{abs_chp}
We introduce the concept of an {\em \context\/} together with some basic examples,
and define the notion of an {\em actor}, which is intended to be an abstract
generalization of the actions of an algebra on its dual.
Then we give an alternative description of \contexts\ in terms of weak comultiplications.
Finally we associate to any \context\ an {\em enveloping algebra\/} of actors.
\end{abs_chp}



\subsection{Basic definitions}
               \label{sect:def_act_context}

\paragraph{Canonical actions}
Every algebra $E$ acts on its dual $E'$ as follows:
\begin{equation}\label{eq:def:canonical_actions}
  \pair{x}{y\lact \om} \:=\: \pair{xy}{\om} \:=\: \pair{y}{\om\ract x}
\end{equation}
for $x,y \in E$ and $\om \in E'$.
With these actions, $E'$ is clearly an \Ebimod\@.

\begin{defn} \label{def:actor_context}
  Consider an algebra $E$ and an \Ebimod\ $\Om$ together with
  a non-degenerate pairing $\pairing : E \times \Om \rarr \kk$.
  Endow $E'$ with its canonical \Ebimod\ structure.
  The pair $(\Om, \pairing)$ is said to be an {\em actor implementation\/} of $E$
  if the canonical embedding $\Om \rarr E'$ is an \Ebimod\ morphism.
  The triplet \EOP\ will be called an {\em\context}.
\end{defn}


\begin{remarks}  \label{rem:topalg}
  \item
     We shall always identify $\Om$ with its image in $E'$, so $\Om$ is in fact
     a sub-\Ebimod\ of $E'$\@. In other words, the above definition
     requires $\Om$ to be invariant under the actions defined by
     (\ref{eq:def:canonical_actions}).
  \item
     We are about to introduce several spaces which \lq extend\rq\ $E$ in
     some sense. Nevertheless the pairing will {\em not\/} be extended
     accordingly until proposition \ref{prop:extended_pairing}\@.
     So we should be careful in dealing with the pairing: for the moment we can only pair
     $\Om$ with $E$ and---of course---with $\Om'$.
  \item
     Both $E$ and $\Om$ are locally convex spaces, if we endow them
     with the weak topologies induced by the duality.
     We will sometimes write $E_\sigma$ or $\Om_\sigma$ to emphasize the
     presence of these weak topologies.
     The above definition could be reformulated in a topological language,
     requiring $E_\sigma$ to be a
     topological\footnote{multiplication in $E$ is separately continuous
     w.r.t.\ the $\sigma(E,\Om)$ topology.} algebra. It follows that $\Om_\sigma$ is a
     topological\footnote{the module structure maps are separately continuous.}
     $E_\sigma$-bimodule.
\end{remarks}


\begin{defn} \label{def:actor}
  Let $\EE \equiv \EOP$ be an \context, and consider linear maps $\lam$ and $\rho$
  from $\Om$ into $\Om$. We say that the pair $(\lam,\rho)$ is a {\em pre-actor\/}
  for \EE\ when $\lam$ is a right $E$-module morphism and
  $\rho$ is a left $E$-module morphism. If moreover
  \begin{equation} \label{eq:bi-actor}
     \pair{x}{\rho(\om \ract y)}  \:=\:  \pair{y}{\lam(x\lact\om)}
  \end{equation}
  for all $x,y \in E$ and $\om \in \Om$, then $(\lam,\rho)$ is said to be
  an {\em actor\/} for \EE\@. The set of all actors for \EE\ will be denoted by \ActE,
  the set of pre-actors by \PreE.
\end{defn}


\begin{remarks} \label{rem:def_actor}
 \item
   In the above definition, $\lam$ can be thought of as a \lq left actor\rq\ and
   $\rho$ as a \lq right actor\rq, whereas (\ref{eq:bi-actor}) expresses their interrelation.
   Equation (\ref{eq:bi-actor}) will be referred to as the {\em bi-actor\/} property.
 \item
   Obviously \PreE\ is a linear space under pointwise operations,
   containing \ActE\ as a subspace.
   Also notice that $(\id_\Om,\id_\Om) \equiv 1_\EE$ belongs to \ActE.
 \item
   If $\Om$ is moreover unital as an \Ebimod, i.e.\ $E \lact \Om = \Om = \Om \ract E$,
   then for $(\lam,\rho)$ to be an actor it is sufficient to require just the \biap;
   the module properties of $\lam$ and $\rho$ will hold automatically.
   Applying the \biap\ twice, we get for instance
   \begin{eqnarray*}
     \pair{x}{\rho(z\lact (\om \ract y))}
          &=&    \pair{x}{\rho((z\lact \om) \ract y)}
          \hspace{0.6em}=\hspace{0.6em}
                 \pair{y}{\lam(x\lact (z\lact \om))}       \\
          &=&    \pair{y}{\lam(xz\lact \om)}
          \hspace{0.6em}=\hspace{0.6em}
                 \pair{xz}{\rho(\om \ract y)}              \\
          &=&    \pair{x}{z \lact \rho(\om \ract y)}
   \end{eqnarray*}
   for all $x,y,z \in E$ and $\om\in \Om$. Since we assume $\Om \ract E = \Om$,
   it follows that $\rho$ is a left $E$-module morphism; similarly we can treat $\lam$.
\end{remarks}



\subsection{Introducing some examples}

We shall follow some examples throughout the development of the theory;
for the moment we focus on the actor {\em contexts\/} rather than the actors themselves.
Our first example has the benefit of its simplicity, but the disadvantage
that some features of the general framework will trivialize here:

\begin{exA} \label{exA:introduction} \rm
  {\bf (function algebras)} \hspace{0.6em}
  Let $S$ be a non-empty set and denote by $\kk^S$ the algebra of all functions
  $f:S \rarr \kk$ with pointwise operations.
  Now let $F$ be any separating algebra of complex functions on $S$,
  i.e.\ a non-zero subalgebra of $\kk^S$ which separates the points in $S$,
  in the sense that for every $s,s_0 \in S$ with $s\neq s_0$ there exists
  a function $f\in F$ such that $f(s_0)=0$ but $f(s)\neq 0$.
  Since $F$ is an algebra, it follows that for any finite non-empty subset $S_0$
  of $S$ and any $s\in S\setminus S_0$ there is an $f\in F$ such that
  $f(S_0)=\{0\}$ but $f(s) \neq 0$\@.
  Next we consider the free linear space $\kk S$ generated
  by the set $S$. Its canonical basis will be denoted
  by $\{\delta_s\}_{s\in S}$\@. Then there is a pairing between
  $\kk^S$ and $\kk S$, given by $\pair{f}{\delta_s} = f(s)$,
  and moreover we have $(\kk S)' \simeq \kk^S$\@. By restriction
  we get a non-degenerate pairing between $F$ and $\kk S$.
  Now $\kk S$ is a bimodule over $\kk^S$, hence also over $F$,
  the actions being determined by
  $ f \lact \delta_s = f(s) \,\delta_s = \delta_s \ract f. $
  With these ingredients, the triple $(F; \kk S, \pairing)$
  is obviously an \context\@. Notice that $\kk S$ is unital
  as an $F$-bimodule.

  Let's look at a more concrete example of this nature: take
  a locally compact Hausdorff space $X$ and consider the algebra $K(X)$
  of all continuous functions $f:X\rarr\kk$ with compact support.
  By Urysohn's lemma, $K(X)$ separates the points in $X$, hence
  $(K(X); \kk X, \pairing)$ is an \context. \hfill $\star$
\end{exA}


\begin{exB} \label{exB:introduction} \rm
  {\bf (C$^*$ and W$^*$ algebras)} \hspace{0.6em}
  Let $A$ be a \Cstar-algebra \cite{Takesaki}\ and denote its
  norm-dual by $A^*$\@. Then $(A; A^*, \pairing)$ is an \context\@.
  When $M$ is a $W^*$-algebra with predual $M_*$, then
  $(M; M_*, \pairing)$ is an \context, since multiplication in a
  $W^*$-algebra $M$ is separately $\sigma(M,M_*)$ continuous.
  In particular also $(A^{**}; A^*, \pairing)$ is an \context.
  \hfill $\star$
\end{exB}


Whenever $A$ is a normed algebra, $A$ is also a normed $A$-bimodule having
a dual Banach $A$-bimodule $A^*$
\cite[\S I.9.12 and \S I.9.13.i-iv]{Bonsall_Duncan}\@. Explicitly:

\begin{exB} \label{exB:normed_algebra_context}  \rm
  {\bf (normed algebras)}  \hspace{0.6em}
  Whenever $A$ is a normed algebra, $(A; A^*, \pairing)$ is an \context;
  moreover we have for $x\in A$ and $\om \in A^*$ that
  \begin{equation} \label{eq:exB:normed_algebra_context}
     \|x \lact \om\| \:\leq\:  \|x\|  \,  \|\om\|    \andspace{10mm}
     \|\om \ract x\| \:\leq\:  \|x\|  \,  \|\om\|.
  \end{equation}
\end{exB}
\begin{proof}
  Observe that
  $ |\pair{y}{x \lact \om}| \,=\, |\pair{yx}{\om}| \,\leq\,  \|y\|  \,  \|x\|  \, \|\om\|$
  for all $y\in A$.
\end{proof}
\vspace{2ex}

{\small
Of course example A is rather trivial, whereas examples B may seem a little artificial.
On the other hand, they do provide a familiar setting to illustrate the notions we shall
introduce later on, and this indeed will be their main purpose.
In \S \ref{chapter:Hopf_systems}\ though, the theory of \contexts\ as
developed here shall be used extensively to study our next example,
which is intended to offer a framework for generalizing Hopf algebra duality:}


\begin{exC}  \label{exC:introduction} \rm
  {\bf (\dpa)}  \hspace{0.4em}
  Let \pairing\ be a non-degen\-erate vector space duality between two algebras
  $A$ and $B$\@. Now if both
  $$\Aa \equiv (A; B, \pairing)   \andspace{10mm}
    \BB \equiv (B; A, \pairing)$$
  are \contexts, then $\pair{A}{B}$ is said to be a {\em \dpa}.
  \hfill $\star$
\end{exC}


  The above merely states that $A$ is invariant under the canonical actions of $B$,
  and vice versa. Equivalently one could require $A$ and $B$ to be topological algebras
  in the weak topologies induced by duality (cf.\ remark \ref{rem:topalg}.iii).
  In any case, these are very natural conditions to impose on such a dual pair.


\begin{exD} \label{exD:introduction} \rm
  Let $E$ be any algebra with non-degenerate product and let $E\reduced = E\lact E' \ract E$
  be the reduction of $E'$ as an \Ebimod\ (\S\ref{sec:conventions}).
  We say $E\reduced$ is the {\em reduced dual\/} of $E$.
  The pairing $\pair{E}{E\reduced}$ is non-degenerate since $E$ is
  assumed to have non-degenerate product.
  Now $(E;E\reduced,\pairing)$ is an \context.
\end{exD}




\subsection{Weak comultiplications}
\label{par:comultiplication}

For a while we will take a \lq dual\rq\ approach towards \contexts,
introducing the notion of a weak comultiplication. Although less convenient
from a technical point of view, we believe this to be helpful in
understanding the link with Hopf algebra theory
\cite{Abe,Sweedler,Fons:DPHA,Fons:MHA}\@.
In this paragraph we will use the weak Fubini tensor product,
as defined in appendix \ref{app:fubini}\@. Usually it will be clear from the
context to which particular vector space dualities a $\fubtens$-tensor product
is related. Recall $\fubtens$ is associative
(cf.\ remark \ref{rem:weak_fubtens}.i), so the following makes sense:

\begin{defn}
  Let $\pair{E}{\Om}$ be a non-degenerate vector space duality. Then by a
  {\em weak comultiplication\/} on $\Om$ we mean a weakly continuous linear mapping
  \mbox{$\Delta: \Om \rarr \Om \fubtens \Om$} which is {\em coassociative\/} in the sense that
  $ (\Delta \fubtens \id)\Delta  = (\id \fubtens \Delta)\Delta $
  (cf.\ remarks \ref{rem:weak_fubtens}.i and \ref{rem:fubtens_of_maps}.iv).
  Observe that this definition involves the duality between $E \tens E$
  and $\Om \fubtens \Om$.
\end{defn}


\begin{prop} \label{prop:induced_actor_context}
  Let\/ $\pair{E}{\Om}$ be a vector space duality.
  If\/ $\Delta: \Om \rarr \Om \fubtens \Om$ is a weak comultiplication
  on\/ $\Om$, then $E$ can be endowed with an algebra structure satisfying
  \begin{equation} \label{eq:duality_m_Delta}
    \pair{xy}{\om}=\pairM{x \tens y}{\Delta(\om)}
  \end{equation}
  for all $x,y \in E$ and $\om \in \Om$\@.
  Moreover $\Om$ becomes an \Ebimod, with actions
  \begin{equation} \label{eq:actions_in_terms_of_Delta}
     x \lact \om = (\id \fubtens f_x)\Delta(\om)  \andspace{10mm}
     \om \ract x = (f_x \fubtens \id)\Delta(\om),
  \end{equation}
  making\/ $\EE_\Delta \equiv \EOP$ into an \context\@.
  \rm Here $f_x$ stands for $\pairdot{x}$.
\end{prop}
\begin{proof}
  Since $\Delta$ is by assumption weakly continuous, it has a weak transpose
  $\Delta^* \equiv m : E \tens E \rarr E$, defining the product on $E$.
  Associativity of $m$ follows easily from the coassociativity of $\Delta$,
  hence $(E,m)$ is an algebra. The remaining assertions are immediate from equation
  (\ref{eq:fub_slice_functional}) in appendix \ref{app:fubini}.
\end{proof}


\begin{exC}  \rm
  A non-degenerate dual pair \pairAB\ of Hopf algebras in the sense of \cite{Fons:DPHA}\
  is also a \dpa\ in the sense of example \ref{exC:introduction}.
  \hfill $\star$
\end{exC}


\begin{prop} \label{prop:induced_comult}
  Let\/ $\EE \equiv \EOP$ be any \context\@.
  Then (\ref{eq:duality_m_Delta}) defines a weak comultiplication
  $\Delta$ on\/ $\Om$. Clearly $\EE_{\Delta} = \EE$,
  so the relations (\ref{eq:actions_in_terms_of_Delta}) between
  actions and comultiplication will hold.
\end{prop}
\begin{proof}
  Since multiplication in $E_\sigma$ is separately continuous,
  (\ref{eq:duality_m_Delta}) defines a map
  $\Delta : \Om \rarr \Om \fubtens \Om$ which
  is easily seen to be a weak comultiplication on $\Om$.
\end{proof}
\vspace{2ex}


{\small
Comultiplications are used extensively in the theory of Hopf algebras;
in that theory however, a comultiplication on a linear space $\Om$ is a map
from $\Om$ into $\Om \tens \Om$, whereas our {\em weak\/} comultiplications
are of a more topological nature, in the sense that they are allowed
to \lq go outside\rq\ the {\em algebraic\/} tensor product.}


\begin{prop} \label{prop:algebraic_and_cont_prod}
  Let\/ $\EE \equiv \EOP$ be an \context\ and let $\Delta$ be the
  induced weak comultiplication on $\Om$. Then the following are equivalent:
  \begin{enumerate}
    \item $\Delta(\Om) \subseteq \Om\tens\Om$,
    \item multiplication in $E_\sigma$ is jointly continuous.
  \end{enumerate}
\end{prop}
\begin{proof}
  $\Om\tens\Om$ identifies with the space of {\em jointly\/} continuous bilinear
  forms on $E\times E$.
  (cf.\ the observations preceding definition \ref{def:weak_fubini_tensor}\
  in appendix \ref{app:fubini}).
\end{proof}


\begin{defn} \label{def:algebraic_context}
  If the conditions of the above proposition are fulfilled,
  then\/ \EE\ is said to be an {\em algebraic\/} \context.
\end{defn}


\begin{exA} \label{exA:algebraic} \rm
  The \context\ $(F; \kk S, \pairing)$ of example A \ref{exA:introduction}\
  is algebraic, since $\Delta$ on $\kk S$ is given by
  $\Delta(\delta_s) = \delta_s \tens \delta_s$ for all $s \in S$.
  \hfill $\star$
\end{exA}


\begin{prop}
  Let\/ $\EE \equiv \EOP$ be an \context, $\Delta: \Om \rarr \Om \fubtens \Om$
  the induced weak comultiplication and $(\lam,\rho) \in \PreE$.
  Then are equivalent:
  \begin{enumerate}
    \item $(\lam,\rho) \in \ActE$, i.e.\ the pair $(\lam,\rho)$ satisfies
          the \biap\ (\ref{eq:bi-actor})
    \item $(\overline{\lam\, \tens}\, \id) \Delta
          =(\id \, \overline{\tens\, \rho}) \Delta$
          (cf.\ remark \ref{rem:fubtens_of_maps}.iii).
  \end{enumerate}
  Moreover, if these conditions are fulfilled, then
  $$  (\overline{\lam \, \tens}\, \id)   \Delta(\Om)
       \,=\, (\id \, \overline{\tens\, \rho}) \Delta(\Om)
       \;\subseteq\: \Om\fubtens\Om. $$
\end{prop}
\begin{proof}
  According to the definition of a slice map like
  $ \overline{\lam\,\tens}\,\id : \Om\fubtens\Om \rarr \overline{\Om\tens\Om},$
  we get for all $x,y \in E$ and $\om \in \Om$ that
  $$ \pair{y \tens x}{(\overline{\lam\,\tens}\,\id)\Delta(\om)}
     \:\stackrel{(\ref{eq:def_Lam1_fubtens_id})}{=}\:
              \pair{y}{\lam(\id \fubtens f_x)\Delta(\om)}
     \:\stackrel{(\ref{eq:actions_in_terms_of_Delta})}{=}\:
              \pair{y}{\lam(x\lact\om)}.  $$
  Similarly we get
  $ \pair{y \tens x}{(\id\, \overline{\tens\, \rho})\Delta(\om)}
       = \pair{x}{\rho(\om \ract y)}. $
  The result follows.
\end{proof}
\vspace{2ex}

{\small
Henceforth the emphasis will be on {\em module\/} structures rather than comultiplications;
as explicit objects the latter are quite redundant, and we will avoid them in
developing our theory of \contexts\@. Nevertheless it may be instructive to interpret
things in terms of the comultiplication from time to time,
especially when one wants to appreciate the relationship with
Hopf algebra theory \cite{Abe,Sweedler}\ or its generalizations \cite{Fons:MHA}.}



\subsection{Enveloping algebras}
\label{par:enveloping_algebras}

In the following, $\EE \equiv \EOP$ will be an \context\@.

\begin{lemma} \label{lem:embedding_of_E}
  Given $x\in E$ we define maps $\lam_x,\rho_x : \Om \rarr \Om$ by
  $\lam_x(\om) = x \lact \om$ and $\rho_x(\om) = \om \ract x$.
  Then $\lamrho{x} \in \ActE$.
\end{lemma}
\begin{proof}
   The pair $\lamrho{x}$ is clearly a pre-actor for \EE, because $\Om$ is an \Ebimod\@.
   The \biap\ (\ref{eq:bi-actor}) follows from associativity in $E$.
\end{proof}

\begin{defn} \label{def:product_of_actors}
  Consider two elements $x_1 \equiv \lamrho{1}$ and $x_2 \equiv \lamrho{2}$ in \PreE\@.
  Clearly $(\lam_1\lam_2,\rho_2\rho_1)$ is again a pre-actor for \EE,
  which will be denoted by $x_1 x_2$.
  This defines a product that makes \PreE\ into a unital algebra.
\end{defn}

Unfortunately the \biap\ (\ref{eq:bi-actor}) is not stable under this operation.
In other words, \ActE\ is in general {\em not\/} a subalgebra of \PreE\@.

{\small
\begin{exB} \label{exB:Act_not_an_algebra} \rm
Let $G$ be any infinite discrete abelian group, and consider its group algebra $A=L^1(G)$.
Since $L^1(G)$ is a Banach algebra, we have an \context\
$\Aa \equiv (L^1(G); L^1(G)^*, \pairing)$ as in example \ref{exB:normed_algebra_context}\@.
We claim that in this case $\Act(\Aa)$ is not closed under multiplication.
To show this assertion, however, we first need to develop our theory;
therefore the actual proof will be deferred until \S \ref{Arens regularity}\@.
For the moment we only mention that this is related to the fact that
$L^1(G)$ is not {\em Arens regular\/} when $G$ is infinite, discrete and abelian
\cite{civin_yood}\@.
\hfill $\star$
\end{exB}
}


Lemma \ref{lem:embedding_of_E}\ allows us to define a mapping
$j:E \rarr \PreE: x \mapsto \lamrho{x}$ which is obviously an algebra morphism;
moreover $j(E)\subseteq \ActE$. Let's investigate whether $j$ is injective:

\begin{lemma} \label{lem:nondeg_context}
  Let\/ \EE\ and $j$ be as above. The following are equivalent:
  \begin{enumerate}
    \item The spaces $E \lact \Om$ and $\Om \ract E$ are weakly dense in $\Om$.
    \item The product in $E$ is non-degenerate.
    \item $(E;\Om_0,\pairing)$ is again an \context, $\Om_0$ being the reduction of $\Om$
       \mbox{\rm  (\S \ref{sec:conventions})}.
       In other words, the pairing $\pair{E}{\Om_0}$ is still non-degenerate.
 \end{enumerate}
 If these assertions hold, then \EE\ is said to be {\em non-degenerate}, and $j$ is injective.
\end{lemma}

The proof is easy. Notice that if $\Om$ is unital as an \Ebimod, then a fortiori
\EE\ is non-degenerate.
When \EE\ is non-degenerate, we can (and will) identify $E$ with its image in \PreE,
so that $E$ is in fact a subalgebra of \PreE\@.

Now \PreE\ is an algebra, hence it is a \PreE-bimodule under multiplication;
since $j:E \rarr \PreE$ is an algebra morphism, \PreE\ is as well an \Ebimod,
the actions of $z\in E$ on a pre-actor $a\equiv (\lam,\rho) \in \PreE$ being given by
$$ za = j(z)\,a = (\lam_z\lam,\rho\rho_z) \andspace{10mm}
   az = a\,j(z) = (\lam\lam_z,\rho_z\rho). $$

\begin{lemma}  \label{lem:Act_is_submodule}
  \ActE\ is a sub-$E$-bimodule of\/ \PreE.
\end{lemma}
\begin{proof}
  Take $z\in E$ and $a\equiv (\lam,\rho) \in \ActE$.
  We have to show that $za$ and $az$, as defined above,
  are again in \ActE\@. Now the pair $(\lam_z\lam,\rho\rho_z) = za$
  indeed satisfies the \biap:
  for every $x,y \in E$ and $\om \in \Om$ we have
  \begin{eqnarray*}
     \pair{x}{(\rho\rho_z)(\om \ract y)}
           &=& \pair{x}{\rho((\om\ract y)\ract z)}
           \hspace{0.6em}=\hspace{0.6em}
               \pair{x}{\rho(\om\ract yz)}              \\
           &=& \pair{yz}{\lam(x\lact\om)}
           \hspace{0.6em}=\hspace{0.6em}
               \pair{y}{z \lact \lam(x\lact\om)}        \\
           &=& \pair{y}{(\lam_z\lam)(x\lact\om)}
  \end{eqnarray*}
  hence $za \in \ActE$\@. Analogously we prove that $az \in \ActE$.
\end{proof}


\begin{defn*} \label{def:env}
  Let $\EE$ be any \context\ and $A$ a subspace of \PreE\@. Denote
  $$  {\rm Env}(\EE\,;A)
         \:=\:  \left\{x\in\PreE  \,\left|\vertM\right.\,
                       xA \subseteq A \supseteq Ax \right\}.  $$
  Given \EE, a natural choice for $A$ would be $A=\ActE$,
  and since this is by far the most important case, we will abbreviate
  $$  \EnvE \: \equiv \: {\rm Env}\left(\EE\, ;\ActE \vertM\right).  $$
\end{defn*}

\begin{remark} \label{rem:Env_in_A}   \rm
  Notice that if $1_\EE \in A$, then ${\rm Env}(\EE\, ;A) \subseteq A$.
  In particular we get $\EnvE \subseteq \ActE$.
  \hfill $\star$
\end{remark}


\begin{lemma}
  Let\/ $\EE$ be any \context\ and $A$ a linear subspace of\/ \PreE\@.
  Then ${\rm Env}(\EE\, ;A)$ is a unital subalgebra of\/ \PreE.
\end{lemma}

\begin{proof}
  Take arbitrary $x,y \in {\rm Env}(\EE\, ;A)$.
  Then by assumption $xA$, $Ax$, $yA$ and $Ay$ are all contained in $A$.
  For all $a\in A$ we have $(xy)a = x(ya) \in xA \subseteq A$,
  hence $(xy)A \subseteq A$\@. Analogously $A(xy)\subseteq Ay \subseteq A$,
  and we conclude that also $xy$ belongs to ${\rm Env}(\EE\,;A)$\@.
  Obviously $1_\EE \in {\rm Env}(\EE\,;A)$.
\end{proof}


\begin{prop}  \label{prop:Env_is_an_algebra}
  Let\/ $\EE\equiv \EOP$ be a non-degenerate \context\ and
  $A$ a sub-$E$-bimodule of\/ \PreE\ w.r.t.\ multiplication.
  Then\/ ${\rm Env}(\EE\,;A)$ is a unital algebra containing $E$ as a subalgebra.
\end{prop}
\begin{proof}
  We already know that both $E$ and  ${\rm Env}(\EE\,;A)$ are subalgebras of \PreE\@.
  By assumption we have $EA \subseteq A \supseteq AE$,
  and hence $E \subseteq {\rm Env}(\EE\,;A)$.
\end{proof}


\begin{cor} \label{cor:Env_is_an_algebra}
   When\/ $\EE\equiv \EOP$ is a non-degenerate \context, then\/ \EnvE\ is a unital
   algebra of actors, containing $E$ as a subalgebra, and
   $$ E \,\simeq\, j(E)\,\subseteq\, \EnvE \,\subseteq\, \ActE \,\subseteq\, \PreE. $$
\end{cor}
\begin{proof}
  Because of lemma \ref{lem:Act_is_submodule}\ we can apply the above with $A=\ActE$.
\end{proof}



\begin{exA} \label{exA:comm_case}
  Recall the \context\/ $\FF \equiv (F; \kk S, \pairing)$ of example
  {\rm A \ref{exA:introduction}}\@.
  We now have  ${\rm Act}(\FF) = {\rm Env}(\FF) \simeq \kk^S.$
\end{exA}

For the proof we refer to \S\ref{par:comm_case}, where we shall study the {\em commutative\/}
case in general. Notice that, starting from any separating algebra of functions on $S$,
we are able to obtain the algebra of {\em all\/} functions on $S$,
by a process which only refers to the underlying set $S$ through the \context\ $\FF$\@.




\begin{exB} \label{exB:univ_env_vNa} \rm
  Take $\Aa \equiv (A; A^*, \pairing)$ as in example \ref{exB:introduction}\@.
  A theorem in operator algebra \cite[\S III.2]{Takesaki}\ states that $A^{**}$
  is a $W^*$-algebra containing $A$ as a subalgebra.
  It is called the universal enveloping von Neumann algebra of $A$\@.
  Now also $(A^{**}; A^*, \pairing)$ is a non-degenerate \context, hence every $a\in A^{**}$
  identifies with an actor $\lamrho{a}$ for the latter context, which is {\em a fortiori\/}
  an actor for \Aa\@. In this way we have embedded $A^{**}$ into $\Act(\Aa)$.
  \hfill $\star$
\end{exB}
