

\begin{abs_chp}
\Little\ $q$-Bessel functions were investigated in \cite{KoornwSwartt,Swarttouw}\
and first related to the quantum $E(2)$ group in \cite{Koelink:QE2,VaksKor:QE2}\@.
In \cite{KoornwSwartt}\ they were used as $q$-integral kernels
in order to construct $q$-Hankel transformation.
\Little\ $q$-Bessel functions are usually defined in terms of $q$-hypergeometric series.
For our purposes, we can rely entirely on a few elementary properties summarized below,
so we shall never have to deal with their definition directly;
for the sake of completeness however, we give an {\em ad hoc\/} definition below.
For the definition of the $q$-shifted factorials and $q$-Exponential function
we refer to appendix \ref{app:qcalc}\@.
\end{abs_chp}



\begin{lemma_sec} \label{lemma:def:qBessel}
Whenever\/ $n \in \ZZ$ and\/ $q\in\RR$ with\/ $0<q<1$, the power series
\begin{equation}\label{eq:def:qBessel}
  J_n(z;q) \:=\; \sum_{k=0}^{\infty} \:
  \frac{(-1)^k\,q^{\frac{1}{2}k(k-1)}\,q^k\,(q^{k+n+1};q)_\infty}{(q;q)_\infty \,(q;q)_k}\:z^{2k+n}
\end{equation}
defines an entire function in\/ $z$, having a zero of order\/ $|n|$ at the origin.

{\rm This function is said to be the {\em \little\ $q$-Bessel\/} function of order $n$.}
\end{lemma_sec}

\begin{proof}
First notice that (\ref{eq:def:qBessel}) is indeed a {\em power\/}
series---not a genuine Laurent series---when $n$ is negative,
because $(q^{k+n+1};q)_\infty=0$ whenever $n < -k \leq 0$.
So if $n<0$, the $(k=|n|)$-term will be the first non-zero term
in the series.
To prove that (\ref{eq:def:qBessel}) converges for all $z\in \CC$,
we appeal to the ratio test: fix any $z\in \CC$ and let
$a_k$ denote the $k$-term of the above series, then $a_k$
is non-zero (provided $k\geq|n|$ when $n$ is negative) and
$$ \left| \frac{a_{k+1}}{a_k} \right| \:=\:
    q^{k+1}\: \frac{(q^{k+n+2};q)_\infty}{(q^{k+n+1};q)_\infty} \:
              \frac{(q;q)_k}{(q;q)_{k+1}}\: |z|^2
    \:=\: \frac{q^{k+1}\, |z|^2}{(1-q^{k+n+1})\, (1- q^{k+1})},  $$
which tends to zero as $k\rarr\infty$, since $0<q<1$.
\end{proof}



\begin{remark_sec} \rm
Notice \little\ $q$-Bessel functions take real values on the real line,
since all the coefficients in (\ref{eq:def:qBessel}) are real.
It should be noted that it is also possible to consider \little\ $q$-Bessel
functions of non-integer order. However, those
of integer order have some interesting properties not shared
by those of non-integer order.
Furthermore, given $q$ we shall usually deal with $q^2$-Bessel functions
henceforth---rather than with $q$-Bessel functions.
\end{remark_sec}



\begin{prop_sec} \label{prop:qbessel:properties}
\Little\/ $q^2$-Bessel functions satisfy the following
recurrence and symmetry relations, for all\/ $n,m \in \ZZ$ and\/ $z\in\CC$,
\begin{eqnarray}
 z\,\J{n-1}{z} &=& \J{n}{z} - q^n\,\J{n}{qz} \vertXL
     \label{eq:qBessel:recurrence1} \\
 -q^{-n} z\, \J{n+1}{z} &=& \J{n}{q^{-1}z} - q^{-n} \J{n}{z} \vertXL
    \label{eq:qBessel:recurrence2} \\
 \J{n}{q^m} &=& \J{m}{q^n} \vertXL
    \label{eq:qBessel:symmetry}\\
 \J{-n}{z} &=& (-q)^n\, \J{n}{q^n z} \vertXL
    \label{eq:qBessel:inversion}
\end{eqnarray}
and respectively Hansen-Lommel and\/ $q$-Hankel orthogonality relations
\begin{eqnarray}
 \sum_{k\in \ZZ} \: q^{2k}\, \J{n+k}{z} \,\J{m+k}{z}
  &=& \delta_{n,m} \,q^{-2n} \hspace{2.5em}  (|qz|<1)
    \hspace{2em}   \label{eq:qBessel:Hansen_Lommel} \\
 \sum_{k\in \ZZ} \: q^{2k} \,\J{r}{q^{n+k}} \,\J{r}{q^{m+k}}
  &=& \delta_{n,m} \,q^{-2n} \hspace{2.5em}  (r\in\ZZ)
    \hspace{2em}  \label{eq:qBessel:qHankel}
\end{eqnarray}
where the sums in (\ref{eq:qBessel:Hansen_Lommel}) and (\ref{eq:qBessel:qHankel})
are absolutely convergent.
We also mention the following estimate:
for every\/ $m\in\ZZ$ there exists a positive number\/ $C_m$ such that
\begin{eqnarray}
  \left|\J{m}{q^k} \vertM\right|
&\leq&
  C_m   \left\{ \!\! \begin{array}{lcl}
              q^{km}                 &\hspace{1em}  &  (k \geq 0)  \\ \vertL
              q^{-km} \, q^{k(k-1)}  &              &  (k \leq 0)
           \end{array} \right. \label{eq:qBessel:estimate}
\\&\leq&
  C_m \: q^{|k|m} \vertXL  \label{eq:qBessel:estimate:bis}
\end{eqnarray}
for all\/ $k\in \ZZ$.
\end{prop_sec}

\begin{proof}
The recurrence relations (\ref{eq:qBessel:recurrence1}) and (\ref{eq:qBessel:recurrence2})
are equivalent with formulas (4.8) and (4.6) in
\cite{KoelinkSwart:qBessel:zeros}, the latter being in turn
related to $q$-derivatives of \little\ $q$-Bessel functions.
It is however not too hard to derive (\ref{eq:qBessel:recurrence1})
directly from the power series expression given in
\cite[proof of prop.\ 2.1]{KoornwSwartt}\@.
The other relations can be found in
\cite{Koelink:thesis,Koelink:QE2,KoornwSwartt,Swarttouw}\@.
Be aware that (\ref{eq:qBessel:Hansen_Lommel}) is only valid when
$|qz|<1$, and observe how (\ref{eq:qBessel:qHankel})
follows from (\ref{eq:qBessel:Hansen_Lommel}) via
(\ref{eq:qBessel:symmetry}) for $r\geq 0$, whereas proving the $r<0$ case of
(\ref{eq:qBessel:qHankel}) moreover involves (\ref{eq:qBessel:inversion}).
Also notice (\ref{eq:qBessel:inversion}) interchanges
(\ref{eq:qBessel:recurrence1}) and (\ref{eq:qBessel:recurrence2}).
\end{proof}
