\section{The duality between actors and multipliers}
\label{sec:multipliers}


\begin{abs_chp}
We investigate the relation between actors and multipliers.
It turns out that every multiplier yields an actor,
and thus the multiplier algebra can and will be embedded in the enveloping algebra
that was introduced in the previous section.
However the latter will usually be much larger than the multiplier algebra.
\end{abs_chp}



\begin{defn_sec} \label{def:MEE}
  If\/ $\EE \equiv \EOP$ is a non-degenerate \context, we write
  $$ M(\EE) \, \equiv \,    {\rm Env}(\EE\, ;E)
            \,=\,  \left\{x\in \PreE  \,\left| \vertM \right. \,
                          xE \subseteq E \supseteq Ex  \right\}. $$
  Recall that $E$ is identified with its image $j(E)$ in \PreE.
\end{defn_sec}

Our goal is to prove that $M(\EE)$ is actually---as one expects---isomorphic to the
multiplier\footnote{For the theory of multipliers we refer to \cite{Fons:MHA}.}\
algebra $M(E)$ of $E$\@. To do so, we need an extra assumption: we will
require $\Om$ to be unital as an \Ebimod, i.e.\ $E \lact \Om = \Om = \Om \ract E$.
Notice that by lemma \ref{lem:nondeg_context}\ this condition implies
non-degeneracy of the product in $E$, so we can indeed consider the multiplier
algebra of $E$. Although our actual approach will be different,
it's worth noticing that half of the assertion is already known:

\begin{remark_sec}
  Let\/ $\EE \equiv \EOP$ be an \context\ and let $\Om$ be unital.
  Then $M(\EE)$ is a unital algebra containing $E$ as an essential two-sided ideal.
\end{remark_sec}
\begin{proof}
  Apply proposition \ref{prop:Env_is_an_algebra}\ with $A=j(E)$\@.
  Clearly $E$ is a two-sided ideal in $M(\EE)$,
  so let's complete this proof showing that it is an {\em essential\/} ideal:
  take e.g.\ a pre-actor $a\equiv (\lam,\rho) \in M(\EE)$ such that $ax=0$,
  i.e.\ $(\lam\lam_x,\rho_x\rho)=0$, for all $x\in E$\@.
  Then for all $x,y \in E$ and $\om \in \Om$ we have
  $\lam(x\lact \om) = \lam(\lam_x(\om)) = 0$ and
  \mbox{$\pair{x}{\rho(y\lact \om)} = \pair{x}{y\lact \rho(\om)}
         = \pair{xy}{\rho(\om)} = \pair{y}{\rho_x(\rho(\om))} = 0$}\@.
  Using $E\lact \Om = \Om$ we conclude that both $\lam$ and $\rho$ are zero,
  hence $a=0$\@.
\end{proof}
\vspace{2ex}

The other half of our assertion would be the statement that $M(\EE)$
is in fact the {\em largest\/} algebra having this property.
Our actual approach however, will be more explicit:
we simply construct a natural isomorphism from $M(E)$ onto $M(\EE)$.

\begin{lemma_sec}  \label{embedding_of_M(E)}
  Let\/ $\EE \equiv \EOP$ be an \context\ such that $\Om$
  is unital as an \Ebimod\@.
  Then the embedding $j: E \rarr \PreE$ extends uniquely to a map
  $$ \jhat : M(E)\rarr \PreE : z \mapsto \jhat(z)\equiv\lamrho{z} $$
  such that the following holds for all $y\in E$, $z\in M(E)$ and $\om\in\Om$:
  \begin{equation}  \label{eq:def_of_jhat}
     \pairM{y}{\lam_z(\om)} \:=\: \pair{yz}{\om}    \itandspace{3em}
     \pairM{y}{\rho_z(\om)} \:=\: \pair{zy}{\om}.
  \end{equation}
  Moreover \jhat\ is still an injective algebra morphism.
\end{lemma_sec}
\begin{proof}
   Let $z\equiv (L,R)$ be a multiplier of $E$. Here $L$ and $R$ are linear maps
   from $E$ into $E$, having (algebraic) transpositions $L\algtp,R\algtp : E' \rarr E'$.
   Now take any $y\in E$ and $\om\in\Om$.
   Identifying $\Om$ with its image in $E'$, we get for all $x\in E$ that
   \begin{equation} \label{eq:mod_prop_mult}
      \pairM{x}{L\algtp(\om\ract y)}
         = \pairM{L(x)}{\om \ract y}
         = \pairM{y(zx)}{\om}
         = \pairM{(yz)x}{\om}
         = \pairM{x}{\om \ract yz}
   \end{equation}
   hence $L\algtp(\om\ract y) = \om \ract yz \in \Om$\@.
   It follows that $L\algtp(\Om) = L\algtp(\Om \ract E) \subseteq \Om$, and
   analogously $R\algtp(\Om) \subseteq \Om$. So we can define linear maps
   $\lam_z$ and $\rho_z$ from $\Om$ into $\Om$ by
   $\lam_z(\om) = R\algtp(\om)$ and $\rho_z(\om) = L\algtp(\om)$,
   clearly satisfying (\ref{eq:def_of_jhat}).
   When $z$ happens to be in $E$ itself, this notation is of course
   compatible with the one we introduced in lemma \ref{lem:embedding_of_E}.
   Now we claim that $\lam_z$ is a right $E$-module morphism; indeed, using
   (\ref{eq:def_of_jhat}) we get for all $\om \in \Om$ and $x,y \in E$ that
   $$ \pairM{x}{\lam_z(\om \ract y)}
            = \pairM{xz}{\om \ract y}
            = \pairM{yxz}{\om}
            = \pairM{yx}{\lam_z(\om)}
            = \pairM{x}{\lam_z(\om) \ract y}.   $$
   Similarly $\rho_z$ is a left $E$-module morphism. Hence for every $z\in M(E)$
   we have a pre-actor $\jhat(z) \equiv \lamrho{z} \in \PreE$
   obeying (\ref{eq:def_of_jhat}).
   The assertion that \jhat\ is an algebra morphism is an easy consequence of
   (\ref{eq:def_of_jhat}) and associativity in $M(E)$.
   Injectivity and uniqueness of \jhat\ are obvious.
\end{proof}



\begin{prop_sec}  \label{prop:ME=MEE}
  Let\/ $\EE \equiv \EOP$ be an \context,
  $\Om$ unital as an \Ebimod, and \jhat\ as above.
  Then $M(\EE) = \jhat(M(E)) \subseteq \EnvE \subseteq \ActE$.
\end{prop_sec}
\begin{proof}
%   The previous lemma would allow us to identify $M(E)$ with its image under \jhat\@.
%   Nevertheless we will postpone this identification for a while,
%   since we are about to consider elements in $M(E)$ both as an
%   actor and a multiplier at the same time. So for the moment
%   we prefer to write \jhat\ explicitly in order to avoid ambiguity.
%%%%%%%%%%%%%%
   We first show that \ActE\ is a sub-$M(E)$-bimodule of \PreE, which is quite similar
    to lemma \ref{lem:Act_is_submodule}, but now considering $M(E)$ instead of $E$\@.
   So let us take any $z\in M(E)$ and $a\equiv (\lam,\rho) \in \ActE$
   and prove that $\jhat(z)a$ and $a\jhat(z)$ are in \ActE\ again.
   The pair $(\lam_z\lam,\rho\rho_z)=\jhat(z)a$ indeed satisfies (\ref{eq:bi-actor}), since
   $$   \pairM{x}{\rho(\rho_z(\om \ract y))}
          \stackrel{(\ref{eq:mod_prop_mult})}{=}
                    \pairM{x}{\rho(\om\ract yz)}
          \stackrel{(\ref{eq:bi-actor})}{=}
                    \pairM{yz}{\lam(x\lact\om)}
          \stackrel{(\ref{eq:def_of_jhat})}{=}
                    \pairM{y}{\lam_z(\lam(x\lact\om))}      $$
  for all $x,y \in E$ and $\om \in \Om$.
  Here we used $\rho_z(\om \ract y) = \om \ract yz$ obtained in (\ref{eq:mod_prop_mult}).
  Similarly for $a\jhat(z)$.  We conclude that $\jhat(M(E)) \subseteq \EnvE$.

  Now we show $M(\EE) = \jhat(M(E))$\@. First take any $a \in M(\EE)$,
  i.e.\ let $a\equiv (\lam,\rho)$ be a pre-actor with $a j(E) \subseteq j(E) \supseteq j(E)a$.
  Then one can construct linear maps $L,R: E \rarr E$ such that
  $j(L(x))= a j(x)$ and $j(R(x))=j(x)a$ for any $x\in E$,
  hence $\lam_{L(x)}=\lam \lam_x$ etc.
  Now $z\equiv (L,R)$ is a multiplier \mbox{of $E$}, and
  for all $x\in E$ we get $\lam_z\lam_x = \lam_{zx} = \lam_{L(x)} = \lam\lam_x$,
  hence $\lam_z$ equals $\lam$ on $E \lact \Om = \Om$. Similarly also $\rho_z =\rho$.
  We conclude that $a = \jhat(z)$.
  The other inclusion is obvious.
\end{proof}



\begin{remarks_sec} \rm
\item
The above allows us to identify $M(E)$ with a subalgebra $M(\EE)$ of \EnvE\
provided that $\Om$ is unital as an \Ebimod\@. Moreover (\ref{eq:def_of_jhat})
exhibits a duality between actors and multipliers in the sense that every
multiplier can be \lq dualized\rq\ into an actor;
however not every actor for \EE\ need to arise in this way,
so \EnvE\ is usually larger than $M(E)$\@.
Also notice that \EnvE\ depends on the actor implementation,
whereas $M(E)$ is completely determined by the algebra $E$ itself.
Furthermore we loose a substantial amount of \lq affiliation\rq\ with $E$,
as shown by example A.
\item
A one-line summary of our results thus far:
$$ E \subseteq M(E) \simeq M(\EE) \equiv {\rm Env}(\EE\vssp;E)
   \subseteq \EnvE \equiv {\rm Env}(\EE\vssp;\ActE)
   \subseteq \ActE. $$
\item
A priori $M(\EE)$ was merely a set of {\em pre\/}-actors satisfying some invariance
condition; then it {\em turned out\/} to be really a space of {\em actors}.
\end{remarks_sec}
