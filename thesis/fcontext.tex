\section{Fourier contexts}


\begin{abs_chp}
We introduce an axiomatic context in which to do harmonic analysis on an algebraic level,
the axioms being inspired to a large extend by the quantum $E(2)$ example that is studied
in the next chapter.
Minimalizing the set of axioms is not our first concern:
we mainly want to outline a setting providing all the ingredients needed to do Fourier analysis.
Within this context, we define the notion of a {\em Fourier transform\/}
and establish its uniqueness.
The underlying structure of such a \lq Fourier context\rq\ basically consists of three
\ahss s---one of which is actually an ordinary dual pair of \mbox{Hopf $^*$-algebras}\@.
\end{abs_chp}



\begin{defn_sec} \label{def:Fourier_context}
Let $\pair{\frakA}{\frakB}$ be a non-degenerate dual pair of Hopf $^*$-algebras and let
$\left\langle \underline{\frakA}\, , B \right\rangle$ and
$\left\langle A, \underline{{\frakB}\!}\,\right\rangle$
be \ahss s (definition \ref{def:algebraic_Hopf_system}).
Then both $A$ and $\frakA$ can be considered as subalgebras of ${\frakB}'$,
hence elements of $A$ can be multiplied with elements of $\frakA$ within ${\frakB}'$
(similarly for $B$ and ${\frakB}$ within ${\frakA}'$).
Assume there exist invariant functionals on $A$ and $B$, say respectively
$\phiA, \phiB$ (left invariant) and $\psiA, \psiB$ (right invariant).
Furthermore, let $A_0$ and $B_0$ be non-trivial subspaces of $A$ and $B$ respectively.
Now assume the following:
\begin{enumerate}
\item
Under multiplication, $A$ is an $\frakA$-bimodule and $B$ is a $\frakB$-bimodule.
\item
$A_0$ is a sub-$\frakB$-bimodule of $A$ (under the actions induced by duality) as
well as a sub-$\frakA$-bimodule (under multiplication). Similarly for $B_0$.
\item
$A_0$ and $B_0$ are invariant under $^*$ and $\conj$, or equivalently,
under $^*$ and $S^{\pm 1}$.
\item
The invariant functionals are all hermitian, positive and faithful.
\item
The invariant functionals also enjoy the following faithfulness properties:
for any $a \in A_0$ and $b\in B_0$ we have
$$ \begin{array}{lccclcc}
   \phiA({\frakA}a) = \{0\} &\; \Rightarrow \;& a=0 & \hspace{3em} &
   \phiB({\frakB}b) = \{0\} &\; \Rightarrow \;& b=0 \\
   \psiA({\frakA}a) = \{0\} & \; \Rightarrow \;& a=0 & \hspace{3em} &
   \psiB({\frakB}b) = \{0\} & \;\Rightarrow \;& b=0
\end{array} $$
\item
The invariant functionals are all weakly {\sc kms}.
The corresponding {\sc kms} automorphisms shall be denoted by $\sigma_{\!\phiA}$ etc.
\item
Moreover assume $\phiA$ to be ${\frakA}$-{\sc kms} on $A_0$ in the sense that
$$ \sigma_{\!\phiA}(A_0) = A_0
        \andspace{3em}
   \phiA(a \alpha) = \phiA \!\left (\alpha \,\sigma_{\!\phiA}(a) \vertM\right) $$
for any $a\in A_0$ and $\alpha\in {\frakA}$. \\
Similarly $\psiA$ is ${\frakA}$-{\sc kms} on $A_0$,
whereas $\phiB$ and $\psiB$ are ${\frakB}$-{\sc kms} on $B_0$.
\item
There exist complex numbers $\zetaA$ and $\zetaB$ of modulus 1 such that
$$ \phiA S = \zetaA \psiA      \hspace{3em}
   \psiA S    = \zetaA \phiA   \hspace{3em}
   \phiB S = \zetaB \psiB      \hspace{3em}
   \psiB S    = \zetaB \phiB.  $$
\item
There exists a $\deltaA \in {\frakA}$ such that
$\phiA S = \phiA(\,\cdot \,\deltaA)$.
Similarly there exists a $\deltaB \in {\frakB}$ such that
$\phiB S = \phiB(\,\cdot \,\deltaB)$.
$\:\deltaA$ and $\deltaB$ are called {\em modular\/} elements.
\end{enumerate}
Under these circumstances, \FourierBABA\ is said to be a {\em Fourier context}\@.
\end{defn_sec}

Let's agree to keep the notation as short as possible, e.g.\ if
$A=A_0$, $B=B_0$, $\phiA=\psiA$ and $\phiB=\psiB$,
then we abbreviate the above by
$\left(A, \phiA; \:{\frakA},{\frakB};\: B, \phiB \vertM \right)$.
\vspace{1ex}

It is also worth noticing that the above definition is {\em self-dual},
in the sense that it is completely symmetric between ${\frakA}$ and  ${\frakB}$,
between $A$ and $B$, etc.


\begin{defn_sec} \label{def:Fourier_transform}
Let \FourierBABA\ be any Fourier context.
A linear map $\FRL:A_0 \rarr B_0$ is said to be an {\em  {\small RL} Fourier transform\/} if
\begin{enumerate}
\item
$\FRL$ is a left $\frakB$-module morphism;
explicitly, for any $\beta\in {\frakB}$ and $a\in A_0$ we have
$\FRL(\beta \lact a) = \beta \, \FRL(a)$.
\item
$\phiB \! \left(\FRL(a) \vertM \right) = \pair{a}{1_{\frakB}}$ for any $a\in A_0$.
\end{enumerate}
Similarly we can define {\sc lr}, {\sc ll} and
{\sc rr}\footnote{The first {\sc l/r} subscript {\em anti\/}-corresponds to the fact
   that the Fourier transform under consideration is a
   left/right $\frakB$-module morphism, the second {\sc l/r} corresponds to the
   {\sc l}eft/{\sc r}ight invariant functional we are dealing with in (ii).}
Fourier transforms from $A_0$ to $B_0$ and vice versa, i.e.\ from $B_0$ to $A_0$.
\end{defn_sec}


\begin{remarks_sec} \label{rem:Fourier_transform}
\item
Requiring $\FRL$ to be a $\frakB$-module morphism expresses
a well-known fact from classical Fourier analysis:
Fourier transformation interchanges differentiation and multiplication.
\item
The main reason to allow these Fourier transforms to be only defined
on subspaces $A_0$ or $B_0$ rather than all of $A$ or $B$, is that
in general $A_0$ and $B_0$ need not to be {\em algebras},
e.g.\ in the case of the quantum $E(2)$ group it certainly won't be easy to
find candidates for $A_0$ and $B_0$ which are moreover algebras
(cf.\ chapter \ref{chapter:Harmonic_analysis_on_quantumE2}).
\item
Observe that the conditions (v) in definition \ref{def:Fourier_context}\
are only \lq one-sided\rq\@. However, we still have the
$^*$-operations\footnote{or the antipodes,
cf.\ condition (viii) of definition \ref{def:Fourier_context}.}\
to change sides.
These conditions will ensure that, given a particular Fourier context,
its Fourier transforms are {\em unique\/}
(cf.\ lemma \ref{lemma:Fourier_transform}\ below).
\item
Obviously the above set of axioms for a Fourier context is far from minimal.
For instance condition (viii) should normally be a proposition
rather than an assumption; also the different types of {\sc kms}
conditions are likely to be related, etc.
However, our main interest here is just to provide all the ingredients we need for
doing Fourier analysis, without getting involved in difficult (though important)
matters like uniqueness and faithfulness of Haar functionals.
Anyway, in our quantum $E(2)$ example all the features required by
definition \ref{def:Fourier_context}\ shall be available.
\item
Observe that requiring $\zetaA$ and $\zetaB$ to have modulus 1 does not
really harm generality, because of corollary \ref{cor:scaling}.
\item
Recall that $A_0$ and $B_0$ were merely assumed to be non-trivial, whereas
maybe it would have been more natural to impose some {\em density\/} requirement on them.
Certainly the latter would not harm, but it turns out that $A_0$ and $B_0$
{\em automatically\/} contain sufficiently many elements---check out lemma
\ref{lemma:properties_of_Fourier_contexts}.ii for details.
\item
Fourier transforms also emerged in the duality theory for multiplier Hopf algebras
with an integral \cite{Fons:AFGD,Fons:pnas}\
and \mhs s (definition \ref{def:mhs:Fourier_transforms}).
In this respect we mention that e.g.\ the {\scriptsize RL} Fourier transform
in definition \ref{def:Fourier_transform}\
is comparable to the {\em inverse\/} of $a \mapsto \hat{a}=\varphi(\,\cdot\, a)$.
\end{remarks_sec}


Let's collect some properties of Fourier contexts in the following

\begin{lemma_sec} \label{lemma:properties_of_Fourier_contexts}
Let \FourierBABA\ be a \mbox{Fourier context}.
\begin{enumerate}
\item
The Haar functionals enjoy the following notions of strong left invariance:
(e.g.) for any\/ $\alpha \in {\frakA}$, $\beta\in {\frakB}$ and\/ $b, d \in B$ we have
\begin{eqnarray*}
\phiB \!\left(\vertM b\,(d \ract \alpha) \right)
      &=&  \phiB \!\left(\vertM (b \ract S(\alpha))\,d \right)       \\
\phiB \!\left(\vertM \beta\,(d \ract \alpha) \right)
      &=&  \phiB \!\left(\vertM (\beta \ract S(\alpha))\,d \right)   \\
\phiB \!\left(\vertM b\,(\beta \ract \alpha) \right)
      &=&  \phiB \!\left(\vertM (b \ract S(\alpha))\,\beta \right)
\end{eqnarray*}
\item
Besides (iv) and (v) in definition \ref{def:Fourier_context},
the invariant functionals also enjoy the following faithfulness property: (e.g.)
$$ \beta \in {\frakB} \itandspace{1em} \phiB(B_0 \beta) = \{0\}
         \hspace{2.5em} \Longrightarrow    \hspace{2em}  \beta=0. $$
\item
The modular elements\/ $\deltaA$ and\/ $\deltaB$ in definition \ref{def:Fourier_context}.ix\
are unique.
\item
Also $\psiA$ and $\psiB$ have a modular property: (e.g.)\/
$\psiA S = \psiA(S(\deltaA) \: \cdot\,)$
\end{enumerate}
\end{lemma_sec}
\begin{proof}
Let ${\mathcal B}$ be the $^*$-subalgebra of $\frakA'$ generated by $B$ and $\frakB$.
In fact ${\mathcal B}$ is then also the linear span of $B \cup {\frakB}$ within ${\frakA}'$,
and $\pair{\underline{\frakA}\,}{\mathcal B}$ is obviously still an \ahss\@.
Now proposition \ref{prop:strong_left_invariance}\ yields (i).

To show (ii), take any $\beta \in {\frakB}$ and assume that
$\phiB(B_0 \beta) = \{0\}$. Using strong left invariance we observe that,
for any $\alpha \in {\frakA}$, $\eta \in {\frakB}$ and $b_0 \in B_0$,
$$ \phiB \!\left(\vertM \eta b_0 \,(\beta \ract \alpha) \right)
    \;=\; \phiB \!\left(\vertM(\eta b_0 \ract S(\alpha))\, \beta \right) = 0. $$
Here we used that $\eta b_0 \ract S(\alpha)$ still belongs to $B_0$.
So $\phiB \!\left(\vertM {\frakB}\, b_0(\beta \ract \alpha) \right) = \{0\}$
and hence $b_0(\beta \ract \alpha)=0$. It follows that
$\pair{1_{\frakA}}{b_0} \pair{\alpha}{\beta}
   = \pair{1_{\frakA}}{b_0(\beta \ract \alpha)} = 0$
for all $\alpha \in {\frakA}$ and $b_0 \in B_0$. Now the
non-degeneracy of the pairings $\pair{\frakA}{\frakB}$
and $\pair{\frakA}{B}$ yields $\beta=0$.
Indeed, if $\beta$ were non-zero, then we would have
$\pair{1_{\frakA}}{B_0}=\{0\}$ and hence
$\pair{\alpha}{b_0} = \pair{1_{\frakA}}{\alpha \lact b_0} = 0$
for all $\alpha \in {\frakA}$ and $b_0 \in B_0$, which would mean
that $B_0$ is trivial. This completes the proof of (ii).
%
Now (iii) follows easily from (ii).
%
Eventually, using \ref{def:Fourier_context}.vi-vii-viii-ix, we get
$$ (\psiA S)(a) \:=\: \zetaA \, \phiA(a)
                \:=\: \zetaA \, \phiA(S^{-1}(a)\,\deltaA)
                \:=\: (\psiA S)(S^{-1}(a)\,\deltaA)
                \:=\: \psiA(S(\deltaA) a)    $$
for all $a\in A$, which proves (iv).
\end{proof}


\begin{lemma_sec} \label{lemma:Fourier_transform}
Let \FourierBABA\ be a \mbox{Fourier context}\@.
For any linear mapping\/ $\FRL:A_0 \rarr B_0$ the following are equivalent:
\begin{enumerate}
  \item $\FRL$ is an {\scriptsize RL} Fourier transform,
  \item $\phiB \!\left(\vertM \beta \FRL(a) \right)= \pair{a}{\beta}$ for all\/
        $\beta\in {\frakB}$ and\/ $a\in A_0$.
\end{enumerate}
\rm Similar properties hold for {\scriptsize LR}, {\scriptsize LL} and {\scriptsize RR}
Fourier transforms. As a corollary we observe that {\em Fourier transforms are unique}.
\end{lemma_sec}
\begin{proof}
(i $\Rightarrow$ ii).
For all $\beta\in {\frakB}$ and $a\in A_0$, we have
$$\phiB \!\left(\vertM\beta \FRL(a) \right)
       \:=\: \phiB \!\left(\vertM\FRL(\beta \lact a) \right)
       \:=\: \pair{\beta \lact a}{1_{\frakB}}
       \:=\: \pair{a}{\beta}.$$
(ii $\Rightarrow$ i).
For all $\beta, \eta\in {\frakB}$ and $a\in A_0$, we have
$$\phiB\!\left(\vertM\eta \beta \FRL(a)\right)
     \:=\: \pair{a}{\eta \beta}
     \:=\: \pair{\beta \lact a}{\eta}
     \:=\: \phiB\!\left(\vertM\eta \FRL(\beta \lact a)\right).$$
Observe that $\beta \FRL(a)$ and $\FRL(\beta \lact a)$ belong to $B_0$,
hence from definition \ref{def:Fourier_context}.v\ it follows that
$\FRL(\beta \lact a) = \beta \FRL(a)$.
Furthermore, $\phiB(\FRL(a)) = \pair{a}{1_{\frakB}}$
is just the $\beta = 1_{\frakB}$ case of (ii).
Uniqueness follows from (ii) and definition \ref{def:Fourier_context}.v.
\end{proof}
\vspace{2ex}

Recall that Fourier transforms from $A_0$ to $B_0$ are, by definition,
$\frakB$-module morphisms.
Now {\em strong\/} invariance (cf.\ lemma \ref{lemma:properties_of_Fourier_contexts}.i)
of the Haar functionals implies these Fourier transforms to have $\frakA$-module
properties as well:


\begin{lemma_sec} \label{lemma:Fourier:extra_module prop}
Take a Fourier context as above and let\/ $\FLL:A_0 \rarr B_0$ be an
{\scriptsize LL} Fourier transform.
Then\/ $\FLL \, S^{-1}: A_0 \rarr B_0$ is a right\/ $\frakA$-module morphism,
i.e. for all\/ $a\in A_0$ and\/ $\alpha \in \frakA$ we have:
$$ (\FLL\, S^{-1})(a\alpha) \:=\: (\FLL\, S^{-1})(a) \ract \alpha. $$
\end{lemma_sec}
\begin{proof}
Using an {\scriptsize LL}-analogue of lemma \ref{lemma:Fourier_transform},
we obtain for any $a\in A_0$, $\alpha \in \frakA$ and $\beta\in {\frakB}$ that
\begin{eqnarray*}
\phiB\!\left((\FLL S^{-1})(a\alpha) \, \beta \vertM\right)
  &=& \pairM{S^{-1}(\alpha)\, S^{-1}(a)}{\beta}      \\
  &=& \pairM{S^{-1}(a)}{\beta \ract S^{-1}(\alpha)}  \\
  &=& \phiB\!\left((\FLL S^{-1})(a) \,
             \left(\beta \ract S^{-1}(\alpha)\vertM\right) \! \vertL\right) \\
  &\stackrel{(*)}{=}&
   \phiB\!\left(\left((\FLL S^{-1})(a) \ract \alpha\vertM\right) \beta \vertL\right).
\end{eqnarray*}
Here $(*)$ relies on {\em strong left invariance\/} of $\phiB$
(cf.\ lemma \ref{lemma:properties_of_Fourier_contexts}.i).
The result follows from the faithfulness assumptions in definition \ref{def:Fourier_context}.v.
\end{proof}



\begin{ex_sec} \label{ex:Schwartz:moment_problem:false} \rm
Recall examples \ref{ex:familiar:hopfR}\ and \ref{ex:familiar:Schwartz}\@.
One may ask whether
$$  \left( \SH, {\textstyle \frac{1}{\sqrt{2\pi}} \int_\RR};\,
          \HopfR, \HopfR  ;\, \SH, {\textstyle \frac{1}{\sqrt{2\pi}} \int_\RR}
           \vertL \right)  $$
is a Fourier context; the answer is no, for the following reason:
condition (v) in definition \ref{def:Fourier_context}\
specializes to the {\em moment problem\/}
\begin{equation}\label{eq:moment_problem}
   f\in \SH \: \mbox{ and}
      \int_\RR t^m f(t)\, dt = 0\; \mbox{ for all }\,m\in \NN
      \hspace{6mm} \stackrel{\textstyle ?}{\Longrightarrow} \hspace{6mm} f=0
\end{equation}
which has {\em negative\/} answer for our Schwartz-like space $\SH$.
Indeed, it is easy to construct a non-zero function $g\in C^\infty(\RR)$ with
compact support and such that $g$ and all its derivatives vanish at the
origin. Now if $f=\hat{g}$ denotes the (usual) Fourier transform of $g$,
then $f\in S(\RR)$ because $g\in C_c^\infty(\RR) \subseteq S(\RR)$.
Furthermore $f$ is entire, for $g$ has compact support.
It follows that $f$ is a non-zero function in $\SH$
with vanishing moments.
This argument immediately reveals a second reason to reject $\SH$
as the domain for our Fourier transforms: indeed
Fourier transformation simply doesn't map $\SH$ into itself.
\hfill $\star$
\end{ex_sec}


\begin{ex_sec} \label{ex:R:Fourier:Schwartz} \rm
Let $\mathcal E$ be the subspace of \HC\ spanned by the functions
$$  \CC \rarr \CC \,: \,x \mapsto\, x^m \exp(-\lambda x^2 + \zeta x)
        \hspace{3em}
        (m\in \NN, \, \lambda\in \RR_0^+, \, \zeta \in \CC). $$
Notice that $\mathcal E$ is actually a $^*$-subalgebra of $\SH$.
We claim that
\begin{equation} \label{eq:ex:Plancherel}
   \left( {\mathcal E} \subseteq \SH,
                {\textstyle \frac{1}{\sqrt{2\pi}} \int_\RR}\,;
                  \HopfR, \, \HopfR \,;\,
          {\mathcal E} \subseteq \SH,
                {\textstyle \frac{1}{\sqrt{2\pi}} \int_\RR}  \vertL \right)
\end{equation}
is a Fourier context.
\end{ex_sec}
\begin{proof}
The observation that $\SH$ is invariant under multiplication with
polynomials yields (i) of definition \ref{def:Fourier_context}\@.
Furthermore $\mathcal E$ is invariant under differentiation and therefore under
the actions (\ref{eq:actions:PR_on_HC}), which means that $\mathcal E$ is a
$\HopfR$-bimodule w.r.t.\ these actions.
Moreover $\mathcal E$ is obviously invariant under multiplication with
polynomials, which proves (ii). To prove (v) we need positive answer
to the moment problem for $\mathcal E$, analogous to (\ref{eq:moment_problem})
in the previous example, which is easily established using the fact that�
Hermite functions form an {\sc onb} for the Hilbert space $L^2(\RR)$.
The remaining items are obvious.
\end{proof}


\begin{prop_sec}  \label{prop:relations_between_Ftransforms}
Let\/ \FourierBABA\ be any Fourier \mbox{context}\ and
let there exist an {\scriptsize RL} Fourier transform\/ $\FRL: A_0 \rarr B_0$.
Then we automatically have the other three Fourier transforms from\/ $A_0$ to\/ $B_0$,
e.g.\footnote{recall that antipodes and {\sc kms} automorphisms are bijective
on $A_0$ and $B_0$ because of (iii) and (vii) in definition \ref{def:Fourier_context}.}
$$  \FLR \:=\: \zetaB\, S^{-1} \FRL S^{-1}       \hspace{3.5em}
    \FLL \:=\: \sigma_{\! \phiB}^{-1} \FRL       \hspace{3.5em}
    \FRR \:=\: \sigma_{\psiB} \FLR.     $$
Another way to construct e.g.\/ $\FRR$ from\/ $\FRL$ is to exploit the $^*$-structure:
\begin{equation} \label{eq:Fourier:stars}
  \FRR \: = \: \zetaB *S \FRL*
\end{equation}
Eventually the presence of modular elements yields formulas like
\begin{equation} \label{eq:Fourier:modular}
 \zetaB\, \FRL(a) \;=\; \deltaB\, \sigma_{\!\phiB} \!\left(\FLR(a)\vertM\right)
 \hspace{3em} \mbox{for any\/ } a\in A_0.
\end{equation}
\end{prop_sec}
\begin{proof}
According to lemma \ref{lemma:Fourier_transform}, we have for any
$\beta\in {\frakB}$ and $a\in A_0$ that
$$ \phiB \! \left( S(\beta) \, \FRL(S^{-1}(a)) \vertM\right)
      \:=\: \pairM{S^{-1}(a)}{S(\beta)}  \:=\: \pair{a}{\beta}. $$
With $\phiB S = \zetaB \psiB$ and the fact that the antipode
(on ${\frakA}' \supseteq {\frakB},B_0$) is an anti-homomorphism, we obtain
$$ \zetaB \, \psiB \! \left((S^{-1} \FRL S^{-1})(a)\, \beta \vertM\right)
         \:=\: \pair{a}{\beta} $$
which (by an analogue of lemma \ref{lemma:Fourier_transform}) means precisely that
$\zetaB\, S^{-1} \FRL S^{-1}$ is an {\scriptsize LR} Fourier transform from $A_0$ to $B_0$.
This shows the first formula.
The second one is an immediate consequence of the ${\frakB}$-{\sc kms}
property of $\phiB$ on $B_0$ (\mbox{cf.\ definition}\ \ref{def:Fourier_context}.vii).
Indeed for any $\beta\in {\frakB}$ and $a\in A_0$ we have
$$ \phiB \! \left( \sigma_{\phiB}^{-1} (\FRL(a)) \, \beta  \vertM\right)
        \:=\: \phiB \! \left(\beta \, \FRL(a)  \vertM\right)
        \:=\: \pair{a}{\beta}. $$
The third formula is analogous to the second one.
To prove (\ref{eq:Fourier:stars}) we start again from lemma \ref{lemma:Fourier_transform}\@.
Using the fact that $\phiB$ is hermitian, we obtain
$$ \phiB \! \left(\FRL(a^*)^* \: S(\beta) \vertM\right)
   \:=\: \overline{\phiB \! \left(S(\beta)^* \:\FRL(a^*)  \vertM\right)}
   \:=\: \overline{\pairM{a^*}{S(\beta)^* }} \:=\: \pair{a}{\beta}  $$
for all $\beta\in {\frakB}$ and $a\in A_0$.
Using $\phiB S = \zetaB \psiB$ we obtain
$$\zetaB \psiB \! \left(\beta \, S^{-1}(\FRL(a^*)^*) \vertM\right) \:=\: \pair{a}{\beta}$$
and (\ref{eq:Fourier:stars}) follows.
Once again using the ${\frakB}$-{\sc kms} property of $\phiB$ on $B_0$, we get
$$ \phiB \! \left(\beta \, \deltaB \, \sigma_{\! \phiB}(\FLR(a)) \vertM\right)
      \:=\:  \phiB \! \left(\FLR(a) \, \beta \deltaB \vertM\right)
      \:=\:  \zetaB\, \psiB \! \left(\FLR(a) \, \beta \vertM\right)
      \:=\:  \zetaB \pair{a}{\beta} $$
for any $\beta\in {\frakB}$ and $a\in A_0$. This proves (\ref{eq:Fourier:modular}).
\end{proof}

\begin{remark_sec} \rm
More generally we have
$S F_{\!\scriptscriptstyle PQ} S =
\zetaB F_{\!\bar{\scriptscriptstyle P} \bar{\scriptscriptstyle Q}}$
for any pair of labels $\scriptstyle P, \, Q$ in the label set $\{{\scriptstyle L,\, R} \}$.
Here $\,\bar{}\,$ is defined by $\bar{\scriptstyle L} =\, \scriptstyle R$
and $\bar{\scriptstyle R} =\, \scriptstyle L$.
\hfill $\star$
\end{remark_sec}
