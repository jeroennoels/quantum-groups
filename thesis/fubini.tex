
\chapter{Weak Fubini tensor products}   \label{app:fubini}

Let $\pair{E_i}{\Om_i}$ ($i=1,\ldots\, n$) be a finite number of
vector space dualities, i.e.\ for every index $i$ we consider two
linear spaces $E_i$ and $\Om_i$ and a non-degenerate pairing
$\pairing : E_i \times \Om_i \rarr \kk$\@.
Endow $E_i$ and $\Om_i$ with the weak topologies induced by the duality.
We have canonical embeddings
$\Om_i \hookrightarrow E_i'\equiv\, \overline{\Om_i\!}\,$
and
$$ \Om_1 \tens \ldots\, \Om_n \:\hookrightarrow\:
   E_1'\tens \ldots\, E_n' \:\hookrightarrow\: (E_1\tens \ldots\, E_n)'
   \:\equiv\:  \overline{\Om_1 \tens \ldots\, \Om_n\!}\, $$
and analogously
$E_1\tens \ldots\, E_n  \hookrightarrow  (\Om_1 \tens \ldots\, \Om_n)'$.
The obvious pairing between $E_1\tens \ldots\, E_n$ and $\Om_1 \tens \ldots\, \Om_n$
is still non-degenerate.
Now $(E_1\tens \ldots\, E_n)'$ identifies with the space of all multilinear forms on
$E_1 \times \ldots\, E_n$, e.g.\ a simple tensor
$\om_1 \tens \ldots\, \om_n \,\in\, \Om_1 \tens \ldots\, \Om_n$
corresponds to the form
$$ E_1 \times \ldots\, E_n \rarr \kk \,: \:
   (x_1, \ldots\, x_n) \:\mapsto\: \pair{x_1}{\om_1} \ldots\, \pair{x_n}{\om_n}. $$
Observe that all multilinear forms which arise from elements in the algebraic
tensor product $\Om_1 \tens \ldots\, \Om_n$ are {\em jointly\/} continuous
w.r.t.\ the weak topologies on $E_1, \ldots\, E_n$\@. Moreover, in the $n=2$ case,
a standard result
(see \cite[\S 42.4]{Treves}, \cite[\S II.41.3.10]{Kothe}, or \cite[\S 15.3.6]{Jarchow})
states that $\Om_1 \tens \Om_2$ is actually linearly isomorphic to the space of all
jointly continuous bilinear forms on $E_1 \times E_2$\@.
Weakening the condition of {\em joint\/} continuity leads us to the following:

\begin{defn_chp} \label{def:weak_fubini_tensor}
  Consider $\pair{E_i}{\Om_i}$ as above ($i=1,\ldots\, n$).
  The linear space of all {\em separately\/} continuous multilinear forms on
  $E_1 \times \ldots\, E_n$ will be denoted by $\Om_1 \fubtens \ldots\, \Om_n$,
  and referred to as the {\em weak Fubini\/} tensor product of the spaces
  $\Om_i$ w.r.t.\ the dualities $\pair{E_i}{\Om_i}$.
\end{defn_chp}

\begin{remarks_chp} \label{rem:weak_fubtens}
%\item
%  Up to identification we get
%  $$ \Om_1 \tens \ldots\, \Om_n               \: \subseteq \:
%     \Om_1 \fubtens \ldots\, \Om_n   \: \subseteq \:
%     \overline{\Om_1 \tens \ldots\, \Om_n}    \: \equiv \:
%     (E_1\tens \ldots\, E_n)' $$
%  so we have a non-degenerate pairing between $E_1\tens \ldots\, E_n$
%  and $\Om_1 \fubtens \ldots\, \Om_n$\@.
\item
  It's easy to see that $\fubtens$ is associative, i.e.
  $$   (\Om_1 \fubtens \Om_2) \fubtens \Om_3
     \:\simeq\: \Om_1 \fubtens \Om_2 \fubtens \Om_3
     \:\simeq\: \Om_1 \fubtens (\Om_2 \fubtens \Om_3) $$
  though we have to take some care with the interpretation:
  besides $\pair{E_i}{\Om_i}$ ($i=1,2,3$) we also need to consider the
  dualities \mbox{$\pair{E_1 \tens E_2}{\Om_1 \fubtens \Om_2}$}\ and
  $\pair{E_2 \tens E_3}{\Om_2 \fubtens \Om_3}$ in order to deal with
  the iterated $\fubtens$-products.
\item
  A weak Fubini tensor product can also be thought of as a space of
  {\em weakly continuous\/} linear operators. In fact we have
  $$ L(E_1,\Om_2) \,\simeq\, \Om_1 \fubtens \Om_2 \,\simeq\, L(E_2,\Om_1),$$
  the identification of any $\om \in \Om_1 \fubtens \Om_2$ with
  operators $S_\om \in L(E_1,\Om_2)$ and $T_\om \in L(E_2,\Om_1)$
  being given by
  $$ \pair{x_2}{S_\om x_1}
               \,=\, \pair{x_1 \tens x_2}{\om} \,=\, \om(x_1,x_2)
                           \,=\, \pair{x_1}{T_\om x_2} $$
  for all $x_1 \in E_1$ and $x_2 \in E_2$\@. Notice $S_\om$ and
  $T_\om$ are each others transpose.
  Within these identifications, the elements of the {\em algebraic\/}
  tensor product $\Om_1\tens\Om_2$ are in 1-1-correspondence
  with the {\em finite rank\/} operators\@; simple tensors
  correspond to rank one operators.
\item
  One should always be aware of the fact that weak Fubini tensor products
  are related to the vector space dualities involved.
\end{remarks_chp}


A nice feature of weak Fubini tensor products is the fact they admit
extension of {\em slice maps,\/} as we will see below;
first we have a closer look at a special case:


\paragraph{Slicing with continuous functionals}
Let $\pair{E}{\Om}$ and $\pair{F}{\Gamma}$ be two vector space dualities
and consider an $x \in E$. Since $E$ identifies with the space $\Om^*$ of
weakly continuous functionals on $\Om$, the element $x$ corresponds to a
functional $\pairdot{x} \equiv f_x \in \Om^*$\@.
When $\phi \in \Om \fubtens \Gamma$ is a separately continuous bilinear
form on $E \times F$, then $\phi(x,\cdot)$ is by assumption a
weakly continuous functional on $F$. Hence $\phi(x,\cdot)$ identifies
with a unique element in $\Gamma$, which will be denoted by $(f_x \fubtens \id)(\phi)$.
Thus we obtain a linear map $f_x \fubtens \id : \Om \fubtens \Gamma \rarr \Gamma$
determined by
\begin{equation} \label{eq:fub_slice_functional}
  \pairM{y}{(f_x \fubtens \id)(\phi)} \:=\: \phi(x,y) \:=\:  \pairM{x \tens y}{\phi}
\end{equation}
for $y\in F$ and $\phi \in \Om \fubtens \Gamma$\@. This map obviously
extends the ordinary algebraic slice map $f_x \tens \id$, and of course
one could also consider slicing \lq from the right\rq\@.
%%% Furthermore it's clear from (\ref{eq:fub_slice_functional})
%%% the obvious Fubini-type property will hold.


\paragraph{General slice maps}
Consider vector space dualities $\pair{E_i}{\Om_i}$ and $\pair{F_i}{\Gamma_i}$,
and linear maps $\Lambda_i: \Om_i \rarr \overline{\Gamma_i\!}\,$ ($i=1,2$).
Assume $\Lambda_2$ to be weakly continuous, i.e.\ having transpose
$\Lambda_2^* : F_2 \rarr E_2$\@.
Notice that the $\Lambda_i$ are allowed to take values in
the weak completions $\overline{\Gamma_i\!}\, \equiv F_i'$ and moreover, that
$\Lambda_1$ is not assumed to be continuous. Now it is easy to see that $$
\Lambda_1 \tens \Lambda_2: \,
      \Om_1 \tens \Om_2 \,\rarr\, \overline{\Gamma_1\!}\, \tens \overline{\Gamma_2\!}\,
      \equiv F_1' \tens F_2' $$
has a canonical extension to the weak Fubini tensor product, say
\begin{equation}
   \label{eq:def_Lam1_fubtens_Lam2}
   \overline{\Lambda_1 \tens}\, \Lambda_2: \, \Om_1 \fubtens \Om_2 \rarr \,
          \overline{\Gamma_1 \tens \Gamma_2\!}\, \equiv(F_1 \tens F_2)',
\end{equation}
such that
$$ \pairM{y_1 \tens y_2}{(\overline{\Lambda_1 \tens}\, \Lambda_2)(\phi)}
     \:=\: \pairM{y_1}{\Lambda_1(\id \fubtens f_x)(\phi)} $$
for $\phi \in \Om_1 \fubtens \Om_2$, $y_1\in F_1$, $y_2\in F_2$,
and $x\equiv\Lambda_2^*(y_2)\in E_2$\@.


\begin{remarks_chp}  \label{rem:fubtens_of_maps}
\item
  Notice we preferred to write
  $\overline{\Lambda_1 \tens}\, \Lambda_2$ rather than
  $\Lambda_1 \fubtens \Lambda_2$. The former notation is intended to
  remind to the fact that $\Lambda_1$ is not assumed to be continuous,
  while the latter is reserved for a \lq proper\rq\ weak Fubini
  tensor product of {\em two\/} continuous linear maps (cf.\ remark iv).
\item
  Of course the roles of $\Lambda_1$ and $\Lambda_2$ can be
  switched---what matters is that at least {\em one\/} of the two maps
  involved is weakly continuous.
\item
  In particular we can take $\pair{E_2}{\Om_2} \equiv \pair{F_2}{\Gamma_2}$ and
  $\Lambda_2\equiv\id$, thus obtaining a slice map
  $ \overline{\Lambda_1 \tens}\, \id :
     \Om_1 \fubtens \Om_2 \rarr \, \overline{\Gamma_1 \tens \Om_2\!}\,$,
  obeying
  \begin{equation}  \label{eq:def_Lam1_fubtens_id}
     \pairM{x \tens y}{(\overline{\Lambda_1 \tens}\, \id)(\phi)}
      \:=\:  \pairM{x}{\Lambda_1(\id \fubtens f_y)(\phi)}
  \end{equation}
  for $\phi \in \Om_1 \fubtens \Om_2$, $x\in F_1$ and $y\in E_2$\@.
  In the special case that $\Lambda_1$ is just a linear {\em functional}
  (i.e.\ if $\Gamma_1 = \CC = F_1$),
  say $\Lambda_1 \equiv g : \Om_1 \rarr \CC$,
  then we obtain a slice map
  $\overline{g \, \tens}\, \id : \Om_1 \fubtens \Om_2 \rarr \overline{\Om_2\!}\,$,
  determined by
  \begin{equation}\label{eq:def:functional_fubtens_id}
    \pairM{y}{(\overline{g \,\tens}\, \id)(\phi)}  \:=\:  g(\id \fubtens f_y)(\phi).
  \end{equation}
\item
  When $\Lambda_1$ and $\Lambda_2$ are actually {\em both\/} weakly
  continuous mappings from $\Om_i$ into $\overline{\Gamma_i\!}\,$
  ($i=1,2$ resp.) then we have a priori two extensions
  $\overline{\Lambda_1 \tens}\, \Lambda_2$ and
  $\Lambda_1 \overline{\tens \,\Lambda_2\!}\,$
  which are in fact equal, since they are both restrictions of
  $$ \overline{\Lambda_1 \tens \Lambda_2\!}\,
     \equiv (\Lambda_1^* \tens \Lambda_2^*)\algtp : \:
     \overline{\Om_1\tens\Om_2\!}\, \,\rarr\, \overline{\Gamma_1 \tens \Gamma_2\!}\,.$$
  If moreover $\Lambda_i(\Om_i) \subseteq \Gamma_i$ ($i=1,2$),
  then also the transposed mappings $\Lambda_i^*: F_i \rarr E_i$ are weakly
  continuous. Only in this case we have a true $\fubtens$ of
  continuous linear maps, being a weakly continuous mapping
  \begin{equation} \label{eq:proper_fubtens}
    \Lambda_1 \fubtens \Lambda_2:\,
         \Om_1 \fubtens \Om_2 \rarr \Gamma_1 \fubtens \Gamma_2 :\,
         \phi \,\mapsto\, \phi\left(\Lambda_1^*(\cdot), \Lambda_2^*(\cdot)\vertM\right)
  \end{equation}
  between the $\fubtens$-products of the spaces involved.
\item
  In view of remark \ref{rem:weak_fubtens}.ii, the action of the map in
  (\ref{eq:proper_fubtens}) can be interpreted in terms of
  {\em composition\/} of continuous linear operators:
  let $\Lambda_i: \Om_i \rarr \Gamma_i$ ($i=1,2$) be weakly continuous,
  hence having transposes $\Lambda_i^*: F_i \rarr E_i$,
  and take any $\om \in \Om_1 \fubtens \Om_2$\@.
  Identifying $\Om_1 \fubtens \Om_2$ and $\Gamma_1 \fubtens \Gamma_2$
  with spaces of continuous operators, we get
  $$ \Lambda_2 S_\om \Lambda_1^*
         \,\simeq\, (\Lambda_1 \fubtens \Lambda_2)(\om)
         \,\simeq\, \Lambda_1 T_\om \Lambda_2^*.         $$
\item
  If $\Lambda_1$ and $\Lambda_2$ are as in (\ref{eq:def_Lam1_fubtens_Lam2}),
  i.e.\ with $\Lambda_2$ being continuous, but now moreover
  $\Lambda_2(\Om_2) \subseteq \Gamma_2$, then it's easy to show
  the following Fubini property:
  $$      (\overline{\Lambda_1 \tens} \,\id)\,(\id \fubtens \Lambda_2)
    \:=\: \overline{\Lambda_1 \tens} \, \Lambda_2
    \:=\: (\overline{\id \tens \Lambda_2\!}\,) \,(\overline{\Lambda_1 \tens}\,\id)$$
\item
  We conclude with a little warning:
  consider linear maps $\Lambda_i: \Om_i \rarr \Gamma_i$ ($i=1,2$),
  {\em none\/} of them weakly continuous. Then according to the
  above it is still possible to construct slice maps on weak
  Fubini tensor products and, though not true in general, it might
  accidentally happen that
  $$ (\overline{\Lambda_1 \tens} \,\id)(\Om_1 \fubtens \Om_2)
            \,\subseteq\, \Gamma_1 \fubtens \Om_2
  \andspace{2em}
     (\id \,\overline{\tens \,\Lambda_2\!}\,)(\Om_1 \fubtens \Om_2)
            \,\subseteq\, \Om_1 \fubtens \Gamma_2, $$
  which would allow us to {\em formulate\/} a Fubini-type property for these maps.
  It does however not imply that such a property actually {\em holds.}
\end{remarks_chp}
