


\subsection{Holomorphic $q$-Hankel transformation}

In the previous section we considered functions living on a discrete space $\RR_q^+$.
Now we shall try to \lq interpolate\rq\ these functions on $\RR_q^+$ by entire functions,
whenever possible.

But first let us recall the  Schwartz-like spaces ${\mathcal S}_r(\RR^+;q)$
of entire functions, as defined in example \ref{ex:Schwartz-like_space}\@.
In case $r=1$, the subscript $r$ will be suppressed in our notation.
Also recall the operators $\Gamma$, $\Phi$, $\Omega$ and $\Psi$ introduced in
\S \ref{par:introducing_shift_and_multiplication_operators}\@.
Let's add one more operator $K: \HC \rarr \HC$, given by $(Kf)(z)=f(z^2)$.
Then the commutation rules $\Omega K = K\Omega^2$ and $\Psi^2 K = K \Psi$ are easily verified.
Also notice that $K$ maps \Swqbis\ into \Swq\@.

Furthermore, let $R_q$ denote the restriction map from \HC\ into
the space of all functions on $\RR_q^+$. Since $\RR_q^+$ admits a
point of accumulation, the identity theorem for holomorphic functions
implies $R_q$ to be injective.
Observe that $R_q$ maps \Swq\ into \Ltwoq\@. The proof of this involves
the fact that entire functions are bounded on e.g.\ the unit interval;
also recall that $\RR_q^+ \equiv (\RR_q^+,m_q)$ was endowed with a particular
measure---{\em not\/} the counting measure.



\begin{abs} \label{abs:holomorphic_qHankel}\rm
Our next aim is to find, for any $m\in\ZZ$, a sufficiently large subspace
$\holS{m}$ of \Swqbis\ together with a linear map $\holH{m}$ from $\holS{m}$
into $\holS{m}$, such that the following diagram commutes:
$$\begin{CD}
  \holS{m} @>\id>>
  \Swqbis  @>{\textstyle K}>>
  \Swq     @>{\textstyle \Psi^{|m|}}>>
  \Swq     @>{\textstyle R_q}>>
  \Ltwoq \\
%%%%%%%%%%%%%%%
  @V{\textstyle\holH{m}}VV @. @. @. @VV{\textstyle H_m}V  \\
%%%%%%%%%%%%%%%
  \holS{m} @>>\id>
  \Swqbis  @>>{\textstyle K}>
  \Swq     @>>{\textstyle \Psi^{|m|}}>
  \Swq     @>>{\textstyle R_q}>
  \Ltwoq
\end{CD}$$
\nopagebreak
\begin{equation}\label{eq:diagram:holomorphic_qHankel}
\diacaption{Holomorphic $q$-Hankel transform}
\end{equation}
The map $\holH{m}$ will then be called the {\em holomorphic\/} $q$-Hankel transform
of \mbox{order $m$}, because it transforms entire functions into entire functions.
However the above scheme involves more than merely the passage to entire functions:
the operators $K$ and $\Psi^{|m|}$ appearing in the diagram will ensure
the system $(\holH{m})_{m\in\ZZ}$ to have the proper behaviour w.r.t.\
$q^2$-differentiation (cf.\ proposition \ref{prop:holqHankel:qdiff}).
Iterating the above diagram immediately reveals that $\holH{m}^2=\id$
(recall that $H_m^2=\id$, \mbox{cf.\ proposition}\ \ref{prop:qHankelSquare_is_id}).
\hfill $\star$
\end{abs}



\begin{defn*} \label{def:Hmpair}
For all $m\in\ZZ$ we define $\auxH{m} : \, \Swqbis \rarr \Ltwoq\,$ by
$$ \auxH{m} \,=\: H_m R_q \Psi^{|m|} K. $$
A pair $(f,g)$ of entire functions is said to be an \Hmpair\ whenever
\begin{enumerate}
\item $f\in \Swqbis$ and $\auxH{m} f = R_q g$
\item $g$ is an even (resp.\ odd) function whenever $m$ is even (resp.\ odd)
\item $g$ has a zero at the origin of order at least $|m|$.
\end{enumerate}
Finally we define $\holS{m}$ to be the following subspace of \Swqbis:
$$ \holS{m} \:= \: \left\{ f\in \Swqbis \:\left|\:
                \begin{array}{l}
                   \mbox{there exists an entire function $g$ such} \\
                   \mbox{that $(f,g)$ is an \Hmpair}
                \end{array}\!\! \right.\right\}$$
\end{defn*}



\begin{remark} \label{rem:not_constructive} \rm
The maps $\auxH{m}$ and the notion of an \Hmpair\ shall have no
role in the eventual picture; however they will turn out to be
very convenient in formulating some intermediate results. Roughly
speaking we want to consider entire functions whose $q$-Hankel
transforms on $\RR_q^+$ admit interpolation by entire functions.
The above approach may seem to be rather descriptive and certainly
not very constructive at this moment, since it does not provide a
criterion to determine whether a function $f$ belongs to
$\holS{m}$ or not---at least not without computing $\auxH{m} f$
first. In this way however, we want to avoid (or at least
postpone) some technical questions which do not have too high
priority from our perspective. Just to give a hint of the kind of
trouble we run into when we actually try to {\em construct\/}
such an \Hmpair\ $(f,g)$, let's write out in detail the equation
$\auxH{m} f = R_q g$ appearing in the above definition:
\begin{equation}\label{eq:remark:trouble:qpower}
 g(q^k) \:=\: (H_m R_q \Psi^{|m|} K f)(q^k)
        \:=\:  \sum_{n\in \ZZ} \: q^{2n} \J{m}{q^{n+k}}\, q^{n|m|} f(q^{2n})
        \hspace{1em}
\end{equation}
for any $k\in\ZZ$. Formally replacing $q^k$ with $z\in\CC$ we obtain
\begin{equation}\label{eq:remark:trouble:entire}
   g(z)  \:=\;  \sum_{n\in \ZZ} \: q^{2n} \J{m}{q^n z}\, q^{n|m|} f(q^{2n})
     \hspace{4em} \mbox{for any $z\in\CC$}. \hspace{1em}
\end{equation}
Now we want (\ref{eq:remark:trouble:entire}) to define an entire
function---so we need the sum to converge uniformly on
compact sets---and that's precisely the point where things become very tricky.
Indeed there turns out to be a huge difference between
(\ref{eq:remark:trouble:qpower}) and (\ref{eq:remark:trouble:entire})
on the technical level. The fact is that our \little\ $q^2$-Bessel
functions are behaving very well as far as we evaluate them in in $q$-powers only,
whereas they are not quite so innocent in the rest of the complex plane.
For instance, the orthogonality relation (\ref{eq:qBessel:qHankel})
reveals that $\J{m}{q^{n+k}}$ tends to zero as $n \rarr -\infty$, but in general,
when $z\in \CC$ is not a $q$-power, $\J{m}{q^n z}$ will grow very
rapidly when $n \rarr -\infty$.

However, the \lq interpolation\rq\ strategy we have chosen for will allow us to
proceed quite far with only {\em partial\/} answers (like for instance lemma
\ref{lemma:compactsupp_in_holS}) to the question of convergence mentioned above,
and therefore avoiding a lot of trouble.
Indeed precisely those results that we are looking for
will come almost for free in our approach, but of course we also pay a price:
some properties one might expect to hold could be hard to prove within our setting
(cf.\ remark \ref{rem:Q:strict_inclusions}).
Let's conclude the present remark by stating that we {\em won't\/} give a full
explicit description of the spaces $\holS{m}$, whereas we {\em will\/}
show---constructively---that they contain plenty of elements.
\hfill $\star$
\end{remark}




\subsection{$q$-Hankel transforms and $q$-differentiation}

Recall the difference operators $\Dqsqr$ and $\nabq{m}$ on \HC\,
emerging in the formulas of proposition \ref{prop:actions:Upsilon}\@.
Notice that \Swqbis\ is not \mbox{$\nabq{m}$-invariant},
though it certainly is \mbox{$\nabq{m} \Omega$-invariant}\@.
Furthermore \Swqbis\ is $\Dqsqr$-invariant,
as was already observed in example \ref{ex:Schwartz-like_space}\@.
Therefore the following makes sense:



\begin{lemma} \label{lemma:auxH:qdiff}
For all\/ $f\in \Swqbis$ and\/ $m,k\in \ZZ$ with\/ $m\geq 1$ we have
\begin{eqnarray}
(\auxH{m} \Dqsqr f)(q^k)
      &=& -\frac{q^{-1}}{q^{-1}-q} \; q^k    \:(\auxH{m-1} \Omega^2 f)(q^k)
       \label{eq:auxH:Dq2:pos}\\
(\auxH{m-1} \nabq{m} \Omega f)(q^k)
      &=& \frac{1}{q^{-1}-q} \; q^{-m}\, q^k \: (\auxH{m} f)(q^k)
       \label{eq:auxH:nabq:pos} \\
(\auxH{-m} \Dqsqr f)(q^k)
      &=&  \frac{q^{-1}}{q^{-1}-q} \; q^k    \: (\auxH{-m+1} f)(q^k)
       \label{eq:auxH:Dq2:neg} \\
(\auxH{-m+1} \nabq{m} \Omega f)(q^k)
      &=& -\frac{1}{q^{-1}-q} \; q^m\: q^k   \: (\auxH{-m} \Omega^2 f)(q^k)
      \hspace{2em}
      \label{eq:auxH:nabq:neg}
\end{eqnarray}
\end{lemma}
\vspace{2ex}

\begin{proof}
Take any $f\in \Swqbis$. Assuming $m \geq 1$, we have for all $k\in \ZZ$
\begin{eqnarray*}
\lefteqn{(\auxH{m} \Dqsqr f)(q^k)}
\\&=&
  (H_m R_q \Psi^{|m|} K \Dqsqr f)(q^k)
\\&=\vertXL&
  \sum_{n\in \ZZ} \: q^{2n} \J{m}{q^{n+k}}\, (R_q \Psi^{|m|} K \Dqsqr f)(q^n)
\\&=&
  \sum_{n\in \ZZ} \: q^{2n} \J{m}{q^{n+k}}\, q^{nm}\,(\Dqsqr f)(q^{2n})
\\&=&
   \sum_{n\in \ZZ} \: q^{2n} \J{m}{q^{n+k}}\, q^{nm}\,
         \frac{f(q^{2n}) - f(q^2 q^{2n})}{(1-q^2) q^{2n}}
\\&\stackrel{(\sharp)}{=}&
  \frac{1}{1-q^2} \left( \sum_{\: n\in \ZZ} \: \J{m}{q^{n+k}}\, q^{nm} \, f(q^{2n})\right.
  \hspace{-7em}
  \raisebox{-5.5ex}{$ \displaystyle
  \left.  -\;\sum_{n\in \ZZ} \: \J{m}{q^{n+k}}\, q^{nm}\, f\!\left(q^{2n+2}\right) \right)
  $}
\\&\stackrel{(*)}{=}&
  \frac{1}{1-q^2}\, \sum_{n\in \ZZ} \:
        \left(q^m \J{m}{q^{n+1+k}} - \J{m}{q^{n+k}}\vertL\right)
        q^{nm} f\!\left(q^{2n+2}\right)
\\&\stackrel{(\ref{eq:qBessel:recurrence1})}{=}&
       -\frac{1}{1-q^2}\, \sum_{n\in \ZZ} \:
         q^{n+k} \J{m-1}{q^{n+k}}\, q^{nm} \, (\Omega^2 f)(q^{2n})
\\&=&
    -\frac{1}{1-q^2}\, q^k \sum_{n\in \ZZ} \:
         q^{2n} \J{m-1}{q^{n+k}} \, q^{n(m-1)}\, (\Omega^2 f)(q^{2n})
\\&=&
    -\frac{q^{-1}}{q^{-1}-q}\, q^k \, (\auxH{m-1} \Omega^2 f)(q^k).
\end{eqnarray*}
$(*)$ relies on replacing $n$ by $n+1$ in only {\em one\/} of the two summations
in the {\sc rhs} of $(\sharp)$. In this respect it is
crucial (!) that both sums in themselves converge
absolutely\footnote{To appreciate this, do consider for a moment the
$m=0$ case which we are excluding in the text above. Indeed, if $m$ were zero, the
sums in the {\sc rhs} of $(\sharp)$ would actually {\em diverge\/}
as $n \rarr +\infty$, unless $f(0)=0$, since $\J{0}{0}\neq 0$.
Nevertheless the {\sc lhs} of $(\sharp)$ would still exist.}. To see this,
consider the following: % observations:
as far as $n \rarr +\infty$ is concerned,
absolute summability follows from the fact that $f$ and $\J{m}{\,\cdot\,}$
are bounded on a neighborhood of the origin, together with the
assumption that $m\geq 1$. To deal with $n \rarr -\infty$, we have to use
that $Kf$ is in the Schwartz-like space \Swq, which roughly means that
$f(q^{2n})= (Kf)(q^n)$ tends to zero very rapidly when $n \rarr -\infty$.
Furthermore we need some bound for $\J{m}{q^{n+k}}$ when $n \rarr -\infty$,
which can easily be obtained from the orthogonality relations
(\ref{eq:qBessel:qHankel}). We conclude that $(\sharp)$ indeed constitutes a
valid operation. In the above computation we also used recurrence relation
(\ref{eq:qBessel:recurrence1}) of proposition \ref{prop:qbessel:properties}\@.
This proves (\ref{eq:auxH:Dq2:pos}).



The proof of (\ref{eq:auxH:nabq:neg}) proceeds similarly:
\begin{eqnarray*}
\lefteqn{\left(\auxH{-m+1} \nabq{m} \Omega f\right)(q^k)}
\\&=&\vertXL
  \left(H_{-m+1} R_q \Psi^{|-m+1|} K \nabq{m} \Omega f\right)(q^k)
\\&=&\vertXL
  \sum_{n\in \ZZ} \: q^{2n} \J{-m+1}{q^{n+k}}\:
             q^{n(m-1)} \left(\nabq{m} \Omega f\right)(q^{2n})
\\&=&
    \frac{1}{q-q^{-1}}\, \sum_{n\in \ZZ} \: q^{2n} \J{-m+1}{q^{n+k}}\, q^{n(m-1)}
           \left( q^m f\left(q^2 q^{2n}\right) -  q^{-m} f(q^{2n})\vertL \right)
\\&=&
    \frac{1}{q-q^{-1}}\, \sum_{n\in \ZZ} \: q^{2n} \, q^{n(m-1)}
        \left(\vertL  q^m  \J{-m+1}{q^{n+k}} \right.
  \hspace{-6em}
  \raisebox{-5.5ex}{$ \displaystyle
    \left. -\: q \J{-m+1}{q^{n+1+k}}\vertL\right) f\!\left(q^{2n+2}\right)
  $}
\\&=&
    \frac{1}{q-q^{-1}} \sum_{n\in \ZZ} \: q^{2n} \, q^{n(m-1)}\, q^m
         \left(\vertL  \J{-m+1}{q^{n+k}}
\right.
  \hspace{-7.2em}
  \raisebox{-5.5ex}{$ \displaystyle
    \left. -\:  q^{-m+1} \J{-m+1}{q^{n+1+k}}\vertL \right)
         (\Omega^2 f)(q^{2n})
  $}
\\&\stackrel{(\ref{eq:qBessel:recurrence1})}{=}&
    \frac{1}{q-q^{-1}}  \sum_{n\in \ZZ} \: q^{2n} \, q^{n(m-1)} \,
          q^m \, q^{n+k}  \J{-m}{q^{n+k}} \, (\Omega^2 f)(q^{2n})
\\&=&
    -\frac{1}{q^{-1}-q}\: q^m \, q^k \, (\auxH{-m} \Omega^2 f)(q^k).
\end{eqnarray*}
Formulas (\ref{eq:auxH:nabq:pos}) and (\ref{eq:auxH:Dq2:neg}) are shown
similarly, this time relying on recurrence relation (\ref{eq:qBessel:recurrence2}).
\end{proof}




\subsection{Construction and basic properties}

We are about to construct the holomorphic $q$-Hankel transform
announced in abstract \ref{abs:holomorphic_qHankel}\@.
We also translate the intermediate results of lemma \ref{lemma:auxH:qdiff}\
into there final form (proposition \ref{prop:holqHankel:qdiff})
and prove some basic properties.



\begin{lemma} \label{lemma:transform_in_schwartz_again}
Take any\/ $m\in\ZZ$. If\/ $(f,g)$ is an \Hmpair, then\/ $g$ belongs to\/ $\Swq$.
\end{lemma}

\begin{proof}
First consider the case $m\geq 0$.
Let $(f,g)$ be an \Hmpair\ and take any $n\in \NN$.
Iterating equation (\ref{eq:auxH:Dq2:pos}) $\,n$ times yields, for all $k\in \ZZ$,
$$ \left(\auxH{m+n} \left(\Dqsqr \Omega^{-2}\right)^n \! f\right)(q^k)
      \:=\:
   \left(-\frac{q^{-1}}{q^{-1}-q}\right)^{\!\! n}  q^{nk} \left(\auxH{m}f\right)(q^k). $$
It follows that for all $n\in \NN$ and $k\in \ZZ$
$$ q^{nk} g(q^k) \:=\: (q^2-1)^n \left(\auxH{m+n} f_n\right)(q^k) $$
where $f_n=(\Dqsqr \Omega^{-2})^n \! f$.
These $f_n$ are still in \Swqbis\ because \Swqbis\ is invariant under
$\Omega^{\pm 2}$ and $\Dqsqr$.
Since $\auxH{m+n}$ maps \Swqbis\ into \Ltwoq, we have
$$ \sum_{k\in\ZZ} \: |q^{nk}g(q^k)|^2\, q^{2k} \:<\, \infty $$
for any $n\in \NN$.
From this one can easily derive that $g$ belongs to \Swq\@.

In case $m$ is negative we proceed analogously, but now using (\ref{eq:auxH:Dq2:neg}).
\end{proof}



\begin{prop} \label{prop:exist:holomorphic_qHankel}
For any\/ $m\in\ZZ$ there exists a unique linear map\/ $\holH{m}$ from\/
$\holS{m}$ into\/ $\holS{m}$, such that diagram (\ref{eq:diagram:holomorphic_qHankel})
in abstract \ref{abs:holomorphic_qHankel}\ commutes.
In particular, if\/ $(f,g)$ is an \Hmpair, then\/ $f\in\holS{m}$ and\/
$\Psi^{|m|} K \holH{m}f =g$. Furthermore we have\/ $\holH{m}^2=\id$.
\end{prop}

\begin{proof}
Uniqueness is obvious (recall that $R_q$ is injective). To prove existence,
take any $m\in\ZZ$ and $f \in\holS{m}$. By definition there exists
an entire function $g$ such that $(f,g)$ is an \Hmpair\@.
Now the function
$$ \CC_0 \rarr \CC : z \mapsto \frac{1}{z^{|m|}}\, g(z)$$
is holomorphic, and has a removable singularity at the origin because
of (iii) in definition \ref{def:Hmpair}\@.
Furthermore this function will always be {\em even\/} because of (ii)
in definition \ref{def:Hmpair}\@.
Hence there exists an entire function, denoted $\holH{m}f$, such that
$$ (\holH{m}f)(z^2)
         \:=\:   \frac{1}{z^{|m|}}\, g(z) \hspace{5em} (z\in \CC_0) $$
or equivalently, $\Psi^{|m|} K \holH{m}f = g$.
Lemma \ref{lemma:transform_in_schwartz_again}\ yields that $g \in \Swq$
and then it follows easily that $\holH{m}f$ belongs to \Swqbis\@.
But we actually want $\holH{m}f\in \holS{m}$, and
in this respect we claim that $(\holH{m}f, \Psi^{|m|} K f)$ is again an \Hmpair\@.
Indeed, items (ii-iii) of definition \ref{def:Hmpair}\ are
obviously fulfilled; to complete (i), observe that
\begin{eqnarray*}
   \auxH{m} \holH{m} f
&=&
   H_m R_q \Psi^{|m|} K \holH{m} f
\\&=&
   H_m R_q g
\\&=&
   H_m \auxH{m} f
\\&=&
   H_m^2 R_q \Psi^{|m|} K f
\\&=&
   R_q \Psi^{|m|} K f.
\end{eqnarray*}
We used that $H_m^2=\id$ (proposition \ref{prop:qHankelSquare_is_id}).
Thus we have constructed a linear map $\holH{m}$ from $\holS{m}$ into
$\holS{m}$ satisfying all our requirements.
\end{proof}




\begin{prop} \label{prop:holS:Omega_invariant}
For any\/ $m\in \ZZ$, the space\/ $\holS{m}$ is\/ $\Omega^{\pm 2}$-invariant and
\begin{equation}\label{eq:holS:Omega_invariant}
  \holH{m} \Omega^2 \;=\; q^{-2(|m|+1)}\, \Omega^{-2} \holH{m}.
\end{equation}
\end{prop}

\begin{proof}
Let $(f,g)$ be any \Hmpair\@; we claim that
$(\Omega^2 f,\, q^{-|m|-2}\Omega^{-1} g)$ is an \Hmpair\ as well.
First observe \Swqbis\ is $\Omega^{\pm 2}$-invariant.
Now from $\Psi^{|m|} K \Omega^2 = q^{-|m|} \Omega \Psi^{|m|} K$
it follows that, for all $k\in \ZZ$,
\begin{eqnarray*}
  (\auxH{m} \Omega^2 f)(q^k)
&=&
  (H_m R_q \Psi^{|m|} K \Omega^2 f)(q^k)
\\&=&
   q^{-|m|}\,(H_m R_q \Omega \Psi^{|m|} K f)(q^k)
\\&=& \vertXL
  q^{-|m|} \sum_{n\in \ZZ} \: q^{2n} \J{m}{q^{n+k}}\,(R_q \Omega \Psi^{|m|} K f)(q^n)
\\&\stackrel{(*)}{=}&
  q^{-|m|} \sum_{n\in \ZZ} \: q^{2(n-1)} \J{m}{q^{n-1+k}}\,(\Psi^{|m|} K f)(q^{n-1} q)
\\&=&
  q^{-|m|-2} \, (\auxH{m} f)(q^{k-1})
\\&=&
  q^{-|m|-2} \, (\Omega^{-1} g)(q^k).
\end{eqnarray*}
In (*) we replaced the summation index $n$ by $n-1$.
This proves item (i) of definition \ref{def:Hmpair}, whereas items (ii-iii)
are obvious here. Proposition \ref{prop:exist:holomorphic_qHankel}\ yields that
$\Omega^2 f$ belongs to $\holS{m}$ and
\begin{eqnarray*}
  \Psi^{|m|} K \holH{m} \Omega^2 f
&=&
  q^{-|m|-2} \: \Omega^{-1} g
\\&=&
  q^{-|m|-2} \: \Omega^{-1} \Psi^{|m|} K \holH{m} f
\\&=&
  q^{-|m|-2} \, q^{-|m|} \: \Psi^{|m|} K \Omega^{-2} \holH{m} f.
\end{eqnarray*}
We used the commutation rules for $\Psi$, $K$, and $\Omega$.
Canceling $\Psi^{|m|} K$ yields the result.
\end{proof}
\vspace{2ex}



Recall that \HC\ is endowed with a $^*$-operation $\til\,$ (cf.\ \S\ref{sec:conventions}).

\begin{prop} \label{prop:qHankel:tilde}
Take any\/ $m\in\ZZ$. The space\/ $\holS{m}$ is\/ $\til$-invariant.
In fact, $\til$ commutes with\/ $\holH{m}$.
\end{prop}

\begin{proof}
Observe that $\til$ commutes with $\Psi$ and $K$.
Furthermore $H_m$ obviously commutes with complex conjugation,
since \little\ $q$-Bessel functions take real values on the real line.
We also have $R_q \tilde{f} = \overline{R_q f}$ for any $f\in\HC$.
It is clear that if $(f,g)$ is an \Hmpair, then so is $(\tilde{f},\tilde{g})$.
The result now follows easily from proposition \ref{prop:exist:holomorphic_qHankel}.
\end{proof}



\begin{prop} \label{prop:qHankel:inverseorder}
Take any\/ $m\in\ZZ$. Then\/ $\holS{-m} = \holS{m}$ and
$$\holH{-m}  \;=\;  (-q)^m q^{m|m|} \:\Omega^{2m} \holH{m}. $$
In this respect we also have two auxiliary results worth mentioning:
\begin{eqnarray}
  (H_{-m} f)(q^k)     &=&  (-q)^m (H_m f)(q^{k+m})
\label{eq:qHankel:inverseorder:aux:1}
\\ \vertL
  (\auxH{-m} g)(q^k)  &=&  (-q)^m (\auxH{m} g)(q^{k+m})
\label{eq:qHankel:inverseorder:aux:2}
\end{eqnarray}
for all\/ $f\in \Ltwoq$, all\/ $g\in \Swqbis$ and any\/ $k\in\ZZ$.
\end{prop}

\begin{proof}
Equation (\ref{eq:qHankel:inverseorder:aux:1}) follows easily from
(\ref{eq:qBessel:inversion}), whereas (\ref{eq:qHankel:inverseorder:aux:2}) is an
immediate consequence of the former.
Now let $(f,g)$ be any \Hmpair\@; then it is easy to see that
$(f, (-q)^m \Omega^m g)$ is a \mbox{$(-m;q)$-Hankel} pair.
Hence according to proposition \ref{prop:exist:holomorphic_qHankel}\
we have $f\in \holS{-m}$ and
\begin{eqnarray*}
   \Psi^{|-m|} K \holH{-m}f
  &=&
   (-q)^m \: \Omega^m g
\\&=&
   (-q)^m \:\Omega^m \Psi^{|m|} K \holH{m}f
\\&=&
   (-q)^m \, q^{m|m|} \: \Psi^{|m|} K \Omega^{2m} \holH{m} f.
\end{eqnarray*}
Canceling $\Psi^{|m|} K$ yields the result.
\end{proof}
\vspace{2ex}

The next proposition will play a key-role henceforth:

\begin{prop} \label{prop:holqHankel:qdiff}
Take any\/ $m\in \NN$ with\/ $m\geq 1$. Then
$$\begin{array}{rcl}
   \Dqsqr \holS{m-1}  &\subseteq&   \holS{m}
\\
   \nabq{m} \Omega \, \holS{m}     &\vertXL\subseteq&   \holS{m-1}
\\
   \Dqsqr \holS{-m+1}             &\vertXL\subseteq&   \holS{-m}
\\
  \nabq{m} \Omega^{-1}  \holS{-m}  &\vertXL\subseteq&   \holS{-m+1}
\end{array}$$
and accordingly:
$$\begin{array}{rcrlcr}
    \holH{m} \Dqsqr f
    &=&
    \displaystyle -\frac{q^{-1}}{q^{-1}-q}  & \holH{m-1} \Omega^2 f
    &\hspace{2em}&
    (f\in \holS{m-1})
\\
    \holH{m-1} \nabq{m} \Omega f
    &=&
    \displaystyle \frac{1}{q^{-1}-q} &   q^{-m}\: \Psi \holH{m}f
    &&
    (f\in \holS{m})
\\
    \holH{-m} \Dqsqr f
    &=&
    \displaystyle  \frac{q^{-1}}{q^{-1}-q} &   \holH{-m+1}f
    &&
    (f\in \holS{-m+1})
\\
    \holH{-m+1} \nabq{m} \Omega^{-1}f
    &=&
    \displaystyle -\frac{1}{q^{-1}-q} &   q^m\: \Psi \holH{-m} f
    &&
    (f\in \holS{-m})
\end{array}$$
\end{prop}
\vspace{2ex}

\begin{proof}
Take any $f\in \holS{m-1}$. Then according to proposition \ref{prop:holS:Omega_invariant},
also $\Omega^2 f$ belongs to $\holS{m-1}$.
Now let $(\Omega^2 f,g)$ be the corresponding \mbox{$(m-1;q)$-Hankel} pair.
With formula (\ref{eq:auxH:Dq2:pos}) of lemma \ref{lemma:auxH:qdiff}\ it follows easily that
$$ \left(\Dqsqr f,  \:  - \frac{q^{-1}}{q^{-1}-q} \,\Psi g \right) $$
is an \Hmpair, and hence $\Dqsqr f$ belongs to $\holS{m}$.
Furthermore, since $m$ and $m-1$ are positive, we have
(cf.\ proposition \ref{prop:exist:holomorphic_qHankel})
$$ \Psi^m K \holH{m} \Dqsqr f
    \;=\;  -\frac{q^{-1}}{q^{-1}-q}\, \Psi g
    \;=\;  -\frac{q^{-1}}{q^{-1}-q}\, \Psi \Psi^{m-1} K \holH{m-1} \Omega^2 f $$
Canceling $\Psi^m K$ yields the first formula. The other formulas
are more or less analogous---let's also have a look at the last one:
take any $f\in \holS{-m}$ and let $(f,g)$ be a \mbox{$(-m;q)$-Hankel} pair.
From (\ref{eq:auxH:nabq:neg}) we obtain that
$$ \left(\nabq{m} \Omega^{-1}f, \; -\frac{1}{q^{-1}-q} \, q^m\, \Psi g \right)$$
is a \mbox{$(-m+1;q)$-Hankel} pair, hence $\nabq{m} \Omega^{-1}f \in \holS{-m+1}$ and
\begin{eqnarray*}
    \Psi^{|-m+1|} K \holH{-m+1} \nabq{m} \Omega^{-1} f
&=&
    -\frac{1}{q^{-1}-q} \, q^m\: \Psi g
\\&=&
    -\frac{1}{q^{-1}-q} \, q^m\:
     \underbrace{\Psi \Psi^{|-m|} K}_{\textstyle \Psi^{m-1} K \Psi} \holH{-m} f
\end{eqnarray*}
Canceling $\Psi^{m-1} K$ yields the last formula.
\end{proof}



\begin{remark} \rm
It is instructive to observe how the combination of
propositions \ref{prop:qHankel:inverseorder}\ and \ref{prop:holS:Omega_invariant}\
transforms the first two formulas of proposition \ref{prop:holqHankel:qdiff}\ into the
remaining two and vice versa. In a similar way
(\ref{eq:qHankel:inverseorder:aux:2}) interchanges
(\ref{eq:auxH:Dq2:pos}-\ref{eq:auxH:Dq2:neg}) as well as
(\ref{eq:auxH:nabq:pos}-\ref{eq:auxH:nabq:neg}).
$\hfill \star$
\end{remark}
