\section{About quantum groups}

Whenever $G$ is a finite group, the space $K(G)$ of complex valued functions on $G$
is a commutative algebra under pointwise operations.
The group structure on $G$ yields a so-called {\em comultiplication\/} on $K(G)$,
which is an algebra homomorphism $\Delta : K(G) \rarr K(G) \tens K(G)$
defined by $(\Delta f)(s,t)=f(st)$ for any $f\in K(G)$ and $s,t \in G$.
This makes sense because $K(G) \tens K(G)$ identifies with the algebra of complex
functions on $G \times G$.
Now the properties of $G$ as a group can be reformulated in terms of this comultiplication.
Thus $K(G)$ becomes a finite dimensional commutative Hopf algebra.
The axioms of Hopf algebra theory, however, do not depend on the commutativity
of the algebra; dropping the commutativity requirement yields the notion of
a finite {\em quantum\/} group.
Such a quantum group can of course only exist on the algebra level---we shall have
to abandon the classical idea of a point-set with some operation upon it.
\vspace{1ex}

Let us return to the group case: there is another way to obtain a
Hopf algebra from a group $G$, considering the group algebra $\CC G$
rather than the function algebra $K(G)$.
Now the Hopf algebras $K(G)$ and $\CC G$ are in a sense {\em dual\/} to each other.
In fact, when $G$ is abelian and $\hat{G}$ denotes its dual group,
then $\CC G$ is actually isomorphic to $K(\hat{G})$ as a Hopf algebra.
Even when $G$ is not abelian, we still have this dual pair of Hopf algebras,
$K(G)$ and $\CC G$. Thus classical Pontryagin duality for finite abelian groups
has been extended to the non-abelian case, provided of course that we allow
quantum groups to enter the picture.
\vspace{1ex}

Unfortunately the above procedure breaks down in case the group $G$ is infinite.
If now $K(G)$ denotes the algebra of functions on $G$ with {\em finite support},
then we still have the identification of $K(G) \tens K(G)$ with $K(G \times G)$.
However $\Delta f$ will have infinite support unless $f$ is zero.
In other words, the comultiplication on $K(G)$ goes outside the algebraic tensor product.
We can get around this problem by allowing $\Delta$ to take values in the
{\em multiplier\/} algebra of $K(G) \tens K(G)$, the latter being
nothing but the algebra of {\em all\/} functions on $G \times G$.
Moreover, the multipliers in the range of $\Delta$ turn out to be \lq covered\rq\
in the sense that $\Delta(f)(1 \tens g)$ and $(f \tens 1)\Delta(g)$
are actually in $K(G) \tens K(G)$ for any $f,g \in K(G)$.
\vspace{1pt}


These observations were the starting point for the theory of multiplier Hopf algebras,
as initiated by A. Van Daele
\cite{Fons:DQG,Fons:MHA,Fons:AFGD,Fons:AFGD:proc}\ and developed further
in collaboration with B. Drabant, J. Kustermans and Y. Zhang
\cite{FonsDra:QD,FonsDraZhang:actions,Kust:Cstar,FonsZhang:DQG,FonsZhang:coactions}\@.
This theory generalizes the notion of a Hopf algebra to the non-unital case and
succesfully incorporates its motivating example of an infinite discrete group.
Moreover it provides a setting for the {\em self-dual\/} category of
regular multiplier Hopf algebras with integrals \cite{Fons:AFGD,Fons:pnas}\@.
The latter category contains all compact \&\ discrete quantum groups \cite{AnnFons,Wor:matrix}\
and admits the construction of a quantum double \cite{FonsDra:QD}\ within the same category.
In \cite{Kust:Cstar,Kust:thesis,Kust:universal}\ it was shown that the \mbox{so-called}\
{\em algebraic\/} quantum groups, i.e.\ multiplier Hopf $^*$-algebras with a positive integral,
can be lifted to the \Cstar-algebra level; in other words, they have a \mbox{\Cstar-envelope}\
fitting into the framework for locally compact quantum groups recently
developed by J. Kustermans and S. Vaes \cite{KV,KV:cr,KV:pnas}\@.
\vspace{1ex}


Although the theory of multiplier Hopf algebras constituted a very substantial
extension of ordinary Hopf algebra theory, there were some indications that even the concept
of a multiplier Hopf algebra was still susceptible to further generalization.
Let us consider e.g.\ the quantum $E(2)$ group,
which is most easily described by a particular dual pair \UqAq\
of Hopf \mbox{$^*$-algebras}\ \cite{Koelink:QE2}\@.
This description however, is not quite satisfying in the sense that it does not admit Haar integrals.
On the \mbox{$C^*$-algebra}\ level, the quantum $E(2)$ group and its dual
were first obtained by S.L. Woronowicz \cite{Wor:QE2,Wor:Affiliated}\@.
Then S. Baaj studied the regular representation and constructed the Haar
weights \cite{Baaj:QE2}\@. Pontryagin duality was investigated by
\mbox{A. Van Daele}\ and S.L. Woronowicz \cite{FonsWor:QE2}\@.
Although the operator-algebraic setting is undoubtedly the most natural,
it is unfortunately also far more complicated than the Hopf algebra picture.
Therefore it would be nice if we could have something in between,
having the technical simplicity of an algebraic approach, yet still incorporating important
features like Haar integrals and Fourier transforms.
\vspace{1ex}


A first step in this direction was taken by \mbox{H.T. Koelink},
who focused on the interplay between quantum groups and $q$-special functions.
In particular \mbox{$q$-Bessel}\ functions have emerged naturally in his study of
the representation theory of quantum $E(2)$.
Also work of L. Vainerman \cite{leonid:vainerman}\ and Vaksman-Korogodski\u{\i}
\cite{VaksKor:QE2}\ indicated a strong relationship between \mbox{$q$-Bessel}\
functions and the quantum $E(2)$ group.
In \cite{Koelink:QE2}\ H.T. Koelink constructed a particular dense subalgebra
$\mathcal{H}$ within the algebraic dual of the Hopf algebra $\Uq$.
What makes this subalgebra so interesting is the fact that,
unlike the Hopf algebra \Aq, the algebra $\mathcal{H}$ does admit a Haar integral.
Now $\mathcal{H}$ turns out to be neither a Hopf algebra or a multiplier Hopf algebra,
the reason being that the comultiplication on $\mathcal{H}$ takes values
outside the multiplier algebra of $\mathcal{H} \tens \mathcal{H}$.
It seems that we need a more general framework here.

\newpage


\section{About this thesis}

Every algebra $E$ acts canonically on its dual $E'$ as follows:
$$  \pair{x}{y\lact \om} \:=\: \pair{xy}{\om} \:=\: \pair{y}{\om\ract x} $$
for $x,y \in E$ and $\om \in E'$.
These actions will play a {\em key-role\/} throughout the whole thesis.
In \mbox{chapter \ref{chapter:Enveloping_algebras}}\ canonical actions will be used to
construct {\em enveloping algebras\/} by a procedure quite similar to the construction
of a multiplier algebra, the main difference being that we now start from the above
actions rather than from multiplication.
This yields the notion of an {\em actor}, or synonymously, a {\em comultiplier}\@.
\vspace{1ex}

The framework developed in \mbox{chapter \ref{chapter:Enveloping_algebras}}\ will
turn out to be indispensable for \mbox{chapter \ref{chapter:Hopf_systems}},
in which we shall introduce the notion of a {\em Hopf System}\@.
The latter is intended to provide a unifying algebraic framework for quantum group
{\em duality}, putting into bigger picture many established notions like e.g.\
locally compact groups,
Hopf algebras
\cite{Schmudgen,Majid,Fons:DPHA},
multiplier Hopf algebras \cite{Fons:MHA}\ with reduced dual,
the duality theory for multiplier Hopf algebras with integrals
\cite{Fons:AFGD,Fons:AFGD:proc,Fons:pnas},
algebraic quantum groups
\cite{Kust:Cstar,Kust:thesis},
etc\footnote{ In this respect we want to draw the reader's attention to
appendix \ref{app:inheritance}, which contains a schematic overview.}\@.
Also the pair $\pair{\Uq}{\mathcal{H}}$ mentioned above will fit in this setting.
\vspace{1ex}

Again canonical actions are playing a crucial role in developing the theory,
because they offer a convenient alternative for the comultiplications
used in Hopf algebra theory. In fact comultiplications become almost {\em obsolete\/}
within our setting, although we will of course consider them from time to time
in order to improve the link with Hopf algebra theory.
\vspace{1ex}

Maybe the most interesting result in \mbox{chapter \ref{chapter:Hopf_systems}}\
is theorem \ref{thm:mhs_yields_mha},
which yields a complete though particularly pleasant {\em characterization\/} of
regular multiplier Hopf algebras with integrals among arbitrary Hopf systems.
\vspace{1ex}

One of the most celebrated results in Hopf algebra theory towards applications
is probably the so-called {\em quantum double\/} construction of Drinfel'd.
At the end of \mbox{chapter \ref{chapter:Hopf_systems}}\ we shall construct
a quantum double within our category of Hopf Systems;
actually the latter category could vaguely be described as the {\em largest\/}
category admitting a quantum double construction on a purely algebraic level.
\vspace{1ex}

Chapter \ref{chapter:algebraic_harmonic_analysis}\ introduces an axiomatic
algebraic framework for harmonic analysis, i.e.\ the study of group duality,
Fourier transformation, Plancherel formulas, etc.
Also here canonical actions are at the very heart of the theory;
in fact the notion of a {\em Fourier transform\/} will be defined in terms of these actions.
\vspace{1ex}

Eventually chapters
\ref{chapter:Fourier_context_for_quantumE2}\ and
\ref{chapter:Harmonic_analysis_on_quantumE2}\
are dealing with the quantum $E(2)$ group, being a concrete example that fits into
the framework of \mbox{chapter \ref{chapter:algebraic_harmonic_analysis}}\@.
The quantum $E(2)$ group is a quantum deformation of the group of Euclidean motions of the plane.
In particular we shall explicitly compute the canonical actions for this example;
once again they seem to be the main issue here.
We believe that it may be interesting to go into a little more detail on this subject:



\section{Harmonic analysis on quantum $E(2)$}

On the Hopf $^*$-algebra level the quantum $E(2)$ group can be described by a
dual pair \UqAq\ where \Aq\ should be thought of as a quantized function algebra,
whereas \Uq\ is a quantized universal enveloping algebra of a Lie algebra.
Here \Uq\ is generated by a self-adjoint invertible element $a$ and a normal
element $b$, satisfying the commutation rule $ab = q \,ba$ for some number $q$ with $0<q<1$.
Similarly \Aq\ is generated by a unitary $\alpha$ and a normal element $\beta$
with $\alpha\beta = q\,\beta\alpha$.
The comultiplications on \Uq\ and \Aq\ are given by
$$ \begin{array}{lcl}
   \Delta(a) = a \tens a
&\hspace{5em}&
   \Delta(\alpha) = \alpha \tens \alpha
\\
   \Delta(b) = a \tens b + b \tens a^{-1}
&&
   \Delta(\beta) = \alpha \tens \beta + \beta \tens \alpha^{-1}
   \vertL
\end{array} $$
and eventually the duality is given by
$$  \pairM{a^p b^r c^s}{\alpha^l \beta^m \gamma^n}
        \;=\;
    \delta_{r,m}\,  \delta_{s,n} \: q^{\frac{1}{2}p(l+m-n)} \,
                         q^{\frac{1}{2}l(m+n)} \,\qfac{m}\, \qfac{n} $$
where $c=b^*$ and $\gamma=-q^{-1}\beta^*$ and $\qfac{\,\cdot\,}$ denotes a $q^2$-factorial
(cf.\ \mbox{appendix \ref{app:qcalc}}).
The problem with these two Hopf algebras is that they do not admit Haar functionals:
indeed on this level we are dealing with {\em polynomial\/} functions in the generators,
whereas to have Haar integrals we would rather need something
like functions {\em tending to zero\/} at infinity. The latter however involve a more
sophisticated functional calculus, which can be obtained in several ways.
One way is to construct a representation on Hilbert space and use operator theory
\cite{Fons:spectral_conditions,FonsWor:QE2,Wor:QE2,Wor:Affiliated,Wor:operatoreq}\@.
Another possibility is to use a holomorphic calculus based on {\em power series}
\cite{Koelink:thesis,Koelink:QE2}\@.
We shall follow the second approach to give a precise meaning to expressions like
$$ \begin{array}{lcr}
    f(\ln a)\, g(b^* b)\: b^m  &  \hspace{3em} \mbox{Fourier} \hspace{3em}&
       \alpha^k  \gamma^n \:  h(\gamma^*\gamma) \\
    f(\ln a)\, g(b^* b)\: c^m  &
      \stackrel{\displaystyle \leftrightarrows}{\mbox{transforms}} &
           \alpha^k  (\gamma^*)^n  \: h(\gamma^*\gamma)
   \end{array}$$
\begin{equation}\label{eq:diagram:Fourier}
\diacaption{Functions in the generators and Fourier transforms between them.}
\end{equation}
where $f$, $g$ and $h$ run through suitable function spaces.
On such elements the left Haar integrals are given by
\begin{eqnarray*}
  \varphi \!\left(  f(\ln a)\, g(b^* b)\: b^m  \vertM\right)
     &=& \delta_{m,0} \!\!\! \sum_{\stackrel{\scriptstyle k,l\,\in\, \ZZ}{\scriptstyle k-l\: {\rm even}}}
            f(k \theta) \, g(\tau q^l) \, q^{k+l}  \\
  \omega \!\left(\alpha^k \gamma^n \:  h(\gamma^*\gamma)\vertM\right)
           &=& \delta_{k,0}\:\delta_{n,0} \: \sum_{j\in\ZZ} \, h(\nu q^{2j})\, q^{2j}
\end{eqnarray*}
provided of course that $f$, $g$ and $h$ satisfy appropriate summability conditions.
Here $\tau$ and $\nu$ are arbitrary positive numbers and \mbox{$\theta = -\frac{1}{2} \ln q$}\@.
These Haar integrals then turn out to be positive, faithful, {\sc kms}, etc.
\vspace{1ex}

Our main objective now, is to construct the {\em Fourier transforms\/} which
transform elements in the left column of diagram (\ref{eq:diagram:Fourier})
into linear combinations of elements in the right column and vice versa.
Although the following formula is not very
precise\footnote{The formula is not completely exact only in the sense that it does
not acknowledge the {\em spectral conditions\/} and their
implications for the summability conditions on $f$ and $g$.},
it may give a good idea of how such a Fourier transform looks like:
$$ \begin{array}{c}
    f(\ln a)\, g(b^* b)\, b^m
\\
\downarrow\vertXL
\\  \displaystyle
    \sum_{k\in\ZZ} \:\: (-1)^m q^{-m} q^{\frac{1}{2}m(k-1)} (q^{-1}-q)^m \:
       f(k \theta) \;  \alpha^{k+m} \, \gamma^m \;  h_{m,k}(\gamma^*\gamma).\vertXL
\end{array}$$
Here $h_{m,k}$ is essentially an $m$-th order $q$-Hankel transform of $g$.
Explicitly we have
$$ h_{m,k}(\nu q^{2r})
       \;=\; \sum_{n\in\ZZ} \: q^{2n} \, q^{m(n-r)}\, \J{m}{q^{n+r}} \, g(\tau q^{2n+k}) $$
with $\J{m}{\,\cdot\;}$ being the \little\ \mbox{$q^2$-Bessel}\ function of order $m$,
given by
$$ \J{m}{z} \:=\; \sum_{k=0}^{\infty} \:  \frac{(-1)^k\,q^{k(k-1)}\,
     q^{2k}\,(q^{2(k+m+1)}; q^2)_\infty}{(q^2;q^2)_\infty  \,(q^2;q^2)_k} \; z^{2k+m} $$
where $(x;q^2)_k = \prod_{j=0}^{k-1} (1-q^{2j} x)$ is some $q$-shifted factorial.
It is instructive to observe the analogy with classical Hankel transforms,
the latter being defined as integral transforms with Bessel function kernels.
\vspace{1ex}

So basically this quantum $E(2)$ Fourier transform amounts to a $q$-Hankel transformation
between the functions $g$ and $h$, which is not completely surprising because
Hankel transformation typically appears when Fourier transformation on the plane
is expressed in {\em polar\/} coordinates, whereas our functional calculus in
$(b,c)$ indeed involves some kind of polar decomposition.

Furthermore it turns out that we can only obtain a Fourier transform when the
\lq dilation\rq\ parameters $\tau$ and $\nu$ are related in a specific way, e.g.
$$   \tau = q^{-1}   \andspace{4em}   \nu = (q^{-1}-q)^{-2} $$
which is quite remarkable.
Once these Fourier transforms are established, we shall prove the {\em Plancherel\/} formula.
At this point it becomes essential to use that peculiar summation
range appearing in the formula for the Haar \mbox{functional $\varphi$},
otherwise the Plancherel formula would simply fail.
Roughly speaking this means that $f\tens g$ actually lives on the set
$$ \left\{ \, (k \theta, \tau q^l)   \, \left|\vertM\right. \,
            k,l\in \ZZ \mbox{ with } k-l \mbox{ even} \, \right\}  $$
which reminds us of the {\em spectral conditions\/} in the \mbox{C$^*$}-algebraic
approach \cite{Wor:QE2}\@.
It also means that it will be convenient to unify $f$ and $g$ into a single object,
\mbox{i.e.\ to}\ consider $f\tens g$ as one function in two variables rather than two functions
in one variable.
\vspace{1ex}

Eventually we shall use our formulas for the Fourier transforms to calculate the
{\em duality\/} between functions in the generators of \Uq\ and \Aq\ respectively, e.g.
\begin{eqnarray*}
      \lefteqn{\pairM{\vertL f(\ln a)\, g(b^* b)\, b^m}{\alpha^l (\gamma^*)^n \,h(\gamma^*\gamma)}}
\\
&\vertXL=&
      \delta_{m,n} \: (-1)^m \, q^{-m}\, q^{-\frac{1}{2}m(m+l+1)}\,
                      (q^{-1}-q)^m \,  \nu^m \, q^{ml} \ldots
\\
&\vertXL&
      \ldots  \sum_{r,k\in \ZZ}   q^{(m+2)(k+r)} \, \J{m}{q^{k+r}} \:
              f(-m\theta-l\theta) \: g(\tau\, q^{2k-m-l})  \: h(\nu\, q^{2r+2l}).
\end{eqnarray*}
