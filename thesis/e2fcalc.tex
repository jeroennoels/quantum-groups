

\section{A functional calculus in 2 variables}
\label{par:Uqext:interesting_subalgebras}


\begin{abs_chp}
  We construct subspaces of \Uqext\ spanned by elements which are formally of type
  $f(\ln a)\, g(bc)\, b^m$ and $f(\ln a)\, g(bc)\, c^m$, where
  $f$ and $g$ run trough suitable spaces of functions and $m\in \NN$.
  We compute the actions of \Aq\ on these particular elements;
  idem for the antipode. Eventually we obtain an \ahss\ in the
  sense of definition \ref{def:algebraic_Hopf_system}\@.
\end{abs_chp}


\subsection{Construction}
Recall the conventions concerning entire functions (cf.\ \S \ref{sec:conventions}).

\begin{defn*}  \label{def:Upsilon}
Define a linear mapping $\Upsilon : \FZ \tens \HC \rarr \Uqext$ by
$$  \hspace{5em}
    \Upsilon(f \tens g)_{r,s} \:=\: \delta_{r,s}\, \mu_r(g) f
    \hspace{4em} (r,s \in \NN). $$
\end{defn*}


\begin{remark} \rm
  In the spirit of remark \ref{rem:formal_power_series},
  an element in \Uqext\ of the form $\Upsilon(f \tens g)$
  could be interpreted {\em formally\/} as
  $$ f(\ln a)\, g(c^* c) \;=\; \sum_{n=0}^\infty \mu_n(g)\,f(\ln a)\, b^n c^n.  $$
  $\Upsilon$ could be thought of as a \lq functional calculus\rq\
  in 2 variables:
\end{remark}


\begin{prop} \label{prop:Upsilon:morphism}
  The map\/ $\Upsilon$ is an injective $^*$-algebra morphism.
\end{prop}
\begin{proof}
  Immediate from definition \ref{def:Upsilon}, formulas (\ref{eq:Uqext:def:product}) and
  the rule for multiplying Taylor series.
\end{proof}
\vspace{2ex}


If we identify \Uq\ with its image in \Uqext\ as in lemma \ref{lemma:embedding:Uq},
then the following makes sense:

\begin{lemma} \label{lemma:Upsilon}
For all\/ $f\in\FZ$, $g\in \HC$ and\/ $r,s,m \in \NN$ we have
\begin{eqnarray*}
  \left( \Upsilon(f \tens g)\,b^m  \vertM\right)_{r,s}
            &=& \delta_{r,s+m} \,\mu_s(g) f  \\
  \left( \Upsilon(f \tens g)\,c^m  \vertM\right)_{r,s}
            &=& \delta_{r+m,s} \, \mu_r(g) f
\end{eqnarray*}
\end{lemma}

\begin{proof}
  Combine (\ref{eq:Uqext:def:product}) with lemma
  \ref{lemma:embedding:Uq}\ and definition \ref{def:Upsilon}.
\end{proof}
\vspace{2ex}


With the same techniques one can show the following commutation rules:

\begin{prop} \label{prop:Upsilon:commutation_rules}
For all\/ $f\in\FZ$, $g\in \HC$, $m \in \NN$ and\/ $p\in \ZZ$ we have
\begin{eqnarray*}
   b^m \, \Upsilon(f \tens g) &=& \Upsilon(\Gamma^{-2m} f \tens g)\, b^m \\
   c^m \, \Upsilon(f \tens g) &=& \Upsilon(\Gamma^{2m} f \tens g) \,c^m
\end{eqnarray*}
\begin{equation} \label{eq:Upsilon:commutation_rules}
 \begin{array}{lclcl}
   \Upsilon(f \tens g)\,(bc)^m
         &=&   \Upsilon(f \tens \Psi^m g)
         &=&   (bc)^m\, \Upsilon(f \tens g)  \\
   \Upsilon(f \tens g)\,a^p
         &=&   \Upsilon(\Phi^p f \tens g)
         &=&    a^p\, \Upsilon(f \tens g)
\end{array}
\end{equation}
\end{prop}




\subsection{Actions \&\ antipode}

Besides the well-known $q$-derivative (see appendix \ref{app:qcalc}) we shall
also need the following $q$-difference operator:

\begin{defn*}
For any $m\in\NN$ we define $\nabq{m}: \HC \rarr \HC$ by
\begin{equation} \label{eq:def:nabq}
 \nabq{m} \:=\: \frac{1}{q-q^{-1}}
        \left( q^m\, \Omega  - q^{-m}\, \Omega^{-1} \vertM\right).
\end{equation}
\end{defn*}


\begin{prop}  \label{prop:actions:Upsilon}
For all\/ $f\in\FZ$, $g\in \HC$ and\/ $m \in \NN$ we have
\begin{eqnarray*}
  \alpha \,\lact \, \Upsilon(f \tens g)\, b^m
     &=& q^{-\frac{1}{2}m}\: \Upsilon \!\left(\Gamma f \tens \Omega^{-1} g \vertM\right) b^m
  \\  \vertL
  \alpha \,\lact \, \Upsilon(f \tens g)\, c^m
     &=& q^{-\frac{1}{2}m}\: \Upsilon \!\left(\Gamma f \tens \Omega^{-1} g \vertM\right) c^m
  \\
  \vertL
  \beta \,\lact \, \Upsilon(f \tens g)\, b^m
     &=& q^{-\frac{1}{2}(m-1)}\:
         \Upsilon \!\left(\Phi\Gamma f \tens \nabq{m} g \vertM\right) b^{m-1}
     \hspace{16mm} (m\geq 1)
  \\  \vertL
  \beta \,\lact \, \Upsilon(f \tens g)\, c^m
     &=& q^{\frac{1}{2}(m+1)}\:
         \Upsilon \!\left(\Phi\Gamma f \tens \Omega^{-1} \Dqsqr g \vertM\right) c^{m+1}
  \\
  \vertL
  \gamma \,\lact \, \Upsilon(f \tens g)\, b^m
     &=& q^{-\frac{1}{2}(m+1)}\:
         \Upsilon \!\left(\Phi\Gamma^{-1} f \tens \Omega^{-1} \Dqsqr g \vertM\right) b^{m+1}
  \\ \vertL
  \gamma \,\lact \, \Upsilon(f \tens g)\, c^m
     &=& q^{\frac{1}{2}(m-1)}\:
         \Upsilon \!\left(\Phi\Gamma^{-1} f \tens \nabq{m} g \vertM\right) c^{m-1}
     \hspace{14mm} (m\geq 1)
  %
  \\
  %
  \vertXL
  \Upsilon(f \tens g)\, b^m \, \ract \, \alpha
     &=& q^{\frac{1}{2}m}\: \Upsilon(\Gamma f \tens \Omega g)\, b^m
  \\  \vertL
   \Upsilon(f \tens g)\, c^m \, \ract \,\alpha
     &=& q^{\frac{1}{2}m}\: \Upsilon \!\left(\Gamma f \tens \Omega g \vertM\right) c^m
  \\
  \vertL
  \Upsilon(f \tens g)\, b^m \, \ract \,\beta
     &=& q^{\frac{1}{2}(m-1)}\:
         \Upsilon \!\left(\Phi^{-1}\Gamma f \tens \nabq{m} g \vertM\right) b^{m-1}
     \hspace{14mm} (m\geq 1)
  \\  \vertL
  \Upsilon(f \tens g)\, c^m \, \ract \,\beta
     &=& q^{-\frac{1}{2}(m+1)}\:
         \Upsilon \!\left(\Phi^{-1}\Gamma f \tens \Omega^{-1} \Dqsqr g \vertM\right) c^{m+1}
  \\
  \vertL
  \Upsilon(f \tens g)\, b^m \, \ract \, \gamma
     &=& q^{\frac{1}{2}(m+1)}\:
         \Upsilon \!\left(\Phi^{-1}\Gamma^{-1} f \tens \Omega^{-1} \Dqsqr g \vertM\right) b^{m+1}
  \\ \vertL
  \Upsilon(f \tens g)\, c^m \, \ract \,\gamma
     &=& q^{-\frac{1}{2}(m-1)}\:
         \Upsilon \!\left(\Phi^{-1}\Gamma^{-1} f \tens \nabq{m} g \vertM\right) c^{m-1}
     \hspace{8mm} (m\geq 1)
\end{eqnarray*}
\end{prop}

\begin{proof}
  Combine proposition \ref{prop:actions_on_Uqext}\ with lemma
  \ref{lemma:Upsilon}\@. The calculations are not too short but
  straightforward.
\end{proof}



\begin{prop} \label{prop:UqFG:antipode}
Recall proposition \ref{prop:Uqext:antipode}\@.
For all\/ $f\in\FZ$, $g\in \HC$ and\/ $m \in \NN$ we have
\begin{eqnarray*}
   S\! \left(\Upsilon(f \tens g) \vertM\right)
        &=&     \Upsilon(f^\bullet \tens g) \\
   S\! \left(\Upsilon(f \tens g)\, b^m \vertM\right)
        &=&     (-q)^{-m}\: \Upsilon \!\left(\Gamma^{-2m} (f^\bullet) \tens g \vertM\right) b^m \\
   S\! \left(\Upsilon(f \tens g)\, c^m \vertM\right)
        &=&     (-q)^m \: \Upsilon \!\left(\Gamma^{2m} (f^\bullet) \tens g \vertM\right) c^m   \\
   S^2\! \left(\Upsilon(f \tens g)\, b^m \vertM\right)
        &=&   q^{-2m} \: \Upsilon(f \tens g)\, b^m   \\
   S^2\! \left(\Upsilon(f \tens g)\, c^m \vertM\right) &=&  q^{2m} \: \Upsilon(f \tens g)\, c^m
\end{eqnarray*}
\end{prop}



\subsection{The space \protect\UqT} \label{subsect:spaceUqT}

Whenever \calL\ is a subspace of $\FZ \tens \HC$, we define \UqT\ to be
the following subspace of \Uqext
\begin{equation}\label{eq:def:UqT}
 \UqT \:=\: \mbox{span} \left(\vertM
              \Upsilon(\calL)\, b^\NN  \:\cup\:
              \Upsilon(\calL)\, c^\NN \right).
\end{equation}
From definition \ref{def:Upsilon}\ and lemma \ref{lemma:Upsilon}\
it is clear that \UqT\ is actually a {\em direct sum\/} of linear spaces:
\begin{equation}\label{eq:UqFG:direct_sum}
 \textstyle \left( \bigoplus_{m=1}^\infty  \Upsilon(\calL)\,b^m  \vertL\right)
                  \: \oplus \:    \Upsilon(\calL)   \: \oplus \:
       \left( \bigoplus_{n=1}^\infty  \Upsilon(\calL)\,c^n \vertL\right)
\end{equation}
It follows immediately that any $x\in \UqT$ can be {\em uniquely\/} written as
$$  x \:=\: \sum_{m=1}^\infty \Upsilon(X_m)\, b^m \:+\:
           \Upsilon(X_0) \:+\;
         \sum_{n=1}^\infty \Upsilon(X_{-n})\, c^n    $$
with only finitely many non-zero $X_m \in \calL\;$ ($m\in \ZZ$).
Concerning {\em uniqueness}, observe that e.g.\ $\Upsilon(X)\, b^m = 0$
for some $X \in \calL$ and $m\in\NN$ implies that
$$  \Upsilon\!\left(\vertM(\id \tens \Psi^m)X \right)
          \: \stackrel{(\ref{eq:Upsilon:commutation_rules})}{=} \:
    \Upsilon(X)\,b^m c^m \:=\: 0,  $$
and recall that $\Upsilon$ is injective. Since also $\Psi$ is injective,
we conclude that $X=0$.


\begin{prop} \label{prop:UqFG}
If\/ \calL\ is a self-adjoint subspace of\/ $\FZ \tens \HC$ which is
invariant under\/
$\Gamma \tens \Omega^{\pm 1}$,
$\Gamma^{-1} \tens \Omega^{\pm 1}$,
$\Gamma^{\pm 2} \tens \id$,
$\Phi^{\pm 1} \tens \id$,
$\id \tens \Psi$,
$\id \tens \Dqsqr$
and\/ $^\bullet \tens \id$, then\/ \UqT\ is
\begin{enumerate}
   \item a sub-\Aq-bimodule of\/ \Uqext
   \item a\/ \Uq-bimodule under multiplication within\/ \Uqext
   \item invariant under\/ $S^{\pm 1}$ and $^*$
\end{enumerate}
If moreover\/ \calL\ is an algebra, then so is\/ \UqT.
\end{prop}

\begin{proof}
  Notice $\Omega^{\pm 1}$-invariance in the second factor implies
  \mbox{$\nabq{m}$-invariance}\@.
  The results follow easily from propositions
  \ref{prop:Upsilon:morphism},
  \ref{prop:Upsilon:commutation_rules},
  \ref{prop:actions:Upsilon}\ and
  \ref{prop:UqFG:antipode}\@.
\end{proof}


\begin{cor}  \label{cor:UqFGAq:Hopf_system}
  Under the assumptions of the previous proposition, including the \lq moreover\rq\ part,
  $\left\langle\vertM\right. \! \UqT, \, \underline{\Aq\!} \left.\vertM\right\rangle$
  is an \ahss\footnote{See definition \ref{def:algebraic_Hopf_system}.},
  provided that\/ \UqT\ separates\/ \Aq\ within the duality.
\end{cor}


{\small
\begin{remark} \label{rem:calculus:finetuning}  \rm
It is worth mentioning that the above construction is susceptible
to some further finetuning: instead of starting from a single space \calL,
one could as well consider an indexed family $(\calL_m)_{m\in\ZZ}$
of subspaces of $\FZ \tens \HC$.
Instead of (\ref{eq:UqFG:direct_sum}) we would then use
$$ \Uq\!\left(\vertM(\calL_m)_{m\in\ZZ}\right) \;=\;
   \textstyle \left( \bigoplus_{m=1}^\infty  \Upsilon(\calL_m)\,b^m  \vertL\right)
                  \: \oplus \:    \Upsilon(\calL_0)   \: \oplus \:
       \left( \bigoplus_{n=1}^\infty  \Upsilon(\calL_{-n})\,c^n  \vertL\right). $$
To obtain an analogue of proposition \ref{prop:UqFG}, one should
require the spaces $\calL_m$ to enjoy the same invariance
conditions that were imposed on \calL\ in proposition \ref{prop:UqFG},
except for invariance under $\id \tens \Dqsqr$, which is to be \mbox{replaced by}
$$  (\id \tens \Dqsqr) \calL_m \,\subseteq\,\calL_{m+1}
      \andspace{4em}
    (\id \tens \Dqsqr) \calL_{-m} \,\subseteq\, \calL_{-m-1}   $$
for all $m \geq 0$.
Moreover one should add the condition that
$$  (\id \tens \nabq{m}) \calL_m  \,\subseteq\, \calL_{m-1}
      \andspace{4em}
    (\id \tens \nabq{m}) \calL_{-m} \,\subseteq\, \calL_{-m+1} $$
for all $m \geq 1$.
Self-adjointness of \calL\ could be replaced by a
condition like $\calL_{-m}^* = \calL_m$ for any $m \in \ZZ$.
Furthermore, to reproduce item (ii) of proposition \ref{prop:UqFG},
we would need some nesting properties like
$$ (\id \tens \Psi) \calL_{m+1} \: \subseteq\: \calL_m  \:\subseteq \: \calL_{m+1} $$
for $m \geq 0$, and similarly for negative indices.
Eventually, if we want $\Uq\!\left(\vertM(\calL_m)_{m\in\ZZ}\right)$
to be an {\em algebra\/}, then we would need something like
$\calL_m \calL_n \subseteq \calL_{m+n}$
for any $m,n\in \ZZ$, rather than requiring all the $\calL_m$ to
be algebras in themselves; so one could say we are dealing with some
$\ZZ$-graded structure here.
It is very likely that all the results in the following sections can be adapted to this
$\ZZ$-graded approach, yielding a theory which would be richer, but unfortunately
also more complicated.
Therefore we shall not elaborate on this point an stick to our original approach.
Nevertheless it might be worth doing the exercise, especially in view of
remark \ref{rem:qHankel:finetuning}\ in chapter \ref{chapter:Harmonic_analysis_on_quantumE2}\@.
\hfill $\star$
\end{remark}
}
