

\section{Haar functionals on \protect\Aqext}
\label{par:Haar_functionals_on_Aqext}


\begin{abs_chp}
The subject of the present section has already been studied by \mbox{H.T.}\ Koelink
\cite{Koelink:thesis,Koelink:QE2}\@. Some of his results have been included
below to fix the notations, which have been slightly modified to match the
language of the previous sections; furthermore the particular choice of function
space\footnote{in \cite{Koelink:thesis,Koelink:QE2} a Schwartz-like space was chosen.}\
will be postponed until chapter \ref{chapter:Harmonic_analysis_on_quantumE2}\@.
In fact the construction of Haar functionals in the previous sections is {\em dual\/}
to the story of \cite{Koelink:thesis,Koelink:QE2}\ sketched below, and actually both
constructions proceed in a similar way, except on one point:
\cite{Koelink:thesis,Koelink:QE2}\ involves a functional
calculus in one variable, whereas in \S\ref{par:Uqext:interesting_subalgebras}\
we were forced into taking \lq functions\rq\ in two variables.
Our main goal, however, is to study the {\em interplay\/}
between both pictures; at this point, Fourier analysis on the quantum $E(2)$
group will emerge (see \S \ref{par:Preview_on_Fourier_transforms}\ and
chapter \ref{chapter:Harmonic_analysis_on_quantumE2}).
\end{abs_chp}



\subsection{Functional calculus}

In order to host \lq functions\rq\ of the generators of \Aq, we shall
first replace \Aq\ by a larger $^*$-algebra \Aqext\@.
From (\ref{eq:fullpairing}) it is clear that any formal series
\begin{equation} \label{eq:Aqext:formal_series}
   \sum_{\stackrel{\scriptstyle l\in\ZZ}{\rm finitely\, many}}
   \sum_{m=0}^{\infty} \;  \sum_{n=0}^{\infty} \:
     w_{\, l,m,n}\, \alpha^l \beta^m \gamma^n
   \hspace{3em}  (w_{\, l,m,n} \in \kk)
   \hspace{2em}
\end{equation}
can be interpreted in a natural way as a well-defined element in $\Uq'$.
Moreover, the commutation rules between the generators of \Aq\ imply that the
formal product and $^*$ of such formal series is again of this form;
the resulting \mbox{$^*$-algebra}\ will be denoted by \Aqext\@.
Recall that $\Uq'$, being the algebraic dual of a Hopf $^*$-algebra, is a $^*$-algebra as well.
Now it is not so hard to prove that we indeed have $^*$-algebra embeddings
$\Aq \,\hookrightarrow \,\Aqext\, \hookrightarrow\, \Uq'$.
Furthermore, also the antipode on \Aq\ extends to a map $S: \Aqext \rarr \Aqext \,$
in a straightforward manner.
\vspace{2ex}



The space of elements of type (\ref{eq:Aqext:formal_series}) with
$w_{\, l,m,n}=0$ whenever $l+m+n$ is odd obviously constitutes a subalgebra of \Aqext\@.
This subalgebra of \lq even\rq\ elements in \Aqext\ shall be denoted by \Aqeven\@.
In \cite{Koelink:thesis,Koelink:QE2}\ it was explained that \Aqext\ should in fact
be considered as a quantized algebra of functions on a {\em double cover\/}
of the $E(2)$ group, whereas \Aqeven\ corresponds to the actual quantum $E(2)$ group.



\begin{defn*}
Whenever \calG\ is a subspace of \HC\ we define the following subspaces of $\Aqext\,$:
\begin{eqnarray*}
\AqG &=&
     \mbox{span} \left\{ \!\! \left.\begin{array}{l}
            \alpha^l \gamma^m \,g(\gamma^*\gamma)   \\ \vertL
            \alpha^l \,(\gamma^*)^m \,g(\gamma^*\gamma)
            \end{array}
            \right|   \begin{array}{l}
            l\in\ZZ, \; m\in\NN, \\
            g\in\calG  \end{array} \! \right\}
\\ \vertXL
\Aqeven(\calG) &=& \AqG \, \cap \, \Aqeven.
\end{eqnarray*}
\end{defn*}



Here, of course, $g(\gamma^*\gamma)$ is merely a transparent way to write the
formal series
\begin{equation} \label{eq:functions_of_gamma*gamma}
 g(\gamma^*\gamma)
    \:=\:  \sum_{n=0}^{\infty} \: \mu_n(g) \:(\gamma^*\gamma)^n
    \:=\:  \sum_{n=0}^{\infty} \: (-q^{-1})^n\: \mu_n(g) \:\beta^n \gamma^n
\end{equation}
which is indeed of type (\ref{eq:Aqext:formal_series}) and obviously the
\lq functional calculus\rq\
$$\HC \rarr \Aqext \: :\: g \mapsto\, g(\gamma^*\gamma)$$
is an injective $^*$-algebra morphism. Furthermore it is straightforward to show



\begin{lemma}  \label{lem:functional_calc:Aq:properties}
For all\/ $g\in\calG$, $l\in\ZZ$ and\/ $m\in\NN$,
  $$  \alpha^l \, g(\gamma^*\gamma)
        \:=\: (\Omega^{2l} g)(\gamma^*\gamma) \:\alpha^l  $$
  $$   (\gamma^*\gamma)^m \, g(\gamma^*\gamma)
        \:=\: (\Psi^m g)(\gamma^*\gamma) \:=\:
        g(\gamma^*\gamma)\, (\gamma^*\gamma)^m. $$
  $$   g(\gamma^*\gamma) \mbox{ commutes with\/ } \gamma \mbox{ and\/ } \gamma^*. $$
\end{lemma}




\begin{lemma} \label{lem:AqG:direct_sum}
$\:\AqG$ can be considered as a direct sum:
$$ \textstyle \bigoplus_{l\in\ZZ}\:
  \left\{ \left( \bigoplus_{m=1}^\infty
     \alpha^l \gamma^m \,\calG(\gamma^*\gamma) \vertL \right)
            \;\bigoplus\;
  \alpha^l \,\calG(\gamma^*\gamma)
           \;\bigoplus\;
  \left(  \bigoplus_{m=1}^\infty  \alpha^l (\gamma^*)^m \,\calG(\gamma^*\gamma)
            \vertL \right)\!\!  \vertXL\right\}. $$
\end{lemma}
\begin{proof}
This is clear from (\ref{eq:fullpairing}) and (\ref{eq:functions_of_gamma*gamma}).
\end{proof}



\begin{cor} \label{cor:Aqeven:description} \rm
Using this direct sum structure and (\ref{eq:functions_of_gamma*gamma}) it follows that
$$\Aqeven(\calG)
   \;=\; \mbox{\rm span}\left\{ \!\! \left.\begin{array}{l}
            \alpha^l \gamma^m \,g(\gamma^*\gamma)   \\ \vertL
            \alpha^l \,(\gamma^*)^m \,g(\gamma^*\gamma)
            \end{array}
            \right| \begin{array}{l}
            l\in\ZZ, \; m\in\NN\: \mbox{ \rm with $l+m$ even,}  \\
            g\in\calG  \end{array}  \! \right\}.  $$
\end{cor}



The following summarizes some results which were already present in
\cite{Koelink:thesis,Koelink:QE2}:

\begin{prop} \label{prop:AqG}
If\/ \calG\ is a non-trivial self-adjoint subspace of\/ \HC\
which is invariant under\/ $\Omega^{\pm 2}$, $\Psi$ and\/ $\Dqsqr$,
then\/ \AqG\ is
 \begin{enumerate}
   \item a sub-\Uq-bimodule of\/ $\Uq'$ under canonical actions (\ref{eq:def:canonical_actions})
   \item an\/ \Aq-bimodule under multiplication within\/ \Aqext
   \item invariant under\/ $S^{\pm 1}$ and $^*$
   \item weakly dense in\/ $\Uq'$ (i.e.\ separates\/ \Uq\ within the duality).
 \end{enumerate}
If moreover\/ \calG\ is an algebra, then so is\/ \AqG.
\end{prop}

\begin{proof}
(i) is shown through explicit calculation \cite{Koelink:thesis,Koelink:QE2}\ of the
actions of \Uq\ on \AqG, involving $\{\Omega^{\pm 2}, \Dqsqr\}$-invariance of \calG\@.
The proof of (ii-iii) relies on lemma \ref{lem:functional_calc:Aq:properties}\@.
(ii) requires $\{\Omega^{\pm 2},\Psi\}$-invariance, whereas (iii) only involves
self-adjointness and $\{\Omega^{\pm 2}\}$-invariance.
(iv) is similar to lemma \ref{lemma:UqTG_separates_Aq}\@.
The \lq moreover\rq\ part follows from the fact that our functional calculus is a
$^*$-algebra morphism and involves $\{\Omega^{\pm 2},\Psi\}$-invariance of \calG\@.
\end{proof}



\begin{cor} \label{cor:UqAqG:Hopf_system}
  Under the assumptions of the previous proposition, including the \lq moreover\rq\ part,
  $\left\langle\vertM\right. \! \underline{\Uq}\, , \, \AqG   \left.\vertM\right\rangle$
  is an \ahss\footnote{See definition \ref{def:algebraic_Hopf_system}.}\@.
\end{cor}



\paragraph{The antipode}
Under the assumptions of proposition \ref{prop:AqG}, the antipode\/ $S$ on\/ \AqG\
is actually a bijection. Furthermore it is straightforward to prove that
\begin{eqnarray*}
S \!\left(\alpha^l \gamma^m \,g(\gamma^*\gamma) \vertM \right) &=& (-q)^m \,
q^{ml} \; \alpha^{-l} \gamma^m \,(\Omega^{2l} g)(\gamma^*\gamma)
\\
S \!\left(\alpha^l (\gamma^*)^m \,g(\gamma^*\gamma) \vertM \right) &=& (-q)^{-m}
\, q^{ml} \; \alpha^{-l} (\gamma^*)^m \,(\Omega^{2l} g)(\gamma^*\gamma)
\\
S^2 \!\left(\alpha^l \gamma^m \,g(\gamma^*\gamma) \vertM \right) &=& q^{2m} \;
\alpha^l \gamma^m \,g(\gamma^*\gamma)
\\
S^2 \!\left(\alpha^l (\gamma^*)^m \,g(\gamma^*\gamma) \vertM \right) &=& q^{-2m}
\; \alpha^l (\gamma^*)^m \,g(\gamma^*\gamma)
\end{eqnarray*}


\subsection{The Haar functional}
\label{par:Haar:Aq:construction}

Throughout this paragraph we fix a real number $\nu$ with $\nu >0$. Furthermore,
let $\calG_\nu$ be any sub{\em algebra\/} of \HC\ satisfying assumptions \ref{assume:Gr}\@.

\begin{defn*}
In view of lemma \ref{lem:AqG:direct_sum}\ it is possible to
define a functional $\omega$ on $\Aq(\calG_\nu)$ as follows:
for any $l\in \ZZ$, $m\in\NN$ and $g\in \calG_\nu$, we set
\begin{equation}\label{eq:def:Haar:AqG}
  \begin{array}{ll}
    \left. \begin{array}{l}
      \omega \!\left(\alpha^l \gamma^m \,g(\gamma^*\gamma) \right) = 0 \\
                       \vertXL
      \omega \!\left(\alpha^l (\gamma^*)^m \,g(\gamma^*\gamma) \right) = 0
    \end{array} \right\}  & \mbox{if either $l\neq 0$ or $m\neq 0$} \\
    \;\: \omega \!\left(g(\gamma^*\gamma)\vertM \right) \vertXXL
           \,=\: \displaystyle \sum_{k\in\ZZ} \, g(\nu q^{2k})\, q^{2k}
           \hspace{4mm} & \mbox{otherwise.}
  \end{array}
\end{equation}
\end{defn*}

%%%%%%(in \cite{Koelink:thesis,Koelink:QE2}\ $\omega$ was denoted by $h$.)


\begin{prop} \label{prop:haar:Aq}
The functional\/ $\omega$ as defined above is both left and right
invariant (see definition \ref{def:invariant_functional}\ and
corollary \ref{cor:UqAqG:Hopf_system}).
Moreover, $\omega$ is hermitian, positive and faithful.
\end{prop}


\begin{prop} \label{prop:Aq:KMS}
$\;\omega$ is weakly {\scriptsize KMS}\@.
The {\scriptsize KMS}-automorphism\/ $\sigma_\omega$ is given by\/
$\sigma_\omega = a^{-2} \lact (\,\cdot\,) \ract a^{-2}$.
Explicitly, for\/ $g\in \calG_\nu$, $l\in \ZZ$ and\/ $m \in \NN$ we have:
$$\sigma_\omega \!\left( \alpha^l \,(\gamma \mbox{ resp.\ } \gamma^*)^m
                   \,g(\gamma^*\gamma)  \vertL \right)
 \;=\;   q^{-2l} \: \alpha^l \, (\gamma \mbox{ resp.\ } \gamma^*)^m
                 \,g(\gamma^*\gamma). $$
Moreover\/ $\omega$ is\/ $\Aq$-{\scriptsize KMS}\ in the sense that\/
$\omega(y \eta) \,=\, \omega(\eta \, \sigma_\omega(y))$ for any\/ $\eta \in \Aq$
and\/ $y \in \Aq(\calG_\nu)$. Furthermore we have\/ $\omega S = \omega$,
and hence the modular element associated with\/ $\omega$ is trivial.
\end{prop}


\begin{proof}
The {\sc kms}-automorphism was already computed in \cite{Koelink:thesis,Koelink:QE2}\@.
Let's have a closer look at the $\Aq$-{\sc kms}\ property.
We have to consider $\eta = \alpha^p \beta^r \gamma^s$ and
$y = \alpha^l \, (\gamma \mbox{ or } \gamma^*)^m \, g(\gamma^*\gamma)$
for any $g\in \calG_\nu$, $l,p \in \ZZ$ and $m,r,s \in \NN$.
We have e.g.
\begin{eqnarray*}
\lefteqn{\omega \!\left( \alpha^l \gamma^m  g(\gamma^*\gamma) \;
            \alpha^p \beta^r \gamma^s \vertL \right)} \\
&\hspace{3em}=& \omega \!\left( \alpha^l \gamma^m \alpha^p
\,(\Omega^{-2p}g)(\gamma^*\gamma)\,
             \beta^r \gamma^s \vertL \right)  \\
&\hspace{3em}=& q^{-pm} \: \omega \!\left( \alpha^{l+p} \beta^r \gamma^{m+s}
           \,(\Omega^{-2p}g)(\gamma^*\gamma) \vertL \right)  \\
&\hspace{3em}=& \delta_{l+p,0} \: \delta_{r,m+s} \: q^{-pm} \: \omega \!\left((-q
\gamma^* \gamma)^r
           \,(\Omega^{-2p}g)(\gamma^*\gamma) \vertL \right)  \\
&\hspace{3em}=& \delta_{l+p,0} \: \delta_{r,m+s} \: q^{-pm} \: (-q)^r \:
\omega \!\left(
           (\Psi^r \Omega^{-2p}g)(\gamma^*\gamma) \vertM \right)  \\
&\hspace{3em}=& \delta_{l+p,0} \: \delta_{r,m+s} \: q^{-pm} \: (-q)^r \:
q^{2pr}\: \omega \!\left(
           (\Omega^{-2p}\Psi^r g)(\gamma^*\gamma) \vertM \right)  \\
&\hspace{3em}\stackrel{(*)}{=}& \delta_{l+p,0} \: \delta_{r,m+s} \: q^{-pm} \:
(-q)^r \: q^{2pr}\:
        q^{2p}\: \omega \!\left((\Psi^r g)(\gamma^*\gamma) \vertM \right)  \\
&\hspace{3em}=& \delta_{l+p,0} \: \delta_{r,m+s} \: q^{-pm} \:  q^{2pr}\:
        q^{2p}\: \omega \!\left( (-q \gamma^* \gamma)^r\, g(\gamma^*\gamma) \vertM \right)  \\
&\hspace{3em}=& \delta_{l+p,0} \: \delta_{r,m+s} \: q^{-(-l)(r-s)} \:
q^{2(-l)r}\:
        q^{2(-l)}\: \omega \!\left( \beta^r \gamma^{m+s} \, g(\gamma^*\gamma) \vertM \right)  \\
&\hspace{3em}=& \delta_{l+p,0} \:  q^{-l(r+s)}\: q^{-2l}\:
        \omega \!\left( \beta^r \gamma^{m+s} \, g(\gamma^*\gamma) \vertM \right)  \\
&\hspace{3em}=& q^{-l(r+s)}\: q^{-2l}\: \omega \!\left( \alpha^{l+p} \beta^r
        \gamma^s \, \gamma^m   \, g(\gamma^*\gamma)   \vertL \right)  \\
&\hspace{3em}=&
 q^{-2l}\: \omega \!\left( \alpha^p \beta^r \gamma^s \alpha^l
             \gamma^m \, g(\gamma^*\gamma)  \vertL \right)  \\
&\hspace{3em}=&
  \omega \!\left( \alpha^p \beta^r \gamma^s  \: \sigma_\omega \! \left( \alpha^l
               \gamma^m \, g(\gamma^*\gamma) \vertM \right) \! \vertL \right).
\end{eqnarray*}
In ($*$) we used the following property of the Haar functional:
$$ \omega \!\left( (\Omega^{2n} g)(\gamma^*\gamma) \vertM \right)
     \:=\:  q^{-2n} \, \omega \!\left( g(\gamma^*\gamma) \vertM \right) $$
for $g\in \calG_\nu$ and $n\in\ZZ$.
The last statement (i.e. $\omega S = \omega$) follows immediately from
the formulas for the antipode on $\Aq(\calG_\nu)$.
\end{proof}
