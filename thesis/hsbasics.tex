\section{Invertible dual pairs}


\begin{abs_chp}
We introduce the notion of an \idpa\ and study its basic properties.
All the \contexts\ involved turn out to be weakly unital; this will be an
important technical advantage throughout the development of the theory.
We also introduce antipodes and comultiplications, and show that they take
values in the enveloping algebras associated to the appropriate
\contexts\ (cf.\ \S \ref{par:enveloping_algebras}).
\end{abs_chp}


Recall example \ref{exC:introduction}, where we defined the notion of a \dpa\@.
Whenever \pairAB\ is such a dual pair, we have \contexts\
$\Aa \equiv (A; B, \pairing)$ and $\BB \equiv (B; A, \pairing)$.
According to lemma \ref{lem:tens_contexts_def}\ we also have an \context
$$ \BBAA  \,\equiv\,  \left(\vertM  B \tens A; \, A \tens B, \pairing \right). $$
Now observe that the pairing \pairing\ between $A$ and $B$ can be viewed
as a linear functional $\pr : A \tens B \rarr \kk: a \tens b \mapsto \pair{a}{b}$.
Thus the following makes sense:

\begin{defn_sec}  \label{def:invertible_dpa}
A \dpa\ \pairAB\ is called {\em invertible\/} whenever its pairing $\pr$ is
$(\BBAA)$-invertible in the sense of definition \ref{def:EE-invertibility}.
\end{defn_sec}

So if \pairAB\ is an \idpa, then $\pr \in (A \tens B)\adl$ induces a
$(\BBAA)$-invertible actor $\theta(\pr) \equiv (\lp\,,\rp)$.
Here \lp\ and \rp\ are bijective linear mappings from $A \tens B$ onto $A \tens B$,
given by (cf.\ proposition \ref{prop:act_by_fuctionals}.iii)
\begin{equation} \label{eq:def:lamP:rhoP}
 \begin{array}{rcl}
     \pairM{d \tens c}{\lp(a \tens b)} &=& \pairM{a \ract d}{b \ract c}
     \\
     \pairM{d \tens c}{\rp(a \tens b)} &\vertL=& \pairM{d \lact a}{c \lact b}
   \end{array}
\end{equation}
for any $a,c \in A$ and $b,d \in B$.
Moreover $(\lp^{-1},\rp^{-1})$ is again an actor for $\BBAA$.


\begin{remark_sec} \rm
It may be instructive to express this in {\em Sweedler\/} \cite{Sweedler}\ notation:
if $\DeltaA$ denotes for instance the induced
comultiplication\footnote{See \S \ref{par:comultiplication}\ for details.}\ on $A$,
then we formally write $\DeltaA(a) = \sweed{a}{1} \tens \sweed{a}{2}$ etc.
According to remark \ref{rem:theta_and_Delta},
the mappings \lp\ and \rp\ may then be expressed as follows:
$$\begin{array}{rcl}
     \lp(a \tens b)  &=&        \pair{\sweed{a}{2}}{\sweed{b}{2}}
                             \; \sweed{a}{1} \tens \sweed{b}{1}
     \\
     \rp(a \tens b)  &=&  \pair{\sweed{a}{1}}{\sweed{b}{1}}
                             \; \sweed{a}{2} \tens  \sweed{b}{2}.\vertL
   \end{array} $$
Similar formulas exist for the inverses of \lp\ and \rp\
(cf.\ remark \ref{rem:sweedler:inverses:lp:rp}).
\hfill $\star$
\end{remark_sec}



\begin{remark_sec}\label{rem:warning:P_strict_cont}  \rm
One should be well aware of the fact that \stricta\ continuity of $\pr : A \tens B \rarr \kk$
is really part of the {\em assumption\/} here (cf.\ definition \ref{def:EE-invertibility}.i).
Strict$\adl$ continuity of $P$ amounts to the fact that the mappings \lp\ and \rp\
given by (\ref{eq:def:lamP:rhoP}) really go into $A \tens B$ rather than just $(B \tens A)'$.
Invertibility in an \context\ was investigated in
lemmas \ref{lem:invertibility_comm_pi}, \ref{lem:comm_implies_invertible},
\ref{lem:invertibility_in_EnvE}\ and \ref{lem:wu:commutation:invertibility},
though one should be careful with the results from \S\ref{par:wu:extra},
since they all {\em assume\/} the \context\ to be {\em weakly unital\/}
(see also: proposition \ref{prop:idpa:wu}\ below).
\hfill $\star$
\end{remark_sec}


\begin{prop_sec} \label{prop:idpa:wu}
If\/ \pairAB\ is an \idpa, then \Aa\ and\/ \BB\ are weakly unital in the sense of
definition \ref{def:weakly_unital}\@.
\rm The counit for \Aa, being a functional on $B$, will be denoted by $\epsB$.
    Similarly $\epsA$ denotes the counit for \BB.
\end{prop_sec}
\begin{proof}
First we show that $B$ is unital as an $A$-bimodule (\S\ref{sec:conventions})
and vice versa; let us for instance prove that $B=B\ract A$.
So choose any $b\in B$ and also take some $c\in A$ and $d\in B$ such that $\pair{c}{d}=1$.
Since \rp\ is surjective, we can write $c \tens b = \sum_i \,\rp (p_i \tens q_i)$
with $p_i \in A$ and $q_i \in B$. Then for all $a\in A$
\begin{eqnarray*}
\pair{a}{b} &=& \pairM{d \tens a}{c \tens b}
      \\  &=&  \textstyle \sum_i \,  \pairM{d \tens a}{\rp (p_i \tens q_i)}
      \\  &=&  \textstyle \sum_i \,  \pairM{d \lact p_i}{a \lact q_i}
      \\  &=&  \textstyle \sum_i \,  \pairM{(d \lact p_i)\, a}{q_i}
      \\  &=&  \textstyle \sum_i \,  \pairM{a}{q_i \ract (d \lact p_i)}
\end{eqnarray*}
hence $b=\sum_i \, q_i \ract (d \lact p_i)$ belongs to $B\ract A$.
It follows that $A \tens B$ is unital as a $(B\tens A)$-bimodule,
hence according to proposition \ref{prop:weakly_unital}.viii, \BBAA\ is weakly unital.
Now proposition \ref{prop:tensor:wu}\ yields the result.
\end{proof}

\begin{cor_sec} \label{cor:hs:everything_is_wu}
Everything in survey \ref{summ:weakly_unital_ids}\ and \S\ref{par:wu:extra}\
applies not only to the \contexts\ \Aa\ and\/ \BB, but also to all the
tensor products constructed from them, e.g.\/ \BBAA, \AAAA, etc.
\rm We will tacitly adopt all the identifications shown in the diagram in
    survey \ref{summ:weakly_unital_ids}, e.g.\ we will no longer explicitly
    distinguish the functional $P$ from the actor $(\lp\,,\rp)$ unless confusion is likely.
    Similarly we identify e.g.\ the counit $\epsA$ with the actor $1_\BB =(\id_A,\id_A)$, etc.
    Furthermore we will extensively rely on (\ref{eq:pairing_with_act}) and
    (\ref{eq:canonical_actions:extended}),
    and whenever it is allowed, we shall exploit the notational convenience offered by
    (\ref{eq:natural_actions_implemented}) and remark \ref{rem:module_notation_for_Env}\@.
\end{cor_sec}


\begin{defn_sec}  \label{def:antipode}
Let \pairAB\ be any \idpa\@. We define linear mappings $\SA: A\rarr B'$
and $\SB: B\rarr A'$ requiring that
\begin{equation}\label{eq:def:antipodes}
 \pair{\SA(a)}{b} \:=\: \pair{P^{-1}}{a \tens b} \:=\: \pair{a}{\SB(b)}
\end{equation}
for all $a \in A$ and $b \in B$.
Observe (\ref{eq:def:antipodes}) makes sense because also $P^{-1}$ is an actor for \BBAA\
(cf.\ definition \ref{def:invertible_dpa}\ and corollary \ref{cor:hs:everything_is_wu}).
We say that $\SA$ and $\SB$ are the {\em antipodes\/} on $A$ and $B$.
\end{defn_sec}



\begin{remark_sec} \label{rem:sweedler:inverses:lp:rp} \rm
Now also the inverses of \lp\ and \rp\ may be expressed in Sweedler notation:
$$\begin{array}{rcl}
     \lp^{-1}(a \tens b)  &=&        \pair{\SA(\sweed{a}{2})}{\sweed{b}{2}}
                                  \; \sweed{a}{1} \tens \sweed{b}{1}
     \\
     \rp^{-1}(a \tens b)  &=&  \pair{\SA(\sweed{a}{1})}{\sweed{b}{1}}
                                  \; \sweed{a}{2} \tens \sweed{b}{2}. \vertL
   \end{array} $$
\end{remark_sec}



\begin{lemma_sec} \label{lem:commutation:id_tens_lamP}
Let\/ \pairAB\ be an\/ \idpa, and consider also any weakly unital actor context\/
$\EE \equiv \EOP$.
Now take any \mbox{pre-actor}\ $(\alpha,\beta)$ for\/ $\EE \tens \BB$
and consider\/ $\id \tens \lp$ and\/ $\beta \tens \id$ as mappings on\/
$\Om \tens A \tens B$.
Then\/ $\id \tens \lp$ commutes with\/ $\beta \tens \id$.
Similarly\/ $\id \tens \rp$ commutes with\/ $\alpha \tens \id$.
\end{lemma_sec}

\begin{proof}
Take any\/ $a,c\in A$ and\/ $b,d\in B$ and\/ $x\in E$ and\/ $\om\in \Om$.
Then we can write $\lp(a \tens b) = \sum_i \,p_i \tens q_i$ with $p_i \in A$ and $q_i \in B$.
An easy computation (similar to the proof of proposition \ref{prop:idpa:wu})
shows that $\sum_i \, \pair{c}{q_i}\,p_i = (b \ract c) \lact a$, hence
\begin{eqnarray*}
     \pairM{x \tens d \tens c}{(\beta \tens \id)(\id \tens \lp)(\om \tens a \tens b)}
  &=&
    \textstyle \sum_i \, \pairM{x \tens d}{\beta(\om \tens p_i)} \pairM{c}{q_i}
\\&=&
    \pairM{x \tens d}{\beta\left(\vertM \om \tens (b \ract c) \lact a \right)}
\\&=&
    \pairM{x \tens d (b \ract c)}{\beta(\om \tens a)}.
\end{eqnarray*}
To obtain the last equality we used an analogue of lemma \ref{lem:action_of_xtens1}\@.

Now write $\beta(\om \tens a)= \sum_j \,\om_j \tens a_j$ with $\om_j\in\Om$ and  $a_j \in A$.
Then we have
\begin{eqnarray*}
 \pairM{x \tens d \tens c}{(\id \tens \lp)(\beta \tens \id)(\om \tens a \tens b)}
   &=&
      \textstyle \sum_j \, \pairM{x}{\om_j} \pairM{d \tens c}{\lp(a_j \tens b)}
 \\&=&
      \textstyle \sum_j \, \pairM{x}{\om_j} \pairM{a_j \ract d}{b\ract c}
 \\&=&
      \textstyle \sum_j \, \pairM{x \tens d(b\ract c)}{\om_j \tens a_j}
\end{eqnarray*}
and the result follows.
\end{proof}

\begin{cor_sec}
Let\/ \pairAB\ be any\/ \idpa\@.
If\/ $(\alpha,\beta)$ is a pre-actor for\/ \BB, then\/ $\lp$ commutes with\/ $\beta \tens\id$,
hence also\/ $\lp^{-1}$ commutes with\/ $\beta \tens \id$.
Similarly\/ $\rp$ and $\rp^{-1}$ commute with\/ $\alpha \tens \id$.
\end{cor_sec}
\begin{proof}
In the above lemma, let \EE\ be the trivial \context\ $(\kk; \kk, \pairing)$.
\end{proof}



\begin{prop_sec}
Let\/ \pairAB\ be an \idpa\@.
Then the antipodes on $A$ and\/ $B$ can be viewed as linear mappings
$$ \SA : A \rarr \Env(\Aa)     \hspace{9em}
   \SB : B \rarr \Env(\BB). $$
In fact, in the sense of notation \ref{not:slicing_on_Act}, we have
for any $a \in A$ and $b \in B$ that
\begin{equation}\label{eq:antipodes:slice:Pinverse}
    \SA(a) \,=\,  (f_a \slice \, \id)(P^{-1})
                     \hspace{5em}
    \SB(b) \,=\,  (\id\, \slice f_b)(P^{-1}).
\end{equation}
\end{prop_sec}

\begin{proof}
Combine the above corollary with propositions
\ref{prop:slicing_on_Act}\ and \ref{prop:slices_in_Env}\@.
\end{proof}


\begin{notation_sec} \rm
We assume the reader to be more or less familiar with the {\em leg-numbering\/} notation
\cite{BS}\ for tensor products, e.g.\ if we consider $\BB \tens \BB \tens \Aa$,
then by $P_{23}$ we mean the actor $\Phi(1_\BB \tens P)$
with $\Phi$ as in lemma \ref{lem:tens_embedding}, etc.
Further\-more we occasionally appeal to proposition \ref{prop:tensor_embedding_Env},
although usually we will suppress the map $\Phi$ in our notation.
\hfill $\bullet$
\end{notation_sec}



\begin{lemma_sec} \label{lem:P13P23actor}
Let\/ \pairAB\ be any \idpa\@.
Then\/ $P_{12} P_{13}$ is an actor for\/ $\BB \tens \Aa \tens \Aa$.
Similarly\/ $P_{13} P_{23}$ is an actor for\/ $\BB \tens \BB \tens \Aa$.
\end{lemma_sec}

\begin{proof}
Use lemma \ref{lem:commutation:id_tens_lamP}\ with $\EE=\Aa$ and
$(\alpha,\beta) = \flip(P)$ where $\flip$ is the obvious flip map from
$\Act(\BBAA)$ into $\Act(\AABB)$.
Rearranging the \lq legs\rq\ a little bit, we conclude that
$(\lp)_{13}$ and $(\rp)_{12}$ commute.
Now invoke lemma \ref{lem:comm_implies_prod_in_act}.
\end{proof}
\vspace{2ex}



\paragraph{Comultiplications}
Before we enter this subject, we like to emphasize that comultiplications
are obsolete in our approach. However we shall pay some attention to them,
thus improving the link with Hopf algebra theory.

Whenever \pairAB\ is an \idpa, we have weakly unital \contexts\ \Aa\ and \BB, and therefore
also weak comultiplications (\S\ref{par:comultiplication}, \S\ref{app:fubini})
$$  \DeltaA : A \rarr A \fubtens A
                 \andspace{4em}
    \DeltaB : B \rarr B \fubtens B.  $$
Recall that $A \fubtens A$ and $\Act(\AAAA)$ are both contained in $(B \tens B)'$.
Therefore the following makes sense:

\begin{prop_sec} \label{prop:comultiplications_in_Env}
Let\/ \pairAB\ be any \idpa\@.
Then the comultiplications on $A$ and\/ $B$ can be viewed as linear mappings
$$ \DeltaA : A \rarr \Env(\AAAA)     \hspace{7em}
   \DeltaB : B \rarr \Env(\BBBB). $$
In fact, in the sense of notation \ref{not:slicing_on_Act}, we have
for any\/ $a \in A$ and\/ $b \in B$ that
\begin{equation}\label{eq:comultiplications:slice:Pinverse}
    \DeltaA(a) \,=\,  (f_a \slice \, \id)(P_{12} P_{13})
                     \hspace{4em}
    \DeltaB(b) \,=\,  (\id\, \slice f_b)(P_{13} P_{23}).
\end{equation}
\rm Observe that these formulas make sense because of lemma \ref{lem:P13P23actor}.
\end{prop_sec}

\begin{proof}
Let's prove the second formula. Take any $b\in B$ and $a,c\in A$, and write
$\lp(c \tens b) = \sum_i \,p_i \tens q_i$ with $p_i \in A$ and $q_i \in B$.
An easy computation using (\ref{eq:def:lamP:rhoP}) shows that
$\sum_i \, \pair{a}{q_i}\, p_i = (b \ract a) \lact c$.

Appealing to proposition \ref{prop:slicing_on_Act}\
(with $\EEone=\BBBB$ and $\EEtwo=\Aa$) we obtain
\begin{eqnarray*}
    \pairM{(\id\, \slice f_b)(P_{13} P_{23})}{a \tens c}
&=&
    \pairM{P_{13} P_{23}}{a \tens c \tens b} \\
&=&
    \pairM{P_{13}}{a \tens \lp(c \tens b)} \\
&=&
    \textstyle \sum_i \, \pairM{P_{13}}{a \tens p_i \tens q_i} \\
&=&
    \textstyle \sum_i \, \pairM{a}{q_i} \pairM{1_\BB}{p_i} \\
&=&
    \pairM{1_\BB}{(b \ract a) \lact c} \\
&=&
    \pairM{b}{ac} \\
&=&
    \pairM{\DeltaB(b)}{a \tens c}.
\end{eqnarray*}
We still have to show that $\DeltaB(b)$ belongs to $\Env(\BBBB)$.
Therefore, take any actor $(\alpha,\beta)$ for $\BBBB$.
From lemma \ref{lem:commutation:id_tens_lamP}\ we obtain that
$\beta \tens \id$ commutes with $(\lp)_{13}$ and $(\lp)_{23}$,
and hence also with $(\lp)_{13} (\lp)_{23}$.
Similarly $\alpha \tens \id$ commutes with $(\rp)_{23} (\rp)_{13}$.
Proposition \ref{prop:slices_in_Env}\ yields the result.
\end{proof}
\vspace{2ex}


Recalling remarks \ref{rem:weak_fubtens}.i and \ref{rem:fubtens_of_maps}.iv\
in appendix \ref{app:fubini}, we can construct mappings
$$ \id\, \fubtens \DeltaA : B \fubtens A  \rarr B \fubtens A \fubtens A
         \andspace{2em}
   \DeltaB \fubtens\, \id : B \fubtens A  \rarr B \fubtens B \fubtens A. $$
Obviously the pairing $P$ belongs to $B \fubtens A$,
hence the following makes sense:

\begin{prop_sec}
If\/ \pairAB\ is an \idpa, then
$$ (\id \,\fubtens \DeltaA)(P) = P_{12} P_{13}
          \andspace{3em}
   (\DeltaB \fubtens \, \id)(P) = P_{13} P_{23}.  $$
\end{prop_sec}

\begin{proof}
Recalling the proof of the previous proposition, we obtain that
$$  \pairM{P_{13} P_{23}}{a \tens c \tens b}
   \:=\:
    \pair{b}{ac}
   \:=\:
    \pairM{P}{ac \tens b}
   \:=\:
    \pairM{(\DeltaB \fubtens\, \id)(P)}{a \tens c \tens b}  $$
for all $b\in B$ and $a,c\in A$.
\end{proof}
\vspace{2ex}


The following lemma will be useful in constructing a quantum double
(\S \ref{par:construction_qdouble}).


\begin{lemma_sec} \label{lem:from_dpa_to_invertible_dpa}
Let\/ \pairAB\ be a \dpa\ and assume\/ \Aa\ and\/ \BB\ to be weakly unital.
If there exist linear bijections\/ $\lam$ and\/ $\rho$ from\/ $A\tens B$ onto\/ $A\tens B$
such that\/ $(\epsA\tens \epsB)\lam^{-1} = (\epsA\tens \epsB)\, \rho^{-1}$ and,
for all\/ $a\in A$ and\/ $b,d \in B$,
$$ \begin{array}{lcl}
   (\epsA \tens \id)\lam(a \tens b) \,=\, a \lact b
&\hspace{1.5em}&
   \lam\!\left((a \ract d) \tens b \vertM\right)
      \,=\, \left(( \,\cdot \,\ract d) \tens \id \vertM\right) \! \lam(a \tens b)
\\
   (\epsA \tens \id)\,\rho(a \tens b) \,=\, b \ract a
&&
   \rho\!\left((d \lact a)  \tens b \vertM\right)
      \,=\, \left((d \lact \,\cdot \,) \tens \id \vertM\right) \! \rho(a \tens b)
   \vertXL
\end{array}$$
then\/ \pairAB\ is an \idpa, with\/ $\lp=\lam$ and\/ $\rp=\rho$.
\end{lemma_sec}
\begin{proof}
Observe that for any $a,c \in A$ and $b,d \in B$ we have
\begin{eqnarray*}
\pairM{d \tens c}{\lam(a \tens b)}
&=&
\pairM{c}{\left(\epsA( \,\cdot \,\ract d) \tens \id \vertM\right) \! \lam(a \tens b)}
\\&=&
\pairM{c}{(\epsA \tens \id) \lam\!\left((a \ract d) \tens b \vertM\right)}
\\&=&
\pairM{c}{(a \ract d) \lact b}
\\&=&
\pairM{c(a \ract d)}{b}
\\&=&
\pairM{a \ract d}{b \ract c}.
\end{eqnarray*}
Similarly we show $\pairM{d \tens c}{\rho(a \tens b)} = \pairM{d \lact a}{c \lact b}$.
It follows that the pairing $P:A \tens B \rarr \CC$ is \stricta\ continuous,
and $\theta(P) = (\lam,\rho)$ is the corresponding actor for \BBAA\@.
Eventually lemma \ref{lem:char:actor:wu}\ yields $(\BBAA)$-invertibility.
\end{proof}
