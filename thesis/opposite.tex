\section{Opposite \contexts}
\label{par:opposite}

The statements in the present paragraph are completely trivial.
However we believe that dealing with opposite products may be quite tricky
notationally, and therefore we shall be rather explicit in this matter.
Opposite \contexts\ will be an important tool to investigate
{\em regularity\/} of a Hopf system (cf.\ \S \ref{par:regularity}).
\vspace{2ex}


Let $\EE \equiv \EOP$ be any \context\@.
Then obviously $\EEop \equiv (E\op; \Om, \pairing)$ is again an \context,
referred to as the {\em opposite\/} \context\@.
When \EE\ is non-degenerate, the trivial anti-isomorphism
$\id\op : E \rarr E\op : x \mapsto x$
extends naturally to an anti-isomorphism
$$ \id\op : \PreE \rarr \Pre(\EEop) : (\lam,\rho) \mapsto (\rho,\lam)  $$
mapping \ActE\ onto $\Act(\EEop)$ and \EnvE\ onto $\Env(\EEop)$.
It follows that $\Env(\EEop) \simeq \EnvE\op$.
Since confusion is likely, we shall use $\lactop$ and $\ractop$
to denote the actions associated with $\EEop$.
So for $x\in E$ and $\om \in \Om$ we have for instance $x \lactop \om = \om \ract x$
(observe we have suppressed $\id\op$ in the notation---as we shall usually
do henceforth).
Of course most properties discussed in the preceding paragraphs
are preserved when we pass to the opposite \context:
\begin{itemize}
\item
   \EE\ is weakly unital if and only if $\EEop$ is weakly unital.
\item
   \EE\ and $\EEop$ induce the same \strictw, \strictm\ and \stricta\
   topologies on $\Om$.
\item
   A functional on $\Om$ is \EEdash invertible if and only if
   it is $\EEop$-invertible.
\item
   If $\EEone$ and $\EEtwo$ are \contexts, then
   $\EEone\op \tens \EEtwo\op = (\EEone \tens \EEtwo)\op$.
\end{itemize}
In the weakly unital case, \EE\ and $\EEop$ share the counit $\eps:\Om\rarr\kk$.
Furthermore the pairings $\pair{\ActE}{\Om}$ and $\pair{\Act(\EEop)}{\Om}$
agree through $\id\op$.
