
\begin{abs_chp}
We introduce a Fourier context for the quantum $E(2)$ group and then explicitly
construct its Fourier transforms, involving holomorphic \mbox{$q$-Hankel}\ transformation.
Of course we then have to {\em prove\/} that the formulas we
propose really are the Fourier transforms; in this respect propositions
\ref{prop:holqHankel:qdiff}\ and \ref{prop:moment:link_with_Hankel}\ will play
a key-role. It also turns out that the dilation and quantization parameters
$\tau$, $\nu$ and $q$ cannot be chosen independently.
\end{abs_chp}



\subsection{A Fourier context for the quantum $E(2)$ group}

Henceforth the positive numbers $\tau$ and $\nu$ will be given by
\begin{equation}\label{eq:tau_nu:value}
   \tau \,=\, q^{-1}   \andspace{3em}   \nu \,=\: (q^{-1}-q)^{-2}. \hspace{1em}
\end{equation}
We shall need some dilation operators on \HC\ similar to $\Omega$,
though for an entirely different purpose: for any $r>0$, let
$\Lambda_r$ be the linear mapping on \HC\ given by
$(\Lambda_r g)(z) = g(r^{-1} z)$.
We are mainly interested in the cases $r=\tau$ and $r=\nu$, and
in particular we have $\Lambda_\tau = \Omega$.
Also observe the commutation rule $\Lambda_r \Psi = r^{-1} \Psi \Lambda_r$.
Accordingly, we shall need dilated versions of the space \Hcore\ in
definition \ref{def:Hcore}\@. Therefore define $\Hcore_r = \Lambda_r \Hcore$,
for any $r>0$. Notice that $\Hcore_r$ is contained in ${\mathcal S}_r(\RR^+;q^2)$.
Furthermore $\Hcore_r$ is non-trivial since $\Hcore$ is non-trivial
(\mbox{cf.\ corollary \ref{cor:qExp_in_Hcore}}).


We are about to invoke \mbox{theorem \ref{thm:combine_results}}, which will provide
a setting for doing harmonic analysis on the quantum $E(2)$ group. From the
above observations and proposition \ref{prop:Hcore:invariance}, it
follows that ${\mathcal G}_r' \equiv \Hcore_r$ satisfies
assumptions \ref{assume:Gr}, for any $r>0$ and in particular for $r=\tau,\nu$.
On the other hand, we already know from \mbox{example \ref{ex:Schwartz-like_space}}\
that ${\mathcal G}_r \equiv {\mathcal S}_r(\RR^+;q^2)$
is a subalgebra of \HC\ \mbox{satisfying}\ assumptions \ref{assume:Gr}\@.
Eventually theorem \ref{thm:qmoment}\ yields that
$\Hcore_\tau$ and $\Hcore_\nu$ enjoy condition (iii) of
\mbox{theorem \ref{thm:combine_results}}\@. We conclude:



\begin{thm} \label{thm:Fourier_context}
The system
$$\left( \Uq\!\left(\calL(\Hcore_\tau)\vertM\right)  \subseteq
         \Uq\!\left(\calL({\mathcal S}_\tau(\RR^+;q^2)) \vertM \right),
                \varphi, \psi; \, \Uq, \Aq; \,
                \Aq(\Hcore_\nu) \subseteq
         \Aq\!\left({\mathcal S}_\nu(\RR^+;q^2)\vertM \right),
          \omega \vertL \right)$$
is a Fourier context (cf.\ chapter \ref{chapter:algebraic_harmonic_analysis}).
\end{thm}



\subsection{Parity structure: an important slice map}

Let $\evzero$ be the linear functional on \FZ\ defined by $\evzero(f)=f(0)$.
Now we consider the slice map $\evzero \tens \id$ from $\FZ \tens \HC$ into \HC\@.
Although trivial, the following lemma is essential for the construction of
Fourier transforms and crucial in understanding the \lq parity\rq\ structure
(cf.\ remark \ref{rem:haar:Uq:concrete})
that entered the construction of the Haar functionals $\varphi$ and $\psi$.
Recall $\calL(\Hcore_\tau)$ is defined as
\begin{equation} \label{eq:def:Hcore:parity}
\calL(\Hcore_\tau) \;=\;  \left(\KZeven \tens \Hcore_\tau^\scripteven \vertL\right)
                                              \:\oplus\:
                          \left(\KZodd  \tens \Hcore_\tau^\scriptodd \vertL\right).
\end{equation}



\begin{lemma} \label{lemma:slice_map:parity}
The slice map\/ $\evzero \tens \id$ maps\/ $\calL(\Hcore_\tau)$ into\/ $\Hcore_\tau$.
\end{lemma}
\begin{proof}
Clearly $\evzero$ annihilates $\KZodd$ and therefore only the \lq even\rq\ component of
(\ref{eq:def:Hcore:parity}) is able to survive $\evzero \tens \id$.
The result follows since $\Hcore_\tau^\scripteven = \Hcore_\tau$.
\end{proof}



\begin{remark}\label{rem:Fourier:construction:parity} \rm
We claim that (\ref{eq:def:Hcore:parity}) offers the most natural way towards the above result.
To appreciate this a little more, let's try to come up with an alternative:
let's suppose for instance, that we would replace
$\calL \! \left({\mathcal S}_\tau(\RR^+;q^2)\vertM \right)$ by the space
$\KZ \tens {\mathcal S}_\tau (\RR^+;q)$ proposed in remark \ref{rem:parity}.iii\@.
In view of (\ref{eq:qSchwartz:not_direct_sum})
the corresponding alternative for $\calL(\Hcore_\tau)$ would be
$\KZ \tens \left(\Hcore_\tau^\scripteven \cap \,\Hcore_\tau^\scriptodd \vertM \right)$.
Also with these choices it should be possible to produce a Fourier context
similar to the one of theorem \ref{thm:Fourier_context}\ above,
and moreover the latter choice would qualify w.r.t.\ lemma \ref{lemma:slice_map:parity}\@.
However we run into trouble by taking the {\em intersection\/} of
$\Hcore_\tau^\scripteven$ and $\Hcore_\tau^\scriptodd$.
The point is that in this operation we probably loose all the interesting
functions supplied by corollary \ref{cor:qExp_in_Hcore}, and in fact
$\Hcore_\tau^\scripteven \cap \, \Hcore_\tau^\scriptodd $ might as well be
trivial---the question is still open. Anyway it should be clear by
now that (\ref{eq:def:Hcore:parity}) is much more natural.

One more try: could we take $\calL(\Hcore_\tau)$ to be the even
component in (\ref{eq:def:Hcore:parity}) only? Again the answer is
negative, since then $\calL(\Hcore_\tau)$ would no longer be invariant under
$\Gamma \tens \Omega^{\pm 1}$ or $\Gamma^{-1} \tens \Omega^{\pm 1}$,
which was essential in proposition \ref{prop:UqFG}\@.
So there seems to be no valuable alternative for (\ref{eq:def:Hcore:parity})
and moreover, our choice shall be manifestly confirmed by further results
(cf.\ \S\ref{sect:plancherel} and \S\ref{sect:inversion}).
\hfill $\star$
\end{remark}




\subsection{Yet another \lq H\rq}

In this paragraph we introduce and study a family $\YaH{m}{k}$ (with $m,k\in\ZZ$)
of linear transforms from $\calL(\Hcore_\tau)$ into $\Hcore_\nu$,
making diagram (\ref{eq:diagram:YaH}) below commute. Basically these
$\YaH{m}{k}$ can still be thought of as some kind of \mbox{$q$-Hankel}\ transform.

\begin{defn} \label{def:YaH}
Whenever $m,k\in\ZZ$, let $\kq(m,k)$ denote the scalar
$$  \kq(m,k) \;=\; (-1)^m q^{-m} q^{\frac{1}{2}|m|(k-1)} (q^{-1}-q)^{|m|}. $$
Furthermore let $\YaH{m}{k}$ be the following linear map from
$\calL(\Hcore_\tau)$ into $\Hcore_\nu$:
$$ \YaH{m}{k}  \;=\;   \kq(m,k) \: \Lambda_\nu \, \holH{m} \, \Lambda_\tau^{-1}\,
         (\evzero \tens \id)\,(\Gamma^{-1} \tens \Omega)_{|\calL(\Hcore_\tau)}^k $$
That this definition makes sense can be seen from the following diagram:
$$\begin{CD}
 \calL(\Hcore_\tau)   @>{\textstyle \hspace{1em}(\Gamma^{-1} \tens \Omega)^k\hspace{1em}}>>
 \calL(\Hcore_\tau)   @>{\textstyle\;\hspace{1em} \evzero \tens \id\hspace{1em}\;}>>
 \Hcore_\tau          @>{\textstyle\hspace{2em}\Lambda_\tau^{-1}\hspace{2em}}>>
 \Hcore  \\
 @V{\textstyle\YaH{m}{k}}VV     @. @.
                      @VV{\textstyle \holH{m}}V \\
 \Hcore_\nu   @<<\hspace{7em}<  \Hcore_\nu
              @<<{\textstyle\;\hspace{1em} \kq(m,k) \hspace{1em}\; }<
 \Hcore_\nu   @<<{\textstyle  \hspace{2em} \Lambda_\nu\hspace{2em}}<
       \Hcore
\end{CD}$$
\nopagebreak
\begin{equation}\label{eq:diagram:YaH}
\diacaption{definition of $\YaH{m}{k}$}
\end{equation}
This diagram in itself does make several statements:
clearly lemma \ref{lemma:slice_map:parity}\ is involved, as well as
proposition \ref{prop:Hcore:invariance}, stating that $\Hcore$ is invariant
under all $\holH{m}$. Furthermore we need $\calL(\Hcore_\tau)$ to be invariant
under $\Gamma^{\pm 1} \tens \Omega^{\mp 1}$, which is an immediate consequence of
(\ref{eq:def:Hcore:parity}). In this respect, notice how
$\Gamma^{\pm 1} \tens \Omega^{\mp 1}$
interchanges\footnote{which is just one of the reasons why {\em both\/}
components should be present.} the even and
odd components in (\ref{eq:def:Hcore:parity}).
\end{defn}



\begin{lemma} \label{lemma:kq:formulas}
For any\/ $m,k\in\ZZ$ we have
\begin{eqnarray}
\kq(m,k)                 &=&        q^{\frac{1}{2}|m|} \: \kq(m,k-1)
       \label{eq:kq:formula1} \\
(q^{-1}-q)\, \kq(m-1,k)  &\vertXL=& -q^{\frac{3}{2} -\frac{1}{2}k}\: \kq(m,k)
                       \hspace{3em}\;   {\rm if}\;\; m\geq 1  \hspace{3em}
       \label{eq:kq:formula2}\\
(q^{-1}-q)\, \kq(-m+1,k) &\vertXL=& -q^{-\frac{1}{2} -\frac{1}{2}k}\: \kq(-m,k)
                       \hspace{2em}     {\rm if}\;\; m\geq 1.  \hspace{3em}
       \label{eq:kq:formula3}
\end{eqnarray}
\end{lemma}
\begin{proof}
Straightforward.
\end{proof}
\vspace{1ex}

In \S \ref{par:Haar:Uq:construction}\ we observed that a space like $\calL(\Hcore_\tau)$
satisfies the conditions of proposition \ref{prop:UqFG}\@.
In particular, $\calL(\Hcore_\tau)$ is invariant under
$$\Gamma^{\pm 1} \tens \Omega^{\mp 1}   \hspace{2em}
\id \tens \Omega^{\pm 2}                \hspace{2em}
\Phi^{\pm 1} \tens \id                  \hspace{2em}
\id \tens \Omega^{\pm 1} \nabq{m}       \hspace{2em}
\id \tens \Dqsqr. $$
So the question arises how the $\YaH{m}{k}$ behave under these operations:



\begin{lemma} \label{lemma:YaH:properties}
For any\/ $m,k\in\ZZ$ we have
$$\begin{array}{lclcl}
\YaH{m}{k} (\Gamma \tens \Omega^{-1})
          &\vertXL =&   q^{\frac{1}{2}|m|}\; \YaH{m}{k-1}  \\
\YaH{m}{k} (\Phi \tens \id)
          &\vertXL =&   q^{-\frac{1}{2}k}\; \YaH{m}{k}     \\
\YaH{m}{k} (\id \tens \Omega^2)
          &\vertXL =&   q^{-2|m|-2}\; \Omega^{-2} \, \YaH{m}{k}     \\
\YaH{m-1}{k} (\id \tens \Omega \nabq{m})
          &\vertXXL =&  -q^{\frac{3}{2} -\frac{1}{2}k -m} \; \Psi \,\YaH{m}{k}
          &  \hspace{1em} &  {\rm if}\;\; m\geq 1     \\
\YaH{-m+1}{k} (\id \tens \Omega^{-1} \nabq{m})
          &\vertXL =&   q^{-\frac{1}{2} -\frac{1}{2}k + m} \; \Psi \, \YaH{-m}{k}
          &  \hspace{1em} &  {\rm if}\;\; m\geq 1     \\
\YaH{m+1}{k} (\id \tens \Dqsqr)
          &\vertXXL =&  q^{-\frac{3}{2} -\frac{1}{2}k} \;
          \YaH{m}{k} (\id \tens \Omega^2)
          &  \hspace{1em} &  {\rm if}\;\; m\geq 0     \\
\YaH{-m-1}{k} (\id \tens \Dqsqr)
          &\vertXL =&  -q^{\frac{1}{2} -\frac{1}{2}k} \; \YaH{-m}{k}
          &  \hspace{1em} &  {\rm if}\;\; m\geq 0.
\end{array}$$
\end{lemma}
\vspace{2ex}

\begin{proof}
The first formula is an immediate consequence of (\ref{eq:kq:formula1})
whereas the second follows from $\evzero \Phi = \evzero$ and the commutation rule
$\Gamma^{-k}\Phi = q^{-\frac{1}{2}k}\,\Phi\Gamma^{-k}$.
The third formula corresponds to (\ref{eq:holS:Omega_invariant}).
The remaining formulas are all consequences of
proposition \ref{prop:holqHankel:qdiff}\ and lemma \ref{lemma:kq:formulas}\@.
Let's for instance check the last but one: using commutation rules like
$\Omega^k \Dqsqr = q^{-k}\, \Dqsqr \Omega^k$, we get
\begin{eqnarray*}
\lefteqn{\YaH{m+1}{k} (\id \tens \Dqsqr)}
\\&=&
\kq(m+1,k) \; \Lambda_\nu \, \holH{m+1} \, \Lambda_\tau^{-1}\,
         (\evzero \tens \id)\,(\Gamma^{-1} \tens \Omega)^k (\id \tens \Dqsqr)
\\ &=&
q^{-k} \, \tau^{-1} \, \kq(m+1,k) \; \Lambda_\nu \, \holH{m+1} \,\Dqsqr\, \Lambda_\tau^{-1}\,
         (\evzero \tens \id)\,(\Gamma^{-1} \tens \Omega)^k
\\ &\stackrel{(*)}{=}&
-q^{-k} q\, \frac{q^{-1}}{q^{-1}-q}\, \kq(m+1,k)
     \; \Lambda_\nu \, \holH{m} \Omega^2  \Lambda_\tau^{-1}\,
         (\evzero \tens \id)\,(\Gamma^{-1} \tens \Omega)^k
\\ &\stackrel{(\ref{eq:kq:formula2})}{=}&
q^{-k}\,q^{-\frac{3}{2} +\frac{1}{2}k}\: \kq(m,k)
     \; \Lambda_\nu \, \holH{m} \, \Lambda_\tau^{-1}\,
         (\evzero \tens \id)\,(\Gamma^{-1} \tens \Omega)^k  (\id \tens \Omega^2)
\\ &=&
q^{-\frac{3}{2} - \frac{1}{2}k}\:  \YaH{m}{k}\, (\id \tens \Omega^2)
\end{eqnarray*}
for $m\geq 0$.
In $(*)$ we used the first formula of proposition \ref{prop:holqHankel:qdiff}\@.
Also notice that the value of $\tau$ was involved in the computation.
In order to appreciate the particular choice (\ref{eq:tau_nu:value}) we made for $\nu$,
we shall also have a closer look at for instance the fifth formula in the list:
for any $m\geq 1$ we have
\begin{eqnarray*}
\lefteqn{ \YaH{-m+1}{k} (\id \tens \Omega^{-1} \nabq{m})}
\\
&\hspace{1em}=&
         \kq(-m+1,k) \; \Lambda_\nu \, \holH{-m+1} \, \Lambda_\tau^{-1}\,
         (\evzero \tens \id)\,(\Gamma^{-1} \tens \Omega)^k
         (\id \tens \Omega^{-1} \nabq{m})
\\
&\hspace{1em}=&
    \kq(-m+1,k) \; \Lambda_\nu \, \holH{-m+1} \, \nabq{m} \Omega^{-1} \Lambda_\tau^{-1}\,
         (\evzero \tens \id)\,(\Gamma^{-1} \tens \Omega)^k
\\
&\hspace{1em}\stackrel{(\sharp)}{=}&
       -\frac{1}{q^{-1}-q} \,q^m\: \kq(-m+1,k) \;
       \Lambda_\nu \, \Psi \holH{-m} \Lambda_\tau^{-1}\,
       (\evzero \tens \id)\,(\Gamma^{-1} \tens \Omega)^k.
\end{eqnarray*}
In $(\sharp)$ we used the last formula of proposition
\ref{prop:holqHankel:qdiff}\@. Now the parameter $\nu$ gets
involved, since we need to use the commutation rule
$$ \Lambda_\nu \Psi \:=\: \nu^{-1} \Psi \Lambda_\nu  \:=\: (q^{-1}-q)^2 \Psi \Lambda_\nu $$
yielding
\begin{eqnarray*}
\cdots &=\vertXL&
      -q^m \,(q^{-1}-q) \kq(-m+1,k) \; \Psi \Lambda_\nu \holH{-m} \Lambda_\tau^{-1}\,
      (\evzero \tens \id)\,(\Gamma^{-1} \tens \Omega)^k
\\
&\stackrel{(\ref{eq:kq:formula3})}{=}&
      q^m \: q^{-\frac{1}{2} -\frac{1}{2}k}\: \kq(-m,k) \;
      \Psi \Lambda_\nu \holH{-m} \Lambda_\tau^{-1}\,
      (\evzero \tens \id)\,(\Gamma^{-1} \tens \Omega)^k
\\&=&
      q^{-\frac{1}{2} -\frac{1}{2}k + m} \; \Psi \, \YaH{-m}{k}.
\end{eqnarray*}
The other cases are similar.
\end{proof}




\begin{remark} \rm
Observe that it is really essential to take $\nu=(q^{-1}-q)^{-2}$
in order to get rid of all the $(q^{-1}-q)$ factors involved in
the last computation. Indeed merely adjusting\footnote{for
instance, $\kq(m,k)=\ldots (q^{-1}-q)^{-|m|}$ instead of $\ldots
(q^{-1}-q)^{|m|}$}\ the definition of the scalars $\kq$ wouldn't
yield the proper effect, since then these $(q^{-1}-q)$ factors
would simply emerge somewhere else (more precisely, in the last
two formulas of lemma \ref{lemma:YaH:properties}). Soon it will
become clear {\em why\/} we actually want these $(q^{-1}-q)$
factors to cancel (cf.\ proof of proposition
\ref{prop:Eq2:Fourier_transforms}). \hfill $\star$
\end{remark}



\begin{lemma} \label{lemma:finite_support}
Take any\/ $X\in \calL(\Hcore_\tau)$ and\/ $m\in \ZZ$.
Then\/ $\YaH{m}{k} X = 0$ except for finitely many\/ $k\in\ZZ$.
\end{lemma}
\begin{proof}
By definition $\calL(\Hcore_\tau) \subseteq \KZ \tens \HC$.
Writing $X=\sum_i f_i \tens g_i$ with $f_i \in \KZ$ and $g_i\in \HC$, we get
$$ (\evzero \tens \id)\,(\Gamma^{-1} \tens \Omega)^k X
       \;=\; \textstyle \sum_i\:  f_i(k\theta) \, \Omega^k  g_i.   $$
Since all the $f_i$ have finite support, the result follows.
\end{proof}



\subsection{Explicit construction of Fourier transforms}

\begin{defn} \label{def:Eq2:Fourier_transforms}
Define linear mappings $\FRabb$ and $\FLabb$ from
$\Uq\!\left( \vertM \calL(\Hcore_\tau)\right)$ into $\Aq(\Hcore_\nu)$ by
\begin{eqnarray*}
 \FRabb \!\left(\vertM\Upsilon(X)\, b^m\right)
    &=& \sum_{k\in\ZZ} \; \alpha^{k+m} \: \gamma^m
    \left(\YaH{m}{k}\,X\right)(\gamma^*\gamma) \\
\nopagebreak[4]
 \FRabb \!\left(\vertM\Upsilon(X)\, c^m\right)
    &=& \sum_{k\in\ZZ} \; \alpha^{k-m} \: (\gamma^*)^m
    \left(\YaH{-m}{k}\,X\right)(\gamma^*\gamma) \\
\nopagebreak[2]
    \vertXL
 \FLabb \!\left(\vertM\Upsilon(X)\, b^m\right)
    &=& \sum_{k\in\ZZ}  \; (q^2 \alpha)^{k+m}\: \gamma^m
    \left(\YaH{m}{k}\,X\right)(\gamma^*\gamma) \\
\nopagebreak[4]
 \FLabb \!\left(\vertM\Upsilon(X)\, c^m\right)
    &=& \sum_{k\in\ZZ}  \; (q^2 \alpha)^{k-m}\: (\gamma^*)^m
    \left(\YaH{-m}{k}\,X\right)(\gamma^*\gamma)
\end{eqnarray*}
for any $X\in \calL(\Hcore_\tau)$ and $m\in \NN$.
Observe the summations over $k\in\ZZ$ are in fact {\em finite\/} sums
because of lemma \ref{lemma:finite_support}. Also notice the
formulae are in agreement when $m=0$.
\end{defn}



\begin{prop} \label{prop:Eq2:Fourier_transforms}
The maps\/ $\FRabb$ and\/ $\FLabb$ are Fourier transforms w.r.t.\ the
Fourier context established in theorem \ref{thm:Fourier_context}\@.
More precisely\/ $\FRabb$ is both an {\scriptsize RL} and an {\scriptsize RR} transform,
whereas\/ $\FLabb$ is an {\scriptsize LR} and {\scriptsize LL} transform.
In particular this means that (cf.\ definition \ref{def:Fourier_transform})
\begin{enumerate}
\item
$\FRabb$ and\/ $\FLabb$ are respectively left and right\/ $\Aq$-module morphisms, and
\item
$\omega \FRabb = \pair{\,\cdot\,}{1} = \omega \FLabb$.
\end{enumerate}
These Fourier transforms are unique within the setting of theorem \ref{thm:Fourier_context}\@.
%
%\rm  Here $\omega$ is the (left {\em and\/} right) Haar functional on
%$\Aq\!\left({\mathcal S}_\nu(\RR^+;q^2)\right)$
%as constructed in \ref{par:Haar:Aq:construction}\@.
\end{prop}

\begin{proof}
{\em Uniqueness\/} of Fourier transforms is a property of any Fourier context
(cf.\ lemma \ref{lemma:Fourier_transform}).
Nevertheless it may be an instructive exercise to observe how in this particular case
uniqueness of Fourier transforms eventually amounts to the \mbox{$q$-moment}\
problem for the space $\Hintersect$ as treated in \S\ref{sec:qHankel}\@.

Now let us show (ii). Observe that for any $X\in \calL(\Hcore_\tau)$ and $m\in \NN$ we have
\begin{eqnarray*}
\lefteqn{ \omega\left(\vertL \FRabb \!\left(\vertM\Upsilon(X)\, b^m\right) \right)}
\\&=&
     \omega\left(\: \sum_{k\in\ZZ} \; \alpha^{k+m} \: \gamma^m
       \left(\YaH{m}{k}\,X\right)(\gamma^*\gamma) \right)
\\&=&
     \delta_{m,0} \; \, \omega \left[
            \left(\YaH{0}{0}\,X\right)(\gamma^*\gamma) \right]
\\&=&
     \delta_{m,0}\: \sum_{n \in\ZZ} \left(\YaH{0}{0} X \right) (\nu q^{2n})\: q^{2n}
\\&=&
     \delta_{m,0}\: \sum_{n\in\ZZ} \: \kq(0,0) \: \left( \Lambda_\nu \, \holH{0}
          \, \Lambda_\tau^{-1}\,(\evzero \tens \id) X \right) (\nu q^{2n})\: q^{2n}
\\&=&
     \delta_{m,0}\: \frac{1}{1-q^2} \int_0^\infty  \left(\holH{0} \, \Lambda_\tau^{-1}
                 (\evzero \tens \id) X \right)(x) \: d_{q^2}x
\\&=&
     \vertXL \delta_{m,0}\: (q^2;q^2)_0  \left(\holH{0}\holH{0} \, \Lambda_\tau^{-1}
                 (\evzero \tens \id) X\right)(0)
\\&=&
     \vertL \delta_{m,0}\: X(0,0)
\\&=&
     \vertL\pair{\Upsilon(X)\, b^m}{1}.
\end{eqnarray*}
Here we used the definitions of $\FRabb$, $\omega$, $\YaH{0}{0}$, $\kq$
and \mbox{$q^2$-integration}\@.
We also used proposition \ref{prop:moment:link_with_Hankel}\ and $\holH{0}^2 = \id$.
The last equality is clear from definition \ref{def:pairing:Uqext:Aq}\
and lemma \ref{lemma:Upsilon}\@.
Similarly we deal with $\Upsilon(X) c^m$ and with $\FLabb$.
\vspace{2ex}


To verify the {\em module\/} properties, it suffices to check the action
of the generators $\alpha,\beta$ and $\gamma$, both on elements of
type $\Upsilon(X)\, b^m$ and of type $\Upsilon(X)\, c^m$, and both
for $\FRabb$ and $\FLabb$. All together this yields 12 cases to check;
we shall have a closer look on only 4 of them.
However the remaining 8 cases are more or less analogous:
they are all consequences of proposition \ref{prop:actions:Upsilon}\
and lemma \ref{lemma:YaH:properties}\@.
In what follows, $X$ is an arbitrary element in $\calL(\Hcore_\tau)$ and $m\in \NN$.
Now consider for instance the following case:
\begin{eqnarray*}
\FRabb \!\left(\vertM  \alpha \, \lact \, \Upsilon(X)\, b^m\right)
&\stackrel{\rm (A)}{=}&
    q^{-\frac{1}{2}m} \, \FRabb \!\left(\vertL
        \Upsilon \! \left(\vertM(\Gamma \tens \Omega^{-1})X\right)\, b^m\right)
\\
&\stackrel{\rm (F)}{=}& \vertXL
    \sum_{k\in\ZZ} \: q^{-\frac{1}{2}m} \: \alpha^{k+m}\: \gamma^m
    \left(\YaH{m}{k}\,(\Gamma \tens \Omega^{-1}) X\right)(\gamma^*\gamma)
\\
&\stackrel{\rm (H)}{=}&
    \sum_{k\in\ZZ} \: q^{-\frac{1}{2}m} q^{\frac{1}{2}|m|} \: \alpha^{k+m}\: \gamma^m
    \left(\YaH{m}{k-1} X\right)(\gamma^*\gamma)
\\
&\stackrel{\rm (S)}{=}&
    \sum_{k\in\ZZ} \; \alpha^{k+1+m}\: \gamma^m
    \left(\YaH{m}{k} X\right)(\gamma^*\gamma)
\\
&\stackrel{\rm (F)}{=}&
    \alpha\: \FRabb \!\left(\vertM\Upsilon(X)\, b^m\right).
\end{eqnarray*}
Let's explain the labeling:
({\sc a}) refers to the formulas for the $\Aq$-actions
as computed in proposition \ref{prop:actions:Upsilon}\@.
({\sc f}) refers to the definition of our Fourier transforms,
i.e.\ definition \ref{def:Eq2:Fourier_transforms}\@.
({\sc h}) links to the properties of the maps $\YaH{m}{k}$
as given in lemma \ref{lemma:YaH:properties}, and
eventually ({\sc s}) means that the summation variable $k$ is substituted
with $k+1$. In what follows, ({\sc c}) will indicate that we use some
commutation rule involving the generators, for instance
$\alpha\beta = q\,\beta\alpha$.
\begin{eqnarray*}
\lefteqn{ \FRabb \!\left(\vertM \,\beta \,\lact \, \Upsilon(X)\, c^m \right)}
\\
\!\!\! &\stackrel{\rm (A)}{=}& \!\!
      q^{\frac{1}{2}(m+1)}\, \FRabb \!\left(\vertL \Upsilon \!
      \left(\vertM(\Phi \tens \id)(\Gamma \tens \Omega^{-1}) (\id \tens \Dqsqr)X
      \right)c^{m+1} \right)
\\
\!\!\! &\stackrel{\rm (F)}{=}& \!\!
      \sum_{k\in\ZZ} \, q^{\frac{1}{2}(m+1)}\,
      \alpha^{k-m-1}\, (\gamma^*)^{m+1}
  \hspace{-3em}\raisebox{-5ex}{$
      \left(\YaH{-m-1}{k} (\Phi \tens \id)(\Gamma \tens \Omega^{-1}) (\id \tens \Dqsqr)X
      \right) \!(\gamma^*\gamma) $}
\\
\!\!\! &\stackrel{\rm (H)}{=}& \!\!
      \sum_{k\in\ZZ} \, q^{\frac{1}{2}(m+1)} q^{-\frac{1}{2}k}\,
      \alpha^{k-m-1} (\gamma^*)^{m+1} \! \left(\YaH{-m-1}{k}
      (\Gamma \tens \Omega^{-1}) (\id \tens \Dqsqr)X
      \right) \! (\gamma^*\gamma)
\\
\!\!\! &\stackrel{\rm (H)}{=}& \!\!
      \sum_{k\in\ZZ} \,  q^{\frac{1}{2}(m+1-k)} \, q^{\frac{1}{2}|-m-1|} \,
      \alpha^{k-m-1}\, (\gamma^*)^{m+1} \left(\YaH{-m-1}{k-1}
      \left( \vertM \id \tens \Dqsqr \right) \! X \right) \! (\gamma^*\gamma)
\\
\!\!\! &\stackrel{\rm (H)}{=}& \!\!
      \sum_{k\in\ZZ} \,  q^{\frac{1}{2}(m+1-k)}\,q^{\frac{1}{2}(m+1)}
      (-1)\, q^{\frac{1}{2} -\frac{1}{2}(k-1)} \,
      \alpha^{k-m-1} (\gamma^*)^{m+1} \! \left(\YaH{-m}{k-1} X
      \right) \! (\gamma^*\gamma)
\\
\!\!\!&=&\!\!
      \sum_{k\in\ZZ} \, (-1)\, q^{m-k+2}
      \alpha^{k-m-1}\,(-q^{-1} \beta)\, (\gamma^*)^m
      \left(\YaH{-m}{k-1} X \right) \! (\gamma^*\gamma)
\\
\!\!\! &\stackrel{\rm (C)}{=}& \!\!
      \sum_{k\in\ZZ} \,   \beta \, \alpha^{k-m-1} \,
      (\gamma^*)^m \left(\YaH{-m}{k-1} X \right) \! (\gamma^*\gamma)
\\
\!\!\! &\stackrel{\rm (S)}{=}& \!\!
      \beta \,  \FRabb \!\left(\vertM \Upsilon(X)\, c^m \right).
\end{eqnarray*}
Our next case deals with right actions and is a little more involved:
\begin{eqnarray*}
\lefteqn{\FLabb \!\left(\vertM \Upsilon(X)\, c^m \, \ract \,\alpha \right)}
\\
\!\!\! &\stackrel{\rm (A)}{=}& \!\!
      q^{\frac{1}{2}m}\, \FLabb \!\left(\vertL
        \Upsilon \! \left(\vertM(\Gamma \tens \Omega)X\right) c^m \right)
\\
\!\!\! &\stackrel{\rm (F)}{=}& \!\!
      \sum_{k\in\ZZ} \, q^{\frac{1}{2}m}\, q^{2(k-m)} \:
      \alpha^{k-m}\, (\gamma^*)^m \left(\YaH{-m}{k}(\Gamma \tens \Omega^{-1})
      (\id \tens \Omega^2) X\right) \! (\gamma^*\gamma)
\\
\!\!\! &\stackrel{\rm (H)}{=}& \!\!
      \sum_{k\in\ZZ} \, q^{\frac{1}{2}m}\, q^{2(k-m)} \, q^{\frac{1}{2}|-m|} \:
      \alpha^{k-m}\, (\gamma^*)^m \left(\YaH{-m}{k-1}
      (\id \tens \Omega^2) X\right) \! (\gamma^*\gamma)
\\
\!\!\! &\stackrel{\rm (H)}{=}& \!\!
      \sum_{k\in\ZZ} \, q^{2(k-m)} \, q^m \, q^{-2|-m|-2} \:
      \alpha^{k-m}\, (\gamma^*)^m
      \left(\Omega^{-2} \YaH{-m}{k-1} X\right) \! (\gamma^*\gamma)
\\
\!\!\! &\stackrel{\rm (S)}{=}& \!\!
      \sum_{k\in\ZZ} \, q^{2(k+1-m)} \, q^{-m-2} \:
      \alpha^{k+1-m}\, (\gamma^*)^m
      \left(\Omega^{-2} \YaH{-m}{k} X\right) \! (\gamma^*\gamma)
\\
\!\!\!&=&\!\!
      \sum_{k\in\ZZ} \, q^{2(k-m)} \, q^{-m} \:
      \alpha^{k-m}\,  \alpha \, (\gamma^*)^m
      \left(\Omega^{-2} \YaH{-m}{k} X\right) \! (\gamma^*\gamma)
\\
\!\!\! &\stackrel{\rm (C)}{=}& \!\!
      \sum_{k\in\ZZ} \, q^{2(k-m)} \: \alpha^{k-m}\,  (\gamma^*)^m
      \left(\YaH{-m}{k} X\right) \! (\gamma^*\gamma) \, \alpha
\\
\!\!\! &\stackrel{\rm (F)}{=}& \!\!
      \FLabb \!\left(\vertM \Upsilon(X)\, c^m \right) \alpha.
\end{eqnarray*}
In the above computation, besides $\alpha\beta = q\,\beta\alpha$,
the label ({\sc c}) also refers to the commutation rule in
lemma \ref{lem:functional_calc:Aq:properties}\@.

Let us consider one more example: assuming $m\geq 1$, we have
\begin{eqnarray*}
\lefteqn{\FLabb \!\left(\vertM \Upsilon(X)\, b^m \, \ract \,\beta \right) } \\
\!\!\! &\stackrel{\rm (A)}{=}& \!\!
       q^{\frac{1}{2}(m-1)}\, \FLabb \!\left(\vertL
       \Upsilon \! \left(\vertM(\Phi^{-1} \tens \id)(\Gamma \tens \Omega^{-1})
       (\id \tens \Omega \nabq{m})X \right)
       b^{m-1} \right)
\\
\!\!\! &\stackrel{\rm (F)}{=}& \!\!
       \sum_{k\in\ZZ} \, q^{\frac{1}{2}(m-1)} \, (q^2 \alpha)^{k+m-1} \, \gamma^{m-1}
    \hspace{-5.5em}\raisebox{-5ex}{$
       \left(\YaH{m-1}{k} (\Phi^{-1} \tens \id)(\Gamma \tens \Omega^{-1})
       (\id \tens \Omega \nabq{m})X  \right) \! (\gamma^*\gamma) $}
\\
\!\!\! &\stackrel{\rm (H)}{=}& \!\!
       \sum_{k\in\ZZ} \: q^{\frac{1}{2}(m-1)}\, q^{2(k+m-1)} \, q^{\frac{1}{2}k}
       \, q^{\frac{1}{2}|m-1|} \: \alpha^{k+m-1}\, \gamma^{m-1}
    \hspace{-5.6em}\raisebox{-5ex}{$
       \left(\YaH{m-1}{k-1} \left(\vertM \, \id \tens \Omega \nabq{m}\right)X
       \right) \! (\gamma^*\gamma) $}
\\
\!\!\! &\stackrel{\rm (H)}{=}& \!\!
       \sum_{k\in\ZZ} \: q^{m-1}\, q^{2(k+m-1)} \, q^{\frac{1}{2}k}
       \, (-1)\, q^{\frac{3}{2} -\frac{1}{2}(k-1) - m}
       \: \alpha^{k+m-1}\, \gamma^{m-1} \:
    \hspace{-4em}\raisebox{-5ex}{$
       \left(\Psi \,\YaH{m}{k-1}X \right) \! (\gamma^*\gamma) $}
\\
\!\!\! &\stackrel{\rm (M)}{=}& \!\!
       \sum_{k\in\ZZ} \:  (-1)\,q^{-1}\, q^{2(k+m)}
       \: \alpha^{k+m-1}\, \gamma^{m-1} \:
       (\gamma^*\gamma) \left(\YaH{m}{k-1}X \right) \! (\gamma^*\gamma)
\\
\!\!\! &\stackrel{\rm (C)}{=}& \!\!
       \sum_{k\in\ZZ} \:  (-1)\, q^{-1}\, q^{2(k+m)}
       \: \alpha^{k+m-1}\, \gamma^m \:
       (-q^{-1} \beta) \left(\YaH{m}{k-1}X \right) \! (\gamma^*\gamma)
\\
\!\!\! &\stackrel{\rm (C)}{=}& \!\!
       \sum_{k\in\ZZ} \: q^{-2} \, q^{2(k+m)}  \: \alpha^{k+m-1}\, \gamma^m \:
       \left(\YaH{m}{k-1}X \right) \! (\gamma^*\gamma)  \, \beta
\\
\!\!\! &\stackrel{\rm (S)}{=}& \!\!
       \FLabb \!\left(\vertM \Upsilon(X)\, b^m \right) \beta.
\end{eqnarray*}
Here ({\sc m}) refers to an obvious property of our functional calculus,
involving the multiplication operator $\Psi$
(cf.\ lemma \ref{lem:functional_calc:Aq:properties}).
\end{proof}




\subsection{Even-odd structure and double covering}
\label{par:double_covering}


Eventually we shall establish a link between the even-odd structure in
(\ref{eq:def:Hcore:parity}) and the algebra $\Aqeven$ which is the one to be
considered when we really want to distinguish the quantum $E(2)$ group from its
double cover---see \S \ref{par:Haar_functionals_on_Aqext}\ for details.


\begin{lemma} \label{prop:double_covering}
The above Fourier transforms\/ $\FRabb$ and\/ $\FLabb$ map
$$\Uq \! \left(\KZeven \tens \Hcore_\tau^\scripteven \vertM\right)
        \hspace{3em} \mbox{into}  \hspace{3em}
  \Aqeven(\Hcore_\nu). $$
\end{lemma}
\begin{proof}
Take any $X \in \KZeven \tens \Hcore_\tau^\scripteven$ and recall the
observations made in the proof of lemma \ref{lemma:slice_map:parity}\@. It is
then clear from definition \ref{def:YaH}\ that $\YaH{m}{k} X = 0$ whenever $k$
is odd, since $\Gamma^{-1} \tens \Omega$ interchanges even and odd parts of
(\ref{eq:def:Hcore:parity}). This means that in the sums appearing in
definition \ref{def:Eq2:Fourier_transforms}, only the terms corresponding to
even $k$ will survive. Corollary \ref{cor:Aqeven:description}\ yields the
result.
\end{proof}
\vspace{2ex}

This investigation shall be continued at the end of \S \ref{sect:inversion}\@.
