\documentclass{book}
\usepackage{amssymb,amscd,fancyheadings}
\includeonly{}

\evensidemargin=24.5mm


\pagestyle{fancyplain}
%\addtolength{\headwidth}{\marginparsep}
%\addtolength{\headwidth}{\marginparwidth}
%\renewcommand{\chaptermark}[1]{\markboth{Chapter \thechapter.\ #1}{}}
\renewcommand{\sectionmark}[1]{\markboth{\textsc
                 Nederlandse samenvatting}{\thesection\hspace{0.6em} #1}}
\lhead[\fancyplain{}{\textsc\thepage}]{\fancyplain{}{\textsc\rightmark}}
\rhead[\fancyplain{}{\textsc\leftmark}]{\fancyplain{}{\textsc\thepage}}
\cfoot{}

%%%%%%%%%%%%%%%%%%%%%%%%
% styling & formatting %
%%%%%%%%%%%%%%%%%%%%%%%%

\parindent=0mm

\renewcommand{\theenumi}{\roman{enumi}}


%%%%%%%%%%%%%%%
% hyphenation %
%%%%%%%%%%%%%%%

\hyphenation{tensor-product tensor-products}


%%%%%%%%%%%%%%%%%%%%%%%%%%%
% frequently used SYMBOLS %
%%%%%%%%%%%%%%%%%%%%%%%%%%%

\newcommand{\qed}{\hfill \rule{4pt}{4pt}}
\newcommand{\Aa}{\ensuremath{\mathbb{A}}}
\newcommand{\BB}{\ensuremath{\mathbb{B}}}
\newcommand{\CC}{\ensuremath{\mathbb{C}}}
\newcommand{\kk}{\ensuremath{\mathbb{C}}}
\newcommand{\KK}{\ensuremath{\mathbb{K}}}
\newcommand{\EE}{\ensuremath{\mathbb{E}}}
\newcommand{\UU}{\ensuremath{\mathbb{U}}}
\newcommand{\RR}{\ensuremath{\mathbb{R}}}
\newcommand{\FF}{\ensuremath{\mathbb{F}}}
\newcommand{\NN}{\ensuremath{\mathbb{N}}}
\newcommand{\XX}{\ensuremath{\mathbb{X}}}
\newcommand{\YY}{\ensuremath{\mathbb{Y}}}
\newcommand{\ZZ}{\ensuremath{\mathbb{Z}}}
\newcommand{\EEdash}{\EE\makebox[1pt]{}-}
\newcommand{\FFdash}{\FF\makebox[1pt]{}-}
\newcommand{\EEone}{\EE_{\hspace{0.5pt}1}\!}
\newcommand{\EEtwo}{\EE_{\hspace{0.5pt}2}\!}
\newcommand{\EEop}{\EE^{\hspace{0.5pt}\rm op\!}}
\newcommand{\BBop}{\BB^{\hspace{0.5pt}\rm op\!}}


\newcommand{\op}{^{\rm op}}
\newcommand{\tens}{\otimes}
\newcommand{\fubtens}{\,\overline{\otimes}\:}
\newcommand{\slice}{\,\hat{\otimes}\,}
\newcommand{\conj}{^\circ}
\newcommand{\reduced}{^\diamond}


\newcommand{\lact}{\triangleright}
\newcommand{\ract}{\triangleleft}
\newcommand{\rtact}{\looparrowleft}
\newcommand{\ltact}{\looparrowright}

\newcommand{\Pop}{P^{\,\rm op}}
\newcommand{\lactop}{\triangleright^{\scriptscriptstyle \! \rm op \!}}
\newcommand{\ractop}{\triangleleft^{\scriptscriptstyle \rm op \!}}


\newcommand{\rarr}{\rightarrow}
\newcommand{\id}{\mbox{\rm id}}
\newcommand{\EOP}{\ensuremath{(E; \Omega, \pairing)}}

\newcommand{\lam}{\lambda}
\newcommand{\om}{\omega}
\newcommand{\Om}{\Omega}
\newcommand{\eps}{\varepsilon}
\newcommand{\lamrho}[1]{\ensuremath{(\lam_#1,\rho_#1)}}


\newcommand{\jhat}{\mbox{\textit{\^{\j}}}}
\newcommand{\vssp}{\hspace{.5pt}}

\newcommand{\andspace}[1]{\hspace{#1} {\rm en} \hspace{#1}}
\newcommand{\itandspace}[1]{\hspace{#1} \mbox{\textit{en}} \hspace{#1}}

\newcommand{\GamL}{\Gamma_{\!\scriptscriptstyle L}}
\newcommand{\GamR}{\Gamma_{\!\scriptscriptstyle R}}

\newcommand{\F}{\mathcal{F}}
\newcommand{\FL}{\mathcal{F}_{\!\scriptscriptstyle L}}
\newcommand{\FR}{\mathcal{F}_{\!\scriptscriptstyle R}}


\newcommand{\FLabb}{F_{\!\scriptscriptstyle L}}
\newcommand{\FRabb}{F_{\!\scriptscriptstyle R}}

\newcommand{\FLL}{F_{\!\scriptscriptstyle LL}}
\newcommand{\FLR}{F_{\!\scriptscriptstyle LR}}
\newcommand{\FRL}{F_{\!\scriptscriptstyle RL}}
\newcommand{\FRR}{F_{\!\scriptscriptstyle RR}}

\newcommand{\GLR}{G_{\!\scriptscriptstyle LR}}
\newcommand{\GLL}{G_{\!\scriptscriptstyle LL}}
\newcommand{\GRR}{G_{\!\scriptscriptstyle RR}}
\newcommand{\GRL}{G_{\!\scriptscriptstyle RL}}

\newcommand{\GL}{\mathcal{G}_{\scriptscriptstyle L}}
\newcommand{\GR}{\mathcal{G}_{\scriptscriptstyle R}}

\newcommand{\phiA}{\varphi_{\!\scriptscriptstyle A}}
\newcommand{\phiB}{\varphi_{\scriptscriptstyle \! B}}
\newcommand{\psiA}{\psi_{\!\scriptscriptstyle A}}
\newcommand{\psiB}{\psi_{\scriptscriptstyle \! B}}

\newcommand{\epsA}{\eps_{\scriptscriptstyle \! A}}
\newcommand{\epsB}{\eps_{\scriptscriptstyle \! B}}

\newcommand{\zetaA}{\zeta_{\scriptscriptstyle A}}
\newcommand{\zetaB}{\zeta_{\scriptscriptstyle B}}

\newcommand{\deltaA}{\delta_{\scriptscriptstyle \! A}}
\newcommand{\deltaB}{\delta_{\scriptscriptstyle \! B}}

\newcommand{\epsX}{\eps_{\scriptscriptstyle \! X}}
\newcommand{\epsY}{\eps_{\scriptscriptstyle Y}}

\newcommand{\DeltaA}{\Delta_{\scriptscriptstyle \! A}}
\newcommand{\DeltaB}{\Delta_{\scriptscriptstyle \! B}}
\newcommand{\DeltaX}{\Delta_{\scriptscriptstyle X}}
\newcommand{\DeltaY}{\Delta_{\scriptscriptstyle Y}}

\newcommand{\SA}{S_{\scriptscriptstyle \! A}}
\newcommand{\SB}{S_{\scriptscriptstyle \! B}}
\newcommand{\SX}{S_{\scriptscriptstyle \! X}}
\newcommand{\SY}{S_{\scriptscriptstyle Y}}

\newcommand{\TL}{T_{\scriptscriptstyle \! L}}
\newcommand{\TR}{T_{\scriptscriptstyle \! R}}

\newcommand{\CL}{\mathcal{C}_{\scriptscriptstyle \! L}}
\newcommand{\CR}{\mathcal{C}_{\scriptscriptstyle \! R}}

\newcommand{\mult}[1]{m_{\scriptscriptstyle #1}}

\newcommand{\idA}{\id_{\scriptscriptstyle A\!}}
\newcommand{\idB}{\id_{\scriptscriptstyle B\!}}
\newcommand{\idX}{\id_{\scriptscriptstyle X\!}}
\newcommand{\idY}{\id_{\scriptscriptstyle Y\!}}

\newcommand{\ttens}[1]{\tens_{_#1\!}}
\newcommand{\TA}[1]{T_{#1}^{\scriptscriptstyle A}}
\newcommand{\TB}[1]{T_{#1}^{\scriptscriptstyle B}}

\newcommand{\muL}{\mu_{\scriptscriptstyle L}}
\newcommand{\HopfR}{R}

\newcommand{\frakA}{\mathfrak A}
\newcommand{\frakB}{\mathfrak B}


\newcommand{\Env}{\mbox{\rm Env}}
\newcommand{\Act}{\mbox{\rm Act}}
\newcommand{\Pre}{\mbox{\rm Pre}}
\newcommand{\ActE}{\ensuremath{\Act(\mathbb{E})}}
\newcommand{\PreE}{\ensuremath{\Pre(\mathbb{E})}}
\newcommand{\EnvE}{\ensuremath{\Env(\mathbb{E})}}
\newcommand{\MME}{\ensuremath{{\mathcal M}(\mathbb{E})}}
\newcommand{\ActA}{\ensuremath{\Act(\mathbb{A})}}
\newcommand{\ActB}{\ensuremath{\Act(\mathbb{B})}}
\newcommand{\BBAA}{\ensuremath{\BB \tens \Aa}}
\newcommand{\BBopAA}{\ensuremath{\BBop \tens \Aa}}
\newcommand{\AABB}{\ensuremath{\Aa \tens \BB}}
\newcommand{\AAAA}{\ensuremath{\Aa \tens \Aa}}
\newcommand{\BBBB}{\ensuremath{\BB \tens \BB}}
\newcommand{\ActBA}{\ensuremath{\Act(\BBAA)}}



% Spacing etc.
% ------------
\newcommand{\vertXS}{\rule{0pt}{5pt}}
\newcommand{\vertS}{\rule{0pt}{1.2ex}}
\newcommand{\vertM}{\rule{0pt}{2ex}}
\newcommand{\vertL}{\rule{0pt}{2.5ex}}
\newcommand{\vertXL}{\rule{0pt}{3.5ex}}
\newcommand{\vertXXL}{\rule{0pt}{4ex}}
\newcommand{\vertUL}{\rule{0pt}{5ex}}


% Duality stuff
% -------------
\newcommand{\pairing}{\ensuremath{\langle\cdot, \cdot\rangle}}
\newcommand{\pair}[2]{\ensuremath{\langle #1,#2 \rangle}}
\newcommand{\dotpair}[1]{\pair{\,\cdot\,}{#1}}
\newcommand{\pairdot}[1]{\pair{#1}{\cdot\,}}
\newcommand{\skalprod}[2]{\ensuremath{\langle #1 \mid #2 \rangle}}
\newcommand{\varpair}[2]{\ensuremath{\{#1 \mid #2\} }}
\newcommand{\pr}{P}
\newcommand{\lp}{\ensuremath{\lam_{_P\!}}}
\newcommand{\rp}{\ensuremath{\rho_{_{\!P}}}}
\newcommand{\lQ}{\ensuremath{\lam_{_Q\!}}}
\newcommand{\rQ}{\ensuremath{\rho_{_{\!Q}}}}
\newcommand{\lpop}{\ensuremath{\lam^{\scriptscriptstyle \rm op}_{_\pr\!}}}
\newcommand{\rpop}{\ensuremath{\rho^{\scriptscriptstyle \rm op}_{_{\!\pr}}}}
\newcommand{\piP}{\ensuremath{\pi_{\scriptscriptstyle \! P}}}

\newcommand{\pairAB}{\pair{A}{B}}
\newcommand{\pairABop}{\pair{A}{B\op}}


\newcommand{\flip}{\chi}
\newcommand{\flipsub}[1]{\flip_{_{\scriptstyle #1}}}

\newcommand{\pairM}[2]{\ensuremath{\left\langle\vertM #1 , \, #2 \right\rangle}}

\newcommand{\vertflip}{\, ^{_{\scriptstyle \mid}}_{_{\!\flip\,}}}

\newcommand{\vertflipM}{|\raisebox{-1.55ex}[0pt]{\hspace{-0.4em}$\scriptstyle \flip$}}

\newcommand{\pairflip}[2]{\ensuremath{ \langle #1 \vertflip #2 \rangle}}

\newcommand{\pairflipM}[2]{\ensuremath{
     \left\langle\vertM #1 \right.  \vertflipM \left. \!  #2 \vertM\right\rangle}}

\newcommand{\sweed}[2]{{#1}_{\scriptscriptstyle (#2)}}


% Compatible topologies etc.
% --------------------------
\newcommand{\wdl}{^\flat}
\newcommand{\mdl}{^\natural}
\newcommand{\adl}{^\sharp}
\newcommand{\strictw}{strict\/$\wdl$}
\newcommand{\stricta}{strict\/$\adl$}
\newcommand{\strictm}{strict\/$\mdl$}
\newcommand{\algtp}{^\tau}
\newcommand{\Omsp}{\Omega_\sharp}
\newcommand{\Omnl}{\Omega_\natural}


% Operators on Hilbert space
% --------------------------
\newcommand{\Hs}{\ensuremath{\mathcal H}}
\newcommand{\BH}{\ensuremath{\mathcal B(\mathcal H)}}
\newcommand{\KH}{\ensuremath{\mathcal K(\mathcal H)}}
\newcommand{\TCH}{\ensuremath{\mathcal T(\mathcal H)}}
\newcommand{\FH}{\ensuremath{\mathcal F(\mathcal H)}}
\newcommand{\Tr}{\ensuremath{\mbox{\rm tr}}}



%%%%%%%%%%%%%%%%%%%%%%%%%%%
% frequently used NOTIONS %
%%%%%%%%%%%%%%%%%%%%%%%%%%%

\newcommand{\unimod}{\mbox{unital}}       %unital modules, i.e. surjective actions.
\newcommand{\Ebimod}{\mbox{$E$-bimodule}}
\newcommand{\context}{actor context}
\newcommand{\contexts}{actor contexts}
\newcommand{\biap}{bi-actor property}
\newcommand{\Cstar}{\ensuremath{C^*}}
\newcommand{\AQGs}{{\sc aqg}$^{\underline{\rm s}}$}
\newcommand{\MHAs}{{\sc mha}$^{\underline{\rm s}}$}
\newcommand{\dpa}{dual pair of algebras}
\newcommand{\idpa}{invertible dual pair of algebras}
\newcommand{\mhs}{multiplier Hopf system}
\newcommand{\mha}{multiplier Hopf algebra}
\newcommand{\ahs}{algebraic Hopf system}
\newcommand{\ahss}{algebraic Hopf $^*$-system}
\newcommand{\Hss}{Hopf $^*$-system}
\newcommand{\Ran}{\mbox{\rm Range}}



\newcommand{\diacaption}[1]{\mbox{\scriptsize {\bf Diagram}\hspace{3pt}\ #1}}

%%%%%%%%%%%%%%%%
% ENVIRONMENTS %
%%%%%%%%%%%%%%%%


% default is subsection-wise numbering, exceptional section-wise:
% -------------------------------------------------------------
\newtheorem{prop}{Propositie}[section]
\newtheorem{defn}[prop]{Definitie}
\newtheorem{lemma}[prop]{Lemma}
\newtheorem{thm}[prop]{Theorema}
\newtheorem{cor}[prop]{Gevolg}
\newtheorem{notation}[prop]{Notatie}
\newtheorem{ex}[prop]{Voorbeeld}

\newcommand{\scal}[2]{\ensuremath{\langle #1 \,|\, #2 \rangle}}
\newcommand{\skalphi}[3]{\ensuremath{\langle #1 \,|\,#2 \rangle_\varphi^{(#3)}}}

\newcommand{\Dqsqr}{D_{\!q^2}}
\newcommand{\qfragile}{$q$}
\newcommand{\qfac}[1]{[#1]_q!}
\newcommand{\Aq}{\ensuremath{{\mathcal A}_q}}
\newcommand{\Uq}{\ensuremath{{\mathcal U}_q}}
\newcommand{\UqAq}{\ensuremath{\pair{\Uq}{\Aq}}}
\newcommand{\Uqext}{\ensuremath{\Uq^{\rm ext}\!}}
\newcommand{\Aqext}{\ensuremath{\Aq^{\rm ext}\!}}
\newcommand{\Aqeven}{\ensuremath{\Aq^{\rm even}}}
\newcommand{\UqextAq}{\ensuremath{\pair{\Uqext}{\Aq}}}
\newcommand{\Zt}{\ensuremath{\ZZ \theta}}
\newcommand{\KZ}{\ensuremath{K(\Zt)}}
\newcommand{\FZ}{\ensuremath{F(\Zt)}}
\newcommand{\HC}{\ensuremath{H(\CC)}}
\newcommand{\nabq}[1]{\nabla_q^{\scriptscriptstyle (#1)}}
\newcommand{\calF}{\ensuremath{\mathcal F}}
\newcommand{\calG}{\ensuremath{\mathcal G}}
\newcommand{\calL}{\ensuremath{\mathfrak{L}}}
\newcommand{\UqT}{\ensuremath{\Uq\!\left(\calL\right)}}
\newcommand{\TG}{\ensuremath{\calL(\Gtau)}}
\newcommand{\UqTG}{\ensuremath{\Uq\!\left(\TG\vertM\right)}}
\newcommand{\AqG}{\ensuremath{\Aq(\calG)}}
\newcommand{\lameven}{\lambda_{\rm \scriptscriptstyle even}}
\newcommand{\lamodd}{\lambda_{\rm \scriptscriptstyle odd}}
\newcommand{\rhoeven}{\rho_{\rm \scriptscriptstyle even}}
\newcommand{\rhoodd}{\rho_{\rm \scriptscriptstyle odd}}
\newcommand{\chieven}{\chi_{\rm \scriptscriptstyle even}}
\newcommand{\chiodd}{\chi_{\rm \scriptscriptstyle odd}}
\newcommand{\ZZeven}{\ZZ_{\rm \scriptscriptstyle even}}
\newcommand{\ZZodd}{\ZZ_{\rm \scriptscriptstyle odd}}
\newcommand{\KZeven}{\ensuremath{K^{\rm \scriptscriptstyle even}(\Zt)}}
\newcommand{\KZodd}{\ensuremath{K^{\rm \scriptscriptstyle odd}(\Zt)}}
\newcommand{\evenodd}{{\stackrel{\scriptscriptstyle
          \rm even}{\scriptscriptstyle \! \rm odd}}}
\newcommand{\scripteven}{{\rm \scriptscriptstyle even}}
\newcommand{\scriptodd}{{\rm \scriptscriptstyle odd}}
\newcommand{\Gtau}{\ensuremath{\calG_\tau}}
\newcommand{\Geven}{\ensuremath{\Gtau^{\rm \scriptscriptstyle even}}}
\newcommand{\Godd}{\ensuremath{\Gtau^{\rm \scriptscriptstyle odd}}}
\newcommand{\FourierBABA}{\ensuremath{
 (A_0 \subseteq A, \phiA, \psiA;  \frakA, \frakB;
        B_0 \subseteq B, \phiB, \psiB )}}
\newcommand{\SH}{\ensuremath{{\mathcal S}_H}}


\newcommand{\J}[2]{J_{#1}(#2; q^2)}
\newcommand{\basis}[2]{e_{#2}^{\scriptscriptstyle (#1)}}
\newcommand{\Swq}{\ensuremath{{\mathcal S}(\RR^+;q)}}
\newcommand{\Swqbis}{\ensuremath{{\mathcal S}(\RR^+;q^2)}}
\newcommand{\Ltwoq}{\ensuremath{L^2(\RR_q^+)}}
\newcommand{\holS}[1]{{\mathcal R}_{#1}}
\newcommand{\holH}[1]{\mathfrak{H}_{#1}}
\newcommand{\auxH}[1]{{\mathcal H}_{#1}}
\newcommand{\Hmpair}{\mbox{$(m;q)$-Hankel} pair}
\newcommand{\Hintersect}{\ensuremath{\mathcal R}}
\newcommand{\Hcore}{\ensuremath{\mathcal E}}
\newcommand{\qqE}{\ensuremath{E_{q^2}^{\scriptscriptstyle \, \bullet}}}
\newcommand{\qE}{\ensuremath{E_q^{\scriptscriptstyle \,\bullet}}}
\newcommand{\til}{\widetilde{\:}}

\newcommand{\varJ}[1]{\mathfrak{J}_{#1}}
\newcommand{\varqfac}[1]{[\![ #1 ]\!]_q}
\newcommand{\varqqfac}[1]{[\![ #1 ]\!]_{q^2}}

\newcommand{\evzero}{\phi_0}
\newcommand{\kq}{\kappa_q}
\newcommand{\YaH}[2]{\mathbb{H}_{\,#1}^{\,(#2)}}

\newcommand{\Htil}[2]{\tilde{\mathbb{H}}_{\,#1}^{\,(#2)}}

\newcommand{\seqS}{S}
\newcommand{\seqM}{M}
\newcommand{\seqR}{R}

\newcommand{\seqX}{{\mathfrak X}_q}
\newcommand{\seqY}{{\mathfrak Y}_q}
\newcommand{\hatseqY}{\widehat{\mathfrak Y}_q}
\newcommand{\CZ}{\CC^{\, \ZZ\!}}

\newcommand{\little}{little}
\newcommand{\Little}{Little}




\renewcommand{\thesection}{\arabic{section}}
\parskip=1ex

\begin{document}
\chapter*{Nederlandse samenvatting}
\section{Inleiding}

Gegeven een algebra $E$ beschouwen we de {\em canonieke acties\/} van $E$ op de
duale vectorruimte $E'$. Deze zijn gedefinieerd door
$$  \pair{x}{y\lact \om} \:=\: \pair{xy}{\om} \:=\: \pair{y}{\om\ract x} $$
voor alle $x,y \in E$ en $\om \in E'$.
Hierdoor wordt $E'$ een bimodule over $E$.
Deze canonieke acties vormen de rode draad doorheen de thesis.
Zo zullen we deze acties bijvoorbeeld benutten om een {\em omhullende algebra\/} te construeren,
via een procedure analoog aan de constructie van de {\em multiplier\/}
algebra---met dit verschil dat we nu abstractie zullen maken van de bovenstaande acties,
daar waar men bij de constructie van de multiplier algebra uitgaat van de
vermenigvuldiging.

Hoewel de theorie van omhullende algebra's op zichzelf niet veel verrassends te bieden heeft,
is hij onmisbaar voor een goed begrip van de zogenaamde {\em Hopf systemen\/} die in deze
thesis worden ge\"{\i}ntroduceerd en bestudeerd.
De notie van een Hopf systeem levert een algemeen algebra\"{\i}sch kader voor dualiteit,
waarbinnen de reeds bekende theorie van {\em multiplier Hopf algebra's met integralen\/}
\cite{Fons:AFGD,Fons:AFGD:proc,Fons:pnas}\ een heel natuurlijke plaats inneemt.
Ook hier weer spelen canonieke acties een essenti\"ele rol in de opbouw van de theorie. \\
Een van de in de toepassingen meest gewaardeerde resultaten uit de klassieke theorie van
Hopf algebra's is de befaamde {\em quantum double\/} constructie van \mbox{Drinfel'd}\@.
Ook voor een ruime klasse van Hopf systemen blijkt het mogelijk een dergelijke
quantum double te construeren.


Tenslotte bestuderen we het voorbeeld van de kwantum $E(2)$ groep, die
verkregen wordt als een kwantum deformatie van de groep van ori\"entatie-bewarende
isometrie\"en van het Euclidisch vlak.
We laten zien hoe deze kwantum groep zich laat beschrijven in termen van een
duaal paar Hopf algebra's en twee Hopf systemen.
Opnieuw is een cruciale rol weggelegd voor de bovenvermelde canonieke acties,
die we hier dan ook concreet zullen uitrekenen.
Bovendien geven we expliciete formules voor de {\em Fourier transformaties\/} tussen
kwantum $E(2)$ en zijn duale, bewijzen de Plancherel formules, etc.
Daarnaast ontwikkelen we ook een axiomatisch algebra\"{\i}sch kader voor harmonische
analyse, gemodelleerd naar dit kwantum $E(2)$ voorbeeld.


\section{De omhullende algebra}

\begin{defn} \rm
Zij gegeven een algebra $E$ en een \Ebimod\ $\Om$, samen met een niet-ontaarde
vectorruimte dualiteit $\pairing : E \times \Om \rarr \kk$.
Zij verder $E'$ uitgerust met zijn canonieke \Ebimod\ structuur.
Als de canonieke inbedding $\Om \rarr E'$ een \Ebimod\ morfisme is,
dan noemen we \EOP\ een {\em\context}\@.
\end{defn}


\begin{defn} \rm
Zij $\EE \equiv \EOP$ een \context, en beschouw twee lineaire afbeeldingen $\lam$ en $\rho$
van $\Om$ naar $\Om$. We zeggen dan dat het paar $(\lam,\rho)$ een {\em pre-actor\/}
is voor \EE\ wanneer $\lam$ een rechts $E$-module morfisme is en $\rho$ een
links $E$-module morfisme. Als bovendien
\begin{equation} \label{biap}
   \pair{x}{\rho(\om \ract y)}  \:=\:  \pair{y}{\lam(x\lact\om)}
\end{equation}
voor alle $x,y \in E$ en $\om \in \Om$, dan noemen we $(\lam,\rho)$
een {\em actor\/} voor \EE\@.
De actoren voor \EE\ vormen duidelijk een vectorruimte, die we zullen noteren met \ActE\@.
De vectorruimte van alle pre-actoren duiden we aan met \PreE\@.
\end{defn}


In wat volgt is $\EE \equiv \EOP$ een willekeurige \context\@.

\begin{lemma} \label{lem:embedding_of_E}
Gegeven een\/ $x\in E$ defini\"eren we afbeeldingen\/ $\lam_x, \rho_x : \Om \rarr \Om$
door\/ $\lam_x(\om) = x \lact \om$ en\/ $\rho_x(\om) = \om \ract x$.
Dan zal\/ $\lamrho{x} \in \ActE$.
\end{lemma}


\begin{defn} \rm \label{def:product_of_actors}
Beschouw twee pre-actoren voor \EE, laat ons zeggen
$x_1 \equiv \lamrho{1}$ en $x_2 \equiv \lamrho{2}$.
Duidelijk is $(\lam_1\lam_2,\rho_2\rho_1)$ opnieuw een pre-actor voor \EE,
die we zullen noteren met $x_1 x_2$.
Dit definieert een vermenigvuldiging die van \PreE\ een algebra maakt
met eenheid $1_\EE \equiv (\id_\Om,\id_\Om)$.
\end{defn}


Helaas is \ActE\ als deelruimte van \PreE\ niet gesloten onder deze bewerking.
Met andere woorden, \ActE\ is geen algebra. Wel hebben we het volgende:

\begin{lemma}
Als de vermenigvuldiging in\/ $E$ niet gedegenereerd is, dan is de afbeelding\/
$E \rarr \PreE: x \mapsto \lamrho{x}$, zoals gedefinieerd door
lemma \ref{lem:embedding_of_E}, een injectief algebra morfisme.
\end{lemma}

In dit geval identificeren we $E$ met de corresponderende deelalgebra van \PreE\@.

\begin{defn} \rm
We defini\"eren de volgende deelruimte van \PreE:
$$ \EnvE  \:=\:  \left\{x\in\PreE  \,\left|\vertM\right.\,
                       x \ActE \subseteq \ActE \supseteq \ActE x \right\}. $$
Als de vermenigvuldiging in $E$ niet-ontaard is, dan defini\"eren we analoog
$$ M(\EE) \:=\:  \left\{x\in\PreE  \,\left|\vertM\right.\,
                       x E  \subseteq E  \supseteq E x \right\}. \hspace{6em} $$
\end{defn}


\begin{lemma} \label{lem:algebras_van_actoren}
Veronderstel dat\footnote{We zeggen dat $\Om$ {\em unitaal\/} is als \Ebimod;
het product in $E$ is dan a fortiori niet-ontaard.}
$E \lact \Om = \Om = \Om \ract E$.
Dan zijn\/ \EnvE\ en\/ $M(\EE)$ unitale algebra's van actoren.
Bovendien is\/ $M(\EE)$ bevat in\/ \EnvE\ en isomorf met de multiplier algebra van $E$.
Samengevat:
$$ E \,\subseteq\, M(E)  \,\simeq\,    M(\EE) \,\subseteq\, \EnvE
     \,\subseteq\, \ActE \,\subseteq\, \PreE. $$
\end{lemma}


\begin{ex} \rm
Zij $S$ een niet-lege verzameling en $\CC S$ de vrije vectorruimte
voortgebracht door $S$. Zij $F$ een separerende algebra van complexe functies
\mbox{op $S$}\@.
De natuurlijke dualiteit tussen $F$ en $\CC S$
is gegeven door $\pair{f}{\delta_s} = f(s)$ waarbij $\{\delta_s\}_{s\in S}$
de canonieke basis is van $\CC S$.
Het triplet $\FF\equiv(F; \kk S, \pairing)$ is dan een \context,
en ${\rm Act}(\FF) = {\rm Env}(\FF)$ is isomorf met de algebra van {\em alle\/}
complexwaardige functies op $S$.
\end{ex}


\begin{defn} \rm
Zij $E$ uitgerust met de zwakke topologie ge\"{\i}nduceerd door de
dualiteit $\pair{E}{\Om}$ en definieer een deelruimte $\Om\adl$ van de
duale van $\Om$ als volgt:
$$ \Om\adl \;=\;
     \left \{ f \in \Om' \left|
          \begin{array}{c}
            \mbox{zowel $\pair{f}{(\cdot) \lact \om}$
                  als $\pair{f}{\om \ract (\cdot)}$ zijn zwak continue}   \\
            \mbox{lineaire functionalen op $E$, en dit voor alle $\om\in \Om$}
          \end{array} \right. \!\!\! \right\}.     $$
\end{defn}


\begin{prop} \label{prop:induced_actor}
Er bestaat een unieke lineaire afbeelding
$$\theta: \Om\adl \rarr \ActE: f \mapsto \theta(f)\equiv\lamrho{f} $$
zodat voor alle\/ $x\in E$, $f \in \Om\adl$ en\/ $\om \in \Om$ geldt dat\/
$$\pair{x}{\rho_f(\om)} = \pair{f}{x \lact \om}
  \andspace{4em}
  \pair{x}{\lam_f(\om)} = \pair{f}{\om \ract x}.$$
\end{prop}


\begin{defn}\label{def:EEinvertible} \rm
Een lineaire functionaal $f:\Om\rarr \kk$ noemen we \EEdash {\em inverteerbaar\/} als
hij voldoet aan de volgende eisen:
\begin{enumerate}
\item
De functionaal $f$ behoort tot $\Om\adl$.
\item
De actor $\theta(f)\equiv\lamrho{f}$ is inverteerbaar binnen de algebra \PreE\@.
De afbeeldingen $\lam_f$ en $\rho_f$ zijn m.a.w.\ bijecties van $\Om$ naar $\Om$.
\item
De pre-actor $\theta(f)^{-1}$ is weer een actor voor \EE, of nog,
het paar $(\lam_f^{-1},\rho_f^{-1})$ voldoet opnieuw aan vergelijking (\ref{biap}).
\end{enumerate}
\end{defn}


\begin{defn} \rm
We noemen $\eps \in \Om'$ een {\em co-eenheid\/} voor \EE\ indien
$$ \pair{\eps}{x\lact\om}    \;=\;  \pair{x}{\om}  \;=\;   \pair{\eps}{\om \ract x} $$
voor alle $\om \in \Om$ en $x\in E$.
Als er voor \EE\ zo een co-eenheid bestaat, en als bovendien
$E \lact \Om = \Om = \Om \ract E$, dan noemen we de \context\ \EE\ {\em zwak unitaal}\@.
In dat geval is de co-eenheid uniek.
\end{defn}


\begin{prop} \label{prop:theta:bijectie}
Als\/ \EE\ zwak unitaal is, dan is de afbeelding\/ $\theta: \Om\adl \rarr \ActE$ een bijectie.
\end{prop}

\begin{cor}
Als\/ \EE\ zwak unitaal is, dan kan de paring $\pair{E}{\Om}$ op natuurlijke
wijze uitgebreid worden tot een paring tussen\/ \ActE\ en\/ $\Om$, als volgt:
$$ \pair{\eps}{\lam(\om)} = \pair{a}{\om} = \pair{\eps}{\rho(\om)},
           \hspace{2em} \mbox{of nog,} \hspace{2em}
     \pair{\theta(f)}{\om} = \pair{f}{\om}  $$
voor alle\/ $a\equiv(\lam,\rho)\in \ActE$, alle\/ $\om\in\Om$ en $f\in\Om\adl$.
\end{cor}

\begin{defn} \rm
Beschouw twee actor contexten
$$ \EEone \equiv (E_1; \Om_1,\pairing)
         \andspace{3em}
   \EEtwo \equiv (E_2; \Om_2,\pairing). $$
Beschouw ook de voor de hand liggende paring tussen $E_1 \tens E_2$ en $\Om_1 \tens \Om_2$.
Het is dan duidelijk dat
$$\EEone \tens\, \EEtwo    \;\equiv\;
          \left(E_1 \tens E_2; \:\Om_1 \tens \Om_2, \pairing\vertM\right)$$
opnieuw een actor context definieert.
\end{defn}


\begin{lemma}
$\;\EEone \tens\, \EEtwo$ is zwak unitaal als en slechts als\/ $\EEone$ en\/ $\EEtwo$
beide zwak unitaal zijn.
\end{lemma}




\section{Hopf systemen}

\begin{defn} \rm
Zij \pairing\ een niet-ontaarde paring tussen twee algebra's $A$ en $B$\@.
Als nu
$$ \Aa \equiv (A; B, \pairing)   \andspace{3em}
   \BB \equiv (B; A, \pairing) $$
beide actor contexten zijn, of met andere woorden, als $A$ en $B$ invariant zijn onder
elkaars canonieke acties, dan noemen we \pairAB\ een {\em duaal paar algebra's}\@.
Interpreteer vervolgens de paring tussen $A$ en $B$ als een lineaire functionaal
$$ P : A \tens B \rarr \kk: a \tens b \mapsto \pair{a}{b}. $$
We zeggen nu dat zo een duaal paar algebra's \pairAB\ {\em inverteerbaar\/} is, wanneer $P$ als
functionaal $(\BBAA)$-inverteerbaar is, in de zin van definitie \ref{def:EEinvertible},
waarbij
$$ \BBAA  \,\equiv\,  \left(\vertM  B \tens A; \, A \tens B, \pairing \right). $$
\end{defn}


In wat volgt zullen we steeds veronderstellen dat \pairAB\ een
inverteerbaar duaal paar algebra's is.

\begin{lemma}
De actor contexten\/ $\Aa$ en\/ $\BB$ zijn zwak unitaal.
\rm De co-eenheden op $A$ en $B$ noteren we met $\epsA$ en $\epsB$.
\end{lemma}


\begin{cor}
Ook tensorproducten van $\Aa$ en\/ $\BB$, bijvoorbeeld\/ $\BBAA$, zijn zwak unitaal.
Overeenkomstig propositie \ref{prop:theta:bijectie}\ kunnen we de paring\/ $P$ identificeren
met de corresponderende actor:
$$ P \,\simeq\, \theta(\pr) \,=\, (\lp\,,\rp).  $$
\end{cor}


Verder weten we dat de afbeeldingen\/ \lp\ en\/ \rp\ bijecties zijn van\/
$A \tens B$ op\/ $A \tens B$, gegeven door
$$ \begin{array}{rcl}
   \pairM{d \tens c}{\lp(a \tens b)} &=& \pairM{a \ract d}{b \ract c}
     \\
     \pairM{d \tens c}{\rp(a \tens b)} &\vertL=& \pairM{d \lact a}{c \lact b}
   \end{array} $$
voor $a,c \in A$ en $b,d \in B$.

Bovendien zal $P^{-1} = (\lp^{-1},\rp^{-1})$ opnieuw een actor zijn voor $\BBAA$.



\begin{prop}
Er bestaan unieke lineaire afbeeldingen
$$ \begin{array}{lcl}
   \DeltaA : A \rarr \Env(\AAAA)  & &  \pairM{\DeltaA(a)}{b \tens d} = \pair{a}{bd}
   \\
   &\hspace{2em} \mbox{\it  zodat} \hspace{2em}& \\
   \DeltaB : B \rarr \Env(\BBBB)  & &  \pairM{a \tens c}{\DeltaB(b)} = \pair{ac}{b}
   \end{array} $$
voor alle\/ $a,c \in A$ en\/ $b,d \in B$.
{\rm We noemen $\DeltaA$ en $\DeltaB$}\ co-vermenigvuldigingen.
\end{prop}



\begin{prop}
Er bestaan unieke lineaire afbeeldingen
$$ \SA : A \rarr \Env(\Aa)     \itandspace{4em}
   \SB : B \rarr \Env(\BB). $$
zodat
$$ \pair{\SA(a)}{b} \:=\: \pair{P^{-1}}{a \tens b} \:=\: \pair{a}{\SB(b)}$$
voor alle\/ $a \in A$ en\/ $b \in B$.
{\rm We noemen $\SA$ en $\SB$} antipoden.
\end{prop}

Het is soms handig om de paring $\pair{B \tens A}{A \tens B}$ te schrijven als
een paring van $A \tens B$ met zichzelf.
Om verwarring te vermijden, zullen we deze laatste noteren met $\pairflip{\,\cdot}{\cdot\,}$,
dit wil zeggen, voor $x,y \in A \tens B$ defini\"eren we $\pairflip{x}{y} = \pair{\flip(x)}{y}$.
Hierbij is $\flip$ de flip $A \tens B \rarr B \tens A$.


\begin{defn} \rm
We noemen de paring $P$ {\em multiplicatief\/} indien
$$ \pairM{ac}{bd} \:=\: \pairflipM{\rp(a \tens b)}{\lp(c\tens d)}$$
voor alle $a,c \in A$ en $b,d \in B$.
\end{defn}


\begin{defn} \rm
Een {\em Hopf systeem\/} is een inverteerbaar duaal paar algebra's met
multiplicatieve paring, in de zin van bovenstaande definitie.
\end{defn}


\begin{prop}
Als\/ \pairAB\ een Hopf systeem is, dan zijn zowel de co-eenheden als
co-vermenigvuldigingen op\/ $A$ en\/ $B$ algebra homomorfismen.
De antipoden zijn anti-homomorfismen.
\end{prop}


De volgende twee proposities tonen hoe multiplier Hopf algebra's passen binnen het
kader van Hopf systemen:


\begin{prop}
Zij\/ $(A,\Delta)$ een willekeurige \mha\ en zij $A\reduced$ zijn
gereduceerde duale---gedefinieerd als $A\reduced = A\lact A' \ract A$.
Dan is $\pair{A}{A\reduced}$ een Hopf systeem.
\end{prop}


\begin{prop} \label{prop:AhatA:Hopf_system}
Zij\/ $(A,\Delta)$ een reguliere \mha\ met niet-triviale invariante functionalen,
en zij\/ $(\hat{A},\hat{\Delta})$ de duale \mha\ zoals gedefinieerd in \cite{Fons:AFGD}\@.
Dan is $\pair{A}{\hat{A}}$ een Hopf systeem.
\end{prop}

Net zoals bij multiplier Hopf algebra's kan men ook voor Hopf systemen een notie
van {\em regulariteit\/} defini\"eren en $^*$-{\em structuur\/} toevoegen.
We gaan hieromtrent niet in detail.
Wel vermelden we dat in het reguliere geval de antipoden $\SA$ en $\SB$
{\em bijecties\/} zijn op $A$ en $B$ respectievelijk, en dat het duaal paar
$\pair{A}{B\op}$ dan eveneens een Hopf systeem is.
Zo is bijvoorbeeld het Hopf systeem uit de vorige propositie automatisch regulier.

Hopf systemen afkomstig van reguliere multiplier Hopf algebra's met integralen,
zoals in propositie \ref{prop:AhatA:Hopf_system}, nemen een zeer natuurlijke
plaats in binnen het bredere kader van de Hopf systemen:


\begin{defn} \rm
We zeggen dat een Hopf systeem \pairAB\ een {\em multiplier\/} Hopf systeem is
wanneer $P$ en $P^{-1}$ behoren\footnote{Zie lemma \ref{lem:algebras_van_actoren}\@.}\
tot $M(B \tens A)$.
\end{defn}


\begin{thm} \label{thm:mhs}
Hopf systemen zoals\/ $\pair{A}{\hat{A}}$ in propositie \ref{prop:AhatA:Hopf_system}\
zijn reguliere multiplier Hopf systemen.
Omgekeerd, als\/ \pairAB\ een regulier multiplier Hopf systeem is,
dan zijn $(A,\DeltaA)$ en\/ $(B,\DeltaB)$ reguliere \mha's met invariante functionalen.
Bovendien zal\/ $B\simeq \hat{A}$ en\/ $A\simeq \hat{B}$ in de zin van \cite{Fons:AFGD}\@.
\end{thm}


Als \pairAB\ een Hopf systeem is, dan beschikken we over een algebra morfisme
$$ \piP : B \tens A \rarr \Pre(\BBAA): y \mapsto P y P^{-1}. $$

\begin{defn} \rm
We zeggen dat een Hopf systeem \pairAB\ {\em gebalanceerd\/} is indien
$(B\tens A)P = P(B\tens A)$.
In dat geval wordt $\piP$ dus een automorfisme van $B \tens A$.
\end{defn}

Zo is bijvoorbeeld elk multiplier Hopf systeem a fortiori gebalanceerd.

Zij \pairAB\ een willekeurig gebalanceerd regulier Hopf systeem.
We gaan nu voor zo een duaal paar een {\em quantum double\/} construeren:

De afbeelding \mbox{$T=\rp\lp^{-1}\flip$}
is een lineaire bijectie van $B \tens A$ op $A \tens B$.
\mbox{Deze $T$}\ is bovendien een twist-afbeelding met betrekking tot de algebra's $A$ en $B\op$.
Zij nu $X=A \ttens{T} B\op$ de corresponderende getwiste tensorproduct algebra,
d.w.z.\ met vermenigvuldiging
$\mult{X} = (\mult{A} \! \tens  \mult{B}\op)(\id \tens T \tens \id)$,
en beschouw anderzijds ook de gewone tensorproduct algebra $Y=B\tens A$.
Zij tenslotte de paring $\pair{X}{Y}$ gegeven door de gewone vectorruimte
dualiteit tussen $A \tens B$ en $B \tens A$.

\begin{thm}
Zij\/ \pairAB\ een gebalanceerd regulier Hopf systeem.
Dan is\/ $\pair{A \ttens{T} B\op}{B \tens A} \,\equiv\, \pair{X}{Y}$
opnieuw een regulier Hopf systeem.\\
Zij nu\/ $Q\equiv(\lQ\, ,\rQ)$ de actor voor\/ $\YY \tens \XX$, afgeleid van
het duaal paar\/ $\pair{X}{Y}$. Dan zijn\/ $\lQ$ en\/ $\rQ$ gegeven door
\begin{eqnarray*}
\lQ &=& (\idX \tens \piP^{-1}) \, (\lp)_{13} \, (\idX \tens \piP) \, (\lpop)_{42}
\\
\rQ  &=& (\idX \tens \piP^{-1}) \, (\rpop)_{42} \, (\idX \tens \piP) \,  (\rp)_{13}
\vertXL
\end{eqnarray*}
waarbij\/ $(\lpop,\rpop)$ de actor is voor\/ $\BB\op \tens\, \Aa$,
afgeleid van het duaal paar\/ $\pair{A}{B\op}$.
De \lq leg-numbering\rq\ notatie heeft betrekking op het tensorproduct\/
$A \tens B \tens B \tens A$. \\
Co-eenheden en antipoden zijn gegeven door
$$\begin{array}{lcl}
  \epsX = \epsA \tens \epsB   &\hspace{2em}&  \SX = T\flip(\SA \tens \SB^{-1})    \\
  \epsY = \epsB \tens \epsA   &            &  \SY = (\SB \tens \SA^{-1})\, \piP. \vertL
\end{array} $$
\end{thm}

\begin{prop} \label{prop:QD:rmhs}
Als\/ \pairAB\ een regulier multiplier Hopf systeem is, dan is de quantum double\/
$\pair{A \ttens{T} B\op}{B \tens A}$ weer een regulier multiplier Hopf systeem.
\end{prop}

Gecombineerd met theorema \ref{thm:mhs}\ levert de bovenstaande propositie ons een
alternatief voor de quantum double constructie voor multiplier Hopf algebra's met
integralen in \cite{FonsDra:QD}\@.



\section{Harmonische analyse op kwantum $E(2)$}

Op het Hopf $^*$-algebra niveau kan men de kwantum $E(2)$ groep beschrijven als
een duaal paar Hopf $^*$-algebra's \UqAq\@. Hierbij dient men \Aq\ te
beschouwen als een gekwantiseerde functie algebra, terwijl \Uq\ gezien moet
worden als een gekwantiseerde universeel omhullende algebra van een Lie
algebra. Als $^*$-algebra wordt \Uq\ gegenereerd door een zelftoegevoegd
inverteerbaar element $a$ en een normaal element $b$, met de commutatieregel
$ab = q \,ba$ voor een re\"eel getal $q$ met $0<q<1$. Analoog wordt \Aq\
gegenereerd door een unitair element $\alpha$ en een normaal element $\beta$
met $\alpha\beta = q\,\beta\alpha$.
De co-vermenigvuldigingen op \Uq\ en \Aq\ zijn gegeven door
$$ \begin{array}{lcl}
   \Delta(a) = a \tens a
&\hspace{5em}&
   \Delta(\alpha) = \alpha \tens \alpha
\\
   \Delta(b) = a \tens b + b \tens a^{-1}
&&
   \Delta(\beta) = \alpha \tens \beta + \beta \tens \alpha^{-1}
   \vertL
\end{array} $$
en de dualiteit tenslotte is gegeven door
$$  \pairM{a^p b^r c^s}{\alpha^l \beta^m \gamma^n}
        \;=\;
    \delta_{r,m}\,  \delta_{s,n} \: q^{\frac{1}{2}p(l+m-n)} \,
                         q^{\frac{1}{2}l(m+n)} \,\qfac{m}\, \qfac{n} $$
waarbij $c=b^*$ en $\gamma=-q^{-1}\beta^*$. De $q$-faculteitsfunctie is hier gedefinieerd door
$$  \qfac{m}  \:=\:  [1]_q [2]_q \ldots [m]_q
                \hspace{5em} \qfac{0} = 1, $$
waarbij
$$  [m]_q  \:=\:  \frac{q^m - q^{-m}}{q - q^{-1}}.  $$
Het probleem met deze twee Hopf algebra's \Aq\ en \Uq\ is echter dat ze geen
Haar functionalen toelaten:
op het Hopf algebra niveau werken we immers met {\em polynomiale\/} functies in de
generatoren, daar waar we---om over Haar integralen te kunnen beschikken---eerder zoiets
nodig hebben als functies die {\em naar nul gaan\/} op oneindig.
De constructie van dit soort \lq integreerbare\rq\ functies in de generatoren
vereist echter een meer geavanceerde calculus, die men op
verscheidene manieren kan verkrijgen. E\'en methode bestaat erin eerst een representatie
op een Hilbertruimte te construeren, om dan vervolgens operatorentheorie in te schakelen
\cite{Fons:spectral_conditions,FonsWor:QE2,Wor:QE2,Wor:Affiliated,Wor:operatoreq}\@.
Een andere mogelijkheid is een holomorfe calculus te ontwikkelen, gebaseerd op
formele {\em machtreeksen}\@.
Het is de tweede strategie die we zullen volgen om een precieze betekenis te
geven aan uitdrukkingen zoals
$$ \begin{array}{lcr}
    f(\ln a)\, g(b^* b)\: b^m  &  \hspace{3em} \mbox{Fourier} \hspace{3em}&
       \alpha^k  \gamma^n \:  h(\gamma^*\gamma) \\
    f(\ln a)\, g(b^* b)\: c^m  &
      \stackrel{\displaystyle \leftrightarrows}{\mbox{transformatie}} &
           \alpha^k  (\gamma^*)^n  \: h(\gamma^*\gamma)
   \end{array}$$
\begin{equation}\label{eq:diagram:Fourier}
\diacaption{Functies in de generatoren}
\end{equation}
waarbij $f$, $g$ en $h$ de gepaste functieruimten doorlopen.
Op dergelijke elementen zijn de linkse Haar integralen dan gegeven door
\begin{eqnarray*}
  \varphi \!\left(  f(\ln a)\, g(b^* b)\: b^m  \vertM\right)
     &=& \delta_{m,0} \!\!\! \sum_{\stackrel{\scriptstyle k,l\,\in\, \ZZ}{\scriptstyle k-l\: {\rm even}}}
            f(k \theta) \, g(\tau q^l) \, q^{k+l}  \\
  \omega \!\left(\alpha^k \gamma^n \:  h(\gamma^*\gamma)\vertM\right)
           &=& \delta_{k,0}\:\delta_{n,0} \: \sum_{j\in\ZZ} \, h(\nu q^{2j})\, q^{2j},
\end{eqnarray*}
uiteraard op voorwaarde dat $f$, $g$ en $h$ aan bepaalde sommeerbaarheidscriteria voldoen.
Hierbij zijn $\tau$ en $\nu$ willekeurige positieve getallen en
\mbox{$\theta = -\frac{1}{2} \ln q$}\@.
Deze Haar integralen zijn positief, trouw, {\sc kms}, etc.


Onze hoofddoelstelling is nu de constructie van de {\em Fourier transformaties\/}
die de elementen in de linkerkolom van diagram (\ref{eq:diagram:Fourier}) transformeren
in lineaire combinaties van elementen in de rechterkolom en vice versa. Hoewel de volgende formule niet helemaal
precies\footnote{Deze formule is niet volledig exact in die zin dat ze onvoldoende
rekening houdt met de zogenaamde spectrale condities en de daarmee
verbonden sommeerbaarheids\-voor\-waarden op $f$ en $g$.
Het zou ons echter te ver voeren om hieromtrent in detail te gaan.}\
is, geeft ze toch een goed idee van hoe zo'n Fourier transformatie er ongeveer uitziet:
$$ \begin{array}{c}
    f(\ln a)\, g(b^* b)\, b^m
\\
\downarrow\vertXL
\\  \displaystyle
    \sum_{k\in\ZZ} \:\: (-1)^m q^{-m} q^{\frac{1}{2}m(k-1)} (q^{-1}-q)^m \:
       f(k \theta) \;  \alpha^{k+m} \, \gamma^m \;  h_{m,k}(\gamma^*\gamma).\vertXL
\end{array}$$
Hierbij is $h_{m,k}$ in essentie een $m$-de orde $q$-Hankel getransformeerde van $g$.
Meer expliciet hebben we
$$ h_{m,k}(\nu q^{2r})
       \;=\; \sum_{n\in\ZZ} \: q^{2n} \, q^{m(n-r)}\, \J{m}{q^{n+r}} \, g(\tau q^{2n+k}) $$
waarbij $\J{m}{\,\cdot\;}$ de Hahn-Exton \mbox{$q^2$-Bessel}\ functie is van orde $m$,
gegeven door
$$ \J{m}{z} \:=\; \sum_{k=0}^{\infty} \:  \frac{(-1)^k\,q^{k(k-1)}\,
     q^{2k}\,(q^{2(k+m+1)}; q^2)_\infty}{(q^2;q^2)_\infty  \,(q^2;q^2)_k} \; z^{2k+m} $$
met $(x;q^2)_k = \prod_{j=0}^{k-1} (1-q^{2j} x)$.

Ruw gesproken komt deze kwantum $E(2)$ Fourier transformatie dus neer op
een $q$-Hankel transformatie tussen de functies $g$ en $h$.
Dit is al bij al niet z\'o verrassend, aangezien Hankel transformatie typisch verschijnt
daar waar Fourier transformatie van het vlak wordt uitgedrukt in {\em pool\/}-co\"ordinaten;
onze calculus in de generatoren $(b,c)$ vertoont inderdaad een soort {\em polaire\/}
decompositie---idem voor de generatoren $(\beta,\gamma)$.


Verder blijkt dat we alleen maar een echte Fourier transformatie verkrijgen
als de parameters $q$, $\tau$ en $\nu$ op een welbepaalde manier
aan elkaar gerelateerd zijn, namelijk
$$   \tau = q^{-1}   \andspace{4em}   \nu = (q^{-1}-q)^{-2}. $$
Vervolgens bewijzen we de {\em Plancherel\/} formules.
Hierbij is het essentieel om het op het eerste zicht wat merkwaardige
sommatie-bereik in de formule voor de Haar \mbox{functionaal $\varphi$} te
gebruiken---anders zouden de Plancherel formules eenvoudigweg niet opgaan.
Dit betekent dat $f\tens g$ in feite \lq leeft\rq\ op de verzameling
$$ \left\{ \, (k \theta, \tau q^l)   \, \left|\vertM\right. \,
            k,l\in \ZZ \mbox{ met } k-l \mbox{ even} \, \right\}  $$
wat doet denken aan de {\em spectrale condities\/} in de \mbox{C$^*$}-algebra\"{\i}sche
aanpak \cite{Wor:QE2}\@. Daarnaast suggereert dit ook dat het handiger en
natuurlijker is om de functies $f$ en $g$ in een enkel object te verenigen,
met andere woorden, om $f\tens g$ echt te beschouwen als \'e\'en functie in twee veranderlijken,
veeleer dan twee functies in \'e\'en veranderlijke.

Uiteindelijk kunnen we onze formules voor de Fourier transformatie gebruiken om
expliciet de {\em dualiteit\/} uit te rekenen tussen \lq integreerbare\rq\ functies
in de generatoren van {\em zowel\/} \Uq\ als \Aq\@.
Zo hebben we bijvoorbeeld:
\begin{eqnarray*}
      \lefteqn{\pairM{\vertL f(\ln a)\, g(b^* b)\, b^m}{\alpha^l (\gamma^*)^n \,h(\gamma^*\gamma)}}
\\
&\vertXL=&
      \delta_{m,n} \: (-1)^m \, q^{-m}\, q^{-\frac{1}{2}m(m+l+1)}\,
                      (q^{-1}-q)^m \,  \nu^m \, q^{ml} \ldots
\\
&\vertXL&
      \ldots  \sum_{r,k\in \ZZ}   q^{(m+2)(k+r)} \, \J{m}{q^{k+r}} \:
              f(-m\theta-l\theta) \: g(\tau\, q^{2k-m-l})  \: h(\nu\, q^{2r+2l}).
\end{eqnarray*}




\newpage

\begin{thebibliography}{99}
\bibitem{Abe}
   {\sc E. Abe},
   {\em Hopf algebras}\@.
   Cambridge Univ.\ Press, London and New York (1977).

\bibitem{Arens} {\sc R. Arens}
   {\em The adjoint of a bilinear operation\/},
   Proc.\ Amer.\ Math.\ Soc.\ {\bf 2}, 839-848 (1951).

\bibitem{Baaj:QE2} {\sc S. Baaj},
   {\em Repr\'esentation r\'eguli\`ere du groupe quantique $E_\mu(2)$ de Woronowicz}\@.
   Comptes Rendus Acad. Sci. Paris, S\'erie I {\bf 314}, 1021-1026 (1992).

\bibitem{BS}  {\sc S. Baaj \&\ G. Skandalis},
   {\em Unitaires multiplicatifs et dualit\'e pour les produits
        crois\'es de C$^*$-alg\`ebres}\@.
   Ann. scient. \'Ec. Norm. Sup., Serie 4, {\bf 26}, 425-488 (1993).

\bibitem{Bonsall_Duncan} {\sc F. F. Bonsall \&\ J. Duncan}
   {\em Complete Normed Algebras}\@.
   Ergebnisse de Mathematik und ihrer Grenzgebiete, Band 80, Springer-Verlag (1973).

\bibitem{civin_yood} {\sc P. Civin \&\ B. Yood},
   {\em The second conjugate space of a Banach algebra  as an algebra \/},
   Pac. J. Math. 11, 847-870 (1961).

\bibitem{FonsDra:QD} {\sc B. Drabant \&\ A. Van Daele},
   {\em Pairing and Quantum Double of Multiplier Hopf Algebras}\@.
   To appear in Algebras and Representation theory.

\bibitem{FonsDraZhang:actions}
   {\sc B. Drabant \&\ A. Van Daele \&\ Y. Zhang},
   {\em Actions of Multiplier Hopf Algebras}\@.
   Comm. in Algebra, {\bf 27}(9), 4117-4172 (1999).

\bibitem{EffrosRuan} {\sc E.G. Effros \&\ Z.-J. Ruan}
   {\em Discrete Quantum Groups I, the Haar measure}\@.
   Int. Journal of Math. 681-723 (1994).

\bibitem{Jarchow} {\sc H. Jarchow},
   {\em Locally Convex Spaces}\@.
   B.G. Teubner Stuttgart (1981).

\bibitem{Schmudgen} {\sc A. Klimyk \&\ K. Schm\"udgen},
   {\em Quantum groups and their representations}\@.
   Texts and Monographs in Physics, Springer-Verlag (1997).

\bibitem{Koelink:thesis} {\sc H.T. Koelink}
   {\em On quantum groups and $q$-special functions}\@.
   Ph. D. thesis, Rijksuniv. Leiden (1991).


\bibitem{Koelink:QE2} {\sc H.T. Koelink},
   {\em The quantum group of plane motions and the Hahn-Exton $q$-Bessel function\/},
        Duke Math. Journal Vol. {\bf 76}, No.\ 2 (1994).

\bibitem{Koelink:Stokman} {\sc H.T. Koelink \&\ J.V. Stokman},
   {\em Fourier transforms on the quantum $SU(1,1)$ group}\@.
    Institut de Math. de Jussieu, Unit\'e Mixte de Recherche 7586,
    Universit\'es Paris VI et Paris VII, {\sc cnrs},
    pr\'epublication 232 (1999).

\bibitem{KoelinkSwart:qBessel:zeros} {\sc H.T. Koelink \&\ R.F. Swarttouw},
   {\em On the zeros of the Hahn-Exton \mbox{$q$-Bessel}\ function and associated $q$-Lommel
   polynomials}\@. J.\ Math.\ Anal.\ Appl.\ {\bf 186}, No. 3, 690-710 (1994).

\bibitem{Koornwinder} {\sc Tom H. Koornwinder},
   {\em Special functions and $q$-commuting variables}\@.
   Fields Institute communications,  Amer. Math. Soc\@.  {\bf 14}, 131-166 (1997).

\bibitem{KoornwSwartt} {\sc T.H. Koornwinder \&\ R.F. Swarttouw},
   {\em On $q$-analogues of the Fourier and Hankel Transforms}\@.
   Trans.\ Amer.\ Math.\ Soc.\ {\bf 333}, 445-461 (1992).

\bibitem{Kothe} {\sc G. K\"othe}
   {\em Topological Vector Spaces, Vol. I \& II}
   Grundlehren der mathematischen Wissenschaften 237,
   Springer-Verlag, New York (1979).

\bibitem{Kust:Cstar} {\sc J. Kustermans \&\ A. Van Daele},
   {\em $C^*$-algebraic quantum groups arising from algebraic quantum groups}\@.
   Int. Journal of Math. Volume 8, {\bf 8}, 1067-1139 (1997).

\bibitem{Kust:thesis} {\sc J. Kustermans},
   {\em $C^*$-algebraic quantum groups arising from algebraic quantum groups}\@.
    Ph.\ D.\ thesis, K.U. Leuven (1996).

\bibitem{Kust:corep} {\sc J. Kustermans},
   {\em Examining the dual of an algebraic quantum group}\@.
   Preprint Odense Univ. \#funct-an/9704004 (1997).

\bibitem{KV} {\sc J. Kustermans \&\ S. Vaes}
   {\em Locally compact quantum groups}\@.
   To appear in: Annales Scient.\ de l'Ec. Norm. Sup. (1999).

\bibitem{KV:cr} {\sc J. Kustermans \&\ S. Vaes}
   {\em A simple definition for locally compact quantum groups}\@.
   C.R.\ Ac.\ Sc.\ Paris, Ser.\ I, {\bf 328}(10), 871-876 (1999).

\bibitem{KV:pnas} {\sc J. Kustermans \&\ S. Vaes}
   {\em The operator algebra approach to quantum groups}\@.
   Proc. Nat. Acad. Sci. U.S.A. (PNAS)
   Vol.\ {\bf 97}, No.\ 2, 547-552 (January 18, 2000).

\bibitem{Kust:universal} {\sc J. Kustermans \&\ A. Van Daele},
  {\em Universal $C^*$-algebraic quantum groups arising from algebraic quantum groups}\@.
   preprint Odense Univ. \#funct-an/9704006 (1997).

\bibitem{AnnFons} {\sc A. Maes \&\ A. Van Daele}
   {\em Notes on compact quantum groups}\@.
   Nieuw archief voor wiskunde, serie 4, deel {\bf 16}, No. 1-2, 73-112 (1998).

\bibitem{Majid} {\sc Shahn Majid}
   {\em Foundations of quantum group theory}\@.
   Cambrige university press (1995).

\bibitem{Montgomery} {\sc Susan Montgomery}
   {\em Hopf algebras and their actions on rings}\@.
   {\sc cbms}\ Regional Conf.\ Series in Math.\ {\bf 82}, AMS (1993).

\bibitem{Jeroen:QE2:Fourier} {\sc J. Noels},
   {\em Constructing Fourier transforms on the Quantum $E(2)$ group}\@.
   Preprint K.U. Leuven (2000)

\bibitem{Jeroen:QE2:haar} {\sc J. Noels},
   {\em An algebraic framework for harmonic analysis on the quantum $E(2)$ group}\@.
   Preprint K.U. Leuven (2000).

\bibitem{Schaefer} {\sc H.H. Schaefer},
   {\em Topological vector Spaces}\@.
   Macmillan Series in Advanced Mathematics and Theor.\ Phys.,
   New York (1966).

\bibitem{Swarttouw} {\sc R.F. Swarttouw},
  {\em The Hahn-Exton $q$-Bessel function}\@.
   Ph. D. thesis, Technical University Delft (1992).

\bibitem{Sweedler} {\sc M.E. Sweedler},
   {\em Hopf algebras}\@.
   Math.\ Lecture Notes Series, Benjamin, New York (1969).

\bibitem{Takesaki} {\sc M. Takesaki},
   {\em Theory of Operator Algebras I}\@.
   Springer-Verlag, New York (1979).

\bibitem{Treves} {\sc F. Treves},
   {\em Topological vector Spaces, Distributions and Kernels}\@.
   Academic Press, New York \&\ London (1967).

\bibitem{leonid:vainerman} {\sc L. Vainerman},
   {\em Gel'fand pair associated with the quantum group of motions of the plane and
        $q$-Bessel functions}\@. Reports of Math.\ Phys. (1995).

\bibitem{Fons:spectral_conditions} {\sc A. Van Daele},
   {\em The operator $a \tens b + b \tens a^{-1}$ when $ab=\lambda ba$}\@.
   Preprint K.U. Leuven (1989).

\bibitem{Fons:DPHA} {\sc A. Van Daele},
   {\em Dual pairs of Hopf $*$-algebras}\@.
   Bull.\ London Math.\ Soc.\ {\bf 25}, 209-230 (1993).

\bibitem{Fons:DQG} {\sc A. Van Daele},
   {\em Discrete Quantum Groups}\@.
   Journal of Algebra {\bf 180}, 431-444 (1996).

\bibitem{Fons:MHA} {\sc A. Van Daele},
   {\em Multiplier Hopf algebras}\@.
   Trans.\ Amer.\ Math.\ Soc.\ {\bf 342}, No.\ 2, 917-932, (1994).

\bibitem{Fons:AFGD} {\sc A. Van Daele},
   {\em An algebraic framework for group duality}\@.
   Adv.\ Math.\ {\bf 140} 323-366 (1998).

\bibitem{Fons:AFGD:proc} {\sc A. Van Daele},
   {\em Multiplier Hopf Algebras and Duality\/}
   Proceedings of the workshop on Quantum groups and quantum spaces
   (Warsaw, Nov.\ 1995).
   Banach Center Publ.\ {\bf 40}, 51-58,
   Inst.\ of Math., Polish Ac.\ of Sciences,
   Warszawa (1997).

\bibitem{Fons:pnas} {\sc A. Van Daele},
   {\em Quantum groups with invariant functionals}\@.
   Proc. Nat. Acad. Sci U.S.A. (PNAS) Vol.\ {\bf 97}, No.\ 2, 541-546 (January 18, 2000).

\bibitem{FonsZhang:DQG} {\sc A. Van Daele \&\ Y. Zhang},
   {\em Multiplier Hopf algebras of Discrete type}\@.
   Journal of algebra {\bf 214}, 400-417 (1999).

\bibitem{FonsZhang:coactions} {\sc A. Van Daele \&\ Y. Zhang},
   {\em Galois theory for Multiplier Hopf algebras with integrals}\@.
   Algebras and Representation theory, {\bf 2}, 83-106 (1999).

\bibitem{FonsWor:QE2} {\sc A. Van Daele \&\ S.L. Woronowicz},
   {\em Duality for the Quantum $E(2)$ Group}\@.
   Pacific Journal of Math.\ {\bf 173}, No.\ 2, 375-385 (1996).

\bibitem{FonsSabine} {\sc A. Van Daele \&\ S. Van Keer},
   {\em The Yang-Baxter and Pentagon equation}\@.
   Compositio Math.\ {\bf 91}, 201-221 (1994).

\bibitem{VaksKor:QE2} {\sc L.L. Vaksman \&\ L.I.
         Korogodski\u{\i}},
         {\em An algebra of bounded functions on the quantum group of motions
              of the plane, and $q$-analogues of Bessel functions}\@.
         Soviet Math. Dokl. {\bf 39}, 173-177 (1989).

\bibitem{wolfram} {\sc S. Wolfram},
   {\em Mathematica, a system for doing mathematics by computer}\@.
   Addison-Wesley (1993).

\bibitem{Wor:matrix} {\sc S.L. Woronowicz},
   {\em Compact Matrix Pseudo Groups}\@.
   Comm.\ Math.\ Phys.\ {\bf 111}, 613-665 (1987).

\bibitem{Wor:QE2} {\sc S.L. Woronowicz},
   {\em Quantum $E(2)$ group and its Pontryagin dual}\@.
   Lett.\ in Math.\ Phys.\ {\bf 23}, 251-263 (1991).

\bibitem{Wor:Affiliated} {\sc S.L. Woronowicz},
   {\em Unbounded elements affiliated with \Cstar-algebras and non-compact quantum groups}\@.
   Comm.\ Math.\ Phys.\ {\bf 136}, 399-432 (1991).

\bibitem{Wor:operatoreq} {\sc S.L. Woronowicz},
   {\em  Operator equalities related to the quantum $E(2)$ group}\@.
         Commun. Math. Phys. {\bf 144}, 417-428 (1992).
\end{thebibliography}

\end{document}
