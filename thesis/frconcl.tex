

\begin{thm_sec}
Take\/ $0<q<1$ and define\/ $\tau = q^{-1}$ and\/ $\nu = (q^{-1}-q)^{-2}$. Then
the system
$$\left( \Uq\!\left(\calL(\Hcore_\tau)\vertM\right)  \subseteq
         \Uq\!\left(\calL({\mathcal S}_\tau(\RR^+;q^2)) \vertM \right),
                \varphi, \psi;  \,  \Uq, \Aq; \,
                \Aq(\Hcore_\nu) \subseteq
         \Aq\!\left({\mathcal S}_\nu(\RR^+;q^2)\vertM \right), \omega \vertL \right) $$
is a Plancherel context (cf.\ definition \ref{def:Plancherel_context}).
\end{thm_sec}

\begin{proof}
Let's show conditions (i-iv) of definition \ref{def:Plancherel_context}\@.
Condition (i) follows from propositions \ref{prop:Eq2:Fourier_transforms},
\ref{prop:Eq2:Fourier:bijections}\ and \ref{prop:Eq2:Fourier:inverse}\@.
(ii) follows from theorem \ref{thm:Eq2:Plancherel}\ together with some results in
chapter \ref{chapter:algebraic_harmonic_analysis}\@.
(iii) was shown in lemma \ref{lemma:Eq2:lambdaiszeta}\@.
Also (iv) is not so hard to prove.
\end{proof}



\paragraph{Further investigation}
From lemmas \ref{lemma:induced_pairing}\ and \ref{lem:duality_revisited}\
we now know there is a natural duality between
$\Uq\!\left(\calL(\Hcore_\tau)\vertM\right)$ and $\Aq(\Hcore_\nu)$,
for instance,
$\pair{x}{y} = \omega(\FLabb(x)\,y)$
for
$x \in \Uq\!\left(\calL(\Hcore_\tau)\vertM\right)$ and $y \in \Aq(\Hcore_\nu)$.
This formula for the pairing can be made very explicit simply
by plugging in definitions \ref{def:Eq2:Fourier_transforms},
\ref{def:YaH}, \ref{def:Hmpair}\ and \ref{def:qHankeltransform:unitary},
together with proposition \ref{prop:exist:holomorphic_qHankel}\
and the definition of the Haar functional $\omega$.
This should yield some very concrete expressions for the pairing, e.g.
\begin{eqnarray*}
  \lefteqn{  \left\langle \vertL\Upsilon(X)\, b^m, \;
     \alpha^l (\gamma^*)^n \,g(\gamma^*\gamma)  \right\rangle
     \hspace{1em}=\hspace{1em}
     \delta_{m,n} \: \kq(m,-m-l)\: \nu^m \, q^{ml} \ldots } \\
  &\ldots & \sum_{r,k\in \ZZ}   q^{(m+2)(k+r)} \: \J{m}{q^{k+r}} \;   g(\nu\, q^{2r+2l}) \;
       X\!\left(-m\theta-l\theta, \: \tau\, q^{2k-m-l} \vertM
           \right) \hspace{1em} \label{eq:Eq2:induced_pairing}
\end{eqnarray*}
for any and $l\in \ZZ$, $m,n\in \NN$, $g \in \Hcore_\nu$ and $X\in
\calL(\Hcore_\tau)$. The interesting point about this formula is that it no
longer depends on the use of {\em entire\/} functions, i.e.\ $X$ and $g$ may be
any functions living on the appropriate (discrete!) sets, provided they satisfy
some elementary summability criterion.
So it may be possible to take the above formula as the {\em point to start\/} the construction
of the quantum $E(2)$ group, i.e.\ to make it into a {\em definition\/} for the pairing\ldots
