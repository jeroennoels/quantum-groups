
\subsection{Eigenfunctions of $q$-Hankel transformation}


Below we shall prove that the function $z \mapsto E_{q^2}(-q^2 z)$
belongs to the space \Hcore\ constructed in the previous paragraph,
and moreover, that it constitutes an eigenfunction of all positive order
holomorphic $q$-Hankel transforms; it follows that \Hcore\ is non-trivial.
This function of $q^2$-exponential type plays a role similar to the
Gaussian\footnote{and as a matter of fact, this $q^2$-exponential does amount
to a $q$-Gaussian if one takes into account the \lq{\em squaring\/} operator\rq\ $K$
appearing in diagram (\ref{eq:diagram:holomorphic_qHankel}).}
in ordinary Fourier analysis on the real line.


\begin{lemma}
Let\/ \qE\ denote the entire function given by\/ $\qE(z) = (z;q)_\infty$. Then
$$   D_q \qE \:=\: -\frac{1}{1-q}\, \Omega \qE. $$
\end{lemma}

\begin{proof}
We could easily derive this from the power series (\ref{eq:qExp:powerseries})
but it is even more convenient to use the product representation:
indeed (\ref{eq:shifted_factorial:infty}) yields
$$\qE(z) \:=\: (1-z)\qE(qz) $$
for all $z\in \CC$, and hence
$$ (D_q \qE)(z) \;=\;  \frac{\qE(z) - \qE(qz)}{(1-q)z}
                \;=\; \frac{-z\,\qE(qz)}{(1-q)z}
                \;=\; -\frac{1}{1-q} \, (\Omega \qE)(z). $$
for $z\neq 0$. Extending the result to $z=0$ by continuity completes the proof.
\end{proof}



\begin{cor} \label{cor:qqderivative:qqE}
Replacing\/ $q$ by\/ $q^2$ and (consequently!)\/ $\Omega$ by\/ $\Omega^2$, we get
$$  \Dqsqr \qqE \:=\: -\frac{1}{1-q^2}\, \Omega^2 \qqE. $$
\end{cor}



\begin{prop} \label{prop:qdifferential_equation:qexp}
$\;\Omega^2 \qqE = (\,\cdot\,q^2 ;q^2)_\infty$ satisfies the\/
$q^2$-differential equation
\begin{equation}\label{eq:qdifferential_equation:qexp}
   \Dqsqr f \;=\; -\frac{q^2}{1-q^2}\, \Omega^2 f
     \hspace{3em} \mbox{($f$ entire).}
\end{equation}
Moreover the solution of this\/ $q^2$-differential equation is
unique in the sense that any entire function\/ $f$
satisfying (\ref{eq:qdifferential_equation:qexp}) must be a scalar
multiple of\/ $\Omega^2 \qqE$.
\end{prop}

\begin{proof}
Combining corollary \ref{cor:qqderivative:qqE}\ with the obvious
commutation rule for $\Dqsqr$ and $\Omega^2$ yields
$$ \Dqsqr \Omega^2 \qqE \:=\: q^2 \Omega^2 \Dqsqr \qqE
             \:=\: -\frac{q^2}{1-q^2}\, \Omega^2 \Omega^2 \qqE. $$
To prove the uniqueness statement, let's evaluate (\ref{eq:qdifferential_equation:qexp})
in $q^{2n}$ for any $n\in\NN$. We get
$$ \frac{f(q^{2n})-f(q^2 q^{2n})}{(1-q^2) \, q^{2n}}
            \;=\;  -\frac{q^2}{1-q^2}\, f(q^2 q^{2n}) $$
Putting $x_n = f(q^{2n})$ for $n\in\NN$, we obtain a sequence $(x_n)_{n=0}^\infty$
of complex numbers satisfying the following recurrence relation:
$$ x_n \,=\: (1-q^{2n} q^2)\, x_{n+1}. $$
Iterating this relation yields
\begin{equation}\label{eq:uniqueness:recurrence_relation}
   x_0 \:=\: (1-q^2)(1-q^2 q^2) \ldots (1-q^{2(n-1)} q^2) \, x_n
       \:=\: (q^2;q^2)_n \, x_n.
\end{equation}
It follows that the sequence $(x_n)_n$ is completely determined by
the value of $x_0$. In other words, (\ref{eq:qdifferential_equation:qexp}) and
$f(1)$ determine the value of $f$ at the points $q^{2n}$ ($n\in\NN$).
Since $q^{2n} \rarr 0$ as $n\rarr\infty$, we may invoke the identity theorem
for holomorphic functions and draw the conclusion (notice that $f(1)=0$ implies $f=0$,
and observe that (\ref{eq:qdifferential_equation:qexp}) is {\em linear\/} in $f$).
\end{proof}
\vspace{2ex}


It is also instructive to observe (\ref{eq:shifted_factorial:infty}) implies that
$$ x_n \:=\: (q^{2n} q^2; q^2)_\infty \:=\: (\Omega^2 \qqE)(q^{2n}) $$
indeed satisfies the above recurrence relation (\ref{eq:uniqueness:recurrence_relation}).


\begin{lemma}
Let\/ $m$ be any non-negative integer.
If\/ $f$ is a solution of (\ref{eq:qdifferential_equation:qexp})
then so is\/ $\holH{m}f$\/
\rm (notice that $f$ belongs to $\Hintersect$ because of proposition
\ref{prop:qdifferential_equation:qexp}\ and corollary \ref{cor:qqE_in_holS},
so the statement makes sense).
\end{lemma}

\begin{proof}
With the commutation rule\/ $\Dqsqr \Omega^2 = q^2 \, \Omega^2 \Dqsqr$
the first formula in proposition \ref{prop:holqHankel:qdiff}\ can be
rewritten in the form
\begin{equation}\label{eq:holqHankel:qdiff:rewritten:bis}
    \Dqsqr \holH{m-1} = -\frac{q^2}{1-q^2}\, \Omega^2 \holH{m}
    \hspace{4em}  (m\geq 1)  \hspace{1em}
\end{equation}
Replacing $m$ with $m+1$ and applying this to a solution $f$ of
(\ref{eq:qdifferential_equation:qexp}) yields
$$  \Dqsqr \holH{m} f
     \:=\: -\frac{q^2}{1-q^2}\, \Omega^2 \holH{m+1} f
     \:\stackrel{(\ref{eq:qdifferential_equation:qexp})}{=}\:
             \Omega^2 \holH{m+1} \Omega^{-2} \Dqsqr f
     \:=\: q^2 \Omega^2 \holH{m+1} \Dqsqr \Omega^{-2} f $$
for any $m\geq 0$. Once again applying the first formula in
proposition \ref{prop:holqHankel:qdiff}\ yields
\begin{equation}\label{eq:HolHmf:solution}
   \Dqsqr \holH{m} f  \:=\: -\frac{q^2}{1-q^2}\, \Omega^2 \holH{m} f
\end{equation}
which means that $\holH{m}f$ is a solution to (\ref{eq:qdifferential_equation:qexp}).
\end{proof}



\begin{remark} \rm
The above lemma does not extend to negative $m$. To see this,
apply proposition \ref{prop:qHankel:inverseorder}\ to (\ref{eq:HolHmf:solution}).
The reason for this remarkable lack of symmetry is not so clear at the moment.
$\hfill \star$
\end{remark}

\begin{prop} \label{prop:qqE:eigenfunction_qHankel}
$\;\Omega^2 \qqE$ is an eigenfunction of all\/ $\holH{m}$ with\/ $m\geq 0$.
\end{prop}

\begin{proof}
Combine the above lemma with proposition \ref{prop:qdifferential_equation:qexp}\@.
\end{proof}
\vspace{2ex}


Below we shall compute the corresponding eigenvalues and search for other
eigenfunctions as well. Before we proceed, however, we wish to emphasize that
our core \Hcore\ is finally known to be non-trivial---indeed it does already contain
an important class of functions:


\begin{cor} \label{cor:qExp_in_Hcore}
For any polynomial\/ $P$ and integer\/ $k$, the entire function
\begin{equation}\label{eq:many_functions_in_Hcore}
  \CC \rarr \CC : \: z \,\mapsto \, P(z)\, E_{q^2}(-q^{2k} z)
\end{equation}
belongs to the space\/ \Hcore\ of definition \ref{def:Hcore}.
\end{cor}


\begin{proof}
We already established that $\Omega^2 \qqE$ belongs to $\Hintersect$
(cf.\ corollary \ref{cor:qqE_in_holS}) and from
proposition \ref{prop:qqE:eigenfunction_qHankel}\
it is clear that $\holH{m} \Omega^2 \qqE$ is in $\Hintersect$ as well,
for any $m\in \NN$. Furthermore, the condition of lemma
\ref{lemma:compactsupp_in_holS}\ is obviously invariant under $q^2$-differentiation,
hence $\Dqsqr^m \Omega^2 \qqE$ still belongs to $\Hintersect$, for all $m\in \NN$.

We conclude that $\Omega^2 \qqE$ is contained in \Hcore\@.
Eventually, the invariance properties of \Hcore\
(cf.\ proposition \ref{prop:Hcore:invariance}) yield the result.
\end{proof}


\begin{notation} \label{not:xi_k} \rm
Henceforth $\xi_0$ will denote the entire function $\Omega^2 \qqE$ that was investigated above,
so $\xi_0(z) = (q^2 z; q^2)_\infty$ for any $z \in \CC$.
Furthermore we denote
\begin{equation}\label{eq:not:xi_k}
   \xi_k   \;=\; q^k \, \Om^{2k} \xi_0
\end{equation}
for any $k\in \ZZ$.
Observe that the $\xi_k$ belong to the space \Hcore\ (cf.\ previous corollary).
\end{notation}



\begin{lemma} \label{lemma:Psi_xi}
$\;\Psi \xi_0 = (\id - \Om^{-2})\xi_0$ and consequently, for any\/ $k\in \ZZ$,
\begin{equation}\label{eq:Psi_xi_k}
  \Psi \xi_k \:=\: q^{-2k} \,(\xi_k - q \,\xi_{k-1}).
\end{equation}
\rm It follows that the functions in (\ref{eq:many_functions_in_Hcore})
belong to the linear span of\/ $\{ \xi_k \mid k\in \ZZ  \}$.
\end{lemma}

\begin{proof}
The first formula follows from a straightforward computation using
the product representation (\ref{eq:shifted_factorial:infty}).
Equation (\ref{eq:Psi_xi_k}) is an immediate consequence.
\end{proof}


\begin{lemma} \label{lemma:xizero:eigenfunction:Hol_m}
$\; \holH{m} \xi_0 = \xi_0$ for all\/ $m \in \NN$.
\end{lemma}

\begin{proof}
We already know from proposition \ref{prop:qqE:eigenfunction_qHankel}\ that
$\xi_0$ is an eigenfunction of $\holH{m}$ for all $m \in \NN$, so it only remains
to show that the corresponding eigenvalues $\lambda_m$ are all equal to $1$.
First we compute the value of $\lambda_0$.
To do so, we shall rely on proposition \ref{prop:moment:link_with_Hankel}\
from the {\em next\/} section---this may be slightly unpleasant, but it does not really raise
a problem since the proof of the latter result is obviously independent of the present lemma
and its consequences. Thus we obtain
$$ (1-q^2) \, (\holH{0}\xi_0)(0)
            \;=\, \int_0^\infty \xi_0(x)\, d_{q^2}x. $$
For details we refer to \S \ref{sec:qHankel:qmoment_problem}\@.
Now recall corollary \ref{cor:qqderivative:qqE}, where we observed that
$\xi_0 = -(1-q^2)\, \Dqsqr \qqE$. It follows that
$$ (\holH{0}\xi_0)(0)
  \;=\; -\! \int_0^\infty (\Dqsqr \qqE)(x)\, d_{q^2}x
  \;=\;  \qqE(0)
  \;=\; (0\vssp ; q^2)_\infty
  \;=\:  1. $$
Observe how we used the relation between $q^2$-integration and $q^2$-differentiation,
combined with the following facts:
\begin{enumerate}
\item[(i)]
$\qqE(q^{2n})$ vanishes for all negative integers $n$.
\item[(ii)]
$\displaystyle \lim_{n \, \rarr \, +\infty} \qqE(q^{2n}) \:=\:   \qqE(0)$
since $\qqE$ is entire and a fortiori continuous.
\end{enumerate}
Using $\xi_0(0)=1$ and $\holH{0}\xi_0 = \lambda_0 \xi_0$ we get $\lambda_0 = 1$.
To show $\lambda_m = 1$ for all $m\in \NN$, we proceed by induction:
using the $2^{\rm nd}$ formula in proposition \ref{prop:holqHankel:qdiff}, we obtain
\begin{eqnarray*}
\Psi \xi_0
&=&
\Psi \holH{m+1} \holH{m+1} \xi_0
\\&=&
\lambda_{m+1} \, \Psi \holH{m+1} \xi_0
\\&=&
\lambda_{m+1} \, (q^{-1}-q) \,q^{m+1} \: \holH{m} \nabq{m+1} \Omega \xi_0
\\&\stackrel{(\ref{eq:def:nabq})}{=}&
- \lambda_{m+1} \,q^{m+1} \: \holH{m}
       (q^{m+1}\, \Omega  - q^{-m-1}\, \Omega^{-1}) \Omega \xi_0
\\&=\vertXL&
- \lambda_{m+1} \: \holH{m} (q^{2(m+1)}\, \Omega^2 - \id) \xi_0
\\&\stackrel{(\ref{eq:holS:Omega_invariant})}{=}&
- \lambda_{m+1} \: (\Omega^{-2} - \id)  \holH{m} \xi_0
\\&=&
\lambda_{m+1} \lambda_m \: (\id - \Omega^{-2}) \xi_0
\\&=&
\lambda_{m+1} \lambda_m \: \Psi \xi_0.
\end{eqnarray*}
The last equality relies on lemma \ref{lemma:Psi_xi}\@.
It follows that $\lambda_{m+1} \lambda_m = 1$ for all $m\in \NN$.
This completes the proof.
\end{proof}



\begin{cor}
An eigenfunction for the\/ $q$-Hankel transformation\/ $H_m$ on\/ $\Ltwoq$ is given by\/
$f_m = R_q \Psi^m K \xi_0$\/ and the eigenvalue is\/ $1$. Explicitly we have:
$$  f_m(q^n) \;=\;   q^{nm} \, (q^{2n+2} ; q^2)_\infty $$
for all\/ $n\in \ZZ$ and\/ $m\in \NN$.
In terms of little\/ $q^2$-Bessel functions, this means that
$$   \sum_{n\in \ZZ} \: q^{(2+m)n} \: (q^{2n+2} ; q^2)_\infty \, \J{m}{q^{n+k}}
     \;=\;   q^{km} \, (q^{2k+2} ; q^2)_\infty.  $$
\end{cor}
\begin{proof}
The first assertion follows immediately from the above lemma, together with
diagram (\ref{eq:diagram:holomorphic_qHankel}) in abstract \ref{abs:holomorphic_qHankel}\@.
The last formula follows from (\ref{eq:qHankeltransform:def}).
\end{proof}



\begin{prop} \label{prop:Hankel:on:basis}
$\; \holH{m} \, \xi_k = \, q^{-2mk} \, \xi_{-k}$ for all\/ $m\in \NN$ and\/ $k\in \ZZ$.
\end{prop}
\begin{proof}
Simply combine lemma \ref{lemma:xizero:eigenfunction:Hol_m}\
with proposition \ref{prop:holS:Omega_invariant}\@.
\end{proof}


\begin{cor}
For any\/ $m\in \NN$ and\/ $k\in \ZZ$, the functions
$$   q^{mk} \, \xi_k  \: \pm \: q^{-mk} \, \xi_{-k}  $$
are eigenfunctions of\/ $\holH{m}$ with respectively eigenvalues $\pm 1$.
\end{cor}

Notice that $\pm 1$ are the only possible eigenvalues because $\holH{m}^2 = \id$.
