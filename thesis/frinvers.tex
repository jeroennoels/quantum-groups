

\begin{defn_sec}
Let $\delta_0$ be the function $\delta_0: \Zt \rarr \CC$ given by
$\delta_0(k\theta)=\delta_{k,0}$ for $k\in\ZZ$. Clearly $\delta_0$ belongs to
\KZeven\@. Next let $T_0$ denote the map from \HC\ into $\FZ \tens \HC$ given
by tensoring with $\delta_0$ on the left, i.e.\ $T_0(g)=\delta_0 \tens g$.
Finally, for any $k\in\ZZ$, let $P_k$ denote the map $P_k : \FZ \rarr \FZ$
given by  $(P_k f)(l\theta)=\delta_{k,l} f(l\theta)$ for $l\in\ZZ$ and
$f\in\FZ$.
\end{defn_sec}


The following lemma is trivial but important;
it is similar to lemma \ref{lemma:slice_map:parity}\@.


\begin{lemma_sec} \label{lemma:T_0:parity}
The mapping\/ $T_0$ maps\/ $\Hcore_\tau$ into\/
$\KZeven \tens \Hcore_\tau^\scripteven \, \subseteq \, \calL(\Hcore_\tau)$.
\end{lemma_sec}



\begin{defn_sec}  \label{def:Htil}
First recall definition \ref{def:YaH}\@.
For any $m,k\in\ZZ$, we define a linear mapping $\Htil{m}{k}$
from $\Hcore_\nu$ into $\calL(\Hcore_\tau)$ as follows:
$$ \Htil{m}{k} \;=\;
   \kq(m,k)^{-1} \, (\Gamma^{-1} \tens \Omega)^{-k}\,
                T_0\, \Lambda_\tau \, \holH{m} \, \Lambda_\nu^{-1}. $$
Observe that because of the previous lemma, $\Htil{m}{k}$ indeed ends up in
$\calL(\Hcore_\tau)$.
\end{defn_sec}




\begin{lemma_sec}
Take\/ $m,k,l\in\ZZ$. Then
$$ \YaH{m}{k} \,\Htil{m}{l} \;=\; \delta_{k,l} \,\id
        \itandspace{3em}
   \Htil{m}{k} \,\YaH{m}{k} \;=\; P_k \tens \id$$
and hence
$$ \sum_{k\in\ZZ}\: \Htil{m}{k}\, \YaH{m}{k} X  \:=\: X $$
for any\/ $X\in \calL(\Hcore_\tau)$.
\end{lemma_sec}
Notice that the last formula makes sense because of lemma \ref{lemma:finite_support}\@.
\vspace{1ex}

{\em Proof.}
Recall that $\holH{m}^{2} = \id$ and observe that
\begin{eqnarray*}
      T_0 (\evzero \tens \id)  &=&  P_0 \tens \id
\\
\vertXL\hspace{1em}
        (\evzero \tens \id) (\Gamma^{-1} \tens \Omega)^{k-l}T_0
             &=& \delta_{k,l} \: \id.   \hspace{2em} \qed
\end{eqnarray*}





\begin{prop_sec}  \label{prop:Eq2:Fourier:bijections}
The Fourier transforms\/ $\FRabb$ and\/ $\FLabb$
(cf.\ definition \ref{def:Eq2:Fourier_transforms}) are bijections from\/
$\Uq\!\left( \vertM \calL(\Hcore_\tau)\right)$ onto\/ $\Aq(\Hcore_\nu)$.
Their inverses are given by
$$\begin{array}{lcr}
     \FRabb^{-1} \!\left(  \alpha^l \gamma^m \, g(\gamma^*\gamma) \vertL \right)
&=&
     \Upsilon \!\left(\Htil{m}{l-m}\,g \right) b^m   \vertXL
\\
     \FRabb^{-1} \!\left(  \alpha^l (\gamma^*)^m \,g(\gamma^*\gamma) \vertL \right)
&=&
     \Upsilon \!\left(\Htil{-m}{l+m}\,g \right) c^m  \vertXL
\\
     \FLabb^{-1} \!\left(  \alpha^l \gamma^m \,g(\gamma^*\gamma) \vertL \right)
&=&
     q^{-2l} \: \Upsilon \!\left(\Htil{m}{l-m}\,g \right) b^m  \vertXL
\\
     \FLabb^{-1} \!\left(  \alpha^l (\gamma^*)^m \,g(\gamma^*\gamma) \vertL \right)
&=&
     q^{-2l} \: \Upsilon \!\left(\Htil{-m}{l+m}\,g \right) c^m  \vertXL
\end{array} $$
\end{prop_sec}



\begin{lemma_sec} \label{lemma:Eq2:lambdaiszeta}
We have\/ $\pair{1_{\Uq}}{\FLabb(x)} = \varphi(x)$
for any\/ $x\in \Uq\!\left(\calL(\Hcore_\tau)\vertM\right)$.
\end{lemma_sec}

\begin{proof}
Using proposition \ref{prop:moment:link_with_Hankel}\ we get for
all $X\in \calL(\Hcore_\tau)$ and $m\in \NN$ that
\begin{eqnarray*}
    \lefteqn{ \left\langle 1_{\Uq} ,\: \FLabb
              \!\left(\vertM\Upsilon(X)\, b^m\right)\right\rangle}
\\&=&
    \sum_{k\in\ZZ} \;  \left\langle 1_{\Uq} ,\: (q^2 \alpha)^{k+m}\, \gamma^m
               \left(\YaH{m}{k}\,X\right)(\gamma^*\gamma) \right\rangle
\\&=&
     \sum_{k\in\ZZ} \; \delta_{m,0} \, q^{2k} \left(\YaH{0}{k}\,X\right)(0)
\\&=&
     \sum_{k\in\ZZ} \; \delta_{m,0} \, q^{2k} \left(\holH{0} \, \Lambda_\tau^{-1}\,
         (\evzero \tens \id)\,(\Gamma^{-1} \tens \Omega)^k X \vertL\right)(0)
\\&=&
     \sum_{k\in\ZZ} \; \delta_{m,0} \, q^{2k}  \frac{1}{1-q^2} \int_0^\infty
     \! \left(\Lambda_\tau^{-1}\, (\evzero \tens \id)\,
           (\Gamma^{-1} \tens \Omega)^k X \vertL\right)(x) \: d_{q^2}x
\\&=&
     \sum_{k\in\ZZ} \; \delta_{m,0} \, q^{2k}  \frac{1}{1-q^2} \int_0^\infty
        X(k\theta,\, \tau q^k x) \: d_{q^2}x
\\&=&
     \sum_{k\in\ZZ} \; \delta_{m,0} \, q^{2k}
     \sum_{n\in\ZZ}   X(k\theta,\, \tau q^k q^{2n}) \, q^{2n}
\\&=&
     \delta_{m,0} \!\! \sum_{(k,l)\, \in \,\mathfrak{S}}
          X(k\theta,\, \tau q^l) \, q^{k+l}
\\&=&
\varphi \!\left(\vertM\Upsilon(X)\, b^m\right).
\end{eqnarray*}
At the last but one equality we rearranged the sums according to the rule $k+2n=l$.
Similarly we can treat $\Upsilon(X)\, c^m$.
\end{proof}
\vspace{2ex}



The previous lemma provides the condition in
proposition \ref{prop:inverse_Fourier_transform}, yielding


\begin{prop_sec} \label{prop:Eq2:Fourier:inverse}
$\;\GLR = S \FLabb^{-1}$ is an {\scriptsize LR} Fourier transform from
$\Aq(\Hcore_\nu)$ to $\Uq\!\left( \vertM \calL(\Hcore_\tau)\right)$.
\end{prop_sec}


From the general observations in chapter \ref{chapter:algebraic_harmonic_analysis}\
we also obtain all the other Fourier transforms from $\Aq(\Hcore_\nu)$ to
$\Uq\!\left( \vertM \calL(\Hcore_\tau)\right)$, as well as the appropriate Plancherel
identities for $\GLR$ and $\GRL$ (observe however that $\GLL$ and
$\GRR$ do {\em not\/} obey a Plancherel formula).



\paragraph{Even-odd structure and double covering, continued}

Let's complete the investigation that was taken up in \S \ref{par:double_covering}\@.

\begin{lemma_sec}
Take\/ $m,k \in \ZZ$. If\/ $k$ is even, then the range of\/ $\Htil{m}{k}$
is contained in\/ $\KZeven \tens \Hcore_\tau^\scripteven$.
\end{lemma_sec}

\begin{proof}
Combining lemma \ref{lemma:T_0:parity}\ and definition \ref{def:Htil}, we obtain
$$ \Htil{m}{k}(\Hcore_\nu)
        \hspace{0.7em} = \hspace{0.7em}
   (\Gamma^{-1} \tens \Omega)^{-k} \,T_0 (\Hcore_\tau)
        \hspace{0.7em}  \subseteq  \hspace{0.7em}
   (\Gamma^{-1} \tens \Omega)^{-k}
       \left(\vertM\KZeven \tens \Hcore_\tau^\scripteven \right). $$
This yields the result, for $\Gamma^{-1} \tens \Omega$ interchanges
even and odd parts of (\ref{eq:def:Hcore:parity}).
\end{proof}


\begin{lemma_sec}
The inverses\/ $\FRabb^{-1}$ and\/ $\FLabb^{-1}$ of the Fourier transforms,
as given in proposition \ref{prop:Eq2:Fourier:bijections}, map
$$   \Aqeven(\Hcore_\nu)
         \hspace{3em} \mbox{into}  \hspace{3em}
     \Uq \! \left(\KZeven \tens \Hcore_\tau^\scripteven \vertM\right). $$
\end{lemma_sec}
\begin{proof}
When $\FRabb^{-1}$ and $\FLabb^{-1}$ are invoked on elements of $\Aqeven(\Hcore_\nu)$ only,
then in the formulas of proposition \ref{prop:Eq2:Fourier:bijections},
we may assume $l \pm m$ to be even.
Now the previous lemma completes the proof.
\end{proof}
\vspace{2ex}


Combining the above lemma with lemma \ref{prop:double_covering}, we obtain

\begin{prop_sec}
The Fourier transforms\/ $\FRabb$ and\/ $\FLabb$ restrict to bijections
$$      \mbox{from}  \hspace{2em}
  \Uq \! \left(\KZeven \tens \Hcore_\tau^\scripteven \vertM\right)
        \hspace{2em} \mbox{onto}  \hspace{2em}
  \Aqeven(\Hcore_\nu). $$
\end{prop_sec}
