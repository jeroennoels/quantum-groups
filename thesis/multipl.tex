\section{Multiplicativity}
\label{par:multiplicativity}


\begin{abs_chp}
Given any \dpa, it would be natural to impose some compatibility condition
on the algebra structures involved, i.e.\ an interaction between the
products in both algebras.
In the Hopf algebra setting this is usually done by requiring comultiplications
to be homomorphisms.
However we shall also express this condition of \lq multiplicativity\rq\ in a
way that does not involve comultiplications.
This yields the notion of a {\em Hopf system}, which is the main subject
in the present chapter.
Eventually we obtain several so-called {\em twist\/} maps which can be used later,
e.g.\ to construct a quantum double or to show that the antipodes are
anti-multiplicative.
\end{abs_chp}



Let \pairAB\ be any \idpa\@.
Sometimes it is convenient to write $\pair{B \tens A}{A \tens B}$
as a pairing of $A \tens B$ with $A \tens B$ itself.
To avoid confusion, the latter will be denoted by
$\pairflip{\,\cdot}{\cdot\,}$, i.e.\ for any $x,y \in A \tens B$
we define $\pairflip{x}{y} = \pair{\flip(x)}{y}$.
Here $\flip$ denotes the flip $A \tens B \rarr B \tens A$.
We shall use the same symbol $\flip$ to denote the flip $B \tens A \rarr A \tens B$,
confusion being unlikely.



\begin{defn_sec}   \label{def:muliplicative_actor}
Let \pairAB\ be any \idpa\@.
An actor $R\equiv(\lam,\rho)$ for \BBAA\ is said to be {\em multiplicative\/} if
\begin{equation}\label{eq:muliplicative_actor}
\pair{R}{xy}  \:=\: \pairflipM{\rho(x)}{\lam(y)}
\end{equation}
for all $x,y \in A \tens B$. Notice that if $R$ belongs to $\Env(\BBAA)$ then
(\ref{eq:muliplicative_actor}) reads
$\pair{R}{xy} \:=\: \pairflip{x \ract R\,}{R \lact y}$.
In particular the pairing $P$ will be multiplicative if
\begin{equation} \label{eq:muliplicative_pairing}
  \pairM{ac}{bd} \:=\: \pairflipM{\rp(a \tens b)}{\lp(c\tens d)}
\end{equation}
for all $a,c\in A$ and $b,d\in B$.
\end{defn_sec}


\begin{remark_sec} \rm
We claim that equation (\ref{eq:muliplicative_pairing}) indeed amounts to the fact that
the comultiplications on $A$ and $B$ are homomorphisms.
Although a rigorous proof shall be given below, it may also be instructive to
appreciate the meaning of (\ref{eq:muliplicative_pairing}) in terms of Sweedler notation:
\begin{eqnarray*}
\pairflipM{\rp(a \tens b)}{\lp(c\tens d)}
&=&
\pairflipM{\pair{\sweed{a}{1}}{\sweed{b}{1}} \: \sweed{a}{2} \tens  \sweed{b}{2}}{
           \pair{\sweed{c}{2}}{\sweed{d}{2}} \: \sweed{c}{1} \tens \sweed{d}{1}}
\\&=&
\pair{\sweed{a}{1}}{\sweed{b}{1}}
\pair{\sweed{a}{2}}{\sweed{d}{1}}
\pair{\sweed{c}{1}}{\sweed{b}{2}}
\pair{\sweed{c}{2}}{\sweed{d}{2}}
\\&=&
\pair{a}{\sweed{b}{1} \sweed{d}{1}}
\pair{c}{\sweed{b}{2} \sweed{d}{2}}
\\&=&
\pair{ac}{bd}.
\end{eqnarray*}
\end{remark_sec}


\begin{prop_sec}  \label{prop:multiplicative:comultiplications}
Let\/ \pairAB\ be any \idpa\@.
Then the following assertions are equivalent:
\begin{enumerate}
  \item $P$ is multiplicative in the sense of the above definition
  \item $\DeltaA : A \rarr \Env(\AAAA)$ is an algebra homomorphism
  \item $\DeltaB : B \rarr \Env(\BBBB)$ is an algebra homomorphism
\end{enumerate}
\end{prop_sec}

\begin{proof}
Take any $a,c\in A$ and $b,d\in B$ and write
$\rp(a \tens b) = \sum_i \,p_i \tens q_i$ with $p_i \in A$ and $q_i \in B$.
Recall that $P_{13}$, $P_{23}$ and $P_{13} P_{23}$ are actors for
$\BBBB \tens \Aa$, whereas $\DeltaB(d) \tens 1_\Aa$ belongs to $\Env(\BBBB \tens \Aa)$.
Also recall remark \ref{rem:module_notation_for_Env}\@.

Now observe the following \lq circle\rq\ of equalities:
\begin{eqnarray*}
      \pairM{\DeltaB(bd)}{a \tens c}
&\stackrel{\rm (iii)}{=}&
      \pairM{\DeltaB(b) \, \DeltaB(d)}{a \tens c}
\\&=&
      \pairM{\DeltaB(b)}{\DeltaB(d) \lact (a \tens c)}
\\&=&
      \pairM{(\id\, \slice f_b)(P_{13} P_{23})}{\DeltaB(d) \lact (a \tens c)}
\\&=&
      \pairM{P_{13} P_{23}}{\left( \vertM \DeltaB(d) \lact (a \tens c)\right) \tens b}
\\&=&
      \pairM{P_{13} P_{23} \! \left( \vertM \DeltaB(d) \tens 1_\Aa \right)}{a \tens c \tens b}
\\&=&
      \pairM{P_{23} \! \left( \vertM \DeltaB(d) \tens 1_\Aa
                   \right)}{ (\rp)_{13}(a \tens c \tens b)}
\\&=&
      \textstyle \sum_i \,
      \pairM{P_{23} \left( \vertM \DeltaB(d) \tens 1_\Aa
                   \right)}{p_i \tens c \tens q_i}
\\&=&
      \textstyle \sum_i \,
      \pairM{P_{23}}{\left( \vertM \DeltaB(d) \lact (p_i \tens c)\right) \tens q_i}
\\&=&
      \textstyle \sum_i \,
      \pairM{1_\BB \tens q_i}{ \DeltaB(d) \lact (p_i \tens c)}
\\&=&
      \textstyle \sum_i \,
      \pairM{(1_\BB \tens q_i) \DeltaB(d)}{p_i \tens c}
\\&=&
      \textstyle \sum_i \,  \pairM{\DeltaB(d)}{p_i \tens (c \ract q_i)}
\\&=&
      \textstyle \sum_i \,  \pairM{p_i(c \ract q_i)}{d}
\\&=&
      \textstyle \sum_i \,  \pairM{c \ract q_i}{d \ract p_i}
\\&=&
      \textstyle \sum_i \,  \pairM{q_i \tens p_i}{\lp(c \tens d)}
\\&=&
      \pairflipM{\rp(a \tens b)}{\lp(c \tens d)}
\\&\stackrel{\rm (i)}{=}&
      \pairM{ac}{bd}
\\&=&
      \pairM{\DeltaB(bd)}{a \tens c}
\end{eqnarray*}
This proves (i) $\Leftrightarrow$ (iii). Similarly we prove (i) $\Leftrightarrow$ (ii).
\end{proof}



\begin{defn_sec}   \label{def:Hopf_system}
An \idpa\ is called a {\em Hopf system\/} whenever its pairing is multiplicative
in the sense of definition \ref{def:muliplicative_actor}.
\end{defn_sec}


\begin{prop_sec}  \label{prop:counits_homomorphism}
If\/ \pairAB\ is a Hopf system, then its counits\/ $\epsA$ and\/ $\epsB$
are algebra homomorphisms into\/ \kk.
\end{prop_sec}

\begin{proof}
Take any $a,c\in A$ and $b,d\in B$. Since \rp\ is surjective, we can write
$a \tens b = \sum_i \,\rp(p_i \tens q_i)$ with $p_i \in A$ and $q_i \in B$.
Then
\begin{eqnarray*}
      \pairM{a(c \ract b)}{d}
&=&
      \pairM{c \ract b}{d \ract a}
\\&=&
      \pairM{b \tens a}{\lp(c \tens d)}
\\&=&
      \textstyle \sum_i \,  \pairflipM{\rp(p_i \tens q_i)}{\lp(c \tens d)}
\\&=&
      \textstyle \sum_i \,  \pairM{p_i c}{q_i d}
\\&=&
      \textstyle \sum_i \,  \pairM{p_i c \ract q_i}{d}
\end{eqnarray*}
and hence $a(c \ract b) = \textstyle \sum_i \,  p_i c \ract q_i$. It follows that
\begin{eqnarray*}
      \pairM{1_\BB}{a(c \ract b)}
&=&
      \textstyle \sum_i \,  \pairM{1_\BB}{p_i c \ract q_i}
\\&=&
      \textstyle \sum_i \,  \pairM{p_i c}{q_i}
\\&=&
      \textstyle \sum_i \,  \pairM{p_i}{c \lact q_i}
\\&=&
      \textstyle \sum_i \,  \pairM{P}{p_i \tens (c \lact q_i)}
\\&=&
      \textstyle \sum_i \,  \pairM{P(1_\BB \tens c)}{p_i \tens q_i}
\\&=&
      \pairM{1_\BB \tens c}{a \tens b}
\\&=&
      \pairM{1_\BB}{a} \pairM{1_\BB}{c \ract b}.
\end{eqnarray*}
Since $A \ract B =A$, the above proves that $\epsA \simeq 1_\BB$ is a homomorphism.
\end{proof}



\begin{defn_sec*}   \label{def:braiding}
Consider two algebras $(A,\mult{A})$ and $(B,\mult{B})$.
A linear map $T : B \tens A \rarr  A \tens B$ is said to be an
$(A,B)$-{\em twisting\/} if
\begin{eqnarray*}
   T(\mult{B} \tens \id)  &=&  (\id \tens \mult{B}) (T \tens \id) (\id \tens T)  \\
   T(\id \tens \mult{A})  &=&  (\mult{A} \tens \id) (\id \tens T) (T \tens \id).
\end{eqnarray*}
\end{defn_sec*}

The following result can be found in almost every textbook or paper \cite{Schmudgen,FonsSabine}\
covering topics like the quantum double or the quantum Yang-Baxter equation:

\begin{prop_sec} \label{prop:twisted_tensorproduct}
If\/ $T : B \tens A \rarr  A \tens B$ is an\/ $(A,B)$-twisting between two
algebras\/ $A$ and\/ $B$, then the vector space\/ $A\tens B$ can be made into an
algebra with product\/ $(\mult{A} \tens \mult{B})(\id \tens T \tens \id)$.
\rm This algebra will be denoted by $A\ttens{T} B$.
\end{prop_sec}


\begin{lemma_sec}   \label{lem:twisting}
Consider two algebras $A$ and\/ $B$, and let\/ $R: A \tens B \rarr A \tens B$
be a linear map. Then\/ $R\flip$ is an $(A,B)$-twisting if and only if
\begin{eqnarray}
   R(\id \tens \mult{B}) &=& (\id \tens \mult{B}) R_{12} R_{13}  \label{eq:lemmma:twisting1}  \\
   R(\mult{A} \tens \id) &=& (\mult{A} \tens \id) R_{23} R_{13}. \label{eq:lemmma:twisting2}
\end{eqnarray}
Furthermore, if\/ $R$ is a bijection and if\/ $R\flip$ is an $(A,B)$-twisting,
then\/ $R^{-1}\flip$ is an $(A\op,B\op)$-twisting.
\end{lemma_sec}


\begin{lemma_sec}   \label{lem:compose_twistings}
Consider two algebras $A$ and\/ $B$, and let\/ $R, Q: A \tens B \rarr A \tens B$
be linear maps such that\/ $R\flip$ and\/ $Q\flip$ are  $(A,B)$-twistings.
If\/ $R_{13}Q_{12} = Q_{12}R_{13}$ and\/ $R_{13}Q_{23} = Q_{23}R_{13}$,
then\/ $RQ\flip$ is again an $(A,B)$-twisting.
\end{lemma_sec}


\begin{prop_sec} \label{prop:twistings:lprp}
Let\/ \pairAB\ be any Hopf system. Then\/
\begin{enumerate}
  \item $\lp\flip$ is an\/ $(A\op,B)$-twisting
  \item $\rp\flip$ is an\/ $(A,B\op)$-twisting.
\end{enumerate}
\end{prop_sec}

\begin{proof}
Take any $a,x\in A$ and $b,d,y\in B$ and write
$\lp(a \tens d) = \sum_i \, p_i \tens q_i$ and $x \tens y = \sum_k \, \rp(v_k \tens w_k)$
with $p_i, v_k \in A$ and $q_i, w_k \in B$. Also observe that
$(q_i \lact x) \tens y = \sum_k \, \rp\!\left((q_i \lact v_k) \tens w_k\vertM\right)$
because of lemma \ref{lem:action_of_xtens1}\@. Now we have
\begin{eqnarray*}
\lefteqn{
  \pairM{y \tens x}{(\id \tens \mult{B}) (\lp)_{12} (\lp)_{13} (a \tens b \tens d)}} \\
&=&
      \textstyle \sum_i \,
      \pairM{y \tens x}{(\id \tens \mult{B}) \left( \lp(p_i \tens b) \tens q_i \vertM\right)}
\\&=&
      \textstyle \sum_i \,
      \pairM{y \tens (q_i \lact x)}{\lp(p_i \tens b)}
\\&=&
      \textstyle \sum_{i,k} \,
      \pairflipM{\rp\!\left((q_i \lact v_k) \tens w_k\vertM\right)}{\lp(p_i \tens b)}
\\&=&
      \textstyle \sum_{i,k} \, \pairM{(q_i \lact v_k) p_i}{w_k b}
\\&=&
      \textstyle \sum_{i,k} \, \pairM{q_i \lact v_k}{p_i \lact w_k b}
\\&=&
      \textstyle \sum_{i,k} \, \pairM{\rp(v_k \tens w_k b)}{q_i \tens p_i}
\\&=&
      \textstyle \sum_{k} \, \pairflipM{\rp(v_k \tens w_k b)}{\lp(a \tens d)}
\\&=&
      \textstyle \sum_{k} \, \pairM{v_k a}{w_k b d}
\\&=&
      \textstyle \sum_{k} \, \pairflipM{\rp(v_k \tens w_k)}{\lp(a \tens b d)}
\\&=&
      \pairM{y \tens x}{\lp(\id \tens \mult{B})(a \tens b \tens d)}
\end{eqnarray*}
which proves (\ref{eq:lemmma:twisting1}) for $R=\lp$. The other cases are similar.
\end{proof}


\begin{cor_sec} \label{cor:twist_map_QD} \hspace{0.5em}
$\rp\lp^{-1}\flip$ is an\/ $(A,B\op)$-twisting.
\end{cor_sec}
\begin{proof}
Both $\rp\flip$ and $\lp^{-1}\flip$ are $(A,B\op)$-twistings.
Analogous to the proof of lemma \ref{lem:P13P23actor}\ we obtain
that $(\rp)_{13}$ commutes with $(\lp)_{12}$ and hence with $(\lp^{-1})_{12}$.
Similarly $(\rp)_{13}$ commutes with $(\lp^{-1})_{23}$.
Now invoke lemma \ref{lem:compose_twistings}.
\end{proof}


\begin{remark_sec} \label{rem:multipl:preview:QD} \rm
Recall that the pairing $P$ is an actor for the weakly unital \context\ $\BBAA$.
Now if $P$ enjoys one of the equivalent conditions (i-iv) in
lemma \ref{lem:wu:commutation:invertibility}, e.g.\ if $\lp$ and $\rp$ commute,
then the map $\rp\lp^{-1}$ appearing in the previous corollary is
actually {\em dual\/} to the \lq inner\rq\ homomorphism
$$ \piP : B \tens A \rarr \Act(\BBAA): y \mapsto P y P^{-1} $$
in the sense that $\pair{\piP(y)}{x} = \pair{y}{\rp\lp^{-1}(x)}$
for all $y \in B \tens A$ and $x \in A \tens B$.
In fact the twist map $\rp\lp^{-1}\flip$ is the one that will be used
later (see  \S \ref{quantum_double}) to construct a {\em quantum double\/}
for the pair $\pair{A}{B\op}$.
\hfill $\star$
\end{remark_sec}
