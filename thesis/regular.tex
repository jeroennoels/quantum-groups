\section{Regularity} \label{par:regularity}



\begin{abs_chp}
Regularity for a Hopf system is mainly about the behaviour of the antipodes.
It is well-known that a Hopf algebra $(A,\Delta)$ has an {\em invertible\/}
antipode if and only if also $(A,\flip\Delta)$ is a Hopf algebra;
in this case the antipodes for $(A,\Delta)$ and $(A,\flip\Delta)$ are each others inverse.
Below we shall obtain a similar result within the Hopf system framework.
\end{abs_chp}



Recall \S \ref{par:opposite}\@.
If \pairAB\ is a \dpa\ in the sense of example \ref{exC:introduction},
then so is \pairABop, obviously. The latter involves the
\contexts\footnote{The superscript $\op$ accompanying $A$ or $B$ will only be
written when it really matters, e.g.\ in (\ref{eq:ABop:contexts_involved}) we wrote
$\Aa \equiv (A; B, \pairing)$ rather than $(A; B\op, \pairing)$
because the product on $B$ is {\em irrelevant\/} at this point.}
\begin{equation} \label{eq:ABop:contexts_involved}
   \Aa \equiv (A; B, \pairing)   \andspace{3em}  \BBop \equiv (B\op; A, \pairing).
\end{equation}
It is not so clear however, whether the {\em invertibility\/} of \pairAB\ does imply
the invertibility of \pairABop\@. On the other hand we do have the following:


\begin{lemma_sec} \label{lem:ABsymmetry}
Let\/ \pairAB\ be a \dpa\@.
Then\/ $\pairAB$ is invertible if and only if\/ $\pair{A\op}{B\op}$ is invertible.
Similarly, $\pair{A}{B\op}$ is invertible if and only if\/
$\pair{A\op}{B}$ is invertible.
\end{lemma_sec}


Let us assume---for a moment---that both \pairAB\ and \pairABop\ are invertible.
Then the pairing between $A$ and $B$ also identifies with an invertible actor for
$$ \BBopAA   \,\equiv\,  \left(\vertM  B\op \tens A; \, A \tens B, \pairing \right), $$
say $P\op = (\lpop,\rpop)$.
So in addition to \lp\ and \rp, we have linear bijections
\lpop\ and \rpop\ from $A \tens B$ onto $A \tens B$, given by

\begin{equation}\label{eq:lamrhoPop:def}
\begin{array}{ccccc}
     \pairM{d \tens c}{\lpop(a \tens b)}
         &=& \pairM{\Pop}{(a \ractop d) \tens (b \ract c)}
         &=& \pairM{d \lact a}{b \ract c}
     \\
     \pairM{d \tens c}{\rpop(a \tens b)}
         &=&        \pairM{\Pop}{(d \lactop a) \tens (c \lact b)}
         &\vertL=&  \pairM{a \ract d}{c \lact b}
\end{array}
\end{equation}

for any $a,c \in A$ and $b,d \in B$. Furthermore, according to lemma \ref{lem:ABsymmetry},
we now have actually {\em four\/} invertible dual pairs,
$$ \pairAB           \hspace{4em}
   \pair{A}{B\op}    \hspace{4em}
   \pair{A\op}{B}    \hspace{4em}
   \pair{A\op}{B\op}$$
{\em each\/} of which induces an antipode either on $A$ or $A\op$, say {\em respectively}
$$ \SA, \, \SA\op : A \rarr \Env(\Aa)
         \hspace{5em}
   ^{\rm op}\!\SA, \, ^{\rm op}\!\SA\op \, : A\op \rarr \Env(\Aa\op) $$
and something similar for the antipodes on $B$ and $B\op$.
Now it is not so hard to see that for instance
$^{\rm op}\!\SA\op = \SA$ and $^{\rm op}\!\SA = \SA\op$,
provided that we either consider these antipodes simply as linear maps from
$A$ into $B'$, or otherwise take into account the appropriate $\id\op$
anti-isomorphisms of \S \ref{par:opposite}\@.
Therefore only $\SA, \SB$ and $\SA\op, \SB\op$ shall be used henceforth,
although the above may be instructive if one wants to
appreciate the {\em symmetry\/} between $A$ and $B$.
Anyway,
$$  \SA\op : A \rarr \Env(\Aa)         \andspace{3em}
    \SB\op : B\op \rarr \Env(\BBop)   $$
are the antipodes associated with \pairABop.


\begin{lemma_sec}
If\/ \pairAB\ and\/ \pairABop\ are invertible dual pairs of algebras, then
\begin{equation} \label{eq:lam_lamop:commutation}
\begin{array}{lcl}
(\lpop)_{21} \mbox{ commutes with\/ } (\lp)_{23}
& \hspace{1em} & (\lpop)_{32} \mbox{ commutes with\/ } (\rp)_{12}
\\
(\rpop)_{21} \mbox{ commutes with\/ } (\rp)_{23}
& \vertL & (\rpop)_{32} \mbox{ commutes with\/ } (\lp)_{12}
  \end{array}
\end{equation}
\end{lemma_sec}

\begin{proof}
Recall that $P\op = (\lpop,\rpop)$ is an actor for $\BBopAA$, hence
$(\rpop,\lpop)$ is an actor for $(\BBopAA)\op= \BB \tens \Aa\op$
(cf.\ \S \ref{par:opposite}). So if we define
$\alpha = \flip \rpop \flip$ and $\beta = \flip \lpop \flip$,
then $(\alpha,\beta)$ is an actor for $\Aa\op \tens \BB$.
Now invoke lemma \ref{lem:commutation:id_tens_lamP}\ with $\EE=\Aa\op$.
This yields the commutation rules in the left column of
(\ref{eq:lam_lamop:commutation}).
Those in the right column are obtained similarly from an appropriate
analogue of lemma \ref{lem:commutation:id_tens_lamP}\
(involving $\BBAA\tens\EE$ rather than $\EE\tens\BBAA$) with $\EE=\BBop$.
\end{proof}



\begin{lemma_sec} \label{lem:ABop_invertible_implies_Hopf_system}
Let\/ \pairAB\ be any Hopf system.
If\/ \pairABop\ is invertible, then\/ \pairABop\ is again a Hopf system.
\end{lemma_sec}
\begin{proof}
This is immediately clear from proposition \ref{prop:multiplicative:comultiplications}\@.
However, we did say that comultiplications are in fact obsolete in our theory,
so we shall give a proof not relying on comultiplications: using the previous lemma
(and rearranging the \lq legs\rq\ if necessary) we obtain
for all $a,c\in A$ and $b,d\in B$ that
\begin{eqnarray*}
      \pairflipM{\rp(a \tens b)}{\lp(c\tens d)}
  &=&
      \pairM{P\op_{14} \, P\op_{32}}{(\rp)_{12} \, (\lp)_{34} \, (a \tens b \tens c \tens d)}
\\&=\vertL&
      \pairM{\eps}{(\rpop)_{14} \, (\lpop)_{32} \,
                   (\rp)_{12}   \, (\lp)_{34}   \,
                   (a \tens b \tens c \tens d)}
\\&\stackrel{(\ref{eq:lam_lamop:commutation})}{=}&
      \pairM{\eps}{(\rp)_{12}   \, (\lp)_{34}   \,
                   (\rpop)_{14} \, (\lpop)_{32} \,
                   (a \tens b \tens c \tens d)}
\\&\stackrel{(*)}{=}&
      \pairM{\eps}{(\rp)_{14}   \, (\lp)_{32}   \,
                   (\rpop)_{12} \, (\lpop)_{34} \,
                   (a \tens d \tens c \tens b)}
\\&=\vertL&
      \pairM{P_{14} \, P_{32}}{(\rpop)_{12} \, (\lpop)_{34} \,
                   (a \tens d \tens c \tens b)}
\\&=\vertL&
      \pairflipM{\rpop(a \tens d)}{\lpop(c\tens b)}
\end{eqnarray*}
where $\eps \equiv \epsA \tens \epsB \tens \epsA \tens \epsB$
(recall that \BB\ and $\BBop$ share the counit $\epsA$).
In $(*)$ we interchanged legs 2 and 4 (notice that this does not affect $\eps$).
\end{proof}


\begin{lemma_sec}  \label{lem:lamrhoPop:antipode}
If\/ \pairAB\ is a Hopf system, then for any\/ $a,c \in A$ and\/ $b,d \in B$
\begin{eqnarray*}
  \pairM{d \tens c}{(\id \tens \SB)\rp^{-1}(a \tens b)}
         &=&  \pairM{d \lact a}{\SB(b) \ract c}
       %   \label{eq:lamPop:antipode1}
  \\
  \pairM{d \tens c}{(\id \tens \SB)\lp^{-1}(a \tens b)}
         &=&  \pairM{a \ract d}{c \lact \SB(b)}.
       %  \label{eq:rhoPop:antipode2}
\end{eqnarray*}
\end{lemma_sec}

\begin{proof}
We show the second formula. Anti-multiplicativity of\/ $\SA$ yields
\begin{eqnarray*}
       \pairM{a \ract d}{c \lact \SB(b)}
  &=&
       \pairM{(a \ract d)c}{\SB(b)}
\\&=&
       \pairM{\SA(c) \, \SA(a \ract d)}{b}
\\&=&
       \pairM{\SA(a \ract d)}{b \ract \SA(c)}
\\&=&
       \pairM{P^{-1}}{(a \ract d) \tens \left(b \ract \SA(c) \vertM\right)}
\\&=&
       \pairM{\left( d \tens \SA(c) \vertM\right) P^{-1}}{a \tens b}
\\&=&
       \pairM{d \tens c}{(\id \tens \SB)\lp^{-1}(a \tens b)}.
\end{eqnarray*}
The other formula is similar.
\end{proof}




\begin{lemma_sec} \label{lem:bijective_S_implies_ABop_invertible}
Let\/ \pairAB\ be any Hopf system.
If\/ $\SB$ is a bijection from\/ $B$ onto\/ $B$, then \pairABop\ is invertible.
By symmetry, the same conclusion applies when\/ $\SA$ is a bijection from\/
$A$ onto\/ $A$
(cf.\ lemma \ref{lem:ABsymmetry}).
\end{lemma_sec}
\begin{proof}
Assume $\SB$ to be a bijection from $B$ onto $B$.
First we have to show that the pairing between $A$ and $B$
indeed induces an {\em actor\/} for \BBopAA, i.e.\ that the functional
$A \tens B \rarr \kk : a \tens b \mapsto \pair{a}{b}$
is \stricta\ continuous within the \context\ \BBopAA\
(cf.\ remark \ref{rem:warning:P_strict_cont}).
In particular we have to prove that given any $a \in A$ and $b \in B$
there exist $x,y \in A \tens B$ such that
(cf.\ equation \ref{eq:lamrhoPop:def})
$$  \pairM{d \lact a}{b \ract c}    \:=\:   \pairM{d \tens c}{x}
           \hspace{5em}
    \pairM{a \ract d}{c \lact b}    \:=\:   \pairM{d \tens c}{y}.     $$
for all $c \in A$ and $d \in B$.
From lemma \ref{lem:lamrhoPop:antipode}\ it is clear that
$$  x = (\id \tens \SB)\rp^{-1} \! \left(a \tens \SB^{-1}(b) \vertM\right)
      \hspace{4em}
    y = (\id \tens \SB)\lp^{-1} \! \left(a \tens \SB^{-1}(b) \vertM\right)  $$
will do the job.
So the pairing indeed corresponds to an actor for \BBopAA,
say $(\lpop,\rpop)$, where
\begin{equation} \label{eq:lamrhoPop:antipodes:bijective}
\begin{array}{ccc}
   \lpop &=& (\id \tens \SB)\rp^{-1} (\id \tens \SB^{-1})
   \\
   \rpop &\vertL=& (\id \tens \SB)\lp^{-1} (\id \tens \SB^{-1})
\end{array}
\end{equation}
are bijections from $A \tens B$ onto $A \tens B$.
Now it only remains to show that the pair $\left((\lpop)^{-1}, (\rpop)^{-1} \vertM\right)$
is still an actor for \BBopAA\@.
To prove this, we shall rely on lemma \ref{lem:char:actor:wu}\@.
By assumption $(\lp,\rp)$ is an actor for $\BBAA$,
and hence $(\epsA \tens \epsB) \lp = (\epsA \tens \epsB) \rp$.
Recall that \BB\ and $\BBop$ share the counit $\epsA$,
and also recall that $\epsB \SB = \epsB$ (corollary \ref{cor:antipode_and_counit}).
Taking inverses in (\ref{eq:lamrhoPop:antipodes:bijective}) yields
\begin{equation}\label{eq:lamrhoPop:antipodes:bijective:bis}
\begin{array}{ccc}
   (\lpop)^{-1} &=& (\id \tens \SB)\rp (\id \tens \SB^{-1})
\\
   (\rpop)^{-1} &\vertL=& (\id \tens \SB)\lp (\id \tens \SB^{-1})
\end{array}
\end{equation}
If we now apply $\epsA \tens \epsB$, we obtain
$(\epsA \tens \epsB) \circ (\lpop)^{-1}  =  (\epsA \tens \epsB) \circ (\rpop)^{-1}$.
The result follows from lemma \ref{lem:char:actor:wu}\@.
\end{proof}



\begin{lemma_sec} \label{lem:ABop_invertible_implies_SSop_equal_id}
Let\/ \pairAB\ be any Hopf system, and assume also\/ \pairABop\ to be a Hopf system.
If\/ $\SB(B) \subseteq B$, then\/ $\SB\op \SB = \id$.
\end{lemma_sec}
\begin{proof}
Combining lemma \ref{lem:lamrhoPop:antipode}\ with (\ref{eq:lamrhoPop:def})
we obtain
\begin{equation} \label{eq:lamrhoPop:antipodes:commutation}
\begin{array}{lll}
   \lpop (\id \tens \SB) &=& (\id \tens \SB)\rp^{-1}
   \\
   \rpop (\id \tens \SB) &=\vertL& (\id \tens \SB)\lp^{-1},
\end{array}
\end{equation}
which is completely similar to (\ref{eq:lamrhoPop:antipodes:bijective}),
though now obtained in a different setting.
By assumption $\lpop$ and $\rpop$ are bijective, hence e.g.\ the second formula in
(\ref{eq:lamrhoPop:antipodes:commutation}) can be rewritten as
$(\id \tens \SB) \lp = (\rpop)^{-1} (\id \tens \SB)$, and
thus we get for any $a,c \in A$ and $b,d \in B$ that
\begin{eqnarray*}
       \pairM{d \tens c}{(\id \tens \SB)\lp(a \tens b)}
  &=&
       \pairM{d \tens c}{(\rpop)^{-1}\!\left(\vertM  a \tens \SB(b)\right)}
\\&=&
       \pairM{(\Pop)^{-1}}{(d \lactop a) \tens \left(\vertM c \lact \SB(b)\right)}
\\&=&
       \pairM{\SA\op(a \ract d)}{c \lact \SB(b)}
\\&=&
       \pairM{\SA\op(a \ract d) \,c}{\SB(b)}
\\&=&
       \pairM{c}{\SB(b) \ract \SA\op(a \ract d)}.
\end{eqnarray*}
On the other hand we also have
\begin{eqnarray*}
       \pairM{d \tens c}{(\id \tens \SB)\lp(a \tens b)}
  &=&
       \pairM{\left(\vertM  d \tens \SA(c)\right)P}{a \tens b}
\\&=&
       \pairM{P}{(a \tens b) \ract \left(\vertM d \tens \SA(c)\right)}
\\&=&
       \pairM{a \ract d}{b \ract \SA(c)}
\\&=&
       \pairM{\SA(c)\,(a \ract d)}{b}
\\&=&
       \pairM{\SA(c)}{(a \ract d) \lact b}
\\&=&
       \pairM{c}{\SB \! \left(\vertM (a \ract d) \lact b \right)}.
\end{eqnarray*}
Since $A \ract B = A$, it follows that for all $a \in A$ and $b \in B$
\begin{equation}\label{eq:lem:SopS:id}
  \SB(b) \ract \SA\op(a)  \:=\: \SB(a \lact b).
\end{equation}
Observe that it was essential to assume $\SB(B) \subseteq B$, in order to be
able to make the above computations.
For the same reason we may now apply $\epsB$ to (\ref{eq:lem:SopS:id}).
Together with the fact that $\epsB \SB = \epsB$
(corollary \ref{cor:antipode_and_counit}) we get
$$ \pairM{a}{\SB\op (\SB(b))}  \:=\: \pairM{\SA\op(a)}{\SB(b)}  \:=\: \pairM{a}{b} $$
and we conclude that $\SB\op(\SB(b)) = b$.
\end{proof}
\vspace{2ex}

Of course a similar result applies to the antipodes on $A$, by symmetry.
On the other hand, interchanging the roles of \pairAB\ and \pairABop\ yields

\begin{cor_sec}
If\/ $\SB\op(B) \subseteq B$, then\/ $\SB \SB\op = \id$.
\end{cor_sec}

\begin{cor_sec}  \label{cor:ABop_invertible_implies_S_bijection}
When both\/ $\SB(B) \subseteq B$ and\/ $\SB\op(B) \subseteq B$,
then\/ $\SB$ and\/ $\SB\op$ are bijections from\/ $B$ onto\/ $B$.
Furthermore\/ $\SB^{-1} = \SB\op$.
\end{cor_sec}



\begin{prop_sec} \label{prop:regular:hs}
Let\/ \pairAB\ be a Hopf system. The following are equivalent:
\begin{enumerate}
\item
$\SA$ and\/ $\SB$ are bijections,
resp.\ from\/ $A$ onto\/ $A$ and from\/ $B$ onto\/ $B$.
\item
The pair \pairABop\ is a Hopf system and the antipodes\/
$\SA$ and\/ $\SA\op$ leave\/ $A$ invariant, whereas\/ $\SB$ and\/ $\SB\op$
leave\/ $B$ invariant, e.g.\ $\SA(A) \subseteq A$, etc.
\end{enumerate}
\end{prop_sec}

\begin{proof}
(i) $\Rightarrow$ (ii).
According to lemma \ref{lem:bijective_S_implies_ABop_invertible},
the pair \pairABop\ is invertible, and therefore a Hopf system
(cf.\ lemma \ref{lem:ABop_invertible_implies_Hopf_system}).
Now lemma \ref{lem:ABop_invertible_implies_SSop_equal_id}\
yields $\SB\op \SB = \id$, and analogously we have $\SA\op \SA = \id$.
Since by assumption $\SA(A)=A$ and $\SB(B)=B$, it follows that
$\SA\op(A) = A$ and $\SB\op(B) = B$.

(ii) $\Rightarrow$ (i). Corollary \ref{cor:ABop_invertible_implies_S_bijection}\
and its analogue for the antipodes on $A$.
\end{proof}


\begin{defn_sec} \label{def:regular:hs}
A Hopf system \pairAB\ is said to be {\em regular\/} whenever it enjoys
the equivalent conditions (i-ii) in the previous proposition.
\end{defn_sec}

An example of a non-regular Hopf system will be given in example
\ref{ex:non_regular_Hopf_system}\@.
