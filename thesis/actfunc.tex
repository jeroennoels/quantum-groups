\section{Actors \& functionals}
\label{par:actors_and_functionals}


Throughout this paragraph $\EE \equiv \EOP$ is an \context\@.

\begin{abs_chp} 
For a while we adopt a rather topological point of view, but as we proceed,
we shall gradually link this to the original algebraic approach.
We show how certain linear functionals on $\Om$ naturally induce actors.
Eventually we investigate a notion of {\em invertibility\/} for actors
and---indirectly---for functionals.
\end{abs_chp}




\subsection{Three natural compatible topologies}

\begin{defn} \label{def:EEcompatible}
  A locally convex topology on $\Om$ is said to be \EEdash {\em compatible\/} if
  it is finer than the $\sigma(\Om,E)$ topology (hence Hausdorff)
  and making $\Om$ into a {\em topological\/} $E_\sigma$-bimodule,
  i.e.\ the structure maps are separately continuous.
\end{defn}

\begin{defn}  \label{def:three_topologies}
 We introduce the following topologies on $\Om$:
 \begin{description}
  \item[\strictw]
   is synonymous for the {\em weak\/} topology on $\Om$, i.e.\ the $\sigma(\Om,E)$ topology.
  \item[\strictm]
   Consider $x \lact (\cdot)$ and $(\cdot) \ract x$, with $x$ running through $E$,
   as a family of linear maps from $\Om_\sigma$ into $\Om$\@.
   These maps induce an inductive {\sc lc} topology \mbox{\cite[\S 6.6]{Jarchow}}\
   on $\Om$, which will be referred to as the {\em \strictm}\ topology on $\Om$.
   In other words, the \strictm\ topology is the {\em finest\/} locally convex
   topology ${\mathcal I}$ on $\Om$ such that the above mappings are continuous
   from $\Om_\sigma$ into $(\Om,{\mathcal I})$.
  \item[\stricta]
   Consider $\om \ract (\cdot)$ and $(\cdot) \lact \om$, with $\om$ running through $\Om$,
   as a family of linear maps from $E_\sigma$ into $\Om$.
   The inductive {\sc lc} topology on $\Om$ defined by these mappings
   will be called the {\em\stricta}\ topology on $\Om$.
 \end{description}
 Whenever $\Om$ is to be considered with one of these topologies,
 we shall write $\Om_\flat\equiv\Om_\sigma$, $\Omnl$ and $\Omsp$ respectively;
 their topological duals will be denoted $\Om\wdl$, $\Om\mdl$ and $\Om\adl$.
 So a {\em sub\/}script following $\Om$ merely indicates which topology
 it is endowed with; on the other hand, if $\Om$ is accompanied by a {\em super\/}script,
 we are always referring to some subspace of the algebraic dual $\Om'$.
\end{defn}


\begin{remarks}  \label{rem:three_topologies}
\item
   If $(\lam,\rho)$ is an actor for \EE, it follows that $\lam$ and $\rho$ are
   {\em continuous\/} mappings from $\Omsp$ into $\Om_\flat$.
 \item
   Let $\Delta$ be the weak comultiplication on $\Om$
   (proposition \ref{prop:induced_comult}). Then
   \begin{eqnarray*}
     \Om\adl &=& \left \{ f \in \Om' \left|
          \begin{array}{c}
            \mbox{for all $\om\in \Om$, both $\pair{f}{(\cdot) \lact \om}$
                  and $\pair{f}{\om \ract (\cdot)}$}   \\
            \mbox{are weakly continuous linear functionals on $E$}
          \end{array} \right. \!\!\! \right\}  \\
        &=& \left\{ f \in \Om'  \left|\vertL\right.\;
          (\overline{f\, \tens}\,\id)\Delta(\Om) \:\subseteq \: \Om \:\supseteq \:
          (\id\, \overline{\tens f})\Delta(\Om) \right\}.
   \end{eqnarray*}
   Slice maps like $\overline{f\, \tens}\, \id: \Om \fubtens \Om \rarr \overline{\Om}$
   are defined in
   \mbox{remark \ref{rem:fubtens_of_maps}.iii and (\ref{eq:def:functional_fubtens_id})}\@.
%%%%%%%%%%%%%%%%%%%%
%   and obey the following Fubini
%   type property:  for any $y\in E$ and $\phi \in \Om \fubtens \Om$ we have
%   $\pair{y}{(\overline{f \tens}\, \id)(\phi)} = \pair{f}{(\id \fubtens f_y)(\phi)}$.
%%%%%%%%%%%%%%%%%%%%
   In particular, if \EE\ is algebraic (definition \ref{def:algebraic_context})
   then $\Om\adl=\Om'$.
\end{remarks}


\begin{prop} \label{prop:three_topologies_are_compatible}
  The \strictw\ and \stricta\ topologies on $\Om$ are \EEdash compatible.
  If\/ $\Om$ is unital as an \Ebimod, then also the \strictm\ topology is \EEdash compatible.
\end{prop}
\begin{proof}
We already knew that $\Om_\sigma$ is a {\em topological\/} $E_\sigma$-bimodule
(remark \ref{rem:topalg}.iii) hence the \strictw\ topology on $\Om$ is \EEdash compatible.
This also implies the \strictm\ and \stricta\ topologies to be finer than the weak topology.
Next we show that $\Omsp$ is a topological $E_\sigma$-bimodule:
appealing to the universal property of \stricta\ as an inductive {\sc lc} topology,
it only remains to show that for all $x\in E$ and $\om \in \Om$
$$  x\lact \left(\vertM (\cdot) \lact \om\right)     \hspace{3em}
    x\lact \left(\vertM \om\ract (\cdot) \right)      \hspace{3em}
    \left(\vertM (\cdot) \lact \om \right) \ract x    \hspace{3em}
    \left(\vertM \om \ract (\cdot) \right) \ract x $$
are continuous mappings from $E_\sigma$ into $\Omsp$, which is easy.
%
Finally, assume $\Om$ to be unital and consider the \strictm\ topology.
For all $x \in E$, the mappings $x\lact (\cdot)$ and $(\cdot) \ract x$
are weakly-\strictm\ continuous and a fortiori \strictm\ continuous.
Now it only remains to show that $(\cdot) \lact \om$ and $\om \ract (\cdot)$
are continuous mappings from $E_\sigma$ into $\Omnl$.
Using $\Om=\Om \ract E$, any $\om \in\Om$ can be written as $\om = \sum_k \om_k \ract x_k$,
hence $(\cdot) \lact \om = \sum_k \left(\vertM (\cdot) \lact \om_k \right) \ract x_k$
enjoys the desired continuity because the $(\cdot) \ract x_k$
are weakly-\strictm\ continuous. For $\om \ract (\cdot)$ we need $\Om=E \lact \Om$.
\end{proof}


\begin{cor}  \label{cor:finest_EEcompatible}
  The \stricta\ topology is the finest \EEdash compatible topology on $\Om$,
  the \strictw\ topology is the coarsest one.
  When $\Om$ is unital, the above topologies compare as follows:
  \mbox{\strictw\ $\preceq$ \strictm\ $\preceq$ \stricta}\@.
  Consequently $\Om\wdl \,\subseteq \,\Om\mdl \,\subseteq \,\Om\adl$.
\end{cor}


\begin{exA}  \rm
  Recall the \context\ $(F; \kk S, \pairing)$ of example \ref{exA:introduction}\@.
  Clearly {\em every\/} {\sc lc} topology on $\kk S$ makes
  $(\cdot) \lact \delta_s  \,=\, \pair{\,\cdot}{\delta_s} \, \delta_s
                           \,=\, \delta_s \ract (\cdot)$
  into continuous mappings from $F_\sigma$ into $\kk S$,
  hence the \stricta\ topology must be the finest {\sc lc} topology on $\kk S$.
  In particular $\kk S$ is \stricta\ {\em complete\/} (a phenomenon which is likely
  to become characteristic for {\em algebraic\/} \contexts).
%%%%%% (definition \ref{def:algebraic_context}\ and example \ref{exA:algebraic}).
%%%%%%  and  \ref{prop:completeness_crit}).
  \hfill $\star$
\end{exA}

%%%%%%%%%%%%%%%%%%%%%%%%
%\begin{exA} \rm
%  Recall the \context\ $(K(X); \kk X, \pairing)$ as introduced in
%  example \ref{exA:introduction}\@. Let $\delta$ be the canonical
%  map $X \rarr \kk X$ where $\kk X$ is endowed with either the
%  weak or the \strictm\ topology. In {\em both\/} cases the initial
%  topology on $X$ induced by $\delta$ coincides with the original
%  topology.
%\end{exA}
%\begin{proof}
%\end{proof}
%%%%%%%%%%%%%%%%%%%%%%%

\begin{exB} \label{exB:weak_is_compatible}   \rm
  Let $\Aa\equiv (A; A^*, \pairing)$ be as in example \ref{exB:normed_algebra_context}\@.
  Then the $\sigma(A^*, A^{**})$ topology is $\Aa$-compatible because of
  (\ref{eq:exB:normed_algebra_context}), hence $A^{**} \subseteq A^{*\sharp}$.
  The norm topology will usually not be $\Aa$-compatible.

{\small
  In the special case that $A$ is the algebra of compact operators on a Hilbert space,
  one can show that $A^{**} = A^{*\sharp}$, hence in this case the norm topology
  on $A^*$ is finer than the \stricta\ topology (the former being a Mackey topology).
}
\end{exB}




\subsection{Actors induced by functionals}
\label{subsec:Acts_ind_by_funcs}

We already encountered several \Ebimod\ structures associated with an
\context\ $\EE \equiv \EOP$. Also $\Om'$ can be made into an \Ebimod,
the actions of $E$ on $\Om'$ (denoted by juxtaposition) being given by
\begin{equation} \label{eq:Om'isEbimod}
   \pair{xf}{\om} = \pair{f}{\om \ract x} \andspace{3em}
   \pair{fx}{\om} = \pair{f}{x\lact\om}
\end{equation}
for $f\in \Om'$, $x\in E$ and $\om \in \Om$.
One could say this \Ebimod\ structure on $\Om'$ is \lq parallel\rq\ to the
one on \ActE, but in some sense dual to those on $\Om$ or $E'$.
It is easy to prove that $\Om\wdl$, $\Om\mdl$ and $\Om\adl$ are sub-$E$-bimodules of $\Om'$.


\begin{prop} \label{prop:act_by_fuctionals}
  There are unique linear mappings
  \begin{enumerate}
    \item $\iota : \Om\wdl \rarr E$
    \item $\mu   : \Om\mdl \rarr M(E)$
                     \hspace{1em}---provided that \EE\ is non-degenerate---
    \item $\theta: \Om\adl \rarr \ActE:
                     f \mapsto \theta(f)\equiv\lamrho{f}$
  \end{enumerate}
  obeying respectively---for every $x\in E$ and $\om \in \Om$,
  \begin{enumerate}
    \item $\pair{\iota(f)}{\om} = \pair{f}{\om}\,$
          for all $f \in \Om\wdl$
    \item $\pair{\mu(f)x}{\om}   = \pair{f}{x \lact \om}\,$ and
          $\,\pair{x\mu(f)}{\om} = \pair{f}{\om \ract x}\,$
          for all $f \in \Om\mdl$
    \item $\pair{x}{\rho_f(\om)}\,   = \pair{f}{x \lact \om}\,$ and
          $\,\pair{x}{\lam_f(\om)}\, = \pair{f}{\om \ract x}\,$
          for all $f \in \Om\adl$.
  \end{enumerate}
\end{prop}
%
\begin{proof}
  (i) $\:\iota$ is nothing but the inverse of the canonical embedding
  $E\hookrightarrow \Om'$.

  (ii) Take any $f\in \Om\mdl$. From definition \ref{def:three_topologies}\ it follows
  that the functionals $fx = \pair{f}{x \lact (\cdot)}$ and $xf = \pair{f}{(\cdot) \ract x}$
  are actually in $\Om\wdl$ for all $x\in E$, which allows us to define a multiplier
  $\mu(f)$ of $E$ by $\mu(f)x = \iota(fx)$ and $x\mu(f) = \iota(xf)$.
%%%%%%%%%%%%%%
%  Since $\Om'$ is an \Ebimod\ and $\iota: \Om\wdl \rarr E$
%  an \Ebimod\ morphism, $\mu(f)$ is indeed a two-sided
%  multiplier of $E$, obviously satisfying equation (ii) above.
%%%%%%%%%%%%%%%

  (iii)  Next consider any $f\in \Om\adl$ and $\om \in \Om$\@. This time it follows that
  $\pair{f}{(\cdot) \lact \om}$ and $\pair{f}{\om \ract (\cdot)}$
  are both continuous functionals on $E_\sigma$\@. Hence they identify with elements in $\Om$,
  respectively denoted by $\rho_f(\om)$ and $\lam_f(\om)$.
  So given $f\in \Om\adl$ we have defined linear maps $\lam_f$ and $\rho_f$ from $\Om$ into $\Om$,
  satisfying (iii). It remains to show that $\lamrho{f}$ is an actor for \EE\@.
  Now for all $x,y \in E$ and $\om \in \Om$ we have
  $$\pair{x}{\lam_f(\om \ract y)}
         = \pair{f}{(\om \ract y) \ract x}
         = \pair{f}{\om \ract yx}
         = \pair{yx}{\lam_f(\om)}
         = \pair{x}{\lam_f(\om) \ract y},  $$
  so $\lam_f$ is a right $E$-module morphism.
  Similarly $\rho_f$ is a left $E$-module map and
  $$\pair{x}{\rho_f(\om \ract y)}
       = \pair{f}{x \lact (\om \ract y)}
       = \pair{f}{(x \lact \om) \ract y}
       = \pair{y}{\lam_f(x\lact\om)} $$
  proves the \biap\ (\ref{eq:bi-actor}). This completes the proof.
\end{proof}


\begin{remark} \label{rem:theta_and_Delta} \rm
 For any $f\in \Om\adl$, the actor $\theta(f)=\lamrho{f}$ is also given by
 $$ \lam_f = (\id \,\overline{\tens f})\Delta    \andspace{3em}
    \rho_f = (\overline{f \tens} \,\id)\Delta   $$
 (cf.\ remark \ref{rem:three_topologies}.ii).
 This should be compared with (\ref{eq:actions_in_terms_of_Delta}).
 \hfill $\star$
\end{remark}

Below we collect some properties of the maps introduced in proposition
\ref{prop:act_by_fuctionals}\@.
First observe that $\Om\wdl$, $\Om\mdl$, $\Om\adl$, $E$, $M(E)$ and \ActE\
are all $E$-bimodules under \lq multiplication\rq\ in one way or the other.
Also recall lemma \ref{embedding_of_M(E)}\@.

\begin{cor} \label{cor:act_by_fuctionals}
  The maps $\iota$, $\mu$ and $\theta$ are \Ebimod\ morphisms;
  $\mu$ extends $\iota$, whereas $\theta$ extends $j\circ\iota$.
  If $\Om$ is unital, then $\mu$ and $\theta$ are injective;
  moreover $\Om\mdl \subseteq \Om\adl$, and $\theta$ extends $\jhat\circ\mu$ accordingly.
  Up to identification we have $\iota \subseteq \mu \subseteq \theta$.
\end{cor}
%
\begin{proof}
  Only the last assertion is not completely trivial.
  So let $\Om$ be unital as an \Ebimod, take any $f\in \Om\mdl$ and denote
  $\jhat(\mu(f))\equiv (\lam,\rho) \in \ActE$.
  By (\ref{eq:def_of_jhat}) we get for all $x\in E$ and $\om\in\Om$ that
  $\pair{x}{\lam(\om)} = \pair{x\mu(f)}{\om} = \pair{f}{\om \ract x},$
  and similarly $\pair{x}{\rho(\om)} = \pair{f}{x \lact \om}$.
  It follows that $\pair{f}{\om \ract (\cdot)}$ and
  $\pair{f}{(\cdot) \lact \om}$ are continuous functionals on $E_\sigma$,
  for any $\om\in\Om$, hence $f$ itself must be \stricta\ continuous.
  So $f\in\Om\adl$ and clearly
  $\theta(f)=\lamrho{f}=(\lam,\rho)=\jhat(\mu(f))$.
\end{proof}
\vspace{2ex}

So the maps $\mu$ and $\theta$ are injective, provided that $\Om$ is unital,
but it would of course be nice if they where actually bijections.
For this we need an extra assumption: the existence of a counit
(see \S \ref{par:weakly_unital_contexts}).




\subsection{Invertibility of actors and functionals}

Given a non-degenerate \context\ \EE, we have several notions
of invertibility. One can for instance consider invertible elements
in the unital algebras $M(\EE)$, $\EnvE$ or $\PreE$\@.
Below we shall see that it is also possible to define a concept of invertibility within \ActE,
although in general \ActE\ is not really an algebra;
in fact the latter notion is the one we will use in
chapter \ref{chapter:Hopf_systems}\ to express \lq group-like\rq\
properties. We start however with a trivial observation:

\begin{lemma} \label{lem:EE-invertibility}
  Let\/ $\EE \equiv \EOP$ be any \context, and let\/ $a\equiv(\lam,\rho)$
  be an actor for\/ \EE\@. Then the following are equivalent:
  \begin{enumerate}
    \item $a$ is invertible within\/ \PreE\ and $a^{-1}$ belongs to \ActE\ again.
    \item $\lam$ and $\rho$ are bijections on\/ $\Om$ and\/
          $(\lam^{-1},\rho^{-1})$ satisfies the \biap.
  \end{enumerate}
\end{lemma}


\begin{defn} \label{def:EE-invertibility}
  Let\/ $\EE \equiv \EOP$ be any \context\@.
  An {\em actor\/} for\/ \EE\ is said to be \EEdash {\em invertible\/}
  when it satisfies the conditions of the previous lemma. On the other hand,
  a linear {\em functional\/} $f:\Om\rarr \kk$ will be called \EEdash {\em invertible\/} if
  \begin{enumerate}
    \item $f$ is \stricta\ continuous,
          i.e.\ $f$ actually belongs to $\Om\adl$, and
    \item $\theta(f)$ is \EEdash invertible as an actor,
  \end{enumerate}
  with $\theta: \Om\adl \rarr \ActE$ as described in \S \ref{subsec:Acts_ind_by_funcs}.
\end{defn}

Whenever a pre-actor $a\equiv(\lam,\rho)$ is invertible in the algebra \PreE\
or, in other words, when $\lam$ and $\rho$ are bijective, we can define an
inner automorphism
$$  \pi_a : \PreE \rarr \PreE : x \mapsto axa^{-1}.  $$

%%%%%%%%%%%%
%Somehow the invariance properties of $\pi_a$ reflect the \lq amount of regularity\rq\
%\mbox{in $a$}\@. For instance when $a$ happens to be an invertible element in \EnvE,
%then $\pi_a$ restricts to an automorphism of \EnvE\@.
%Or one could require $\pi_a$ to leave $E$ invariant, as is the case when $a$ is
%invertible within the algebra $M(\EE)$ introduced in \S \ref{sec:multipliers}\@.
%In the following lemma though, we study a much weaker condition:
%%%%%%%%%%%%%%

\begin{lemma} \label{lem:invertibility_comm_pi}
  Let\/ $\EE \equiv \EOP$ be any \context\ with\/ $\Om$ unital.
  Let $a\equiv(\lam,\rho)$ be an actor for\/ \EE\ such that $\lam$ and $\rho$
  are bijections, and consider
  \begin{enumerate}
     \item $\pi_a(E) \subseteq \ActE$
     \item $a$ is \EEdash invertible
     \item $\lam$ and $\rho$ commute.
  \end{enumerate}
  Whenever two of these statements are satisfied, all three of them will hold.
\end{lemma}
\begin{proof}
  Under the circumstances, (ii) merely states that $(\lam^{-1},\rho^{-1})$
  satisfies the \biap\ (\ref{eq:bi-actor}).
  Similarly (i) means that $(\lam\lam_x\lam^{-1}, \, \rho^{-1}\rho_x\rho)$
  enjoys this property for all $x\in E$.
  Furthermore $\lam$ is a right \mbox{$E$-module}\ morphism, hence so is $\lam^{-1}$.
  We thus obtain, for all $x,y,z \in E$ and $\om \in \Om$, the following \lq circle\rq :
  $$ \begin{array}{lclcl}
          \pairM{x}{(\rho\lam^{-1})(y\lact \om \ract z)}
      \!\!\!&=& \!\!\!
          \pairM{x}{\rho(\lam^{-1}(y\lact \om) \ract z)}
      \!\!\!&=& \!\!\!
            \pairM{z}{\lam(x \lact \lam^{-1}(y\lact \om))}  \\
      \!\!\!&=& \!\!\!
            \pairM{z}{(\lam\lam_x\lam^{-1})(y\lact \om)}
      \!\!\!&\stackrel{\rm (i)}{=}&\!\!\!
            \pairM{y}{(\rho^{-1}\rho_x\rho)(\om \ract z)}   \\
      \!\!\!&=&\!\!\!
            \pairM{y}{\rho^{-1}(\rho(\om \ract z) \ract x)}
      \!\!\!&\stackrel{\rm (ii)}{=}&\!\!\!
            \pairM{x}{\lam^{-1}(y \lact \rho(\om \ract z))}  \\
      \!\!\!&=&\!\!\!
            \pairM{x}{(\lam^{-1}\rho)(y \lact \om \ract z)}
      \!\!\!&\stackrel{\rm (iii)}{=}&\!\!\!
            \pairM{x}{(\rho\lam^{-1})(y\lact \om \ract z)}.
  \end{array} $$
  Since by assumption $E \lact \Om \ract E = \Om$ and $\rho(\Om \ract E)=\Om$,
  the result follows.
\end{proof}
