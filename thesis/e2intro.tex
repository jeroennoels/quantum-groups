
\paragraph{Conventions}
Throughout this chapter and the next one, $q$ denotes a fixed real number
with \mbox{$0<q<1$}.
Furthermore, let us agree that comultiplications, counits and antipodes
will always be denoted by $\Delta$, $\eps$ and $S$, respectively.


\section{Introduction}
\label{par:E2:introduction}


\begin{abs_chp}
We recall the description of the quantum $E(2)$ group and its Pontryagin dual
on the Hopf $^*$-algebra level. In particular we introduce a dual pair \UqAq\
of Hopf $^*$-algebras \cite{Koelink:thesis,Koelink:QE2,FonsWor:QE2,Wor:Affiliated}\@.
Furthermore we introduce the operators $\Gamma,\Phi,\Omega,\Psi$ which will
be used throughout chapters
\ref{chapter:Fourier_context_for_quantumE2}\ and
\ref{chapter:Harmonic_analysis_on_quantumE2}\@.
\end{abs_chp}


\subsection{Introducing two Hopf $^*$-algebras}

Let \Aq\ be the algebra with unit $1$, generators $\alpha,\beta,\gamma,\delta$
and relations $$ \alpha\beta = q\,\beta\alpha     \hspace{3em}
   \alpha\gamma = q\,\gamma\alpha   \hspace{3em}
   \beta\delta = q\,\delta\beta     \hspace{3em}
   \gamma\delta = q\,\delta\gamma   \hspace{3em}
   \beta\gamma = \gamma\beta $$
$$   \alpha\delta = 1 = \delta\alpha.  $$
We make \Aq\ into a $^*$-algebra as follows:
$$ \begin{array}{lcl}
     \alpha^* = \delta   &\hspace{1em} & \beta^* = -q\,\gamma \\
     \gamma^* = -q^{-1} \beta  & & \delta^* = \alpha
   \end{array}$$
In other words, $\alpha$ is unitary whereas $\beta$ is normal.
Eventually \Aq\ is made into a Hopf $^*$-algebra by defining $\Delta, \eps$ and $S$
on the generators:
$$ \begin{array}{lcl}
   \Delta(\alpha) = \alpha \tens \alpha & \hspace{5em} &
                       \Delta(\beta) = \alpha \tens \beta + \beta \tens \delta  \\
   \Delta(\gamma) = \gamma \tens \alpha + \delta \tens \gamma & &
                       \Delta(\delta) = \delta \tens \delta
\end{array} $$
$$ \begin{array}{lclclcl}
  \eps(\alpha) = 1 & & \eps(\beta) = 0  &\hspace{6em} &
                S(\alpha) = \delta  & &  S(\beta) = -q^{-1} \beta \\
  \eps(\gamma) = 0 & & \eps(\delta) = 1 & &
                S(\gamma) = -q\,\gamma & & S(\delta) = \alpha
\end{array}$$

Next, let \Uq\ be the algebra with unit $1$, generators $a,b,c,d$
and relations
\begin{equation}\label{eq:relations:Uq}
   ab = q \,ba   \hspace{3em}
   ac = q^{-1}ca \hspace{3em}
   bc = cb       \hspace{3em}
   ad = 1 = da   \hspace{1em}
\end{equation}
Notice $bc$ is central in \Uq\@. The $^*$-structure on \Uq\ is
given by
$$ \begin{array}{lcl}
     a^* = a  & \; & b^* = c \\
     c^* = b  &    & d^* = d
   \end{array} $$
Now also \Uq\ becomes a Hopf $^*$-algebra by defining
$$ \begin{array}{lcl}
  \Delta(a) = a \tens a  &\hspace{5em} & \Delta(b) = a \tens b + b \tens d \\
  \Delta(c) = a \tens c + c \tens d   &  & \Delta(d) = d \tens d
  \end{array}  $$
$$ \begin{array}{lclclcl}
  \eps(a) = 1 & &  \eps(b) = 0 & \hspace{6em} & S(a) = d & & S(b) = -q^{-1}b \\
  \eps(c) = 0 & &  \eps(d) = 1 &  & S(c) = -q\,c & & S(d) = a
\end{array} $$

It is not difficult to show that \Aq\ and \Uq\ are indeed
Hopf $^*$-algebras. In \cite{Koelink:thesis,Koelink:QE2}\ \Aq\ was
denoted by $\Aq(\widetilde{M}(2))$ and $\Uq$ by $\Uq(\mathfrak{m}(2))$. Since
we are dealing with these two Hopf $^*$-algebras
exclusively, we prefer the shorthand form.


\subsection{The pairing}
We want the pairing \UqAq\ to take the following values on the generators:
\begin{equation}\label{eq:pairing:generators}
  \begin{array}{c}
   \begin{array}{lclcl}
     \pair{a}{\alpha} = q^\frac{1}{2} & \hspace{1em} & \pair{a}{\delta}= q^{-\frac{1}{2}} &
                           \hspace{1em}  & \pair{b}{\beta} = 1 \\
     \pair{c}{\gamma} = 1  &  & \pair{d}{\alpha} =q^{-\frac{1}{2}}&
                                   & \pair{d}{\delta} = q^\frac{1}{2}
   \end{array} \\
   \pair{x}{\xi} =0 \vertXL\; \mbox{ for other choices $x, \xi$ of generators.}
 \end{array}
\end{equation}
Observe that
$\{\alpha^l \beta^m \gamma^n \mid l \in \ZZ;\, m,n \in \NN \}$ is
a basis for the linear space \Aq\ whereas
$\{ a^p b^r c^s \mid p \in \ZZ;\, r,s \in \NN \}$
is a basis for \Uq\@. Of course we assume $\alpha^{-n} = \delta^n$
and $a^{-n} = d^n$ for $n \in \NN$. In \cite{Koelink:thesis,Koelink:QE2}\
H. T. Koelink calculated the full pairing on these basis elements;
we turn his result into a definition:


\begin{defn}
Let $\pairing : \Uq \times \Aq \rarr \kk$ be the bilinear form such that
\begin{equation} \label{eq:fullpairing}
 \pair{a^p b^r c^s}{\alpha^l \beta^m \gamma^n}
      \:=\: \delta_{r,m}\,  \delta_{s,n} \, q^{\frac{1}{2}p(l+m-n)} \,
                         q^{\frac{1}{2}l(m+n)} \,\qfac{m}\, \qfac{n}
\end{equation}
for all $p,l \in \ZZ$ and $m,n,r,s \in \NN$.
For the $q$-factorials, see appendix \ref{app:qcalc}\@.
\end{defn}


\begin{prop} \label{prop:dual_pair}
The pairing defined in (\ref{eq:fullpairing}) is non-degenerate,
it makes \UqAq\ into a duality of Hopf $^*$-algebras and
specializes to (\ref{eq:pairing:generators}) on generators.
\end{prop}
\begin{proof}
See \cite{Koelink:thesis,Koelink:QE2,Fons:DPHA}\@.
\end{proof}


\subsection{Introducing shift \& multiplication operators}
\label{par:introducing_shift_and_multiplication_operators}


Throughout the remainder of this text we set \mbox{$\theta = -\frac{1}{2} \ln q$}\@.
Since $0<q<1$ we have $\theta>0$, and $e^{-\theta} = q^\frac{1}{2}$.
Let \FZ\ be the $^*$-algebra of all complex functions on
$\Zt = \{ k \theta \mid k\in \ZZ \} \subseteq \RR$,
with pointwise operations.

\begin{defn}
We define two linear mappings
$\Gamma, \Phi : \FZ \rarr \FZ$ by
$$  (\Gamma f)(x) = f(x-\theta)     \andspace{4em}     (\Phi f)(x) = e^x f(x) $$
for $f\in \FZ$ and $x\in \Zt$.
\end{defn}

Obviously $\Gamma$ and $\Phi$ are bijective, which will allow us occasionally
to write powers $\Gamma^k$ or $\Phi^k$ for any $k\in\ZZ$.
Actually $\Gamma$ is a $^*$-isomorphism, and the commutation rule
$\Gamma \Phi = q^\frac{1}{2} \, \Phi \Gamma$ is easily verified.
\vspace{2ex}

Now recall the $^*$-algebra \HC\ of entire functions (\S \ref{sec:conventions}).
Again we introduce two linear operators, $\Omega$ and $\Psi$,
which play a role similar to $\Gamma$ and $\Phi$ above:

\begin{defn}
Let $\Omega,\Psi : \HC \rarr \HC$ be defined by
$$(\Omega g)(z) = g(qz)       \andspace{4em}      (\Psi g)(z) = z\, g(z) $$
For $g\in\HC$ and $z\in\CC$.
\end{defn}

Here $\Omega$ is a bijection, but $\Psi$ is not, so one should be aware not
to take {\em negative powers\/} of $\Psi$.
Since $q$ is real, $\Omega$ is a $^*$-isomorphism.
Notice that $\Omega\Psi = q\, \Psi\Omega$ involves a factor $q$,
whereas the $\Gamma,\Phi$ commutation rule involved $q^\frac{1}{2}$.
