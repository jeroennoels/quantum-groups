
\section{Duality}



\begin{abs_chp}
Whenever \FourierBABA\ is a Fourier context, we have several \lq Hopf type\rq\ dualities,
as represented in the diagram below:
{\normalsize
$$ \begin{array}{ccccl}
A_0 & \subseteq & A & \stackrel{\rm dual}{\vertS \longleftrightarrow} & {\frakB} \\
 {\scriptstyle ?} \updownarrow\hspace{1em} &  &  {\scriptstyle ?}\updownarrow\;\:
           & & \,\updownarrow {\scriptstyle \rm dual}      \\
B_0 & \subseteq & B &  \stackrel{\rm dual}{\vertS \longleftrightarrow} & {\frakA}
\end{array}$$}

Thus the question arises whether this situation induces a natural
duality between $A$ \mbox{and $B$}, or more likely, between $A_0$ and $B_0$.
This will be investigated below.
\end{abs_chp}




\begin{lemma_sec}  \label{lemma:induced_pairing}
Let \FourierBABA\ be a Fourier context and assume there exist Fourier transforms\/
$\FLL$, $\FRL \,\ldots$ from\/ $A_0$ to\/ $B_0$. Then
$$                           \phiB\!\left(\FLL(a)\, b \vertM\right)
   \:\stackrel{(i)}{=}\:     \phiB\!\left(b \,\FRL(a)\vertM\right)
   \:\stackrel{(ii)}{=}\:    \psiB\!\left(\FLR(a) \, b\vertM\right)
   \:\stackrel{(iii)}{=}\:   \psiB\!\left(b \, \FRR(a)\vertM\right)
   \:\stackrel{\rm (def)}{=}\: \pair{a}{b}
$$
for all\/ $a \in A_0$ and\/ $b \in B_0$.
This defines a vector space duality between\/ $A_0$ and\/ $B_0$,
which we shall once again denote by \pairing\@.
If the above Fourier transforms are bijections, then this pairing\/
$\pair{A_0}{B_0}$ is non-degenerate.
\end{lemma_sec}
\begin{proof}
(i) follows immediately from the {\sc kms} property of $\phiB$
and the formula $\sigma_{\phiB} \FLL = \FRL$
shown in proposition \ref{prop:relations_between_Ftransforms}\@.
(iii) is analogous, whereas (ii) relies on the modular elements:
indeed, using the {\sc kms} property of $\phiB$ we obtain
$$ \zetaB \, \phiB\!\left(b \, \FRL(a)\vertM\right)
           \stackrel{(\ref{eq:Fourier:modular})}{=}
   \phiB\!\left(b \, \deltaB\, \sigma_{\!\phiB}(\FLR(a))\vertM\right)
          \:=\:
   \phiB\!\left(\FLR(a)\, b \, \deltaB \vertM\right)
          \:=\:
   \zetaB \, \psiB\!\left(\FLR(a)\, b \vertM\right) $$
for all $a \in A_0$ and $b \in B_0$.
Now let's assume $\FRL$ to be bijective and prove the non-degeneracy statement:
first, let $a \in A_0$ such that $\pair{a}{b}=0$ for all $b \in B_0$.
Then in particular $\pair{a}{\FRL(a)^*}=0$ and hence
$\phiB\! \left(\FRL(a)^*\,\FRL(a)\vertM\right) =0$.
Since $\phiB$ is faithful, it follows that $\FRL(a)=0$ and hence $a=0$.
Next, let $b \in B_0$  such that $\pair{a}{b}=0$ for all $a \in A_0$.
This means that $\phiB(b B_0)=\{0\}$ and in particular $\phiB(b b^*)=0$, hence $b=0$.
\end{proof}
\vspace{2ex}

Recall that $A_0$ and $B_0$ are bimodules over both the algebras
$\frakA$ and $\frakB$, either under multiplication or under the actions
induced by the dualities $\pair{\frakA}{B}$ and $\pair{A}{\frakB}$.
The following lemma tells us that the \lq induced\rq\ pairing $\pair{A_0}{B_0}$
is compatible with these bimodule structures. Also antipodes and $^*$-structures
satisfy the usual relations:

\begin{lemma_sec}
The pairing\/ $\pair{A_0}{B_0}$ defined in lemma \ref{lemma:induced_pairing}\ enjoys
\begin{eqnarray}
   \pair{\alpha a}{b} \,=\, \pair{a}{b \ract \alpha} & \hspace{3em} &
   \pair{a \alpha}{b} \,=\, \pair{a}{\alpha \lact b}
\label{eq:induced_pairing:actions:1} \\ \vertL
   \pair{a}{\beta b}  \,=\, \pair{a \ract \beta}{b}  & &
   \pair{a}{b \beta}  \,=\, \pair{\beta \lact a}{b}
\label{eq:induced_pairing:actions:2}
\end{eqnarray}
for any\/ $a \in A_0$, $b \in B_0$, $\alpha \in {\frakA}$ and\/ $\beta\in {\frakB}$.
Furthermore we have
$$   \pair{S(a)}{b} \:=\: \pair{a}{S(b)}
           \itandspace{4em}
     \pair{a}{b^*} \:=\: \overline{\pair{S(a)^*}{b}}.  $$
\end{lemma_sec}
\begin{proof}
Combining lemma \ref{lemma:Fourier:extra_module prop}\
with lemma \ref{lemma:properties_of_Fourier_contexts}.i we obtain
\begin{eqnarray*}
\pair{\alpha a}{b}
&=&
\phiB\!\left(\FLL(\alpha a)\, b \vertM\right) \\
&=&
\phiB\!\left((\FLL S^{-1})\left(S(a) S(\alpha)\vertM\right) b \right) \\
&=&
\phiB\!\left(\left[(\FLL S^{-1})(S(a)) \ract S(\alpha) \vertM\right] b \right)   \\
&=&
\phiB\!\left(\left[\FLL(a) \ract S(\alpha) \vertM\right] b\right)  \\
&=&
\phiB\!\left(\FLL(a)\, (b \ract \alpha) \vertM\right) \\
&=&
\pair{a}{b \ract \alpha}
\end{eqnarray*}
The second formula in (\ref{eq:induced_pairing:actions:1}) is similar,
whereas the ones in (\ref{eq:induced_pairing:actions:2}) are trivial, e.g.
$$ \pair{a}{\beta b}
   \;=\; \phiB\!\left(\FLL(a)\, \beta b \vertM\right)
   \;=\; \phiB\!\left(\FLL(a \ract \beta) \, b \vertM\right)
   \;=\; \pair{a \ract \beta}{b}.   $$
Using $S \FRL S = \zetaB \FLR$ we obtain
$$ \pair{S(a)}{b}
    \:=\: \phiB\!\left(b \,\FRL S(a) \vertM\right)
    \:=\: \zetaB \, \phiB\!\left(b \,S^{-1} \FLR(a) \vertM\right)
    \:=\: \psiB\!\left(\FLR(a) \, S(b) \vertM\right)
    \:=\: \pair{a}{S(b)} $$
The last formula follows easily from (\ref{eq:Fourier:stars}).
\end{proof}



\section{Inversion formulas}

Until now we only considered Fourier transforms from $A_0$ to $B_0$.
However our theory is clearly fully symmetric, hence one could as well
consider Fourier transformation from $B_0$ to $A_0$.
To avoid any confusion about the left-right nomenclature,
we give an example below, although it is completely analogous to
definition \ref{def:Fourier_transform}.

\begin{defn_sec} \label{def:inverse_Fourier_transform}
Let \FourierBABA\ be any Fourier \mbox{context}\@.
A linear map $\GLR:B_0 \rarr A_0$ is called an {\scriptsize LR}
{\em Fourier transform\/} if
\begin{enumerate}
\item
$\GLR$ is a {\em right\/} $\frakA$-module morphism;
explicitly, for any $\alpha \in {\frakA}$ and $b\in B_0$ we have
$\GLR(b \ract \alpha) \,=\, \GLR(b)\, \alpha$
\item
$\psiA\! \left(\GLR(b)\vertM\right) \,=\, \pair{1_{\frakA}}{b}$ for any $b\in B_0$.
\end{enumerate}
So the subscript {\scriptsize L} in $\GLR$ {\em anti\/}-corresponds to the fact
we are dealing with a {\em right\/} module morphism in (i),
whereas the {\scriptsize R} in $\GLR$ refers to the {\em right\/}
invariant functional $\psiA$ in (ii).
\end{defn_sec}


\begin{prop_sec} \label{prop:inverse_Fourier_transform}
Let \FourierBABA\ be any Fourier context and
let\/ $\FLL: A_0 \rarr B_0$ be a bijective {\scriptsize LL} Fourier transform.
Assume there exists a scalar\/ $\lambda\in \CC$ such that\/
$\pair{1_{\frakA}}{\FLL(a)} = \lambda \,\phiA(a)$ for any\/ $a\in A_0$.
Then
$$ \GLR \:=\: \lambda \zetaA^{-1} \, S \FLL^{-1} $$
is an {\scriptsize LR} Fourier transform from\/ $B_0$ to $A_0$.
\end{prop_sec}
\begin{proof}
Condition (i) of definition \ref{def:inverse_Fourier_transform}\ is
fulfilled because of lemma \ref{lemma:Fourier:extra_module prop},
whereas (ii) follows from our assumption.
\end{proof}


\begin{remark_sec} \rm
The assumption that $\pair{1_{\frakA}}{\FLL(\,\cdot \,)}$
is a scalar multiple of $\phiA$ (restricted to $A_0$)
is a very natural one---in fact it would hold automatically if we
would have shown an appropriate result on uniqueness of Haar functionals
(an important though difficult matter we didn't elaborate on).
Indeed the functional $\pair{1_{\frakA}}{\FLL(\,\cdot \,)}$ on $A_0$
behaves just like a left invariant functional:
$$  \pairM{1_{\frakA}}{\FLL(a \ract \beta)}
      \:=\: \pairM{1_{\frakA}}{\FLL(a)\, \beta}
      \:=\: \eps(\beta) \: \pairM{1_{\frakA}}{\FLL(a)}  $$
for all $a \in A_0$ and $\beta\in {\frakB}$.
Here we used that $\FLL$ is a right $\frakB$-module morphism.
\end{remark_sec}
