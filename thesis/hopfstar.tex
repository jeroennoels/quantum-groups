\section{\Hss s}


\begin{prop_sec} \label{prop:Hopf_star_system}
Let\/ \pairAB\ be a \dpa\ (cf.\ example \ref{exC:introduction}) and moreover assume
$A$ and\/ $B$ to be endowed with $^*$-operations making them into $^*$-algebras.
Then the following are equivalent:
\begin{enumerate}
\item
\Aa\ and\/ \BB\ are involutive {\rm(\S \ref{par:involutive contexts})} \contexts\@.
The pairing\/ $P: A \tens B \rarr \kk$ is \stricta\ continuous w.r.t.\ \BBAA\
and the corresponding actor is unitary in the sense of definition \ref{def:unitary_actor},
and multiplicative as in definition \ref{def:muliplicative_actor}\@.
\item
The pair \pairAB\ is a regular Hopf system and $\pair{a}{b^*} = \overline{\pair{\SA(a)^*}{b}}$
for any\/ $a \in A$ and\/ $b \in B$.
\end{enumerate}
\end{prop_sec}

\begin{proof}
First assume (i).
Then \pairAB\ is clearly a Hopf system, and $P^{-1}=P^*$.
Using lemma \ref{lem:weakly_unital_involutive_context}\ it follows that
for all $a \in A$ and $b \in B$
\begin{equation}\label{eq:prop:Hopfstar:Pstar}
\pairM{P^*}{a \tens b\conj}
     \:=\:  \overline{\pairM{P}{(a \tens b\conj)\conj}}
     \:=\:  \overline{\pair{a\conj}{b}}
     \:=\:  \pair{a}{b^*}.
\end{equation}
On the other hand, recall that $\SA(A)\subseteq \Env(\Aa)$,
so again using lemma \ref{lem:weakly_unital_involutive_context}, we obtain
\begin{equation}\label{eq:prop:Hopfstar:Pinv}
\pairM{P^{-1}}{a \tens b\conj}
     \:=\:  \pairM{\SA(a)}{b\conj}
     \:=\:  \overline{\pairM{\SA(a)^*}{b}}
\end{equation}
and hence $\pair{a}{b^*} = \overline{\pair{\SA(a)^*}{b}}$.
This formula can be rewritten as $\SA(a)=(a\conj)^*$.
It follows that $\SA$ is a bijection from $A$ onto $A$,
and analogously $\SB$ is a bijection from $B$ onto $B$.
Hence \pairAB\ is regular, which proves (ii).

Now assume (ii).
First observe that, under the circumstances, the formula
$\pair{a}{b^*} = \overline{\pair{\SA(a)^*}{b}}$
automatically implies its $\SB$-analogue:
indeed, replacing $a$ with $\SA^{-1}(a^*)$ and $b$ with $\SB(b)^*$,
we obtain for all $a \in A$ and $b \in B$
$$  \pair{a^*}{b}
      \:=\:  \pairM{\SA^{-1}(a^*)}{\SB(b)}
      \:=\:  \overline{\pairM{a}{\SB(b)^*}}. $$
It follows that $a\conj = \SA(a)^*$ and $b\conj = \SB(b)^*$.
Hence $A$ and $B$ are $\conj$-invariant, or in other words, \Aa\ and \BB\ are
involutive \contexts\@.
Equations (\ref{eq:prop:Hopfstar:Pstar}) and (\ref{eq:prop:Hopfstar:Pinv}) still hold
and imply that $P^{-1} = P^*$, so $P$ is unitary. This proves (i).
\end{proof}



\begin{defn_sec} \label{def:Hopf_star_system}
A pair \pairAB\ satisfying the conditions in proposition \ref{prop:Hopf_star_system}\
is said to be a {\em \Hss}.
\end{defn_sec}



\begin{prop_sec}
If\/ \pairAB\ is a \Hss, then the comultiplications\/ $\DeltaA$ and\/ $\DeltaB$,
and the counits\/ $\epsA$ and\/ $\epsB$, are all $^*$-homomorphisms.
Furthermore $\SA(\SA(a)^*)^* = a^{\circ\circ} = a$ and\/
$\SB(\SB(b)^*)^* = b^{\circ\circ} = b$ for any\/ $a \in A$ and\/ $b \in B$.
\end{prop_sec}
\begin{proof}
Take any $a \in A$ and $b,d \in B$.
Recall that $b\conj = \SB(b)^*$ and $d\conj = \SB(d)^*$.
Using lemma \ref{lem:weakly_unital_involutive_context}\ and the anti-multiplicativity
of $\SB$, we obtain
\begin{eqnarray*}
\pairM{\DeltaA(a)^*}{b \tens d}
&=&
\overline{\pairM{\DeltaA(a)}{b\conj \tens d\conj}}
\\&=&
\overline{\pairM{\DeltaA(a)}{\SB(b)^* \tens  \SB(d)^*}}
\\&=&
\overline{\pairM{a}{\SB(b)^* \SB(d)^*}}
\\&=&
\overline{\pairM{a}{\SB(bd)^*}}
\\&=&
\pairM{a^*}{bd}
\\&=&
\pairM{\DeltaA(a^*)}{b \tens d}
\end{eqnarray*}
hence $\DeltaA(a)^* = \DeltaA(a^*)$.
Combining corollary \ref{cor:antipode_and_counit}\ with
lemma \ref{lem:weakly_unital_involutive_context}, it follows that
$\epsA(a^*) = \overline{\epsA(a)}$.
Similarly for $\DeltaB$ and $\epsB$.
\end{proof}
