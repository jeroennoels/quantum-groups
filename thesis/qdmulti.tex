
\subsection{Multiplicativity}

Again adopt setting \ref{setting:qdouble}\ and recall \S \ref{par:multiplicativity}\@.
By now we know that $\pair{X}{Y}$ is an \idpa, but of course we want it to be a Hopf system.
The shortest way towards multiplicativity is to show that the comultiplication
$$ \DeltaY: \, Y = B\tens A \:\rarr\: \Env(\YY \tens \YY)
                 = \Env(\BBAA \tens \BBAA) $$
is an algebra homomorphism and then invoke proposition \ref{prop:multiplicative:comultiplications}\@.
Since $\DeltaY$ is dual to
$\mult{X} = (\mult{A} \! \tens  \mult{B}\op)(\id \tens T \tens \id)$, we have
$\DeltaY  = \Lambda \Phi (\DeltaB \tens \flip\DeltaA)$
where
$$ \DeltaA : A \rarr \Env(\AAAA)     \andspace{3em}
   \DeltaB : B \rarr \Env(\BBBB) $$
are the comultiplications (cf.\ proposition \ref{prop:comultiplications_in_Env}) on $A$ and $B$,
$$ \Phi :\, \Env(\BBBB) \,\tens\, \Env(\AAAA)  \,\rarr\,  \Env(\BBBB \tens \AAAA) $$
is the embedding established in proposition \ref{prop:tensor_embedding_Env}, and
$$ \Lambda :\, \Pre(\BBBB \tens \AAAA)  \,\rarr\,  \Pre(\BBAA \tens \BBAA) $$
is given by
$\Lambda = \flipsub{23}(P_{23} \,\cdot\, P^{-1}_{23})$.
Of course the flip map involved here should be extended properly
to the pre-actor algebra, but this does not present a problem.
On the other hand we do have to be careful with $\Lambda$, since we need
\begin{equation}\label{eq:extended_dual_of_twist}
  \pairM{(\id \tens T \tens \id)(u)}{v}  \:=\:  \pairM{u}{\Lambda(v)}
\end{equation}
to hold for any $u \in A \tens B \tens A \tens B$ and $v \in \Env(\BBBB \tens \AAAA)$.
For the right hand side of (\ref{eq:extended_dual_of_twist}) to make
sense, however, $\Lambda$ should map $\Env(\BBBB \tens \AAAA)$ into $\Act(\BBAA \tens \BBAA)$.
Fortunately $P_{23} \in \Act(\BBBB \tens \AAAA)$ enjoys (iv) of
\mbox{lemma \ref{lem:wu:commutation:invertibility}},
and statement (iii) of this lemma will ensure (\ref{eq:extended_dual_of_twist}).

Now $\DeltaA$, $\DeltaB$, $\Phi$ and $\Lambda$ are all algebra homomorphisms,
hence so is $\DeltaY$.
\vspace{2ex}

The above argument, however, is not very loyal to our duality approach;
indeed it would be nice if we could show multiplicativity
(in the sense of \mbox{definition \ref{def:muliplicative_actor}}) {\em directly\/} from
the formulas (\ref{eq:QD:alpha}) and (\ref{eq:QD:beta}) for $\lQ$ and $\rQ$
(cf.\ notation \ref{not:QD:pairing:Q})
i.e.\ without the intervention of comultiplications. In this respect we have:

\begin{prop}  \label{prop:qdouble:multiplicative}
Adopt setting \ref{setting:qdouble}\ and notation \ref{not:QD:pairing:Q}\@. Then
$$ \pairflipM{\rQ(x_1 \tens y_1)}{\lQ(x_2 \tens y_2)}
        \:=\:  \pairM{x_1 \!\star x_2}{y_1 y_2} $$
for all\/ $x_1,x_2 \in X$ and\/ $y_1,y_2 \in Y$.
\end{prop}
\begin{proof}
Take any $x_1,x_2 \in X$ and\/ $y_1,y_2 \in Y$, and denote
$$ \xi \:=\: \pairflipM{\rQ(x_1 \tens y_1)}{\lQ(x_2 \tens y_2)}. $$
Before we start, let us agree that the leg-numbering notation always refers to the
lowest level, i.e.\ the legs are always in $A$ or in $B$ (rather than in $X$ or $Y$).
From time to time we shall use the fact that $\piP : Y \rarr Y$ is multiplicative,
which will be indicated with an asterisk ($*$).
Observe that for any $x\in X$ and $y\in Y$
\begin{eqnarray*}
\lefteqn{\pairflipM{\rQ(x_1 \tens y_1)}{(\idX \tens \piP^{-1})(x \tens y)}}
\\&=&
\pairM{\rQ(x_1 \tens y_1)}{\piP^{-1}(y) \tens x}
\\&=&
\pairM{\piP^{-1}(y) \lact x_1}{x \ltact y_1}
\\&=&
\pairM{x_1}{\mult{Y} \! \left((x \ltact y_1) \tens \piP^{-1}(y) \vertM\right)}
\\&\stackrel{(\ref{eq:QD:left_twisted_action})}{=}&
\pairM{x_1}{\mult{Y} (\epsX \tens \idY \tens \idY)
             (\lQ \tens \idY)\left(x \tens y_1 \tens \piP^{-1}(y) \vertM\right)}
\\&=&
\pairM{x_1}{(\epsX \tens \mult{Y}\!)
                (\lQ \tens \piP^{-1})(x \tens y_1 \tens y)}
\\&\stackrel{(\ref{eq:QD:alpha})}{=}&
\left\langle x_1,\: (\epsX \tens \mult{Y}\!)
      (\idX \tens \piP^{-1} \tens \piP^{-1})
\right. \\&& \hspace{5em}\left.
   (\lp)_{13} (\idX \tens \piP \tens \idY)
      (\lpop)_{42} (x \tens y_1 \tens y)  \right \rangle
\\&\stackrel{(*)}{=}&
\pairM{x_1}{\piP^{-1} (\epsX \tens  \mult{Y}\!) (\lp)_{13}
       (\idX \tens \piP \tens \idY)  (\lpop)_{42} (x \tens y_1 \tens y)}
\\&=&
\pairM{R^{-1}(x_1)}{(\epsX \tens  \mult{Y}\!) (\lp)_{13}
       (\piP)_{34}  (\lpop)_{42} \flipsub{35} \flipsub{46} (x \tens y \tens y_1)}.
\end{eqnarray*}
Now we replace $x\tens y$ with $(\lp)_{13}  (\idX \tens \piP)  (\lpop)_{42}(x_2 \tens y_2)$
and obtain:
\begin{eqnarray*}
\xi &\!\!\!=\!\!\!&\!
      \left\langle R^{-1}(x_1),
      (\epsX \tens  \mult{Y}\!) (\lp)_{13}  (\piP)_{34}  (\lpop)_{42}  (\lp)_{15}
      (\piP)_{56}  (\lpop)_{62} (x_2 \tens y_1 \tens y_2)\right \rangle
\\&\!\!\!=\!\!\!&\!
      \left\langle R^{-1}(x_1),
      (\epsX \tens  \mult{Y}\!) (\lp)_{13} (\lp)_{15}
      (\piP)_{34}  (\piP)_{56}  (\lpop)_{42}
      (\lpop)_{62} (x_2 \tens y_1 \tens y_2)\right \rangle
\end{eqnarray*}
Recall that $\lp\flip$ is an\/ $(A\op,B)$-twisting (proposition \ref{prop:twistings:lprp}).
Using (\ref{eq:lemmma:twisting1})
we get\footnote{The presence of $\mult{A}$ in (\ref{eq:QD:multipl:direct:twisting:lp})
is irrelevant and does not interfere with the use of (\ref{eq:lemmma:twisting1}).}
\begin{equation}\label{eq:QD:multipl:direct:twisting:lp}
    (\idX \tens  \mult{B \tens A}) (\lp)_{13}  (\lp)_{15}
          \:=\:  (\lp)_{13} (\idX \tens  \mult{B \tens A})
\end{equation}
and our computation proceeds as follows:
\begin{eqnarray*}
\xi &=& \left\langle R^{-1}(x_1),\:
      (\epsX \tens \idY) (\lp)_{13} (\idX \tens  \mult{Y}\!) \right.
\\&& \hspace{7em}\left.
       (\piP)_{34}   (\piP)_{56}  (\lpop)_{42}
      (\lpop)_{62} (x_2 \tens y_1 \tens y_2)\right \rangle
\\&\stackrel{(*)}{=}&
\pairM{R^{-1}(x_1)}{(\muL\tens \idA)(\idA \tens \epsB \tens  \piP \mult{Y}\!)
          (\lpop)_{42}  (\lpop)_{62} (x_2 \tens y_1 \tens y_2)}.
\end{eqnarray*}
We used the fact that $(\epsA \tens \id)\lp = \muL : A \tens B \rarr B$
is the left action $\lact$ of $A$ on $B$.
For mere notational convenience we shall assume \mbox{$x_1,\, x_2,\, y_1$}\ and $y_2$
to be simple tensors, say $x_i = a_i\tens b_i$ and $y_i = d_i\tens c_i\,$ ($i=1,2$).
Furthermore we write $\lpop(c_2 \tens b_2) = \sum_k \, p_{k} \tens q_{k}$
and proceed:
\begin{eqnarray*}
\xi &=& \textstyle \sum_k \:
\left\langle R^{-1}(x_1),\:
(\muL\tens \idA)(\idA \tens \epsB \tens  \piP \mult{Y}\!) \right.
\\&& \hspace{8em}\left.
          (\lpop)_{42} (a_2 \tens q_k \tens d_1 \tens c_1 \tens d_2 \tens p_k)\right  \rangle.
\end{eqnarray*}
An easy computation shows that $(\id \tens \epsB)\lpop(c_1 \tens q_k) = c_1 \ract q_k$,
hence
\begin{eqnarray*}
\xi &=& \textstyle \sum_k \,
    \pairM{R^{-1}(a_1 \tens b_1)}{(\muL\tens \idA)
             \left[ a_2 \tens \piP \!
                     \left(d_1 d_2 \tens (c_1 \ract q_k)p_k\vertM\right)
                  \right]}
\\ &=& \textstyle \sum_k \,
    \pairM{(\mult{A}\op \tens \idB)R^{-1}_{23}(a_2 \tens a_1 \tens b_1)}{
               \piP \! \left(d_1 d_2 \tens (c_1 \ract q_k)p_k\vertM\right)}
\\&\stackrel{(\ref{eq:lemmma:twisting2})}{=}& \textstyle \sum_k \,
  \pairM{R^{-1}(\mult{A}\op \tens \idB)R_{13}(a_2 \tens a_1 \tens b_1)}{
               \piP \! \left(d_1 d_2 \tens (c_1 \ract q_k)p_k\vertM\right)}
\\ &=& \textstyle \sum_k \,
  \pairM{(\mult{A}\op \tens \idB)R_{13}(a_2 \tens a_1 \tens b_1)}{
               d_1 d_2 \tens (c_1 \ract q_k)p_k}.
\end{eqnarray*}
Now observe that for any $b\in B$ we have
\begin{eqnarray*}
\textstyle \sum_k \,\pairM{(c_1 \ract q_k)p_k}{b}
&=&
\textstyle \sum_k \,\pairM{c_1 \ract q_k}{p_k \lact b}
\\&=&
\textstyle \sum_k \,\pairM{\rpop(c_1 \tens b)}{q_{k} \tens p_{k}}
\\&=&
\pairflipM{\rpop(c_1 \tens b)}{\lpop(c_2 \tens b_2)}
\\&=&
\pairM{c_1 c_2}{b_2 b}
\end{eqnarray*}
and hence $\sum_k \,(c_1 \ract q_k)p_k = c_1 c_2 \ract b_2$. We proceed:
\begin{eqnarray*}
\xi &=&
\pairM{(\mult{A}\op \tens \idB)R_{13}(a_2 \tens a_1 \tens b_1)}{
               d_1 d_2 \tens c_1 c_2 \ract b_2}
\\&=&
\pairM{(a_1 \tens b_2) R(a_2 \tens b_1)}{d_1 d_2 \tens c_1 c_2}
\\&=&
\pairM{x_1 \! \star x_2}{y_1 y_2}.
\end{eqnarray*}
This completes the proof.
\end{proof}
