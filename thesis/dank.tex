
Vooraleer van start te gaan, wil ik eerst enkele mensen bedanken
die een belangrijke rol gespeeld hebben bij het tot stand komen van dit werkje.
In de eerste plaats natuurlijk mijn promotor Fons van Daele,
die me zes jaar geleden de kans gaf het wiskundig onderzoek in
te stappen---het begin van zes fijne jaren in de afdeling analyse---en wiens
suggesties de kiem vormden waaruit later deze thesis zou groeien.
En dan is er uiteraard die andere \textsl{boy-scout}, Johan Quaegebeur,
altijd bereid onze alma-uurtjes op te luisteren met oeverloze discussies
over verheven en soms minder verheven onderwerpen;
onvermoeibaar kruisvaarder voor het huwelijk ook---ach dat zal de
mid-life crisis wel zijn. Het zou mij verder geenszins verwonderen mocht hij
op dit eigenste ogenblik het meest-sluwe-aller-plannen bekokstoven om mijn
verdediging een ludieke toets te geven.

Met Johan \lq de kust\rq\ Kustermans in de afdeling stonden we kniehoog in de
subtiele, fijnzinnige en genuanceerde opvattingen.
Met hem kon ik dan ook converseren \textsl{op nivo\/}:
The Simpsons, Voyager, Koken met Kust, \lq\lq Welk OS gaan we vandaag weer
installeren?\rq\rq\ldots\
maar ook over de gewone dingen des levens zoals kwantum groepen en niet-commutatieve
meetkunde.
Mijn buro-genoot, de Pieter, wist mij steeds te vertellen wie \textsl{de koers\/} gewonnen had.
Ann heb ik zeer recent nog omschreven als een \lq wolvin in schaapsvacht\rq\
maar dat is al bij al toch wat overdreven, nietwaar.
Ook Stefaan, partner in crime bij \textsl{infinit}, stond altijd klaar om een of
andere grap uit te halen.
Iets langer geleden, maar nog lang niet vergeten: Roby, Rupert en Jan.
Mag hier ook niet ontbreken: Noeki.
\vspace{1ex}

Verder dank ik Erik Koelink, met wie ik tijdens de eindsprint nog heel wat
e-mails uitgewisseld heb, voor zijn tips en suggesties.
\vspace{1ex}

Natuurlijk zijn ook mijn ouders, broer, bomma en de rest van mijn familie niet
weg te denken uit dit dankwoord.
Hetzelfde geldt voor Greet en haar familie, waarbij ik me altijd heel welkom heb mogen
voelen. Tenslotte wil ik ook mijn vriendenkring bedanken voor de vele aangename uurtjes.
