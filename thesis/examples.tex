\section{Examples}
\label{par:Examples}


\begin{abs_chp}
We show how Hopf algebras \cite{Schmudgen,Fons:DPHA}\
and multiplier Hopf algebras \cite{Fons:MHA}\ fit into the Hopf system picture.
Also the duality theory for multiplier Hopf algebras with integrals as developed
by A.\ Van Daele \cite{Fons:AFGD,Fons:AFGD:proc,Fons:pnas}\ fits into this framework.
Then we introduce a special subclass of Hopf systems, the so-called {\em algebraic\/} ones,
having an \lq algebraic\rq\ comultiplication\footnote{i.e.\ taking values in the
algebraic tensor product, rather than some envelope of it.
See definition \ref{def:algebraic_context}\@.}\
on at least one side; these are the kind of Hopf systems that enter e.g.\
the algebraic description of the quantum $E(2)$ group covered in chapter
\ref{chapter:Fourier_context_for_quantumE2}\@.
For the moment we stick to a fairly trivial but very familiar (hence instructive) example
derived from the group $(\RR,+)$.
We conclude with an example of a {\em non-regular\/} Hopf system.
\end{abs_chp}




\subsection{Multiplier Hopf algebras}

\begin{lemma}  \label{lem:mha:Hopf_system}
Let\/ \pairAB\ be a \dpa\@. Assume\/ $B$ to be unital as an $A$-bimodule.
Let $A$ be endowed with a comultiplication $\Delta : A \rarr M(A\tens A)$
making\/ $(A,\Delta)$ into a \mha\@. If for any\/ $a,c \in A$ and\/ $b,d \in B$
\begin{eqnarray}
  \pairM{T_1(a \tens c)}{b\tens d}  &=&   \pairM{a}{b(c \lact d)}
       \label{eq:ex:mha:T1:mB}  \\
  \pairM{T_2(a \tens c)}{b\tens d}  &=&   \pairM{c}{(b \ract a)d}
       \label{eq:ex:mha:T2:mB}
\end{eqnarray}
then \pairAB\ is a Hopf system.
If\/ $(A,\Delta)$ is regular as a \mha\ and if\/ $\SB(B)=B$,
then\/ \pairAB\ is regular in the sense of definition \ref{def:regular:hs}\@.
\end{lemma}


{\small
\begin{remarks*} \label{rem:mha:Hopf_system}
\item
The comultiplication, counit and antipode associated with $A$ in the sense of
\cite{Fons:MHA}\ are consistent\footnote{i.e.\ if we take into account that
$M(A) \subseteq \Env(\Aa)$ and $M(A \tens A) \subseteq \Env(\AAAA)$.}\
with the objects $\DeltaA$, $\epsA$ and $\SA$ derived from the Hopf system \pairAB\@.
For the comultiplication and the counit, this assertion is easily shown;
for the antipode it follows from the proof below.
\item
Observe that conditions (\ref{eq:ex:mha:T1:mB}) and (\ref{eq:ex:mha:T2:mB})
are actually equivalent.
They express that the comultiplication $\Delta$ is dual to the product in $B$.
Here $T_1$ and $T_2$ are the bijections from $A\tens A$ onto $A\tens A$ as
defined in \cite{Fons:MHA}\@. Explicitly:
$$  T_1(a \tens c) \:=\: \Delta(a)(1 \tens c)   \hspace{6em}
    T_2(a \tens c) \:=\: (a \tens 1)\Delta(c).  $$
\end{remarks*} } % end small


\begin{proof}
Before we start, recall definition \ref{def:invertible_dpa}\ and
remark \ref{rem:warning:P_strict_cont}\@.
Take any $a,c,e \in A$ and $b,d \in B$, and write $T_1(a \tens e) = \sum_i\, p_i \tens r_i$.
Now (\ref{eq:ex:mha:T1:mB}) yields
\begin{eqnarray*}
\pairM{a \ract d}{(e\lact b) \ract c}
  &=&
\pairM{a}{d(e\lact b \ract c)}
\\&=&
\pairM{T_1(a \tens e)}{d \tens (b \ract c)}
\\&=&
\textstyle \sum_i \, \pair{p_i}{d} \pair{r_i}{b \ract c}
\\&=&
\textstyle \sum_i \, \pair{p_i}{d} \pair{c}{r_i \lact b}
\\&=&
\textstyle \sum_i \, \pairM{d\tens c}{p_i \tens (r_i \lact b)}
\end{eqnarray*}
Since $A\lact B = B$, it follows that given any $a \in A$ and $b \in B$,
there exists an $x\in A\tens B$ such that for all  $c \in A$ and $d \in B$
$$ \pairM{d \tens c}{x}  \:=\: \pairM{a \ract d}{b \ract c}. $$
Similarly we show the existence of an $y\in A\tens B$, only depending on $a$ and $b$,
such that
$$ \pairM{d \tens c}{y}  \:=\:  \pairM{d \lact a}{c \lact b}. $$
This means that the pairing $\pr : A \tens B \rarr \kk$ is indeed \stricta\ continuous
within the \context\ \BBAA, hence identifying with an actor $P \simeq (\lp\,,\rp)$ for
\mbox{\BBAA\@. Now let}\
$\muL:A\tens B \rarr B$ denote the left action of $A$ on $B$, and observe
we have actually shown that (cf.\ equations \ref{eq:def:lamP:rhoP})
$$  \lp(\id \tens \muL)  \:=\:  (\id \tens \muL)(T_1 \tens \id).  $$
Because $\muL$ and $T_1$ are surjective, it follows that $\lp(A\tens B) = A \tens B$.
The next step is to show that $\lp$ is injective.
Therefore, consider the linear mapping $\eta$ from $A\tens B$ into $(B\tens A)'$ defined by
\begin{equation}\label{eq:ex:mha:inverse_of_lp}
 \pairM{d \tens c}{\eta(a \tens b)} \:=\: \pairM{c S(a \ract d)}{b}
\end{equation}
where $S:A\rarr M(A)$ is the antipode associated to the {\sc mha} $(A,\Delta)$.
Again take any $a,c,e \in A$ and $b,d \in B$,
and write $T_1(a \tens e) = \sum_i\, p_i \tens r_i$ as before.
An easy computation shows that
$T_1 \! \left((a\ract d) \tens e \vertM\right) = \sum_i\, (p_i\ract d) \tens r_i$,
hence
$$  \textstyle \sum_i  S(p_i \ract d) \, r_i
           \;\stackrel{(*)}{=}\;
    \textstyle \sum_i \, (\eps \tens \id)\, T_1^{-1} \!\left((p_i\ract d) \tens r_i \vertM\right)
    \,=\:
    \eps(a\ract d) \, e
    \,=\:
    \pair{a}{d} \, e  $$
where $\eps$ is the counit on $A$.
In $(*)$ we used \cite[definition 4.1]{Fons:MHA}\@. Thus we obtain
\begin{eqnarray*}
\pairM{d \tens c}{\eta\lp \! \left(a \tens (e\lact b) \vertM\right)}
  &=&
\textstyle \sum_i \, \pairM{d \tens c}{\eta \! \left(p_i \tens (r_i \lact b) \vertM\right)}
\\&=&
\textstyle \sum_i \, \pairM{c S(p_i \ract d) r_i}{b}
\\&=&
\pair{a}{d} \pair{ce}{b}
\\&=&
\pairM{d \tens c}{a \tens (e\lact b)}
\end{eqnarray*}
and it follows that $\eta\lp$ is the identity map on $A\tens B$, hence $\lp$ is
a bijection from $A\tens B$ onto $A\tens B$.
Similarly also $\rp$ is bijective.
Now we would like to use lemma \ref{lem:char:actor:wu}\ to prove that
$(\lp^{-1},\rp^{-1})$ is again an actor for $\BBAA$. To do this, however,
we first need to establish weak unitality:
using (\ref{eq:ex:mha:T1:mB}) and (\ref{eq:ex:mha:T2:mB})
together with the surjectivity of $T_1$ and $T_2$, it is not so hard to see that
$A$ is unital as a $B$-bimodule.
From lemma \ref{lem:counit:existence}\ it follows that \BBAA\ is weakly unital,
hence so are \Aa\ and \BB\ (proposition \ref{prop:tensor:wu}).
Now $S(A)\subseteq M(A) \subseteq \Env(\Aa)$ can be paired with $B$,
and (\ref{eq:ex:mha:inverse_of_lp}) can be rewritten as
$$ \pairM{d \tens c}{\lp^{-1}(a \tens b)} \:=\: \pairM{c}{S(a \ract d) \lact b}. $$
Fix $b\in B$ for a while; using weak unitality we obtain for any $a\in A$ and $d\in B$
$$ \pairM{d}{(\id \tens \epsB)\lp^{-1}(a \tens b)}  \:=\: \pairM{f}{a \ract d} $$
where $f:A\rarr \kk\,$ is defined by $f(x)=\pair{S(x)}{b}$.
Analogously we obtain
$$ \pairM{d}{(\id \tens \epsB)\, \rp^{-1}(a \tens b)}  \:=\: \pairM{f}{d \lact a}. $$
It follows that $f \in A\adl$, hence $\theta(f) = \lamrho{f}$
is an actor for \BB\ given by
$$ \lam_f(a) \:=\: (\id \tens \epsB)\lp^{-1}(a \tens b)    \hspace{4em}
   \rho_f(a) \:=\: (\id \tens \epsB)\,\rp^{-1}(a \tens b).  $$
(cf.\ proposition \ref{prop:act_by_fuctionals}.iii).
Applying $\epsA$ and using (\ref{eq:pairing_with_act}) we obtain
\begin{equation}\label{eq:S_of_MHA_is_S_of_pairAB}
  (\epsA \tens \epsB) \lp^{-1}(a \tens b)
       \:=\: \pair{S(a)}{b}
       \:=\: (\epsA \tens \epsB)\, \rp^{-1}(a \tens b).
\end{equation}
Lemma \ref{lem:char:actor:wu}\ yields that $(\lp^{-1},\rp^{-1})$ is an
actor for $\BBAA$, which proves that \pair{A}{B}\ is an \idpa\@.
From proposition \ref{prop:multiplicative:comultiplications}\ it follows that
\pair{A}{B}\ is a Hopf system.
Comparing (\ref{eq:S_of_MHA_is_S_of_pairAB}) with (\ref{eq:def:antipodes})
and (\ref{eq:pairing_with_act}),
we conclude that $S$ coincides with the antipode $\SA$ derived from the
Hopf system \pairAB\ (as announced in remark \ref{rem:mha:Hopf_system}.i).

Recall from \cite{Fons:MHA}\ that the antipode on a {\em regular\/} \mha\ $A$
is a bijection from $A$ onto $A$. Assuming $\SB(B)=B$, it follows that
$\SB$ is a bijection from $B$ onto $B$. Hence \pair{A}{B}\ is regular.
\end{proof}




\begin{prop}
Let\/ $(A,\Delta)$ be any \mha\@. Let $A\reduced$ be the reduced dual of $A$
(defined as $A\reduced = A\lact A' \ract A$).
Then $\pair{A}{A\reduced}$ is a Hopf system.
If\/ $(A,\Delta)$ is regular, then $\pair{A}{A\reduced}$ is regular.
Finally, if\/ $(A,\Delta)$ is a multiplier Hopf $^*$-algebra, then
$\pair{A}{A\reduced}$ is a Hopf $^*$-system.
\end{prop}
\begin{proof}
Recall example \ref{exD:introduction}\@. Since the product in $A$ is non-degenerate,
it follows that $\Aa\equiv (A;A\reduced,\pairing)$ is an \context\@.
Furthermore $A\reduced$ is unital as an $A$-bimodule because $A^2=A$.
In \cite[\S 6]{Fons:MHA}\ it was shown that $A\reduced$ can be made into an algebra
in such a way that for all $a,c_i, e_i \in A$ and $\om_i \in A'$ ($i=1,2$)
\begin{equation}\label{eq:def:product:reduced_dual}
   \pairM{(c_1 \tens c_2)\Delta(a)(e_1 \tens e_2)}{\om_1 \tens \om_2}
      \:=\:  \pairM{a}{\nu_1 \nu_2}
\end{equation}
where $\nu_i = e_i \lact \om_i \ract c_i$ ($i=1,2$).
Now it is easy to see that also $(A\reduced;A,\pairing)$ is an \context,
and hence $\pair{A}{A\reduced}$ is a \dpa\@.
For this pair, the conditions (\ref{eq:ex:mha:T1:mB}) and (\ref{eq:ex:mha:T2:mB})
can be derived from (\ref{eq:def:product:reduced_dual}).
Lemma \ref{lem:mha:Hopf_system}\ then yields that $\pair{A}{A\reduced}$ is a Hopf system.
If\/ $(A,\Delta)$ is regular, then $S: A\rarr A$ is a bijection,
hence so is its transpose $S^{\tau}: A'\rarr A'$. Using the anti-multiplicativity of $S$,
it is not so hard to show that for all $c,e \in A$ and $\om \in A'$
$$  S^{\tau} \! \left( S(c) \lact \om \ract S(e)  \vertM\right)
    \:=\:  e \lact S^{\tau}(\om) \ract c. $$
It follows that $S^{\tau}(A\reduced) = A\reduced$,
hence $\pair{A}{A\reduced}$ is regular.

Recall from \cite{Fons:MHA}\ that a multiplier Hopf $^*$-algebra is automatically regular,
and furthermore $A\reduced$ can be made into a $^*$-algebra \cite[proposition 6.6]{Fons:MHA}\
in such a way that $\pair{A}{A\reduced}$ becomes a Hopf $^*$-system.
\end{proof}


\begin{cor} \label{cor:Hopf_algebra_and_full_dual}
If $A$ is a Hopf algebra, then $\pair{A}{A'}$ is a Hopf system.
When $A$ has invertible antipode, then $\pair{A}{A'}$ is regular.
If $A$ is a $^*$-Hopf algebra, then $\pair{A}{A'}$ is a \Hss\@.
\end{cor}



\begin{prop} \label{prop:AhatA:Hopf_system}
Let\/ $(A,\Delta)$ be a regular \mha\ with non-trivial invariant functionals.
Let\/ $(\hat{A},\hat{\Delta})$ be the dual object defined in \cite{Fons:AFGD:proc,Fons:AFGD}\@.
Then $\pair{A}{\hat{A}}$ is a regular Hopf system.
\end{prop}
\begin{proof}
Let $\varphi$ be a non-trivial left invariant functional on $A$, and consider
the Fourier transform $A \rarr \hat{A}: a \mapsto \hat{a} = \varphi(\cdot\,a)$.
The duality between $A$ and $\hat{A}$, given by
$\pair{a}{\hat{c}} = \varphi(ac)$, is non-degenerate because $\varphi$ is faithful.
Now it is easy to show that
e.g.\footnote{for the right action we need to use another Fourier transform, of
course.}\ $a \lact \hat{c} = \widehat{ac\,}\!$.
Since $A^2=A$ it follows that $\hat{A}$ is unital as an $A$-bimodule.
On the other hand we have for instance
$\hat{c} \lact a = (\id \tens \varphi)T_1(a\tens c)$.
It follows that $A$ is an $\hat{A}$-bimodule, and hence $\pair{A}{\hat{A}}$ is a \dpa\@.
The remaining conditions of lemma \ref{lem:mha:Hopf_system}\ are easily verified.
\end{proof}




\subsection{Algebraic Hopf systems}

\begin{defn}  \label{def:algebraic_Hopf_system}
Let $(A,\Delta,\eps,S)$ be any Hopf algebra, $B$ any algebra and
\mbox{$\pairing : A \times B \rarr \kk\,$}\ a non-degenerate vector space duality.
Embed $B$ in $A'$. If
\begin{enumerate}
  \item $B$ is invariant under canonical actions of $A$ on $A'$, and
  \item for all $a \in A$ and $b,d \in B$ we have
        $\pair{a}{bd} = \pair{\Delta(a)}{b \tens d}$
\end{enumerate}
then the pair $\pair{\underline{A}\,}{B}$ is said to be an {\em \ahs}\@. If
moreover
\begin{enumerate}
\item[iii.] $A$ is a Hopf $^*$-algebra and $B$ is a $^*$-algebra,
\item[iv.]  $B$ is $\conj$-invariant (\S \ref{par:involutive contexts}) within $A'$
\item[v.]   for any $a \in A$ and $b \in B$ we have $\pair{a}{b^*} = \overline{\pair{S(a)^*}{b}}$
\end{enumerate}
then $\pair{\underline{A}\,}{B}$ is called an {\em \ahss}.
\end{defn}

Here the {\em underlining\/} in $\pair{\underline{A}\,}{B}$ is merely a convenient
way to indicate which one of the two algebras involved is a genuine Hopf algebra;
indeed we wouldn't like this to depend on the {\em order\/} in which we write the pairing.
The terminology is justified by the following:

\begin{prop}
An \ahs\ $\pair{\underline{A}\,}{B}$ in the sense of the above definition is indeed
a Hopf system as defined in the beginning of chapter \ref{chapter:Hopf_systems}\@.
An \ahss\ is a Hopf $^*$-system in the sense of definition \ref{def:Hopf_star_system}.
\end{prop}

\begin{proof}
Observe that $A$ is invariant under canonical actions of $B$ on $B'$ because
$\Delta(A) \subseteq A\tens A$ (cf.\ proposition \ref{prop:induced_actor_context}).
Hence $\pair{A}{B}$ is a \dpa\@. Recall that $A$ has an identity;
lemma \ref{lem:mha:Hopf_system}\ yields that \pairAB\ is a Hopf system.

Assume that $\pair{\underline{A}\,}{B}$ moreover enjoys conditions (iii-iv-v)
of definition \ref{def:algebraic_Hopf_system}\@.
We claim that $\pair{A}{B}$ is regular as a Hopf system:
indeed $\SA$ is bijective because $A$ is a Hopf $^*$-algebra,
whereas (v) can be rewritten as $\SB(b\conj) = b^*$,
hence $\SB$ is a bijection from $B$ onto $B$. The result follows.
\end{proof}



\begin{remark}  \rm
An \ahs\ $\pair{\underline{A}\,}{B}$ might fail to be regular,
even when $A$ has invertible antipode; see example \ref{ex:non_regular_Hopf_system}\@.
\hfill $\star$
\end{remark}



\begin{ex} \rm
Let $G$ be any locally compact group, and let $\CC \vssp G$ denote its group algebra,
with canonical basis $\{\delta_s\}_{s\in G}$.
It is well-known that $(\CC \vssp G,\Delta,\eps,S)$ is a Hopf algebra with
$$ \Delta(\delta_s) \,=\, \delta_s \tens \delta_s   \hspace{3em}
   \eps(\delta_s)   \,=\, 1                         \hspace{3em}
   S(\delta_s)      \,=\, \delta_{s^{-1}}           \hspace{3em}
   (s \in G).   $$
Also consider the space $K(G)$ of all continuous complex functions on $G$
with compact support. $K(G)$ becomes an algebra under pointwise multiplication.
Defining the pairing as in example \ref{exA:introduction}, the pair
$\pairM{K(G)}{\underline{\CC \vssp G\!}\,}$ is easily seen to be a regular
(algebraic) Hopf system with non-trivial invariant functionals on both sides:
indeed integration w.r.t.\ the Haar measures on $G$ yields invariant
functionals on $K(G)$, whereas on $\CC \vssp G$ we have a Haar functional $\varphi$
given by $\varphi(\delta_s) = 0$ if $s\neq e$ and $\varphi(\delta_e) = 1$,
where $e$ denotes the identity in $G$.
\hfill $\star$
\end{ex}


\begin{ex} \label{ex:DPHA_is_Hopfstarsystem}
If\/ $\pair{A}{B}$ is a dual pair of Hopf $^*$-algebras \cite{Fons:DPHA}
with non-degenerate pairing,
then $\pair{\underline{A}\,}{\underline{B\!}\,}$ is an \ahss\@.
\hfill $\star$
\end{ex}



\subsection{A familiar example}

Let's have a look at probably the most familiar example of a (non-compact, non-discrete)
locally compact group: $(\RR,+)$.

Let $\HopfR$ be the unital $^*$-algebra generated by a single self-adjoint element $h$.
Then $\HopfR$ can be made into a Hopf $^*$-algebra by defining
$$ \Delta(h) = h \tens 1 + 1 \tens h \hspace{3em}
   \eps(h) = 0  \hspace{4em}  S(h) = -h.  \hspace{5em} $$
In fact $\HopfR$ is nothing but the algebra of polynomial functions on \RR,
endowed with the usual comultiplication, counit and antipode:
\begin{equation}\label{eq:ex:R:comult}
 (\Delta f)(s,t) = f(s+t)  \hspace{3em} \eps(f)=f(0)
     \hspace{3em} (Sf)(t) = f(-t) \hspace{1em}
\end{equation}
for $s,t \in \RR$ and $f : \RR \rarr \kk$ a polynomial function.
Clearly $\Delta f$ is a polynomial in two variables, hence
identifying with an element in $\HopfR \tens \HopfR$.
Next we define a pairing of $\HopfR$ with $\HopfR$ itself, as follows:
\begin{equation}\label{eq:ex:R:pairing}
  \pair{h^n}{h^m} \:=\: \delta_{n,m}\, n!\,(-i)^n   \hspace{5em} (n,m \in\NN).
\end{equation}
Then $\pair{\HopfR}{\HopfR}$ is a dual pair of Hopf $^*$-algebras \cite{Fons:DPHA}\@.
Now we wish to consider all entire\footnote{
Of course entire functions are defined on all of \CC, but since
they are determined by their values on \RR, we can still think of them
as functions on our group \RR.}
functions rather than just polynomial ones, so recall the conventions concerning
entire functions (\S \ref{sec:conventions}).
The main observation in this respect, is that (\ref{eq:ex:R:comult}) no longer
defines an \lq algebraic\rq\ comultiplication:
indeed, if $f$ is an entire function, then $\Delta f$ will generally
not identify with an element in the {\em algebraic\/} tensor product $\HC \tens \HC$.
In other words, we cannot make \HC\ into a Hopf algebra.
Nevertheless we can extend the pairing (\ref{eq:ex:R:pairing}) on one side,
defining a bilinear form $\pairing :  \HopfR \times \HC  \rarr \CC$ by
\begin{equation}\label{eq:ex:PR:extendedpairing}
  \pair{h^n}{f} \,=\: \mu_n(f)\, n!\, (-i)^n  \hspace{4em} (f\in \HC,\, n \in \NN).
\end{equation}


\begin{ex} \label{ex:familiar:hopfR}
$\pairM{\underline{\HopfR}\,}{\HC}$ is an \ahss\@.
The actions of\/ $\HopfR$ on\/ \HC\ are essentially given by differentiation:
\begin{equation}\label{eq:actions:PR_on_HC}
  h \lact f \:=\:  -i f'  \:=\:  f \ract h.
\end{equation}
\end{ex}

\begin{proof}
Non-degeneracy of (\ref{eq:ex:PR:extendedpairing})\ is clear.
Also (\ref{eq:actions:PR_on_HC}) and conditions (i-iii-v) of
definition \ref{def:algebraic_Hopf_system}\ are easily verified.
(ii) involves the binomial theorem and the rules for multiplying Taylor series,
whereas (iv) follows from $f\conj(z)=\tilde{f}(-z)$.
\end{proof}
\vspace{2ex}

Now let $S(\RR)$ denote classical Schwartz space and define
$$ \SH \:=\: \left\{ f\in \HC \left|\vertM \right.
                               f_{|\RR}  \in S(\RR) \right\}.$$
Clearly $\SH$ is a $^*$-subalgebra of \HC, containing e.g.\ the Hermite functions.


\begin{ex} \label{ex:familiar:Schwartz}
$\pairM{\underline{\HopfR}\,}{\SH\!}$ is an \ahss\@.
Lebesgue integration over\/ \RR\ yields a functional\/ $\int_\RR$ on\/ $\SH$ which is both
left and right invariant.
\end{ex}

\begin{proof}
$\SH$ is invariant under differentiation, hence also under the actions
(\ref{eq:actions:PR_on_HC}). The pairing is still non-degenerate.
The fundamental theorem of calculus yields
$$  \int_\RR f \! \ract h
         \;=\:  \int_\RR h \lact \! f
         \;=\:  -i \int_\RR f'
         \;=\:  0
         \:=\:  \eps(h) \int_\RR f   \hspace{3em} (f\in\SH). $$
Hence $\int_\RR$ is both left and right invariant in the sense of
definition \ref{def:invariant_functional}.
\end{proof}
\vspace{2ex}


\paragraph{A non-regular Hopf system}
Let $E$ be the subspace of \HC\ spanned by the functions
$\CC \rarr \CC : x \mapsto  x^n \exp(\alpha x)$ with $n\in \NN$ and $\alpha > 0$.
Clearly $E$ is a $^*$-subalgebra of \HC\@.
Since $E$ is not invariant under the antipode, we obtain


\begin{ex} \label{ex:non_regular_Hopf_system}
$\pairM{\underline{\HopfR}\,}{\!E}$ is an \ahs, but it is not regular
in the sense of definition \ref{def:regular:hs}\@.
\end{ex}
