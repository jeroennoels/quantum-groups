
\section{Haar functionals on \protect\Uqext}

\begin{abs_chp}
We construct positive and faithful Haar functionals on \UqT\ for a suitable choice of \calL\@.
In their definition we recognize the {\em spectral conditions\/} of
\cite{Fons:spectral_conditions,Wor:QE2}\@.
Furthermore we compute {\scriptsize KMS} automorphisms and modular elements.
\end{abs_chp}

Throughout this section we fix a real number $\tau$ with $\tau>0$.


\subsection{Construction \&\ invariance}
\label{par:Haar:Uq:construction}

We start by making a specific choice for the space \calL\ appearing
in \S\ref{subsect:spaceUqT}\@.


\begin{assume} \label{assume:Gr} \rm
Whenever $r$ is a real number with $r>0$, we assume $\calG_r$ to be a
non-trivial self-adjoint subspace of \HC\ such that
\begin{enumerate}
\item $\calG_r$ is invariant under $\Omega^{\pm 2}$, $\Psi$ and $\Dqsqr$
\item $\sum_{k \in \ZZ} \, g(r q^{2k})\, q^{2k} $
      is absolutely summable for any $g\in \calG_r$.
\end{enumerate}
$\calG_r$ is {\em not\/} assumed to be an algebra unless explicitly stated.
\hfill $\bullet$
\end{assume}


\begin{ex} \label{ex:Schwartz-like_space}
\rm A possible choice for $\calG_r$ could be the following: first define
$$ {\mathcal S}_r(\RR^+;q)
    \:=\: \left\{ g \in \HC \left|\vertL \right.
          \mbox{for all $n \in \NN$, the set
          $\left\{\,g(r q^k)\,q^{nk} \right\}_{_{\scriptstyle k \in \ZZ}}$
          is bounded} \right\}.$$
Now it is not hard to
prove\footnote{To show (ii) and \mbox{$\Dqsqr$-invariance},
one should take into account the following observation:
if $g$ is entire, then so is $\Dqsqr g$. Hence $g$ and $\Dqsqr g$
are bounded on the interval $[0,r]$.}\
that ${\mathcal S}_r(\RR^+;q^2)$ satisfies our requirements on $\calG_r$.
Moreover ${\mathcal S}_r(\RR^+;q^2)$ is in fact an {\em algebra\/}, and furthermore
it is easy to prove that it is actually equal to the \lq Schwartz-like\rq\ space
that was used in \cite{Koelink:thesis,Koelink:QE2}\ for similar purposes.
\hfill $\star$
\end{ex}


Given a space \Gtau\ satisfying assumptions \ref{assume:Gr},
we define subspaces \Geven\ and \Godd\ of \HC\ by
$\Geven = \Gtau$ and $\Godd = \Omega\Gtau$.
Now it is easy to see that both \Geven\ and \Godd\ are self-adjoint and invariant
under $\Omega^{\pm 2}$, $\Psi$ and $\Dqsqr$.
In this respect, observe that $\Dqsqr \Omega = q\,\Omega \Dqsqr$.
\vspace{1ex}

Let us agree that $\evenodd$ means: even and odd {\em respectively}\@.
Henceforth $K^\evenodd(\Zt)$ will denote the space of all
functions on $\Zt$ with finite support contained in $\ZZ_\evenodd\theta$.
Here $\ZZ_\evenodd$ denotes the $\evenodd$ integers.
Clearly $K^\evenodd (\Zt)$ are self-adjoint subalgebras of \FZ\ which are
moreover invariant under $\Gamma^{\pm 2}$, $\Phi^{\pm 1}$ and $^\bullet$.
\vspace{2ex}

Notice that \KZeven\ and \KZodd\ are in direct sum within \FZ,
and consequently, $\KZeven \tens \Geven$ and $\KZodd \tens \Godd$
are in direct sum within $\FZ \tens \HC$. Therefore we can define
\begin{equation}\label{eq:def:TG}
  \calL(\Gtau) \,=\,  \left(\KZeven \tens \Geven \vertM\right) \oplus
                      \left(\KZodd  \tens \Godd  \vertM\right)
  \, \subseteq \,  \KZ \tens \HC. \;
\end{equation}
Now it is easy to verify that $\calL=\calL(\Gtau)$ satisfies the conditions of
proposition \ref{prop:UqFG}, except for the \lq moreover\rq\ part,
the latter being satisfied as well {\em provided\/} we assume \Gtau\ to be an
algebra; indeed if \Gtau\ is an algebra, then so is $\calL(\Gtau)$,
because $\KZeven$ and $\KZodd$ annihilate one another under multiplication.


\begin{remarks} \label{rem:parity} \rm
\item
The fact we actually have a {\em direct\/} sum in (\ref{eq:def:TG}) comes
merely from the \lq first leg\rq\ of the tensor product. Indeed
$\Geven$ and $\Godd$ need not to be in a direct sum position within \HC\@.
To illustrate this, consider the example
$\Gtau={\mathcal S}_\tau (\RR^+;q^2)$ and observe that
\begin{equation}\label{eq:qSchwartz:not_direct_sum}
    {\mathcal S}_\tau^{\rm \scriptscriptstyle even} (\RR^+;q^2) \:\cap\:
    {\mathcal S}_\tau^{\rm \scriptscriptstyle odd}  (\RR^+;q^2)  \;=\;
    {\mathcal S}_\tau (\RR^+;q).
\end{equation}
\item
Notice that in (\ref{eq:def:TG}) {\em both\/} the even and odd components
are necessary: otherwise $\calL(\Gtau)$ would not be invariant under
$\Gamma \tens \Omega^{\pm 1}$ or $\Gamma^{-1} \tens \Omega^{\pm 1}$
(which was an essential condition in proposition \ref{prop:UqFG}).
\item
Why so complicated? Can't we simply take
${\calL} = \KZ \tens {\mathcal S}_\tau (\RR^+;q)\,$?
First observe that the latter space is actually {\em smaller\/} than
$\calL\!\left(\vertM {\mathcal S}_\tau (\RR^+;q^2) \right)$
because of (\ref{eq:qSchwartz:not_direct_sum}).
Furthermore it indeed satisfies all the requirements of \mbox{proposition \ref{prop:UqFG}},
and as a matter of fact one would get quite far with it.
However, this is {\em not at all\/} the proper choice if we want
to do harmonic analysis, as we will explain later
(cf.\ remarks \ref{rem:haar:Uq:concrete}\ and \ref{rem:Fourier:construction:parity}).
\end{remarks}


We are about to invoke corollary \ref{cor:UqFGAq:Hopf_system}\@.
In this respect only one condition remains to be shown:
\UqTG\ should separate \Aq\ within the duality.
In order not to slow down our construction, we defer this technical matter to
appendix \ref{app:technical}\@.
The result of all this is the following:


\begin{prop} \label{prop:UqGT:Aq:summary}
 Whenever\/ \Gtau\ is a subalgebra of\/ \HC\ satisfying assumptions \ref{assume:Gr},
 the pair
 $\left\langle\vertM\right.\!   \UqTG, \, \underline{\Aq\!}  \left.\vertM\right\rangle$
 is an \ahss\footnote{See definition \ref{def:algebraic_Hopf_system}.}\@.
\end{prop}


The next step is to construct Haar functionals on \UqTG\@.
\begin{defn*} \label{def:lam_rho_chi}
Let us define functionals $\lambda_{_\evenodd}$ and
$\rho_{_\evenodd}$ on $K^\evenodd(\Zt)$ by
$$ \displaystyle
     \lambda_{_\evenodd}(f)\: = \!\! \sum_{k\,\in\,\ZZ_\evenodd}
        \hspace{-0.4em}  f(k\theta)\,q^k
   \andspace{3em}
     \rho_{_\evenodd}(f)   \: = \!\! \sum_{k\,\in\,\ZZ_\evenodd}
        \hspace{-0.4em}  f(k\theta)\,q^{-k}  $$
whereas on $\Gtau^\evenodd$ we define linear functionals $\chi_{_\evenodd}$ by
$$  \chi_{_\evenodd}(g) \;= \! \sum_{l\,\in\,\ZZ_\evenodd}
               \hspace{-0.3em} g(\tau q^l)\, q^l. $$
\end{defn*}


\begin{lemma} \label{lemma:lam_rho_chi_nabq}
We have
$$ \begin{array}{lcl}
   \begin{array}{l}
  \lameven\Gamma = q \lamodd  \\
  \lamodd\Gamma\,= q \lameven
\end{array} & \hspace{15mm}&
\begin{array}{l}
  \rhoeven\Gamma = q^{-1} \rhoodd  \\
  \rhoodd\Gamma\,= q^{-1} \rhoeven
\end{array} \\ \rule{0pt}{5ex}
\begin{array}{l}
  \chieven\Omega = q^{-1} \chiodd \\
  \chiodd\Omega\,= q^{-1} \chieven
\end{array} & &
\;\,\chi_{_\evenodd} \nabq{1} = 0.
\end{array}$$
\rm This makes sense because $\KZeven = \Gamma \KZodd$ and $\Geven = \Omega\Godd$.
\end{lemma}


\begin{defn}
In view of (\ref{eq:UqFG:direct_sum}) we can define
linear functionals $\varphi$ and $\psi$ on \UqTG\ as follows:
putting $\varphi$ and $\psi$ equal to zero on
\begin{equation} \label{eq:haar:Uq:zero}
    \Upsilon\!\left(\TG \vertM\right) b^{\NN_0}
       \andspace{10mm}
    \Upsilon\!\left(\TG \vertM\right) c^{\NN_0},
\end{equation}
it only remains to define $\varphi$ and $\psi$ on $\Upsilon\!\left(\TG \vertM\right)$.
In this respect we require
\begin{eqnarray} \label{eq:def:haar:Uq:phi}
   \varphi \Upsilon &=& (\lameven \tens \chieven) \oplus  (\lamodd \tens \chiodd) \\
   \psi \Upsilon    &=& (\rhoeven \tens \chieven) \oplus (\rhoodd \tens \chiodd)
\end{eqnarray}
where we considered $\Upsilon$ restricted to \TG\ as defined in (\ref{eq:def:TG}).
\end{defn}


\begin{prop} \label{prop:haar:Uq:invariance}
 The functionals\/ $\varphi$ and\/ $\psi$ as defined above are respectively
 left and right invariant (cf.\ definition \ref{def:invariant_functional}) for\/
 $\left\langle\vertM\right.\!  \UqTG, \, \underline{\Aq\!} \left.\vertM\right\rangle$.
\end{prop}
\begin{proof}
 Let's consider the case of $\varphi$.
 Since $\eps(\alpha)=1$ and  $\eps(\beta) = \eps(\gamma) = 0$,
 it suffices to show
 \begin{eqnarray*}
 \varphi\left(\Upsilon(X)\, \{b \mbox{ or } c\}^m \,\ract \, \alpha
     \vertM\right)
   &=& \varphi\left(\Upsilon(X)\, \{b \mbox{ or } c\}^m  \vertM\right)
     \\ \vertXL
  \varphi\left(\Upsilon(X)\,
          \{b \mbox{ or } c\}^m \,\ract\,\beta  \vertM\right) &=& 0
    \\  \vertXL
  \varphi\left(\Upsilon(X)\,
         \{b \mbox{ or } c\}^m\, \ract \,\gamma  \vertM\right) &=& 0
\end{eqnarray*}
for $X\in \TG$ and $m\in \NN$.
Observe the action of $\alpha$ does not alter the exponent of $b$ or $c$,
whereas it is lowered or raised by 1 under the actions of $\beta$ and $\gamma$
(\mbox{cf.\ proposition \ref{prop:actions:Upsilon}}). Since $\varphi$ vanishes
on (\ref{eq:haar:Uq:zero}) only a few cases remain to be investigated:
$$\begin{array}{lclcl}
   \varphi\left(\Upsilon(X)\,\ract \, \alpha  \vertM\right)
     &=& \varphi\left(\Upsilon \!\left(\Gamma \tens \Omega \vertM \right)X \vertM\right)
     &=& \varphi\left(\Upsilon(X) \vertM\right)
     \\ \vertXXL
   \varphi\left(\Upsilon(X)\,b\, \ract\,\beta  \vertM\right)
     &=& \varphi\left(\Upsilon \!\left(\Phi^{-1}\Gamma \tens \nabq{1}\right)X
      \vertM\right) &=& 0 \\  \vertXXL
   \varphi\left(\Upsilon(X) \, c\, \ract \, \gamma  \vertM\right)
     &=& \varphi\left(\Upsilon \!\left(\Phi^{-1}\Gamma^{-1} \tens \nabq{1}\right)X
       \vertM\right)
     &=& 0.
\end{array} $$
Here we used proposition \ref{prop:actions:Upsilon},
lemma \ref{lemma:lam_rho_chi_nabq}\ and of course
(\ref{eq:def:haar:Uq:phi}).
%Right invariance of $\psi$ is shown similarly.
\end{proof}


\begin{remark} \label{rem:haar:Uq:concrete} \rm
Elements in $\FZ \tens \HC$ can be considered as functions in
2 variables, i.e.\ from $\Zt \times \CC$ into \CC\@.
When definition \ref{def:lam_rho_chi}\ is plugged into (\ref{eq:def:haar:Uq:phi})
we thus obtain the following expression for $\varphi$ on $\Upsilon\!\left(\TG \vertM \right)$:
\begin{equation}\label{eq:haar:Uq:concrete}
  \varphi \left(\Upsilon(X) \vertM\right)
     \:=\: \sum_{(k,l)\, \in \,\mathfrak{S}}  X(k\theta, \tau q^l)\, q^{k+l}.
\end{equation}
Here $\mathfrak{S}=(\ZZeven \times \ZZeven) \cup (\ZZodd \times \ZZodd)$
and $X\in \TG$ was implicitly identified with the corresponding
function on $\Zt \times \CC$.
This formula however, isn't always convenient because of
the peculiar nature of the summation range; one might ask why we don't simply
take a sum over $\ZZ^2$ anyway, and indeed, as far as the
construction of invariant functionals is concerned, there's
absolutely no need to work with $\mathfrak{S}$ instead of $\ZZ^2$
(cf.\ remark \ref{rem:parity}.iii).
Nevertheless there are good reasons to use $\mathfrak{S}$:
in chapter \ref{chapter:Harmonic_analysis_on_quantumE2}\ we shall construct Fourier
transforms and prove a Plancherel formula which relates the
Haar functionals $\varphi$ and $\psi$ with the Haar functional
on the dual as constructed in
\cite[see also \S\ref{par:Haar_functionals_on_Aqext}\ below]{Koelink:thesis,Koelink:QE2}\@.
Precisely at this point it becomes essential to use this particular
\lq parity\rq\ structure, otherwise our proof for this Plancherel formula would fail.
Moreover only (\ref{eq:def:TG}) will supply sufficiently many elements which
behave \lq nicely\rq\ under Fourier transformation.
Another motivation comes from the operator-algebraic versions
of the quantum $E(2)$ group \cite{FonsWor:QE2,Wor:QE2,Wor:Affiliated}\@.
The so-called \lq spectral conditions\rq\
\cite{Fons:spectral_conditions,Wor:QE2} emerging at this
operator-theoretical level are indeed closely related to our
choice $\mathfrak{S}$ for the summation range in (\ref{eq:haar:Uq:concrete}).
\hfill $\star$
\end{remark}


\subsection{Positivity \&\ faithfulness}

Let \Gtau\ be a subalgebra of \HC\ satisfying assumptions \ref{assume:Gr}\@.

\begin{lemma}  \label{lem:Haar_on_UpStarUp}
  Take any\/ $X,Y\in \TG$ and\/ $r, s \in \NN$. Put\/ $m=\min\{r,s\}$.
  Then we have
\begin{eqnarray}
   \left(\Upsilon(Y)\, b^r \vertM\right)^*\!
     \left(\Upsilon(X)\, b^s
           \vertM\right)
   \!\!&=&\!\!
    \Upsilon\!\left((\Gamma^{2r} \tens \Psi^m)(Y^* X) \vertM\right)
     b^{s-m} c^{r-m}
  \label{eq:UpStarUp:bb} \\
  \left(\Upsilon(Y)\, b^r  \vertM\right)^* \!
  \left(\Upsilon(X)\, c^s  \vertM\right)
   \!\!&=&\!\!
  \Upsilon\!\left((\Gamma^{2r} \tens \id)(Y^* X) \vertM\right)
      c^{r + s}
  \label{eq:UpStarUp:bc} \\
  \left(\Upsilon(Y)\, c^r \vertM\right)^* \!
  \left(\Upsilon(X)\, b^s  \vertM\right)
   \!\!&=&\!\!
   \Upsilon\!\left((\Gamma^{-2r} \tens \id)(Y^* X) \vertM\right)
      b^{r + s}
  \label{eq:UpStarUp:cb} \\
  \left(\Upsilon(Y)\, c^r  \vertM\right)^* \!
  \left(\Upsilon(X)\, c^s  \vertM\right)
  \!\!&=&\!\!
   \Upsilon\!\left((\Gamma^{-2r} \tens \Psi^m)(Y^* X) \vertM\right)
      b^{r-m} \, c^{s-m} \hspace{3em}
  \label{eq:UpStarUp:cc}
\end{eqnarray}
\end{lemma}
\begin{proof}
  Combine the commutation rules of proposition
  \ref{prop:Upsilon:commutation_rules}\ with the fact that $\Upsilon$
  is a $^*$-algebra morphism (proposition \ref{prop:Upsilon:morphism}).
\end{proof}



\begin{cor} \label{cor:Haar_on_UpStarUp}
  By definition the functionals\/ $\varphi$ and\/ $\psi$ vanish on the elements
  (\ref{eq:UpStarUp:bc}) and (\ref{eq:UpStarUp:cb}) unless both\/ $r$ and
  $s$ are zero. Moreover\/ $\varphi$ and\/ $\psi$ also vanish on the elements
  (\ref{eq:UpStarUp:bb}) and (\ref{eq:UpStarUp:cc}) except when\/ $r=s=m$.
  In the latter case we actually have for instance
\begin{eqnarray*}
    \varphi\left(
       \left(\Upsilon(Y)\, b^m  \vertM\right)^* \!
       \left(\Upsilon(X)\, b^m  \vertM\right)\right)
    &=& \;\: q^{2m} \,\varphi \left(
      \Upsilon\!\left(\id \tens \Psi^m \right)(Y^* X) \vertL\right) \\
   \varphi\left(
       \left(\Upsilon(Y)\, c^m  \vertM\right)^* \!
       \left(\Upsilon(X)\, c^m  \vertM\right)\right)
    &=& q^{-2m}  \,\varphi \left(
      \Upsilon\!\left(\id \tens \Psi^m \right)(Y^* X)  \vertL\right)
\end{eqnarray*}
\end{cor}

\begin{proof}
  Observe (\ref{eq:UpStarUp:bb}) and (\ref{eq:UpStarUp:cc}) only
  involve {\em even\/} (!)\ powers of $\Gamma$.
  On the other hand, from lemma \ref{lemma:lam_rho_chi_nabq}\ it follows that
  $\lambda_{_\evenodd} \Gamma^2 =  q^2\,\lambda_{_\evenodd}$.
\end{proof}


\begin{defn*} \label{def:skalprodphi}
 For any $m\in \NN$ we define a sesquilinear form $\skalphi{\cdot}{\cdot}{m}$
 on $\TG \times \TG$ by
 $$ \skalphi{X}{Y}{m}  \: =\: \varphi \left(
      \Upsilon\!\left(\id \tens \Psi^m \right)(Y^* X)  \vertL\right)$$
\end{defn*}



\begin{lemma} \label{lemma:skalprodphi:positive}
  The sesquilinear forms defined above are scalar products.
\end{lemma}

\begin{proof}
First recall (\ref{eq:def:TG}) and observe that the spaces $\KZeven\tens \Geven$
and $\KZodd \tens \Godd$ are mutually orthogonal w.r.t.\ $\skalphi{\cdot}{\cdot}{m}$.
Therefore they can be treated separately. Let's for instance take any
$X=\sum_{i=1}^r  f_i \tens g_i$ with $f_1,\ldots, f_r$ in $\KZeven$
and $g_1,\ldots, g_r$ in $\Geven$. Then
\begin{eqnarray*}
  \skalphi{X}{X}{m} &=& \sum_{i,j=1}^r\;
           \varphi\left(\Upsilon \!\left(\overline{f_i}f_j
           \tens \Psi^m(\widetilde{g}_i g_j) \right) \vertL\!\right) \\
 &=& \sum_{i,j=1}^r \lameven(\overline{f_i}f_j)\:
             \chieven\!\left(\Psi^m(\widetilde{g}_i g_j)\vertM\right) \\
 &=& \sum_{i,j=1}^r \;\sum_{k \,\in \,\ZZeven}
       (\overline{f_i}f_j)(k\theta)\,q^k  \sum_{l \,\in \,\ZZeven}
       \left(\Psi^m(\widetilde{g}_i g_j)\vertM\right)(\tau q^l)\: q^l\\
 &=& \sum_{k,l \,\in\, \ZZeven} \; \sum_{i,j=1}^r \;
       \overline{f_i(k\theta)\,g_i(\tau q^l)}\; f_j(k\theta)\,
         g_j(\tau q^l)\; \tau^m\, q^{l(m+1)+k} \\
 &=& \sum_{k,l \,\in\, \ZZeven} \;
       \left| \sum_{i=1}^r f_i(k\theta)\,g_i(\tau q^l) \right|^2  \tau^m\,
       q^{l(m+1)+k}
\end{eqnarray*}
Notice that the summation over $k$ in the above expressions de
facto runs over a {\em finite\/} number of terms, so the only
infinite sum involved here, is the summation over $l$. Clearly
the sums converge absolutely, because of the assumptions \mbox{on \Gtau}\@.
The last expression above is obviously non-negative; it is zero
if and {\em only\/} if
\begin{equation}\label{eq:skalprod:definite}
  \left(\textstyle \sum_{i=1}^r \, f_i(k\theta)\,g_i \vertL\right)(\tau q^l)= 0
  \hspace{3em}\mbox{for all } k,l \in \ZZeven.
  \hspace{1em}
\end{equation}
Since $\sum_{i=1}^r f_i(k\theta)\,g_i$ is entire and
vanishes on a set having a point of accumulation, it follows that
$\sum_{i=1}^r f_i(k\theta)\,g_i=0$ for any $k\in\ZZeven$, and hence $X=0$.
The \lq odd\rq\ case is similar.
\end{proof}


\begin{prop} \label{prop:haar:Uq:positive}
$\;\varphi$ and\/ $\psi$ are hermitian, positive and faithful.
\end{prop}

\begin{proof}
Choose any element $x\in \UqTG$. Recalling (\ref{eq:UqFG:direct_sum}) we can write
\begin{equation} \label{eq:expansion_element_of_UqTG}
 x \:=\: \sum_{m=0}^\infty \Upsilon(X_m)\, b^m \:+\:
         \sum_{n=1}^\infty \Upsilon(Y_n)\, c^n
\end{equation}
with only finitely many non-zero $X_m, Y_n \in \TG$.
Corollary \ref{cor:Haar_on_UpStarUp}\ and definition \ref{def:skalprodphi} yield
$$  \varphi(x^*x)\; = \;\sum_{m=0}^\infty q^{2m}\,  \skalphi{X_m}{X_m}{m}
                  \;+\; \sum_{n=1}^\infty q^{-2n}\, \skalphi{Y_n}{Y_n}{m}$$
Now positivity and faithfulness of $\varphi$ follows since the
$\skalphi{\cdot}{\cdot}{m}$ are scalar products (cf.\ lemma \ref{lemma:skalprodphi:positive}).
Furthermore $\varphi$ is hermitian because $\Upsilon$ is a $^*$-morphism.
\end{proof}



\subsection{KMS properties \&\ modular element}

Recall definitions \ref{def:KMS}\ and \ref{def:Fourier_context}\@.
Again (besides assumptions \ref{assume:Gr}) we assume \Gtau\ to be an {\em algebra}\@.

\begin{prop} \label{prop:Uq:KMS}
The Haar functionals\/ $\varphi$ and\/ $\psi$ are weakly {\scriptsize KMS}\@.
The {\scriptsize KMS}-automorphisms are given by\/ $\sigma_\varphi = S^2$
and\/ $\sigma_\psi = S^{-2}$. Moreover\/ $\varphi$ and\/ $\psi$ are\/
\Uq-{\scriptsize KMS}\ in the sense that for instance\/
$\varphi(x \xi) = \varphi(\xi \sigma_{\!\varphi}(x))$
for any\/ $\xi \in \Uq$ and\/ $x\in \UqTG$.
Furthermore, we have\/ $\varphi S = \psi$ and\/ $\psi S = \varphi$.
Here\/ $S$ is the antipode on\/ $\UqTG$.
Eventually we observe that the modular element for\/ $\varphi$ is given by\/ $a^4\in\Uq$.
\end{prop}

\begin{proof}
We have to show that $\varphi(xy) = \varphi(y S^2(x))$ for any $x,y \in \UqTG$.
Now in view of (\ref{eq:UqFG:direct_sum}) it suffices to consider
\begin{equation}\label{eq:proof:Uq:KMS:xy}
   x=\Upsilon(X)\, \{ b \mbox{ or } c \}^r  \andspace{3.5em}
   y=\Upsilon(Y)\, \{ b \mbox{ or } c \}^s
\end{equation}
for $r,s\in\NN$ and $X,Y \in \TG$.
From propositions \ref{prop:Upsilon:morphism},
\ref{prop:Upsilon:commutation_rules}\ and \ref{prop:UqFG:antipode},
together with the invariance properties of $\TG$, it follows easily that
both $\varphi(xy)$ and $\varphi(y S^2(x))$ vanish for most
combinations of $x$ and $y$ of type (\ref{eq:proof:Uq:KMS:xy}).
Only the following cases remain to be shown:
\begin{equation}\label{eq:proof:Uq:KMS:bc}
       \varphi\left(\Upsilon(X)\,b^m\; \Upsilon(Y)\,c^m      \vertL \right)
   \;=\; \varphi\left(\Upsilon(Y)\,c^m\;
                 S^2\!\left(\Upsilon(X)\,b^m \vertM \right) \! \vertL \right)
\end{equation}
\begin{equation}\label{eq:proof:Uq:KMS:cb}
   \varphi\left(\Upsilon(X)\,c^m\; \Upsilon(Y)\,b^m          \vertL \right)
   \;=\; \varphi\left(\Upsilon(Y)\,b^m\;
                 S^2\!\left(\Upsilon(X)\,c^m \vertM \right) \! \vertL \right)
\end{equation}
for all $m\in\NN$. Now, again with propositions \ref{prop:Upsilon:morphism},
\ref{prop:Upsilon:commutation_rules}\ and \ref{prop:UqFG:antipode},
and together with the fact that $\TG$ is a (commutative!) algebra
which is invariant under $\Gamma^{\pm 2} \tens \id$ and $\id \tens \Psi$, we obtain
\begin{eqnarray*}
\varphi\left(\Upsilon(X)\,b^m\: \Upsilon(Y)\,c^m  \vertL \right)
&=&
\varphi\left(\Upsilon(X)\,\Upsilon \!\left((\Gamma^{-2m} \tens \id)Y\right) b^m c^m  \vertL \right)
\\ &=&
\varphi\left(\Upsilon \!\left\{\left(\Gamma^{-2m} \tens \id \right) \left(
       \left[(\Gamma^{2m} \tens \id)X\right] Y \vertL \right)\right\}  (bc)^m \right)
\\ &=&
\varphi\left(\Upsilon \!\left\{\left(\Gamma^{-2m} \tens \Psi^m \right) \left(
       \left[(\Gamma^{2m} \tens \id)X\right] Y \vertL \right)\right\} \right)
\\ &\stackrel{(*)}{=}&
q^{-2m}\; \varphi\left(\Upsilon \!\left\{\left(\id \tens \Psi^m \right) \left(
       \left[(\Gamma^{2m} \tens \id)X\right] Y \vertL \right)\right\} \right)
\\ &=&
q^{-2m}\; \varphi\left(\Upsilon \!\left(Y (\Gamma^{2m} \tens \id) X \right) (bc)^m \vertL \right)
\\ &=&
q^{-2m}\; \varphi\left(\Upsilon(Y) \, \Upsilon
       \left((\Gamma^{2m} \tens \id) X \right) c^m b^m \vertL\right)
\\ &=&
q^{-2m}\; \varphi\left(\Upsilon(Y)\, c^m \: \Upsilon(X)\, b^m \vertL\right)
\\ &=&
\varphi\left(\Upsilon(Y)\,c^m\;
        S^2\!\left(\Upsilon(X)\,b^m \vertM \right) \! \vertL\right)
\end{eqnarray*}
In $(*)$ we used that
$$ \varphi \left(\Upsilon(\Gamma^k \tens \id)Z\vertM\right)
        \:=\: q^k \,\varphi\left(\Upsilon(Z)\vertM\right) $$
for any $Z \in \TG$ and any {\em even\/} integer $k$.
This proves (\ref{eq:proof:Uq:KMS:bc}). Similarly (\ref{eq:proof:Uq:KMS:cb}).
Also the $\Uq$-{\sc kms}\ statement is quite straight\-forward to prove;
then again, the computation differs significantly from the one above,
so let's have a look anyway:
for any $X \in \TG$, $p \in \ZZ$ and $m,r,s \in \NN$ we have
\begin{eqnarray*}
\varphi\left(\Upsilon(X)\, b^m \; a^p b^r c^s \vertL \right)
\!\!\!&=\!\!&
q^{-mp} \: \varphi\left(\Upsilon(X)\, a^p b^m b^r c^s \vertL\right) \\
&=\!\!&
q^{-mp} \: \varphi\left(\Upsilon \!\left(
        (\Phi^p \tens \id)X \vertM\right) b^{m+r} c^s \vertL \right) \\
&=\!\!&
\delta_{m+r,s} \: q^{-mp} \: \varphi\left(
     \Upsilon \!\left((\Phi^p \tens \Psi^{m+r})
     X\right) \! \vertL\right) \\
&=\!\!&
\delta_{m+r,s} \: q^{-mp} \, q^{-2m} \: \varphi\left(
     \Upsilon \!\left((\Gamma^{2m} \tens \id)(\Phi^p \tens \Psi^{m+r})
     X\right) \! \vertL\right) \\
&=\!\!&
\delta_{m+r,s} \: q^{-(s-r)p} \, q^{-2m} \: \varphi\left(
    (bc)^{m+r} \:\Upsilon \!\left((\Gamma^{2m} \Phi^p \tens \id)
    X\right)\! \vertL\right)  \\
&=\!\!&
    q^{(r-s)p} \, q^{-2m} \: \varphi\left(
    b^{m+r} c^s \: \Upsilon \!\left((\Gamma^{2m} \Phi^p \tens \id)
    X\right)\! \vertL\right)  \\
&=\!\!&
    q^{(r-s)p} \, q^{-2m} \: \varphi\left(
    b^r c^s \: \Upsilon \!\left((\Phi^p \tens \id) X \vertM \right)  b^m \vertL\right)  \\
&=\!\!&
    q^{(r-s)p} \: q^{-2m} \: \varphi\left( b^r c^s a^p \: \Upsilon(X)\, b^m \vertL\right)  \\
&=\!\!&
    q^{-2m} \, \varphi\left(a^p  b^r c^s \: \Upsilon(X) \, b^m \vertL\right) \\
&=\!\!&
\varphi\left(a^p  b^r c^s \; S^2\!\left(\Upsilon(X) \, b^m \vertM \right) \! \vertL \right).
\end{eqnarray*}
Notice the above computation again involved several invariance requirements on
$\TG$, which are all satisfied here, of course. The remaining cases are similar.
The behaviour w.r.t.\ the antipode ($\varphi S = \psi$ etc.)
is an easy consequence of proposition \ref{prop:UqFG:antipode}\ and the fact that
$\lambda_{_\evenodd} \bullet = \rho_{_\evenodd}$.
Eventually the formula for the modular element follows easily from one of the
commutation rules in proposition \ref{prop:Upsilon:commutation_rules}.
\end{proof}
