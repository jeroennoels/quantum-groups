
\section{Special cases}

\subsection{The commutative case} \label{par:comm_case}

When $\Om$ is a bimodule over some algebra $E$, we shall use $\mbox{\rm End}_E(\Om)$
to denote the algebra of all \Ebimod\ morphisms from $\Om$ into $\Om$.
If $E$ is {\em commutative}, then $ \mbox{\rm End}_E(\Om)$ is again an \Ebimod,
the actions (denoted by juxtaposition) of an element $x\in E$ on a morphism
$\gamma \in  \mbox{\rm End}_E(\Om)$ being given by
$$ (x \gamma)(\om) \,=\,  x \lact \gamma(\om)   \andspace{2em}
   (\gamma x)(\om) \,=\,  \gamma(\om) \ract x   \hspace{3em}  (\om \in \Om). $$


\begin{prop} \label{prop:comm_case}
  Let\/ $\EE \equiv \EOP$ be an \context, $E$ commutative and\/ $\Om$ unital as an
  \Ebimod\ \mbox{\rm (\S \ref{sec:conventions})}\@. Then we have natural isomorphisms
  $$ \ActE  \,\simeq\, \mbox{\rm End}_E(\Om)    \itandspace{3em}
     \EnvE  \,\simeq\: \mbox{\rm Center}\left(\mbox{\rm End}_E(\Om)\vertM\right) $$
  of $E$-bimodules and algebras respectively
  (recall lemma \ref{lem:Act_is_submodule}\ and corollary \ref{cor:Env_is_an_algebra}).
  In particular we observe that\/ \EnvE\ is again commutative.
\end{prop}

\begin{proof}
  Since $(\Om,\pairing)$ is an actor implementation of a commutative algebra $E$,
  the left and right $E$-module structures on $\Om$ must coincide,
  i.e.\ $x\lact \om = \om \ract x$ for all $x \in E$ and $\om \in \Om$\@.
  Let $a\equiv(\lam,\rho)$ be any actor for \EE\@.
  Since $\lam$ is now both a left and right $E$-module morphism,
  we get for all $x,y \in E$ and $\om \in \Om$ that
  $$ \pair{x}{\rho(\om \ract y)}    \:\stackrel{(\ref{eq:bi-actor})}{=}\:
                \pair{y}{\lam(x\lact\om)}
          \,=\, \pair{y}{x \lact \lam(\om)}
          \,=\, \pair{yx}{\lam(\om)}
          \,=\, \pair{x}{\lam(\om \ract y)}. $$
  Since by assumption $\Om \ract E = \Om$, it follows that $\lam=\rho$. Hence $a$ is of
  the form $(\gamma,\gamma)$ with $\gamma \equiv \lam=\rho \in {\rm End}_E(\Om)$\@.
  Thus we get a well-defined linear bijection
  $$\Phi : {\rm End}_E(\Om) \rarr \ActE : \gamma \mapsto (\gamma,\gamma)$$
  which is easily seen to be an isomorphism of $E$-bimodules,
  and moreover restricts to an algebra isomorphism
  of ${\rm Center}({\rm End}_E(\Om))$ onto \EnvE\@.
  Indeed, for any $\gamma_1,\gamma_2 \in {\rm End}_E(\Om)$ we observe that
  $ \Phi(\gamma_1) \Phi(\gamma_2) = (\gamma_1\gamma_2, \gamma_2\gamma_1) $
  belongs to \ActE\ again if and only if $\gamma_1$ and $\gamma_2$ commute.
  The result follows.
\end{proof}


\begin{exA} \label{exA:comm_case:revisited}
  Recall the \context\/ $\FF \equiv (F; \kk S, \pairing)$ of example
  {\rm A \ref{exA:introduction}}\@.
  We now have  ${\rm Act}(\FF) = {\rm Env}(\FF) \simeq \kk^S.$
\end{exA}

\begin{proof}
First observe that \FF\ is nothing but a \lq restricted\rq\ version of
the \context\ $(\kk^S; \kk S, \pairing)$.
For every function $g\in \kk^S$ we consider a linear map $\gamma_g : \kk S \rarr  \kk S$
defined by $\gamma_g(\delta_s) = g(s)\,\delta_s$. Thus we get an injective algebra morphism
$\Gamma : \kk^S \rarr  {\rm End}_F(\kk S) : g \mapsto \gamma_g$.
To prove that $\Gamma$ is surjective, take any $\gamma \in {\rm End}_F(\kk S)$
and define $g\in \kk^S$ by $g(s) = \pair{1}{\gamma(\delta_s)}$.
Here $1$ denotes the identity in $\kk^S$.
Using the $F$-bimodule property of $\gamma$ and the non-degeneracy of $\pair{F}{\kk S}$
it follows easily that $\gamma_g=\gamma$.
Now $\Gamma$ is an algebra isomorphism, hence ${\rm End}_F(\kk S) \simeq \kk^S$ is
commutative; proposition \ref{prop:comm_case}\ yields the result.
Also observe that $\Phi\Gamma$ (with $\Phi$ as in the above proof)
is an algebra isomorphism from $\kk^S$ onto ${\rm Env}(\FF)$ extending
the natural embedding of $F$ in ${\rm Env}(\FF)$.
\end{proof}




\subsection{Pseudo-discrete \contexts}

\begin{defn} \label{def:pseudo_discrete}
A non-degenerate \context\ \EE\ is called {\em pseudo-discrete\/} if $M(\EE) = \ActE$.
\end{defn}

In chapter \ref{chapter:Hopf_systems}, theorem \ref{thm:mhs_yields_mha}, we
shall encounter a large class of pseudo-discrete \contexts, arising from \mha s
with invariant functionals. The terminology is inspired by the following

\begin{exA}
Let\/ $X$ be a locally compact Hausdorff space and consider the \context\/
$\KK_X \! \equiv (K(X); \, \kk X, \pairing)$ as introduced in example \ref{exA:introduction}\@.
Then\/ $\KK_X \!$ is pseudo-discrete if and only if\/ $X$ is discrete.
\end{exA}
\begin{proof}
According to example \ref{exA:comm_case:revisited}\ we have $\Act(\KK_X \!) \simeq \kk^X \!$.
On the other hand $M(\KK_X \!) \simeq M(K(X)) \simeq C(X)$.
Hence $M(\KK_X \!) = \Act(\KK_X \!)$ if and only if
every complex function on $X$ is continuous.
\end{proof}
