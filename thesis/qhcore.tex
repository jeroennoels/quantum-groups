

\subsection{The spaces $\holS{m}$ do contain many functions}

The above theory would of course collapse in case the spaces $\holS{m}$ would
turn out to be trivial; fortunately we have the following:


\begin{lemma} \label{lemma:compactsupp_in_holS}
Let\/ $f$ be an entire function having the property that there
exists an integer\/ $n_0$ such that\/ $f(q^{2n})=0$ for all integers\/ $n<n_0$.
Then\/ $f \in \holS{m}$ for any\/ $m\in \ZZ$.
\end{lemma}

\begin{proof}
Observe that our assumptions imply $f$ to be in \Swqbis\ in the first place.
Then fix any $m\in \ZZ$. We claim that the series
\begin{equation}\label{eq:lemma:qqE:series}
    g(z) \:=\: \frac{1}{z^{|m|}}\: \sum_{n\in\ZZ} \:
                 q^{2n} \, \J{m}{q^n z} \, q^{|m|n} \, f(q^{2n})
\end{equation}
converges\footnote{Notice that convergence in (\ref{eq:lemma:qqE:series})\
is quite a different matter than for instance in (\ref{eq:qHankeltransform:def}).
Indeed in (\ref{eq:lemma:qqE:series}) our $q$-Bessel functions are
to be evaluated in any complex number, whereas (\ref{eq:qHankeltransform:def})
only deals with powers of $q$.
See also remark \ref{rem:not_constructive}}\ absolutely for all $z\in \CC_0$ and defines an
{\em entire\/} function $g$ (the singularity at the origin being removable).
Once this fact established, it will be easy to see that $(f, \Psi^{|m|} g)$
is an \Hmpair, and hence $f \in \holS{m}$.
So let's investigate the above series; according to lemma \ref{lemma:def:qBessel}\
it is possible to define, for any integer $n$, an entire function $g_n$ such that
$$ g_n (z) \:=\:  (q^n z)^{-|m|} \, \J{m}{q^n z} \,f(q^{2n}) $$
for $z\in \CC_0$.
The series (\ref{eq:lemma:qqE:series}) can now be rewritten as follows:
\begin{equation}\label{eq:lemma:qqE:series:rewrite}
  g \:=\: \sum_{n=n_0}^\infty \:  q^{2n(|m|+1)}\,g_n
\end{equation}
Since every $g_n$ is entire, it suffices  to show the latter series converges
uniformly on compact sets. Therefore, pick any number $r>0$ and let $D(r)$
denote the disk $\{z\in \CC \mid |z| \leq r\}$. Once again appealing to lemma
\ref{lemma:def:qBessel}, it is clear that there exists a bound, say $M_r>0$,
such that $|x^{-|m|}\J{m}{x}| \leq M_r$ whenever $0 < |x| \leq q^{n_0}r$. On
the other hand also $f$ is entire, hence bounded on compact sets; so we can
find a bound $N>0$ such that $|f(x)| \leq N$ for all $x$ in the interval
$[0,q^{2n_0}]$. Since $0<q<1$, it follows that $|g_n(z)| \leq M_r N$ for any $n
\geq n_0$ and any $z\in D(r)$. In other words, the family
$\{g_n\}_{n=n_0}^\infty$ is uniformly bounded on $D(r)$. Since
$\sum_{n=n_0}^\infty q^{2(|m|+1)n}$ is a convergent geometric series,
(\ref{eq:lemma:qqE:series:rewrite}) yields an entire function $g$ which
satisfies (\ref{eq:lemma:qqE:series}). Now we still have to show that $(f,
\Psi^{|m|} g)$ is indeed an \Hmpair\@. Only item (ii) of definition
\ref{def:Hmpair}\ requires some explanation; from the power series
(\ref{eq:def:qBessel}) of the \little\ $q$-Bessel functions, it is clear that
$\J{m}{\,\cdot\:}$ is even (resp.\ odd) whenever $m$ is even (resp.\ odd). It
follows that $g_n$ is always even (for any $n$) and consequently also $g$ is
even. Eventually, $\Psi^{|m|} g$ satisfies item (ii) of definition
\ref{def:Hmpair}, whereas (i) is straight\-forward and (iii) is obvious.
\end{proof}



\begin{remark} \rm
With a more thorough study using the proper estimates for the \little\
$q$-Bessel functions involved here, it may be possible to relax the conditions
on  the function $f$ a little bit, yet still ensuring the proper convergence in
(\ref{eq:lemma:qqE:series}). From the theory of entire functions and canonical
products, it is nevertheless clear that there exist plenty of functions
satisfying the conditions of the previous lemma. The $q$-analogue of the
exponential function described in appendix \ref{app:qcalc}\ provides an
important example: $\hfill \star$
\end{remark}



\begin{cor} \label{cor:qqE_in_holS}
Let\/ \qqE\ denote the entire function given by
$$  \qqE(z)   \:=\:   E_{q^2}(-z)
              \:=\:   (z;q^2)_\infty
              \:=\;   \prod_{k=0}^{\infty} \, (1-q^{2k} z). $$
Then\/ $\qqE \in \holS{m}$ for any\/ $m\in \ZZ$.
\end{cor}




\subsection{One single core susceptible to $q$-Hankel transforms of any order}

The problem with the system $(\holS{m},\holH{m})_{m\in \ZZ}$
is that we have to keep track of the order $m$ at any time.
It would certainly be convenient to have some kind of order-independent
domain on which all the $\holH{m}$ can act nicely;
so let's try to find such a \lq core\rq.



\begin{prop} \label{prop:qHankel:nesting}
If\/ $n,m\in \ZZ$ and\/ $|n| \leq |m|$, then\/ $\holS{n} \subseteq \holS{m}$.
\end{prop}

\begin{proof}
Recalling proposition \ref{prop:qHankel:inverseorder},
it suffices to prove that $\holS{m-1} \subseteq \holS{m}$ for all $m\geq 1$.
So let's take any $m\in \NN$ with $m\geq 1$ and any $g\in \holS{m-1}$.
Then $\Omega^{-2} \holH{m-1} \,g$ still belongs to $\holS{m-1}$,
hence we can apply the first formula of proposition \ref{prop:holqHankel:qdiff}\
with $f=\Omega^{-2} \holH{m-1} \, g$, yielding
\begin{equation}\label{eq:qHankel:nesting:proof}
  \holH{m} \Dqsqr \Omega^{-2} \holH{m-1}\, g
      \;=\; -\frac{q^{-1}}{q^{-1}-q} \,\holH{m-1} \Omega^2 \Omega^{-2} \holH{m-1}\,g
      \;=\; -\frac{q^{-1}}{q^{-1}-q} \, g.
      \;\;
\end{equation}
We conclude that $g$ belongs to $\holS{m}$.
\end{proof}



\begin{remark} \label{rem:Q:strict_inclusions} \rm
At present we lack an example which shows the inclusions in
proposition \ref{prop:qHankel:nesting} to be {\em strict\/} inclusions,
and as a matter of fact the spaces $\holS{m}$ for various $m\in\ZZ$ are not unlikely
to coincide. For instance from the second formula in proposition
\ref{prop:holqHankel:qdiff}\ one can derive that
$\Psi \holS{m} \subseteq \holS{m-1}$ for any $m\geq 1$, which yields an indication
for the reversed inclusions. In our approach to holomorphic $q$-Hankel transformation
however, it won't be easy to settle this question; but do we really
care?\footnote{no, not really; it will become clear very soon that we can perfectly
well proceed without knowing whether the $\holS{m}$ coincide or not. }
$\hfill \star$
\end{remark}



\begin{defn*} \label{def:Hcore}
We introduce two more subspaces of \HC\ as follows:
\begin{eqnarray*}
  \Hintersect &=& \bigcap_{m\in\ZZ} \holS{m}  \\
  \Hcore      &=& \left\{ f \in \Hintersect \left| \vertL \right.
                  \holH{m}f, \, \Dqsqr^m f \in \Hintersect
                  \mbox{ for all } m\in \NN  \right\} \\
\end{eqnarray*}
\end{defn*}



Observe that $\Hcore  \subseteq \Hintersect \subseteq \Swqbis$.
Also notice the definition of \Hcore\ only refers to {\em positive\/}
order $q$-Hankel transforms, which is justifiable in view of
propositions \ref{prop:qHankel:inverseorder}\ and \ref{prop:holS:Omega_invariant}\@.
Of course $\Hintersect$ is nothing but $\holS{0}$ (cf.\ proposition \ref{prop:qHankel:nesting}).
However we prefer to use this new symbol $\Hintersect$ (rather than $\holS{0}$)
to emphasize that it will be considered in relation to $q$-Hankel transforms
of {\em any\/} order---not just $\holH{0}$.
Now the problem with $\Hintersect$ is that it might
(cf.\ remark \ref{rem:Q:strict_inclusions})
not be invariant under $\holH{m}$ when $m \neq 0$, and moreover it
is not clear either whether $\Hintersect$ is invariant under $\Dqsqr$.
This is of course exactly the reason why we have introduced the space \Hcore\@.
Indeed we can prove the following:



\begin{prop} \label{prop:Hcore:invariance}
\hspace{2pt} \Hcore\ is invariant under\/ $\holH{m}$ for any\/ $m\in \ZZ$.
Furthermore, \Hcore\ is also invariant under\/ $\til$, $\Omega^{\pm 2}$,
$\Psi$ and\/ $\Dqsqr$.
\end{prop}

\begin{proof}
Obviously $\Hintersect$ is $\Omega^{\pm 2}$-invariant
(cf.\ proposition \ref{prop:holS:Omega_invariant}).
From (\ref{eq:holS:Omega_invariant}) and the commutation rule
$\Dqsqr \Omega^2 = q^2 \Omega^2 \Dqsqr$, it follows that also \Hcore\ is
invariant \mbox{under $\Omega^{\pm 2}$}\@.

Next we will show $\Dqsqr$-invariance; take any $f\in\Hcore$ and put $g=\Dqsqr f$.
Choose any $m\in \NN$ and first assume that $m\geq 1$.
Since $f\in \Hintersect \subseteq \holS{m-1}$, it follows
from the first formula in proposition \ref{prop:holqHankel:qdiff}\ that
$\holH{m}g = ({\rm scalar})\, \holH{m-1} \Omega^2 f$.
Now the latter belongs to $\Hintersect$ because $f\in\Hcore$.
Now consider the $m=0$ case. Since $f\in\Hcore$, we have
$g = \Dqsqr f \in \Hintersect = \holS{0}$, and hence
$\holH{0} g \in \Hintersect$.
So we have shown that $\holH{m}g \in \Hintersect$ for any $m\in\NN$,
whereas clearly $\Dqsqr^m \, g = \Dqsqr^{m+1} f \in \Hintersect$
for any $m\in\NN$. We conclude that $g\in\Hcore$.

Now we will show by induction on $m\in\NN$ that \Hcore\ is invariant under
$\holH{m}$. Notice it is sufficient to consider positive $m$ only,
because of proposition \ref{prop:qHankel:inverseorder}\@.
Let's first consider the $m=0$ case; take any $f\in\Hcore$, put $g=\holH{0}f$
and observe that $g\in\holS{0}=\Hintersect$.
We shall prove that $g$ belongs to \Hcore\ again;
therefore, choose any $n\in \NN$ and consider $\holH{n} g$ and $\Dqsqr^n g$.
When $n=0$, we obtain $\holH{0} g = \holH{0}^2 f = f \in \Hintersect$.
To deal with $n\geq 1$, rewrite the first formula in proposition
\ref{prop:holqHankel:qdiff}\ as
\begin{equation}\label{eq:holqHankel:qdiff:rewritten}
   \holH{n}  \:=\: ({\rm scalar})\, \Dqsqr \Omega^{-2} \holH{n-1}
\end{equation}
(see also: equation (\ref{eq:qHankel:nesting:proof}) above). Iterating this $n$ times yields
\begin{equation} \label{eq:holqHankel:qdiff:iterated}
   \holH{n}  \:=\: ({\rm scalar})\, (\Dqsqr \Omega^{-2})^n \holH{0}
             \:=\: ({\rm scalar})\, \Omega^{-2n} \Dqsqr^n \holH{0}
\end{equation}
We used the commutation rule for $\Dqsqr$ and $\Omega^{-2}$.
The exact value of the scalars involved here could be computed easily, though
they are not relevant for our purposes (except for the fact they are all non-zero).
Since $g$ belongs to $\holS{0}$ we may apply (\ref{eq:holqHankel:qdiff:iterated})
to it, yielding
$$ \holH{n} g     \:=\:    ({\rm scalar})\, \Omega^{-2n} \Dqsqr^n f. $$
Now the latter is in $\Hintersect$ because $f\in\Hcore$.
We have shown that $\holH{n} g \in \Hintersect$ for all $n\in \NN$.
On the other hand, (\ref{eq:holqHankel:qdiff:iterated}) is also useful in
the other direction:
$$ \Dqsqr^n \,g
       \:=\:   \Dqsqr^n \holH{0} f
       \:=\:   ({\rm scalar})\, \Omega^{2n} \holH{n}f. $$
It follows that $\Dqsqr^n \, g \in \Hintersect$ for all $n\in \NN$,
which completes the $m=0$ case. Now we proceed by induction on $m$.
Let's assume \Hcore\ to be invariant under $\holH{m-1}$ for some $m\geq 1$,
and prove that this implies invariance under $\holH{m}$.
Applying (\ref{eq:holqHankel:qdiff:rewritten}) to an arbitrarily chosen $f\in\Hcore$ yields
$$  \holH{m} f \:=\: ({\rm scalar})\, \Dqsqr \Omega^{-2} \holH{m-1} f. $$
By hypothesis $\holH{m-1} f$ belongs to \Hcore\ again, and since
we have already shown that \Hcore\ is invariant under $\Omega^{-2}$ and
$\Dqsqr$, the result follows.

Invariance under $\til$ follows easily from proposition \ref{prop:qHankel:tilde}\
together with the fact that $\til$ commutes with $\Dqsqr$.
It only remains to prove that \Hcore\ is $\Psi$-invariant.
Therefore consider the $m=1$ case of the second formula in
proposition \ref{prop:holqHankel:qdiff}\@. Plugging in (\ref{eq:def:nabq}) and
canceling some scalars, the formula can be rewritten as
$$ \holH{0} (\id - q^2 \Omega^2) \holH{1} g     \:=\:   \Psi g $$
for $g\in\holS{1}$ and a fortiori for $g\in\Hcore$.
Since we have already established that \Hcore\ is invariant under
$\holH{0}$, $\holH{1}$ and $\Omega^2$, the result follows.
\end{proof}



{\small
\begin{remark} \label{rem:qHankel:finetuning} \rm
Again (cf.\ remark \ref{rem:not_constructive}) our approach
is far from being explicit: still we lack any criterion which
describes the spaces \Hintersect\ and \Hcore\ in a direct way; we
merely {\em define\/} \Hcore\ to be the largest space on which one
can happily take iterated \mbox{$q$-Hankel}\ transforms of any order.
It is likely that one could do better than this, unraveling all the
details of \mbox{$q$-Hankel}\ transformation: for instance we
can perhaps obtain a richer theory if we stick to the original
system $(\holS{m},\holH{m})_{m\in \ZZ}$, keeping track of the
orders at any time\ldots (cf.\ \mbox{remark \ref{rem:calculus:finetuning}}).
In this respect, observe the striking similarity between the properties of
$(\holS{m})_{m\in \ZZ}$ given in proposition \ref{prop:holqHankel:qdiff},
and the conditions on the \lq second leg\rq\ of
$(\calL_m)_{m\in\ZZ}$ in remark \ref{rem:calculus:finetuning}).
Our main concern however, is to proceed into harmonic analysis on the quantum $E(2)$ group.
Nevertheless we will show---constructively---that \Hcore\ contains an
important class of functions.
$\hfill \star$
\end{remark}
}
