

\begin{abs_chp}
We prove Plancherel formulas for the Fourier transforms
constructed in the previous section, relating the Haar functional
on the quantum $E(2)$ group with the ones on its Pontryagin dual.
At the heart of the proof we encounter---not quite surprisingly---the
orthogonality relations (\ref{eq:qBessel:qHankel}) for
\little\ $q$-Bessel functions.
\end{abs_chp}



The following lemma is of the same nature as lemma \ref{lem:Haar_on_UpStarUp}\@.

\begin{lemma_sec}
Take any\/ $X,Y \in\calL(\Hcore_\tau)$ and\/ $r, s \in \NN$. Put\/ $m=\min\{r,s\}$.
Then
\begin{eqnarray*}
   \FRabb \!\left(\Upsilon(Y)\, b^r \vertM\right)^*
   \FRabb \!\left(\Upsilon(X)\, b^s \vertM\right)
\!\!&=&\!\!
    \sum_{k,l\in\ZZ}  \, (q^{-r} \alpha)^{l-k+s-r} \,
    (\gamma^*)^{r-m} \, \gamma^{s-m}
\\
\!\!&&\!\!
    \left\{ \! \Psi^m \!
    \left[ \left(\Omega^{-2(l-k+s-r)}\, \YaH{r}{k}Y\right)^{\!\!\sim} \!\!
           \left(\YaH{s}{l}X\right)
    \right] \vertXL \! \right\} \! (\gamma^*\gamma)
\\ \vertXL
   \FRabb \!\left(\Upsilon(Y)\, b^r \vertM\right)^*
   \FRabb \!\left(\Upsilon(X)\, c^s \vertM\right)
\!\!&=&\!\!
    \sum_{k,l\in\ZZ}  (q^{-r} \alpha)^{l-k-s-r} \, (\gamma^*)^{r+s}
\\
\!\!&&\!\!
    \left[ \left(\Omega^{-2(l-k-s-r)}\, \YaH{r}{k}Y\right)^{\!\!\sim} \!\!
           \left(\YaH{-s}{l}X\right)
    \right]   \!(\gamma^*\gamma)
\end{eqnarray*}
and similar formulas for
$$ \begin{array}{l}
    \FRabb\! \left(\Upsilon(Y)\, c^r \vertM\right)^*
    \FRabb\! \left(\Upsilon(X)\, b^s \vertM\right)
\\ \vertXL
    \FRabb \!\left(\Upsilon(Y)\, c^r \vertM\right)^*
    \FRabb \!\left(\Upsilon(X)\, c^s \vertM\right).
\end{array} $$
\rm Notice the summations over $k,l\in \ZZ$ are in fact {\em finite\/} sums
   (cf.\ lemma \ref{lemma:finite_support}).
\end{lemma_sec}


\begin{proof}
Using the commutation rules $\alpha\beta = q\,\beta\alpha$ and
lemma \ref{lem:functional_calc:Aq:properties}, we obtain
\begin{eqnarray*}
\lefteqn{\FRabb\!\left(\Upsilon(Y)\, b^r \vertM\right)^*
         \FRabb\!\left(\Upsilon(X)\,b^s \vertM\right) }  \\
&=&
    \left(\sum_{\;\:k\in\ZZ} \: \alpha^{k+r} \: \gamma^r
          \left(\YaH{r}{k}Y\right)\!(\gamma^*\gamma)
    \right)^{\!\!*}
    \left(\sum_{\;\:l\in\ZZ} \: \alpha^{l+s} \: \gamma^s
          \left(\YaH{s}{l}X\right)\!(\gamma^*\gamma) \right)  \\
&=&
    \sum_{k,l\in\ZZ} \:
    \left(\YaH{r}{k}Y\right)^{\!\!\sim} \!\!\!(\gamma^*\gamma) \;
    (\gamma^*)^r \, \alpha^{-k-r} \, \alpha^{l+s} \, \gamma^s
          \left(\YaH{s}{l}X\right)\!(\gamma^*\gamma) \\
&=&
    \sum_{k,l\in\ZZ} \: q^{-r(l-k+s-r)} \: \alpha^{l-k+s-r} \,
    (\gamma^*)^r  \, \gamma^s
 \hspace{-4em}\raisebox{-5ex}{$
    \left[ \left(\Omega^{-2(l-k+s-r)}\, \YaH{r}{k}Y\right)^{\!\!\sim} \!\!
           \left(\YaH{s}{l}X\right)
    \right]   \!(\gamma^*\gamma)  $}
\end{eqnarray*}
With lemma \ref{lem:functional_calc:Aq:properties}\ the first formula follows.
The other ones are similar.
\end{proof}
\vspace{2ex}



For the other Fourier transform, the situation is slightly different---although
the proof is based on the same techniques:


\begin{lemma_sec} \label{lemma:pre_plancherel:left}
Take any\/ $X,Y\in\calL(\Hcore_\tau)$ and\/ $r, s \in \NN$. Put\/ $m=\min\{r,s\}$.
Then
\begin{eqnarray*}
\lefteqn{\FLabb\! \left(\Upsilon(X)\, b^r \vertM\right) \,
         \FLabb\! \left(\Upsilon(Y)\, b^s \vertM\right)^*
\;=\;
               \sum_{k,l\in\ZZ}  \; q^{2(k+l+r+s)} \: q^{(r+s)(l+s)}   }\\
& \hspace{3em} &
    \alpha^{k-l+r-s} \, \gamma^{r-m} \, (\gamma^*)^{s-m}  \left\{ \! \Psi^m \, \Omega^{2(l+s)}
    \left[ \left(\YaH{r}{k}X\right) \! \left(\YaH{s}{l}Y\right)^{\!\!\sim}\,
    \right] \vertXL \! \right\} (\gamma^*\gamma).
\end{eqnarray*}
and similar formulas for
$$ \begin{array}{l}
   \FLabb\! \left(\Upsilon(X)\, b^r \vertM\right) \,
   \FLabb\! \left(\Upsilon(Y)\, c^s \vertM\right)^*
\\ \vertXL
   \FLabb\! \left(\Upsilon(X)\, c^r \vertM\right) \,
   \FLabb\! \left(\Upsilon(Y)\, b^s \vertM\right)^*
\\ \vertXL
   \FLabb\! \left(\Upsilon(X)\, c^r \vertM\right) \,
   \FLabb\! \left(\Upsilon(Y)\, c^s \vertM\right)^*
\end{array} $$
\end{lemma_sec}
\vspace{2ex}



The essence of our Plancherel formula lies within the proof of the next lemma.
Recall that we had fixed a Haar functional $\omega$ on
$\Aq\!\left({\mathcal S}_\nu(\RR^+;q^2)\right)$ and observe the following makes sense
because ${\mathcal S}_\nu(\RR^+;q^2)$ and
$\Aq\!\left({\mathcal S}_\nu(\RR^+;q^2)\vertM \right)$
are $^*$-algebras, containing respectively $\Hcore_\nu$ and $\Aq(\Hcore_\nu)$.



\begin{lemma_sec}
Take any\/ $X,Y\in\calL(\Hcore_\tau)$ and\/ $r, s \in \NN$. Then\/ $\omega$ vanishes on
\begin{eqnarray*}
   \FRabb\!\left(\Upsilon(Y)\, b^r \vertM\right)^*
   \FRabb\!\left(\Upsilon(X)\, c^s \vertM\right)
& \hspace{2em}  &
   \FRabb\!\left(\Upsilon(Y)\, c^r \vertM\right)^*
   \FRabb\!\left(\Upsilon(X)\, b^s \vertM\right)
\\
   \FLabb\! \left(\Upsilon(X)\, b^r \vertM\right) \:
   \FLabb\! \left(\Upsilon(Y)\, c^s \vertM\right)^*
& &
   \FLabb\! \left(\Upsilon(X)\, c^r \vertM\right) \:
   \FLabb\! \left(\Upsilon(Y)\, b^s \vertM\right)^*
\end{eqnarray*}
unless both $r$ and $s$ are zero. Moreover $\omega$ also vanishes on
\begin{eqnarray*}
   \FRabb\!\left(\Upsilon(Y)\, b^r \vertM\right)^*
   \FRabb\!\left(\Upsilon(X)\, b^s \vertM\right)
& \hspace{2em}  &
   \FRabb\!\left(\Upsilon(Y)\, c^r \vertM\right)^*
   \FRabb\!\left(\Upsilon(X)\, c^s \vertM\right)
\\
   \FLabb\! \left(\Upsilon(X)\, b^r \vertM\right) \:
   \FLabb\! \left(\Upsilon(Y)\, b^s \vertM\right)^*
&&
   \FLabb\! \left(\Upsilon(X)\, c^r \vertM\right) \:
   \FLabb\! \left(\Upsilon(Y)\, c^s \vertM\right)^*
\end{eqnarray*}
except when\/ $r=s=m$. In the latter case we actually have
\begin{eqnarray}
   \omega\! \left(\vertL  \FRabb\!\left(\Upsilon(Y)\, b^m \vertM\right)^*
                          \FRabb\!\left(\Upsilon(X)\, b^m \vertM\right)   \right)
\!\!&=&\!\!
   q^{-2m} \: \psi \! \left(\Upsilon \!\left(\id \tens \Psi^m \right)(Y^* X) \vertL\right)
\hspace{3em}
\label{eq:pre_plancherel:essence:Rb}
\\
    \omega\! \left(\vertL  \FRabb\!\left(\Upsilon(Y)\, c^m \vertM\right)^*
                          \FRabb\!\left(\Upsilon(X)\, c^m \vertM\right)   \right)
\!\!&=&\!\!
   q^{2m} \: \psi \! \left(\Upsilon \!\left(\id \tens \Psi^m \right)(Y^* X) \vertL\right)
\label{eq:pre_plancherel:essence:Rc}
\\
    \omega\! \left(\vertL   \FLabb\! \left(\Upsilon(X)\, b^m \vertM\right) \,
                           \FLabb\! \left(\Upsilon(Y)\, b^m \vertM\right)^*   \right)
\!\!&=&\!\!
    \varphi \!\left(\Upsilon \!\left(\id \tens \Psi^m \right)(X Y^*) \vertL\right)
\label{eq:pre_plancherel:essence:Lb}
\\
    \omega \! \left(\vertL   \FLabb\! \left(\Upsilon(X)\, c^m \vertM\right) \,
                           \FLabb\! \left(\Upsilon(Y)\, c^m \vertM\right)^*   \right)
\!\!&=&\!\!
    \varphi \!\left(\Upsilon \!\left(\id \tens \Psi^m \right)(X Y^*)  \vertL\right).
\label{eq:pre_plancherel:essence:Lc}
\end{eqnarray}
\end{lemma_sec}


\begin{proof}
All statements follow from the definition of $\omega$ and the previous lemmas.
Only the last set of formulas is not immediately clear. Therefore, observe that
for all $m\in\NN$
\begin{eqnarray*}
\lefteqn{\omega \left(\vertL  \FRabb\!\left(\Upsilon(Y)\, b^m \vertM\right)^*
\FRabb\!
                \left(\Upsilon(X)\, b^m \vertM\right) \right)}  \\
&=&
    \omega \left(\sum_{\;\:k\in\ZZ} \left\{ \Psi^m
    \left[ \left(\YaH{m}{k}Y\right)^{\!\!\sim} \!\!
           \left(\YaH{m}{k}X\right)
    \right] \vertXL \! \right\}  (\gamma^*\gamma) \right) \\
&=&
    \sum_{k\in\ZZ}\: \sum_{n\in\ZZ} \; (\nu q^{2n})^m
    \left[ \left(\YaH{m}{k}Y\right)^{\!\!\sim} \!\! \left(\YaH{m}{k}X\right)
    \right] \! (\nu q^{2n})\: q^{2n}   \\
&=&
    \sum_{k\in\ZZ}\: \kq(m,k)^2 \:
    \sum_{n\in\ZZ}\; q^{2n} \: (\nu q^{2n})^m \;
    (\Lambda_\nu \holH{m} \,g_k)^\sim (\nu q^{2n}) \;\,
    (\Lambda_\nu \holH{m} \, f_k)(\nu q^{2n})
\end{eqnarray*}
Here $f_k,  g_k \in \Hcore$ are given by (cf.\ definition \ref{def:YaH})
\begin{eqnarray*}
   f_k  &=&  \Lambda_\tau^{-1}\, (\evzero \tens \id)\, (\Gamma^{-1} \tens \Omega)^k X
\\
   g_k  &=&  \Lambda_\tau^{-1}\, (\evzero \tens \id)\, (\Gamma^{-1} \tens \Omega)^k \,Y
\end{eqnarray*}
for any $k\in \ZZ$. Below $\scal{\cdot}{\cdot}$ will denote the scalar product on
$L^2(\RR_q^+,m_q)$ as introduced at the beginning of \S\ref{sec:qHankel}\@.
Recall $H_m$ was a
unitary\footnote{Notice that unitarity of $H_m$ amounts to the orthogonality relations
(\ref{eq:qBessel:qHankel}).}\
operator on this Hilbert space (cf.\ definition \ref{def:qHankeltransform:unitary}).
We proceed with our computation:
\begin{eqnarray*}
&=& \sum_{k\in\ZZ} \: \nu^m \: \kq(m,k)^2 \:
    \sum_{n\in\ZZ} \: q^{2n} \:
    \overline{(R_q \Psi^m K \holH{m} \, g_k)(q^n)} \;
              (R_q \Psi^m K \holH{m} \, f_k)(q^n)   \\
&=& \sum_{k\in\ZZ} \: \nu^m \: \kq(m,k)^2 \:
    \sum_{n\in\ZZ} \: q^{2n} \; \overline{(\auxH{m} g_k)(q^n)}\; (\auxH{m} f_k)(q^n) \\
&=& \sum_{k\in\ZZ} \: \nu^m \: \kq(m,k)^2 \; \scal{\auxH{m} f_k}{\auxH{m} g_k} \\
&=& \sum_{k\in\ZZ} \: \nu^m \: \kq(m,k)^2 \;
      \left\langle H_m R_q \Psi^m K f_k  \,\left|\vertM\right.
                                            H_m R_q \Psi^m K g_k \right\rangle \\
&=& \sum_{k\in\ZZ} \: \nu^m \: \kq(m,k)^2 \;
      \left\langle R_q \Psi^m K f_k  \,\left|\vertM\right.
                                            R_q \Psi^m K g_k \right\rangle \\
&=& \sum_{k\in\ZZ} \: \nu^m \: \kq(m,k)^2 \:
    \sum_{n\in\ZZ} \: q^{2n} \; \overline{(\Psi^m K g_k)(q^n)}\; (\Psi^m K f_k)(q^n) \\
&=& \sum_{k\in\ZZ} \: \nu^m \: \kq(m,k)^2 \:
    \sum_{n\in\ZZ} \: q^{2n} \: q^{2nm} \; \overline{g_k(q^{2n})}\; f_k(q^{2n}) \\
&=& \sum_{k\in\ZZ} \: \nu^m \: \kq(m,k)^2 \:
    \sum_{n\in\ZZ} \: q^{2n} \: q^{2nm} \;
             \overline{Y(k\theta, \tau q^{2n+k})} \; X(k\theta, \tau q^{2n+k})    \\
&=& \sum_{k,n\in\ZZ} \:  \nu^m \: \kq(m,k)^2 \:  q^{2n} \: q^{2nm} \;
       (Y^*X)(k\theta, \tau q^{2n+k})    \\
&=& \sum_{(k,l)\in\mathfrak{S}} \:  \nu^m \: \kq(m,k)^2 \:  q^{l-k} \: q^{(l-k)m} \;
       (Y^*X)(k\theta, \tau q^l)    \\
\end{eqnarray*}
where
\begin{equation}\label{eq:spectral:summ_range}
   \mathfrak{S}  \:=\:  (\ZZeven \times \ZZeven) \,\cup\, (\ZZodd \times \ZZodd).
\end{equation}
So in the last step we performed a rearrangement of our summation variables,
according to $l=2n+k$.
Furthermore $X$ and $Y$ were interpreted as functions in two variables---in the obvious way.
Before we proceed with the computation, let's have a closer look at
some of the scalars involved here:
\begin{eqnarray*}
\lefteqn{\nu^m \: \kq(m,k)^2 \: q^{(l-k)m}}
\\&=&
(q^{-1}-q)^{-2m} \left(\vertL(-1)^m \, q^{-m} \, q^{\frac{1}{2}|m|(k-1)} \,
                         (q^{-1}-q)^{|m|} \right)^2  q^{(l-k)m}
\\&=&
q^{-3m} \: q^{lm}
\\&=&
q^{-2m} \: (\tau q^l)^m.
\end{eqnarray*}
Notice that we used our particular choices for $\tau$ and $\nu$. Eventually we obtain
\begin{eqnarray*}
    \omega \!\left(\vertL  \FRabb \!\left(\Upsilon(Y)\, b^m \vertM\right)^*
                           \FRabb \!\left(\Upsilon(X)\, b^m \vertM\right) \right)
\!\!&=&\!\!\!
    q^{-2m}  \!\!\! \sum_{(k,l)\in\mathfrak{S}}  \! (\tau q^l)^m \:
                       (Y^*X)(k\theta, \tau q^l) \, q^{-k+l}
\\
\!\!&=& \!\!\!
    q^{-2m} \; \psi \! \left(\Upsilon \!\left(\,\id \tens \Psi^m \right)(Y^* X) \vertL\right).
\end{eqnarray*}
Here we used the $\psi$-analogue of (\ref{eq:haar:Uq:concrete})\@.
This proves (\ref{eq:pre_plancherel:essence:Rb}).
Equation (\ref{eq:pre_plancherel:essence:Rc}) is shown similarly, though one
should be careful with the signs, since here {\em negative\/} order $q$-Hankel
transforms are involved:
\begin{eqnarray*}
\lefteqn{ \omega \!\left(\vertL \FRabb \!\left(\Upsilon(Y)\, c^m \vertM\right)^*
                \FRabb \!\left(\Upsilon(X)\, c^m \vertM\right) \right)}
\\&\hspace{4em}=&
    \omega \! \left(\sum_{\:k\in\ZZ} \left\{ \Psi^m \!
    \left[ \left(\YaH{-m}{k}Y\right)^{\!\!\sim} \!\!
           \left(\YaH{-m}{k}X\right)
    \right] \vertXL \! \right\}  \! (\gamma^*\gamma) \right)
\;=\;  \ldots
\end{eqnarray*}
So now we have to use $\kq(-m,k)$ instead of $\kq(m,k)$.

To prove (\ref{eq:pre_plancherel:essence:Lb}) we start from
lemma \ref{lemma:pre_plancherel:left}:
\begin{eqnarray*}
\lefteqn{ \omega \left(\vertL
     \FLabb\! \left(\Upsilon(X)\, b^m \vertM\right) \,
     \FLabb\! \left(\Upsilon(Y)\, b^m \vertM\right)^*   \right)  }\\
&=&\!\!
    \omega \left( \sum_{\,k \in\ZZ}  \, q^{4(k+m)} \, q^{2m(k+m)}
    \left\{ \! \Psi^m \, \Omega^{2(k+m)}
    \left[ \left(\YaH{m}{k}X\right) \! \left(\YaH{m}{k}Y\right)^{\!\!\sim}\,
    \right] \vertXL \! \right\} (\gamma^*\gamma) \right)
\\&=&\!\!
    \sum_{k \in\ZZ}   q^{4(k+m)}  q^{2m(k+m)}
    \sum_{n\in\ZZ}  q^{2n}  (\nu q^{2n})^m \!
    \left[ \left(\YaH{m}{k}X\right) \! \left(\YaH{m}{k}Y\right)^{\!\!\sim}\,
    \right] \! \left(\nu q^{2(n+k+m)}\right)
\\
&\stackrel{(*)}{=}&\!\!
    \sum_{k\in\ZZ} \, q^{4(k+m)} \, q^{2m(k+m)} \,
    \sum_{n\in\ZZ} \, q^{2(n-k-m)} \, \nu^m \, q^{2(n-k-m)m}
 \hspace{-4em}\raisebox{-4.5ex}{$
    \left[ \left(\YaH{m}{k}X\right) \! \left(\YaH{m}{k}Y\right)^{\!\!\sim}\,
    \right] (\nu q^{2n})  $}
\\&=&\!\!\vertXXL
    \sum_{k \in\ZZ} \: q^{2(k+m)} \,
    \sum_{n\in\ZZ} \, q^{2n} \, (\nu q^{2n})^m
    \left[ \left(\YaH{m}{k}X\right) \! \left(\YaH{m}{k}Y\right)^{\!\!\sim}\,
    \right]  (\nu q^{2n})
\\
&\stackrel{(\sharp)}{=}&\!\!
    \sum_{k\in\ZZ} \; q^{2(k+m)} \, \nu^m \, \kq(m,k)^2 \,
    \sum_{n\in\ZZ} \,  q^{2n} \, q^{2nm} \:  (X Y^*)(k\theta, \tau q^{2n+k})
\\&=&\!\!
    \sum_{(k,l)\in\mathfrak{S}} q^{2(k+m)} \,  q^{-2m} \, (\tau q^l)^m \;
    (X Y^*)(k\theta, \tau q^l) \: q^{-k+l}
\\&=&\!\!
    \sum_{(k,l)\in\mathfrak{S}} (\tau q^l)^m \;
          (X Y^*)(k\theta, \tau q^l) \; q^{k+l}
\\&=&\!\!
    \varphi \! \left(\Upsilon \!\left(\id \tens \Psi^m \right)(X \, Y^*) \vertL\right).
\end{eqnarray*}
In $(*)$ we substituted $n$ with $n-k-m$. From $(\sharp)$ on, the computation
proceeds completely analogous to the one made above in proving
(\ref{eq:pre_plancherel:essence:Rb}).
Thus we have shown (\ref{eq:pre_plancherel:essence:Lb}).
Equation (\ref{eq:pre_plancherel:essence:Lc}) is analogous.
\end{proof}



\begin{remark_sec} \rm
Observe how the \lq spectral conditions\rq\ (cf.\ remark \ref{rem:haar:Uq:concrete})
incorporated by (\ref{eq:spectral:summ_range})
emerge naturally in the above computation, although here it has nothing
to do with any operator theoretic  considerations whatsoever; this is quite remarkable.
Also notice that---once again---the particular values for the
\lq dilation\rq\ parameters $\tau$ and $\nu$ chosen in (\ref{eq:tau_nu:value}) seem
to be quite indispensable in our computations.
\hfill $\star$
\end{remark_sec}



\begin{thm_sec} \label{thm:Eq2:Plancherel}
The Fourier transforms\/ $\FRabb$ and\/ $\FLabb$ given in definition
\ref{def:Eq2:Fourier_transforms}\ enjoy the following Plancherel formulas:
$$  \psi(y^* x)     \;=\;  \omega\left(\vertM \FRabb(y)^* \FRabb(x) \right)
                \andspace{2em}
    \varphi(x y^*)  \;=\;  \omega\left(\vertM \FLabb(x) \, \FLabb(y)^* \right) $$
for any\/ $x,y\in \Uq\!\left(\vertM \calL(\Hcore_\tau)\right)$.
\end{thm_sec}

\begin{proof}
Recall that any $x,y\in \Uq\!\left(\vertM \calL(\Hcore_\tau)\right)$ can be written as
\begin{eqnarray*}
  x  &=&  \sum_{m=0}^\infty \Upsilon(X_m)\, b^m \,+\,
          \sum_{m=1}^\infty \Upsilon(X_{-m})\, c^m
\\
  y  &=&  \sum_{m=0}^\infty \Upsilon(Y_m)\, b^m \;+\,
          \sum_{m=1}^\infty \Upsilon(Y_{-m})\, c^m
\end{eqnarray*}
with only {\em finitely many\/} non-zero $X_n, Y_n \in \calL(\Hcore_\tau)$.
Applying consecutively the $\psi$-analogon of (\ref{eq:haar:Uq:concrete})
and the previous lemma, we obtain
$$ \psi(y^* x) \;=\; \sum_{n\in\ZZ}  q^{-2n} \: \psi \!
          \left(\Upsilon(\,\id \tens \Psi^{|n|})(Y_n^* X_n) \vertL\right)   \\
\;=\; \omega\left(\vertM \FRabb(y)^* \FRabb(x) \right).  $$ Notice how the
$n\geq 0$ and $n<0$ cases have been conveniently merged into a single summation
over all $n\in\ZZ$.

The second Plancherel formula is analogous; however, instead of the formulas in
lemma \ref{lem:Haar_on_UpStarUp}\ we need analogous formulas like for instance
$$\left(\Upsilon(X)\, b^m  \vertM\right)
          \left(\Upsilon(Y)\, b^m  \vertM\right)^*
    \;=\; \Upsilon \!\left(\id \tens \Psi^m \right)(X Y^*)
    \;=\; \left(\Upsilon(X)\, c^m  \vertM\right)
          \left(\Upsilon(Y)\, c^m  \vertM\right)^*  $$
which are equally easy to prove.
\end{proof}




\begin{remark_sec}  \rm
Notice that in dealing with $\FRabb$ we have used expressions of type $y^*x$
whereas concerning $\FLabb$ we considered expressions of type $x y^*$. It is an
instructive exercise to see what goes wrong when e.g.\ one tries to use $y^*x$
in dealing with $\FLabb$.
\hfill $\star$
\end{remark_sec}
