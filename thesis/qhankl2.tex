

\begin{abs_chp}
$q$-analogues of Hankel transformation based on \little\ \mbox{$q^2$-Bessel}\
functions first appeared explicitly in \cite{KoornwSwartt}, whereas
in \cite{leonid:vainerman}\ they were mentioned in \mbox{relation}\ to
the quantum $E(2)$ group. The starting point for our theory will
be the definition proposed in \cite{KoornwSwartt},
which we will recall below---reformulated in an \mbox{$L^2$-language}\@.
Then we study  the possibility of transforming {\em entire\/} functions into entire
functions, a feature which is
crucial\footnote{in the sense that we want to stick to our algebraic description
of the quantum $E(2)$ group, as outlined in chapter \ref{chapter:Fourier_context_for_quantumE2},
the algebraic nature of which depends heavily on the use of entire functions
in constructing the functional calculus.}\ for our applications to the
quantum $E(2)$ group.
Furthermore we investigate the behaviour of \mbox{$q$-Hankel}\ transformation
w.r.t.\ \mbox{$q$-differentiation}\ and multiplication.
Eventually we search for {\em eigenfunctions\/} of \mbox{$q$-Hankel}\ transformation;
these turn out to be functions of \mbox{$q^2$-Exponential}\ type.
\end{abs_chp}



\subsection{The $L^2$-theory}

Let $\RR_q^+ \equiv (\RR_q^+,m_q)$ be the discrete set $\{ q^k \,|\, k\in\ZZ \}$
endowed with the measure $m_q$ which assigns weight $q^{2k}$ to the
point $q^k$. Thus integration w.r.t.\ $m_q$ yields
$$ \int_{\RR_q^+} f\,dm_q
               \:=\: \sum_{k\in \ZZ} f(q^k)\,q^{2k}
               \hspace{3em} \mbox{for} \hspace{1em}  f\in L^1(\RR_q^+).  $$
Now the orthogonality relations (\ref{eq:qBessel:qHankel})
can be reformulated in an $L^2$-language:


\begin{prop}
Fix any\/ $m\in\ZZ$ and define functions\/ $\basis{m}{k}$ on\/ $\RR_q^+$ by
$$  \basis{m}{k}(x) \:=\: q^k \,\J{m}{q^k x}
                       \hspace{5em} (k\in \ZZ,\: x\in\RR_q^+). $$
Then\/ $(\basis{m}{k})_{k\in \ZZ}$ is an orthonormal basis ({\scriptsize ONB}) in
the Hilbert space\/ \Ltwoq\@.
\end{prop}


On the other hand, we have a canonical {\sc onb}\ for \Ltwoq,
say $(d_k)_{k\in \ZZ}$ with $d_k = q^{-k} \delta_{q^k}$.
Here $\delta_{q^k}$ denotes the characteristic function of the singleton $\{q^k\}$.


\begin{defn} \label{def:qHankeltransform:unitary}
Given $m\in\ZZ$, let $H_m$ be the unitary transformation of \Ltwoq\
which maps the {\sc onb} $(\basis{m}{k})_{k\in \ZZ}$
into the {\sc onb} $(d_k)_{k\in \ZZ}$.
Explicitly:
$$ H_m f \:=\:\sum_{k\in \ZZ} \:\scal{f}{\basis{m}{k}}\,d_k , $$
for $f\in \Ltwoq$ and $k\in\ZZ$, or equivalently,
\begin{equation}\label{eq:qHankeltransform:def}
  (H_m f)(q^k) \;=\;  q^{-k} \scal{f}{\basis{m}{k}}
               \;=\;  \sum_{n\in \ZZ} \: q^{2n} \J{m}{q^{n+k}} f(q^n).
  \hspace{1em}
\end{equation}
$H_m$ is said to be the {\em $q$-Hankel transform\/} of order $m$.
\end{defn}



\begin{prop} \label{prop:qHankelSquare_is_id}
$\: H_m^2 = \id$ for all\/ $m\in\ZZ$.
\end{prop}
\begin{proof}
It suffices to show that $H_m d_j = \basis{m}{j}$ for all $j\in \ZZ$.
Now observe that
\begin{eqnarray*}
(H_m d_j)(q^k)
&=&
\textstyle \sum_{n\in \ZZ} \: q^{2n} \: \J{m}{q^{n+k}} \: d_j(q^n)
\\&=&
q^{2j} \: \J{m}{q^{j+k}} \: q^{-j}
\\&=&
q^j \: \J{m}{q^j q^k}
\\&=&
\basis{m}{j}(q^k)
\end{eqnarray*}
for all $k \in \ZZ$.
\end{proof}
